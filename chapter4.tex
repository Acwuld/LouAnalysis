\chapter{导数与微分}
\section{导数与微分}
\precis{导数的几何物理背景,一元向量值函数的导数,左导数,右导数,导数与单调性,方向导数与偏导数,全导数,微分,线性变换/线性算子,微商,梯度,导数的四则运算}
\begin{quiza}
\woe 设\(x_0\in (a,b),\, f:(a,b)\rightarrow\mathbb{R}\)在点\(x_0\)连续.
\begin{quizs}
\item 证明: \(f\)在点\(x_0\)可导当且仅当\(\lim_{\substack{t,s\rightarrow 0\\ts<0}}\frac{f(x_0+t)-f(x_0+s)}{t-s}\)存在.
\begin{proof}
\[\begin{split}
\lim_{\substack{t,s\rightarrow 0\\ ts<0}}\frac{f(x_0+t)-f(x_0+s)}{t-s}&=\lim_{t\rightarrow 0^+}\lim_{s\rightarrow 0^-}\frac{f(x_0+t)-f(x_0+s)}{t-s}\\&=\lim_{t\rightarrow 0^+}\frac{f(x_0+t)-f(x_0)}{t}=f_+'(x_0).
\end{split}\]同理可证\[\lim_{\substack{t,s\rightarrow 0\\ ts<0}}\frac{f(x_0+t)-f(x_0+s)}{t-s}=f'_-(x_0).\]
\end{proof}
\item 试讨论\(f\)在点\(x_0\)的可导性与极限\(\lim_{\substack{t,s\rightarrow 0\\ ts>0,t\ne s}}\frac{f(x_0+t)-f(x_0+s)}{t-s}\)的存在性之间的关系.
\tcbline
若\(f\)在点\(x_0\)可导, 则其左右导数都存在且相等, 可以推出题设二次极限存在. 仿照上述过程, \(f\)在点\(x_0\)可导可以推出\(\lim_{\substack{t,s\rightarrow 0\\ ts>0,t\ne s}}\frac{f(x_0+t)-f(x_0+s)}{t-s}\)存在, 反之不然.
\end{quizs}
\woe 设\(x_0\in(a,b),\,f:(a,b)\rightarrow\mathbb{R}\). 试讨论\(f'_+(x_0)\)与\(f'(x_0^+)\)之间的关系(包括他们的存在性).
\tcbline
根据导数定义\[f_+'(x_0)=\lim_{x\rightarrow x_0^+}\frac{f(x)-f(x_0)}{x-x_0},\qquad f'(x_0^+)=\lim_{t\rightarrow x_0^+}\lim_{x\rightarrow t}\frac{f(x)-f(t)}{x-t}.\]这样如果导数存在且连续, 则\(f_+'(x_0)\)与\(f'(x_0^+)\)存在且相等. 如果导数在\(x_0\)处不存在, 则二者可能不等, 如取\(f(x)=|x|,x=0\), 则前者为1而后者不存在.
\woe 设\(\boldsymbol{\lambda}\in\mathbb{R}^n\)为常向量, \(\boldsymbol{f}(\boldsymbol{x})=\boldsymbol{x}\cdot\boldsymbol{\lambda}(\boldsymbol{x}\in\mathbb{R}^n)\). 试计算\(f_{\boldsymbol{x}},\nabla f\).
\begin{solution}
\(f_{\boldsymbol{x}}=(\nabla f)^\mathrm{T}=\boldsymbol{\lambda}\).
\end{solution}
\woe 设\(\boldsymbol{A}\)为\(m\times n\)常数矩阵, \(\boldsymbol{f}(\boldsymbol{x})=\boldsymbol{Ax}(x\in\mathbb{R}^n)\). 试计算\(\boldsymbol{f_x}\).
\begin{solution}
\(\boldsymbol{f_x}=\boldsymbol{A}\).
\end{solution}
\woe 设\(\boldsymbol{A}\)为\(n\)阶常数方阵, \(f(\boldsymbol{x})=\boldsymbol{x}^T\boldsymbol{A}\boldsymbol{x}(x\in\mathbb{R}^n)\). 试计算\(f_{\boldsymbol{x}},\nabla f\).
\begin{solution}
\(\nabla f=(f_{\boldsymbol{x}})^\mathrm{T}=(\boldsymbol{A}+\boldsymbol{A}^\mathrm{T})\boldsymbol{x}\).
\end{solution}
\woe 设\(\boldsymbol{A}\)为\(m\times k\)常数矩阵, \(n=k+m,\,f(\boldsymbol{x})=(x_1,x_2,\cdots,x_m)\boldsymbol{A}(x_{m+1},x_{m+2},\cdots,x_n)^T(x\in\mathbb{R}^n)\). 试计算\(f_{\boldsymbol{x}},\nabla f\).
\begin{solution}
\(\nabla f=(f_{\boldsymbol{x}})^\mathrm{T}=[\boldsymbol{A}(x_{m+1},\cdots,x_{n});\boldsymbol{A}^\mathrm{T}(x_1,\cdots,x_m)]\).
\end{solution}
\woe 利用\(\lim_{x\rightarrow 0}\frac{\sin x}{x}=1\)求\[\lim_{n\rightarrow+\infty}\left(\sin\frac{1}{n^2}+\sin\frac{3}{n^2}+\cdots+\sin\frac{2n-1}{n^2}\right).\]
\begin{solution}
注意到\[\lim_{n\rightarrow+\infty}\frac{f\left(\displaystyle\frac{k}{n}\right)-f(0)}{\displaystyle\frac{k}{n^2}}=f'(0),\]其中\(f(x)=\sin x\), 于是\(\sin\frac{k}{n^2}=\frac{k}{n^2}+o\left(\frac{k}{n^2}\right)\), 于是原极限为1.
\end{solution}
\woe 设\(f(x)=a_1\sin x+a_2\sin 2x+\cdots+a_n\sin nx\), 且\(|f(x)|\leqslant|\sin x|\), 其中\(a_1,a_2,\cdots,a_n\)为常数. 求证: \(|a_1+2a_2+\cdots+na_n|\leqslant 1\).
\begin{proof}
由于\[f'(x)=a_1+2a_2\cos 2x+\cdots+na_n\cos nx,\]而\(\lim_{x\rightarrow 0}\left|\frac{f(x)}{\sin x}\right|=|f'(0)|\leqslant 1.\)从而得证.
\end{proof}
\end{quiza}
\begin{quizb}
\woe 试构造\(\mathbb{R}\)上的实函数\(f\)使得它仅在点\(0\)连续, 且在点0的导数为1.
\begin{solution}
设\[f(x)=\begin{cases}
x,\quad &x\text{是有理数},\\
x^2+x,&x\text{是无理数}.
\end{cases}\]容易验证\(f\)仅在\(x=0\)处连续且可导, 并且\(f'(0)=1\).
\end{solution}
\woe 设\(f\)在区间\((a,b)\)处处可导, 证明: 存在区间\([\alpha,\beta]\subset (a,b)\)使得\(f'\)在\([\alpha,\beta]\)上有界.
\begin{proof}
取\(\delta=\frac{b-a}{6}\), 令\(a_1=a+\delta,b_1=b-\delta\), 若能证得存在区间\([\alpha,\beta]\subset[a_1,b_1]\), 则题设结论得证. 下面我们证明这个结论.

令\(a_2=\frac{a_1+b_1}{2}\), 若\(f'\)在\([a_1,a_2]\)或\([a_2,b_1]\)任何一个区间内有界, 则结论得证, 否则令\(b_2=\frac{a_2+b_1}{2}\), 若\(f'\)在\([a_2,b_2]\)或\([b_2,b_1]\)任何一个区间有界, 则结论得证, 否则令\(a_3=\frac{a_2+b_2}{2}\), 若\(f'\)在\([a_2,a_3]\)或\([a_3,b_2]\)任何一个区间有界, 则结论得证, 否则重复这个操作,即令\(a_{n+1}=\frac{a_n+b_n}{2},b_{n+1}=\frac{a_{n+1}+b_n}{2}\). 我们断言: 在有限步内一定能找到区间使得\(f'\)在该区间内有界. 否则, 这一系列操作中产生的任何区间都使\(f'\)无界, 特别地, \(f\)在闭区间列\(\{[a_n,b_n]\}\)上始终无界, 易见\([a_{n+1},b_{n+1}]\subset [a_n,b_n]\)并且\(\lim_{n\rightarrow+\infty}\left(b_n-a_n\right)=0\), 从而\(\{[a_n,b_n]\}\)形成闭区间套, 依闭区间套定理, \(\exists\xi\in[a_n,b_n]\),且\(\lim_{n\rightarrow+\infty}a_n=\lim_{n\rightarrow+\infty}b_n=\xi\).  并且\(f'(\xi)=\infty\)与\(f\)处处可导矛盾.
\end{proof}
\woe 设\(n\)元实函数\(f\)在点\(0\)的一个邻域内有定义, 满足\(f(\boldsymbol{0})=0\). 进一步, 对任何\(\boldsymbol{e}\in S^{n-1}\), 成立\(\lim_{r\rightarrow 0^+}\frac{f(r\boldsymbol{e})}{r}=0\)以及\(\lim_{\substack{(r,\boldsymbol{\sigma})\rightarrow (0,\boldsymbol{e})\\ r>0,\boldsymbol{\sigma}\in S^{n-1}}}\frac{f(r\boldsymbol{\sigma})-f(r\boldsymbol{e})}{r}=0\). 证明: \(f\)在点\(\boldsymbol{0}\)可微.
\begin{proof}
	
\end{proof}
\woe 设\(\boldsymbol{A}\)为\(n\)阶实方阵, 则存在\(\delta>0\), 使得当\(0<|s|<\delta\)时, 矩阵\[\boldsymbol{A}(s)=\boldsymbol{A}+\left(\begin{matrix}
    s&0&\cdots&0\\0&s^2&\cdots&0\\\vdots&\vdots&&\vdots\\0&0&\cdots&s^n\\
\end{matrix}\right)\]的特征值互不相同.

提示: 先证明对充分大的\(s\), \(\boldsymbol{A}(s)\)的特征值互不相同.
\begin{proof}
先证当\(s\)充分大时, \(\boldsymbol{A}(s)\)有\(n\)个不同的特征值. 不妨记\(\boldsymbol{A}=(a_{ij})\), 由Gerschgorin圆盘第一定理, 可知\(\boldsymbol{A}(s)\)的特征值落在下列圆盘中:\[D_i=\left|z-a_{ii}-s^i\right|\leqslant R_i=\sum_{j=1,j\ne i}^{n}|a_{ij}|,\quad 1\leqslant i\leqslant n.\]取\(s\)充分大, 使得\(s^n\gg s^{n-1}\gg \cdots\gg s\). 注意到\(R_i\)的值固定, 故\(D_i\)的圆心之间的距离大于半径\(R_i\), 从而\(D_i\)互不相交, 各自构成了一个连通分支. 再由第二圆盘定理, 每个连通分支\(D_i\)中有且仅有一个特征值, 于是\(\boldsymbol{A}(s)\)有\(n\)个不同的特征值.

设\(f_s(\lambda)=\left|\lambda\boldsymbol{I}_n-\boldsymbol{A}(s)\right|\)是\(\boldsymbol{A}(s)\)的特征多项式, 则其判别式\(\Delta\left(f_s(\lambda)\right)\)是关于\(s\)的多项式. 由前面的讨论可知, 当\(s\)充分大时, \(f_s(\lambda)\)无重根, 从而\(\Delta\left(f_s(\lambda)\right)\ne 0\), 即\(\Delta\left(f_s(\lambda)\right)\)是关于\(s\)的非零多项式. 

若\(\Delta\left(f_s(\lambda)\right)\)的所有复根都是零, 则任取一个正数\(\delta\); 若\(\Delta\left(f_s(\lambda)\right)\)的复根不全为零, 则可取\(\delta\)为\(\Delta\left(f_s(\lambda)\right)\)的非零复根的模长的最小值. 于是对任意的\(s\in(0,\delta)\), \(s\)都不是\(\Delta\left(f_s(\lambda)\right)\)的根, 即\(\Delta\left(f_s(\lambda)\right)\ne 0\), 从而\(f_s\left(\lambda\right)\)都无重根, 即\(\boldsymbol{A}(s)\)都有\(n\)个不同的特征值.
\end{proof}
\woe 设\(D\subseteq\mathbb{C}\)为开集, \(f:D\rightarrow\mathbb{C}\)可表示为\(f(x+\mathrm{i}y)=P(x,y)+\mathrm{i}Q(x,y)\), 其中\(P,Q\)均为实值函数. 设\(x_0+\mathrm{i}y_0\in D\), 试讨论\(P,Q\)满足什么充要条件时, 极限\[\lim_{(x,y)\rightarrow(x_0,y_0)}\frac{f(x+\mathrm{i}y)-f(x_0+\mathrm{i}y_0)}{(x+\mathrm{i}y)-(x_0+\mathrm{i}y_0)}\]存在.
\begin{solution}
	参考一般的复变函数教材。
\end{solution}
\woe 设\(f\)为\([0,1]\)上处处可导的实函数, 0和1均为\(f\)的不动点, 且\(f'(0)>1,f'(1)>1\), 证明:
\begin{quizs}
\item \(f\)在\((0,1)\)中也有不动点.
\item 存在\(f\)的不动点\(\xi\in(0,1)\)使得\(f'(\xi)\leqslant 1\).
\end{quizs}
\begin{proof}
(1)设\(F(x)=f(x)-x\), 则\(F(0)=F(1)=0\). 由于\(F'(0)=f'(0)-1>0\), 则有\(\varepsilon_1>0\)使得\(F(\varepsilon_1)>F(0)=0\); 同理由\(F'(1)>0\), 有\(\varepsilon_2>0\)使得\(F(1-\varepsilon_2)<F(1)=0\), 由连续函数的介值定理可知有\(\xi\in(\varepsilon_1,1-\varepsilon_2)\)使得\(F(\xi)=0\), 即为\(f(x)\)在\((0,1)\)中的不动点.

(2) 由(1)知\(f\)在\((0,1)\)上有不动点, 记\(\xi\)为这些不动点中最大的那一个. 若\(f'(\xi)>1\), 即\(F'(\xi)>0\), 则仿(1)可证有\(\xi'\in(\xi,1)\)使得\(F(\xi')=0\), 即\(\xi'\)为比\(\xi\)更大的不动点, 矛盾.
\end{proof}
\woe 试仿照习题\(\boldsymbol{\mathcal{A}}\)第7题和第8题编写一些习题.
\end{quizb}
\section{反函数, 复合函数和隐函数的导数}
\precis{一元实函数反函数的可导性及求导公式,复合函数的导数,链式法则,一阶微分形式不变性,隐函数求导,基本初等函数的导数,对数求导法}
\begin{quiza}
\woe 证明对任何实数\(x\)成立\(\ee^x\geqslant 1+x\)以及\(\ee^x\geqslant \ee x\).
\begin{proof}
记\(f(x)=\ee^x-1-x\),由\[f'(x)=\ee^x-1\geqslant f'(0)=0,\]易见\(x=0\)是\(f(x)\)的最小值, 即\(f(x)\geqslant f(0)=0\), 从而得证. 后者只需置\(x\)为\(x-1\)带入即得\(\ee^{x-1}\geqslant 1+(x-1)\)即\(\ee^{x}\geqslant\ee x\).
\end{proof}
\woe 试改进不等式\(\frac{x}{1+x}<\ln(1+x)<x(x>0)\).
\begin{quizs}
\item 证明对于\(x>0\), 成立\(\ln (1+x)>\frac{x}{1+x/2}\).

 进一步, 试求最小的\(\alpha\), 使得对于任何\(x>0\)成立\(\ln(1+x)>\frac{x}{1+\alpha x}\).
\item 证明对于\(x>0\), 成立\(\ln (1+x)<\frac{x}{2}+\frac{x}{2(1+x)}\).

进一步, 试求最小的\(a\), 使得对于任何\(x>0\), 成立\(\ln (1+x)<ax+(1-a)\frac{x}{1+x}\).
\end{quizs}
\begin{proof}
(1)记\(F(x)=\ln(1+x)-\frac{x}{1+\alpha x}\), 则\[F'(x)=\frac{1}{1+x}-\frac{1}{(1+\alpha x)^2}=\frac{x(\alpha^2x-(1-2a))}{(1+x)(1+\alpha x)^2},\]
由于\(F(0)=0\), 而当\(\alpha=\frac{1}{2}\)时\(F'(x)>0\), 从而题设不等式成立. 进一步, \(\alpha\)的最小值便是\(\frac{1}{2}\), 否则\(F(x)\)在\(\left(0,\frac{1-2\alpha}{\alpha^2}\right)\)单调递减.

(2)记\(G(x)=\ln(1+x)-ax-(1-a)\frac{x}{1+x}\), 则\[G'(x)=\frac{1}{1+x}-a-\frac{1-a}{(1+x)^2}=\frac{x-a(1+x^2+2a)+a}{(1+x)^2}=\frac{-x(ax-(1-2a))}{(1+x)^2},\]当\(a=\frac{1}{2}\)时, \(G'(x)<0\)而\(G(0)=0\), 从而不等式成立. 进一步, \(a\)的最小值是\(\frac{1}{2}\), 否则\(G(x)\)在\(\left(0,\frac{1-2a}{a}\right)\)上单调递增.
\end{proof}
\woe 试证明对于\(x\in\left(0,\frac{\pi}{2}\right)\), 成立\(x<\frac{\sin x}{\sqrt{\cos x}}\).\\ 进一步, 试求最小的\(\alpha\), 使得对于任何\(x\in\left(0,\frac{\pi}{2}\right)\), 成立\(x<\frac{\sin x}{\cos^\alpha x}\).
\begin{proof}
设\(F(x)=x-\frac{\sin x}{\cos^{\alpha}x}\), 那么\[F'(x)=1-\frac{\cos^{\alpha+1}x+\alpha\cos^{\alpha-1}x\sin^2x}{\cos^{2\alpha}x}=\frac{\cos^{\alpha+1}x-(1-\alpha)\cos^2x-\alpha}{\cos^{\alpha+1}x},\]考虑\(f(t)=\cos^{\alpha+1}x-(1-\alpha)\cos^2x-\alpha=(1-t)^{\alpha+1}-(1-\alpha)(1-t)^2-\alpha\), 其中\(t=1-\cos x\), 注意到\(F(0)=0\), 由于\(F(x)\)在\(\left(0,\frac{\pi}{2}\right)\)上连续, 并且\(f(0)=0\), 考虑\[f'(t)=2(1-\alpha)(1-t)-(\alpha+1)(1-t)^\alpha,\]\(f'(0)=1-3\alpha\), 若\(f'(0)>0\), 则存在\(\delta>0\)使得\(F(x)\)在\((0,\delta)\)上单调递增, 不符题设, 从而\(\alpha\geqslant\frac{1}{3}\).
\end{proof}
\woe 试求以下双曲函数的导数与反函数, 并以此求得这些反函数的导数. 将结果与相应的三角函数比较:\[\sinh x=\frac{\ee^x-\ee^{-x}}{2},\qquad \cosh x=\frac{\ee^x+\ee^{-x}}{2},\qquad \tanh x=\frac{\sinh x}{\cosh x},\]其中\(\cosh x\)的反函数在\(x>0\)的范围内考虑.
\begin{solution}
对于\(\sinh x\), 其导数\(\left(\sinh x\right)'=\frac{\ee^x+\ee^{-x}}{2}=\cosh x\), 反函数记\(\sinh y=x\), 即\(\ee^{2y}-2x\ee^{y}-1=0\), 解得\(e^{y}=x+\sqrt{x^2+1}\), 即\(y=\ln(x+\sqrt{x^2+1})\). 又有\[\frac{\dd y}{\dd x}=\frac{1}{(\sinh y)'}=\frac{2}{\ee^{y}+\ee^{-y}}=\frac{2}{x+\sqrt{x^2+1}+\displaystyle\frac{1}{x+\sqrt{x^2+1}}}=\frac{1}{\sqrt{x^2+1}}.\]

对于\(\cosh x\), 其导数\((\cosh x)'=\frac{\ee^x-\ee^{-x}}{2}=\sinh x\), 反函数记\(\cosh y=x\), 即\(\ee^{2y}-2x\ee^{y}+1=0\), 解得\(\ee^y=x+\sqrt{x^2-1}\), 即\(y=\ln(x+\sqrt{x^2-1})\). 又有\[\frac{\dd y}{\dd x}=\frac{1}{(\cosh y)'}=\frac{2}{\ee^y-\ee^{-y}}=\frac{1}{\sqrt{x^2-1}}.\]

对于\(\tanh x\), 其导数\(\left(\tanh x\right)'=\frac{\cosh^2x-\sinh^2x}{\cosh^2x}=\frac{1}{\cosh^2x}\), 反函数即\(\tanh y=x\), 同理可以得到\(\ee^y=\sqrt{\frac{x+1}{1-x}}\), 即\(y=\ln\sqrt{\frac{x+1}{1-x}}\). 又有\[\frac{\dd y}{\dd x}=\frac{1}{(\tanh y)'}=\cosh^2y=\frac{1}{1-x^2}.\]

跟三角函数差不多.
\end{solution}
\woe 设某区间上的连续可微函数\(y=y(x),z=z(x)\)由方程组\[\begin{cases}
     x^2+3y^2+3z^2=1,\\ x^3+3y^3+9z^3=0
 \end{cases}\]确定. 试求\(y'(x), z'(x)\).
\begin{solution}
对\(x\)求导可得\[\begin{cases}
2x+6yy'+6zz'=0,\\3x^2+6y^2y'+27z^2z'=0,
\end{cases}\]解得\(y'(x)=\frac{x^2z-3xz^2}{9yz^2-2y^2z},\,z'(x)=\frac{2xy^2-3x^2y}{27yz^2-6y^2z}\).
\end{solution}
\woe 设某区域内的连续可微函数\(u=u(x,w),v=v(x,w)\)由方程组\[\begin{cases}
    x^2+2u^2+3v^2+w^4=1,\\x+u+v+w=0
\end{cases}\]确定. 试求\(\frac{\partial u}{\partial x},\frac{\partial u}{\partial w},\frac{\partial v}{\partial x},\frac{\partial w}{\partial x}.\)
\begin{solution}
分别对\(x,w\)求偏导得到\[\begin{cases}
2x+4u\frac{\partial u}{\partial x}+6v\frac{\partial v}{\partial x}=0,\\
1+\frac{\partial u}{\partial x}+\frac{\partial v}{\partial x}=0,
\end{cases}\quad\begin{cases}
4u\frac{\partial u}{\partial w}+6v\frac{\partial v}{\partial w}+4w^3=0,\\
\frac{\partial u}{\partial w}+\frac{\partial v}{\partial w}+1=0,
\end{cases}\]解得\(\frac{\partial u}{\partial x}=\frac{x-3v}{3v-2u},\frac{\partial u}{\partial w}=\frac{3v-2w^3}{2u-3v},\frac{\partial v}{\partial x}=\frac{2u-x}{3v-2u},\frac{\partial v}{\partial w}=\frac{2w^3-2u}{2u-3v}\). 于是\[\frac{\partial w}{\partial x}=\frac{1}{\partial u/\partial w}\cdot\frac{\partial u}{\partial x}=\frac{2u-3v}{3v-2w^3}\cdot\frac{x-3v}{3v-2u}=\frac{3v-x}{3v-2w^3}.\qedhere\]
\end{solution}
\woe 设某区域内的连续可微函数\(v=v(x,u),w=w(x,u)\)由方程组\[\begin{cases}
    x^2+2u^2+3v^2+w^4=1,\\x+u+v+w=0
\end{cases}\]确定. 试求\(\frac{\partial u}{\partial x},\frac{\partial w}{\partial u},\frac{\partial v}{\partial x},\frac{\partial w}{\partial x}.\)
\begin{solution}
分别对\(x,u\)求偏导得到\[\begin{cases}
2x+6v\frac{\partial v}{\partial x}+4w^3\frac{\partial w}{\partial x}=0,\\
1+\frac{\partial v}{\partial x}+\frac{\partial w}{\partial x}=0,
\end{cases}\quad\begin{cases}
4u+6v\frac{\partial v}{\partial u}+4w^3\frac{\partial w}{\partial u}=0,\\
1+\frac{\partial v}{\partial u}+\frac{\partial w}{\partial u}=0,
\end{cases}\]解得\(\frac{\partial v}{\partial u}=\frac{2w^3-2u}{3v-2w^3},\frac{\partial w}{\partial u}=\frac{2u-3v}{3v-2w^3},\frac{\partial v}{\partial x}=\frac{2w^3-x}{3v-2w^3},\frac{\partial w}{\partial x}=\frac{x-3v}{3v-2w^3}.\) 于是\[\frac{\partial u}{\partial x}=\frac{1}{\partial w/\partial u}\cdot\frac{\partial w}{\partial x}=\frac{x-3v}{2u-3v}.\qedhere\]
\end{solution}
\woe 设某区间上的连续可微函数\(y=y(x)\)由参数方程组\[\begin{cases}
    x=2\ee^t-\cos t,\\ y=t-\sin t,
\end{cases}t\in [0,2\pi]\]确定. 试求\(y'(x)\).
\begin{solution}
我们有\(\frac{\dd y}{\dd t}=1-\cos t,\,\frac{\dd x}{\dd t}=2\ee^t+\sin t\), 于是\[\frac{\dd y}{\dd x}=\frac{\dd y}{\dd t}\cdot\frac{1}{\dd x/\dd t}=\frac{1-\cos t}{2\ee^t+\sin t}.\qedhere\]
\end{solution}
\end{quiza}
\begin{quizb}
\woe 若\(\Omega\)是\(\mathbb{R}^n\)中的区域, \(\boldsymbol{A}:\Omega\rightarrow\mathbb{R}^{n\times n}\)以及\(\boldsymbol{g}:\Omega\rightarrow\mathbb{R}^n\)可微, \(\boldsymbol{f}(\boldsymbol{x})=\boldsymbol{A}(\boldsymbol{x})\boldsymbol{g}(\boldsymbol{x})\). 试计算\(\boldsymbol{f_x}\).
\begin{solution}

\end{solution}
\woe 试构造一个在\([0,1]\)上处处可导的函数, 使得其导函数在\([0,1]\)上无界.
\begin{solution}
设\(f(x)=\begin{cases}
x^2\sin\frac{1}{x^2},&x\ne 0,\\
0,&x=0
\end{cases}\). 可得\(f'(0)=\lim_{x\rightarrow 0}\frac{f(x)}{x}=\lim_{x\rightarrow 0}x\sin\frac{1}{x^2}=0\), 而\(x\ne 0\)时\[f'(x)=2x\sin\frac{1}{x^2}-\frac{1}{x^2}\cos\frac{1}{x^2}\]无界.
\end{solution}

\woe 试仿照习题\(\boldsymbol{\mathcal{A}}\)第2题编写一些习题.
\end{quizb}
\section{高阶导数}
\precis{高阶导数,Leibniz公式,微分算子 D,\(C^k,C^{k,\alpha}\)函数类,光滑函数,H\"{o}lder条件,Lipschitz条件,多重指标,多重零点}
\begin{quiza}
\woe 计算以下函数在\(x=0\)处的各阶导数, 或给出递推公式:\vspace{8pt}\\
\begin{tabular}{lcl}
\((1)\,y=\tan^2 x\);&\qquad\qquad\qquad&\((2)\,y=\arctan x\);\vspace{0.3cm}\\
\((3)\,y=\arcsin x\);&&\((4)\,y=\frac{x^2+3x+3}{x^2-3x+2}\).\vspace{0.3cm}\\
\((5)\,y=\cos^3x\);&&\((6)\,y=\frac{x}{x^2+x+1}\).\vspace{0.3cm}\\
\((7)\,y=(1+x)^{2+x}\);&&\((8)\,y=\begin{cases}
              \ee^{-x^{-2}},&x\ne 0,\\0,&x=0.
          \end{cases}\).
\end{tabular}
\begin{solution}
(1)我们有\(y'(x)=2\tan x\sec^2x,\,y''(x)=2\sec^4x+4\tan^2x\sec^2x=6y^2+8y+2\), 于是\(y^{(n+2)}=6\sum_{k=0}^{n}C_n^ky^{(k)}y^{(n-k)}+8y^{(n)}\).

(2)此问使用6.6节相关内容解法十分容易, 因为有\(\arctan x=\sum_{n=0}^{\infty}\frac{(-1)^nx^{2n+1}}{2n+1}\), 所以\[y^{(2n+1)}(0)=\frac{(-1)^n(2n+1)!}{2n+1}=(-1)^n(2n)!,\quad y^{(2n)}=0.\]然而, 我们也可以用12题的结论, 因为注意到\[\left(\arctan x\right)'=\frac{1}{1+x^2}=\frac{1}{(x+\ii)(x-\ii)}=\frac{1}{2\ii}\left(\frac{1}{x-\ii}-\frac{1}{x+\ii}\right),\]于是\[\begin{split}
y^{(n+1)}(0)=\left.\frac{1}{2\ii}\left(\frac{(-1)^{n}n!}{(x-\ii)^{n+1}}-\frac{(-1)^{n}n!}{(x+\ii)^{n+1}}\right)\right|_{x=0}=\frac{(-1)^nn!}{2}\left(\frac{\ii^{n+1}-(-\ii)^{n+1}}{\ii}\right).
\end{split}\]

(3)我们有\(\sin y=x\), 得\(y'\cos y=1\), 进一步有\[y''\cos y-y'\sin y=0\Rightarrow y''=y'\tan y,\]于是有\(y^{(n+2)}=\sum_{k=0}^{n}C_n^ky^{(k+1)}\left(\tan y\right)^{(n-k)}\).

(4)我们有\[f(x)=\frac{x^2+3x+3}{x^2-3x+2}=1+\frac{13}{x-2}-\frac{7}{x-1},\]因此\(f^{(n)}(x)=(-1)^n\cdot n! \cdot\left(\frac{13}{(x-2)^{n+1}}-\frac{7}{(x-1)^{n+1}}\right),\) 于是\(f^{(n)}(0)=(7-\frac{13}{2^{n+1}})n!\).

(5)

(6)

(7)

(8)
\end{solution}
\woe 设\(f(x)=\begin{cases}
      ax^2+b\cos x,\qquad &x<0,\\\ee^{x}+\ln(1+x)+cx,&x\geqslant 0.
  \end{cases}\)问当\(a,b,c\)取何值时, \(f\)是\(\mathbb{R}\)上的二阶连续可导函数.
\begin{solution}
由于\(f\)连续, 有\[\lim_{x\rightarrow 0^-}f(x)=\lim_{x\rightarrow 0^+}f(x)\Rightarrow b=1,\]由于\(f\)一阶导函数存在且连续, 即对于\(f'(x)=\begin{cases}
2ax-b\sin x,\qquad &x<0\\
\ee^x+\frac{1}{1+x}+c,&x\geqslant 0
\end{cases},\)有\[\lim_{x\rightarrow 0^-}f'(x)=\lim_{x\rightarrow 0^+}f'(x)\Rightarrow c=-2,\]并且\(f'(0)=0\). 最后, \(f\)二阶导函数连续, 则\[\lim_{x\rightarrow 0}\frac{2ax-b\sin x}{x}=\lim_{x\rightarrow 0}\frac{1}{x}\left(\ee^x+\frac{1}{1+x}+c\right)=\lim_{x\rightarrow 0}\frac{1}{x}\left(\left(\ee^x-1\right)+\left(\frac{1}{1+x}-1\right)\right)=0,\]即\(2a-b=0\), 于是\(a=\frac{1}{2}\).
\end{solution}
\woe 设某区间内的连续可微函数\(u=u(x,w),v=v(x,w)\)由方程组\[\begin{cases}
    x^2+2u^2+3v^2+w^4=1,\\x+u+v+w=0.
\end{cases}\]确定, 试求\(\frac{\partial^2u}{\partial x^2},\frac{\partial^2u}{\partial x\partial w}\).
\begin{solution}
结合4.2节\(\boldsymbol{\mathcal{A}}\)第\textbf{6}题, 我们得到\[\begin{cases}
2+4\left(\frac{\partial u}{\partial x}\right)^2+4u\frac{\partial^2u}{\partial x^2}+6\left(\frac{\partial v}{\partial x}\right)^2+6v\frac{\partial^2v}{\partial x^2}=0,\\
\frac{\partial^2u}{\partial x^2}+\frac{\partial^2v}{\partial x^2}=0,
\end{cases}\]解得\(\frac{\partial^2u}{\partial x^2}=\frac{1}{3v-2u}\left(1+2\left(\frac{\partial u}{\partial x}\right)^2+3\left(\frac{\partial v}{\partial x}\right)^2\right)=\frac{16u^2-12uv-12ux+27v^2-12vx+5x^2}{(3v-2u)^3}.\)

又有\[\begin{cases}
4\frac{\partial u}{\partial w}\frac{\partial u}{\partial x}+4u\frac{\partial^2u}{\partial x\partial w}+6\frac{\partial v}{\partial w}\frac{\partial v}{\partial x}+6v\frac{\partial^2v}{\partial x\partial w}=0,\\
\frac{\partial^2u}{\partial x\partial w}+\frac{\partial^2v}{\partial x\partial w}=0,
\end{cases}\]解得\(\frac{\partial^2u}{\partial x\partial w}=\frac{1}{3v-2u}\left(2\frac{\partial u}{\partial w}\frac{\partial u}{\partial x}+3\frac{\partial v}{\partial w}\frac{\partial v}{\partial x}\right)=\frac{6(u-w^3)(2\!\:u-x)+2(3v-2w^3)(3v-x)}{(3v-2u)^3}.\)
\end{solution}
\woe 设某区间\(\Omega\subset\mathbb{R}^n\)内, \(n\)阶方阵值函数\(\boldsymbol{A}(\cdot)=\left(a_{ij}(\cdot)\right)\)连续可微, 实函数\(f\)连续, 实函数\(u\)在\(\Omega\)内二阶连续可微且满足方程\[\sum_{i,j=1}^{n}\frac{\partial}{\partial x_i}\left(a_{ij}\left(\boldsymbol{x}\right)\frac{\partial u(\boldsymbol{x})}{\partial x_j}\right)=f(x).\]
作变量代换\(\boldsymbol{x}=\boldsymbol{Ps}\), 其中\(\boldsymbol{P}\)为可逆常数矩阵, 试将上述方程化为关于\(v(\boldsymbol{s})=u(\boldsymbol{Ps})\)的方程.
\begin{solution}

\end{solution}
\woe 设\(f(x,y)=\ln\sqrt{x^2+y^2+z^2}\left((x,y)\ne (0,0)\right)\). 试计算\(f_{xx}(x,y)+f_{yy}(x,y)\).
\begin{solution}
容易算得\[f_x(x,y)=\frac{x}{x^2+y^2+z^2},\quad f_{xx}(x,y)=\frac{y^2+z^2-x^2}{\left(x^2+y^2+z^2\right)^2},\]考虑对称性易得\(f_{yy}(x,y)=\frac{x^2+z^2-y^2}{\left(x^2+y^2+z^2\right)^2},\) 从而\(f_{xx}(x,y)+f_{yy}(x,y)=\frac{2z^2}{(x^2+y^2+z^2)^2}.\)
\end{solution}
\woe 设\(f(x,y,z)=\frac{1}{\sqrt{x^2+y^2+z^2}}\left((x,y,z)\ne(0,0,0)\right)\). 试计算\(f_{xx}(x,y,z)+f_{yy}(x,y,z)+f_{zz}(x,y,z)\).
\begin{solution}
容易算得\[f_x(x,y,z)=\frac{-x}{\left(x^2+y^2+z^2\right)^{3/2}},\quad f_{xx}(x,y,z)=\frac{2x^2-y^2-z^2}{(x^2+y^2+z^2)^{5/2}},\]结合\(x,y,z\)的对称性可得\[f_{xx}(x,y,z)+f_{yy}(x,y,z)+f_{zz}(x,y,z)=0.\qedhere\]
\end{solution}
\woe 设二元实函数\(f\)在不包含原点的一个区域内有两阶的连续偏导数. 作变量代换\(\begin{cases}
    x=r\cos\theta,\\y=r\sin\theta,
\end{cases}\)其中\(r>0,\theta\in\left[0,2\pi\right)\). 试用\(f\)关于\(r,\theta\)的二阶偏导数表示\(f_{xx}+f_{yy}\).
\begin{solution}
注意到\[r^2=x^2+y^2,\quad \tan\theta=\frac{y}{x},\]先对\(x\)微分有\[\begin{split}
2r\frac{\partial r}{\partial x}&=2x\Rightarrow\frac{\partial r}{\partial x}=\frac{x}{r}\Rightarrow\frac{\partial^2r}{\partial x^2}=\frac{r-x(\partial r/\partial x)}{r^2}=\frac{y^2}{r^3},\\
\frac{\partial\theta}{\partial x}&=\frac{1}{1-(y/x)^2}\left(-\frac{y}{x^2}\right)=-\frac{y}{r^2}\Rightarrow\frac{\partial^2\theta}{\partial x^2}=\frac{2y}{r^3}\frac{\partial r}{\partial x}=\frac{2xy}{r^4},
\end{split}\]按照类似的过程, 对\(y\)微分可以得到\[\frac{\partial r}{\partial y}=\frac{y}{r},\quad\frac{\partial^2r}{\partial y^2}=\frac{x^2}{r^3},\quad\frac{\partial\theta}{\partial y}=\frac{x}{r^2},\quad\frac{\partial^2\theta}{\partial y^2}=-\frac{2xy}{r^4},\]从以上结果, 可以推出\[\frac{\partial^2\theta}{\partial x^2}+\frac{\partial^2\theta}{\partial y^2}=0,\quad\frac{\partial r}{\partial x}\frac{\partial \theta}{\partial x}+\frac{\partial r}{\partial y}\frac{\partial \theta}{\partial y}=0.\]利用复合函数求导链式法则有\[\frac{\partial f}{\partial x}=\frac{\partial f}{\partial r}\frac{\partial r}{\partial x}+\frac{\partial \theta}{\partial r}\frac{\partial \theta}{\partial x},\]再次对\(x\)微商得\[\begin{split}
&\frac{\partial^2f}{\partial x^2}=\frac{\partial}{\partial x}\left(\frac{\partial f}{\partial r}\right)\frac{\partial r}{\partial x}+\frac{\partial f}{\partial r}\frac{\partial^2r}{\partial x^2}+\frac{\partial}{\partial x}\left(\frac{\partial f}{\partial \theta}\right)\frac{\partial \theta}{\partial x}+\frac{\partial f}{\partial \theta}\frac{\partial^2\theta}{\partial x^2}\\=&\left(\frac{\partial f^2}{\partial r^2}\frac{\partial r}{\partial x}+\frac{\partial^2f}{\partial r\partial\theta}\frac{\partial\theta}{\partial x}\right)\frac{\partial r}{\partial x}+\frac{\partial f}{\partial r}\frac{\partial^2r}{\partial x^2}+\left(\frac{\partial f^2}{\partial r\partial\theta}\frac{\partial r}{\partial x}+\frac{\partial^2f}{\partial\theta^2}\frac{\partial\theta}{\partial x}\right)\frac{\partial\theta}{\partial x}+\frac{\partial f}{\partial\theta}\frac{\partial^2\theta}{\partial x^2}\\
=&\frac{\partial^2f}{\partial r^2}\left(\frac{\partial r}{\partial x}\right)^2+2\frac{\partial^2f}{\partial r\partial\theta}\frac{\partial r}{\partial x}\frac{\partial\theta}{\partial x}+\frac{\partial f}{\partial r}\frac{\partial^2r}{\partial x^2}+\frac{\partial^2f}{\partial \theta^2}\left(\frac{\partial\theta}{\partial x}\right)^2+\frac{\partial f}{\partial\theta}\frac{\partial^2\theta}{\partial x^2}.
\end{split}\]
对于\(y\)有同样的结果\[\frac{\partial^2f}{\partial y^2}=\frac{\partial^2f}{\partial r^2}\left(\frac{\partial r}{\partial y}\right)^2+2\frac{\partial^2f}{\partial r\partial\theta}\frac{\partial r}{\partial y}\frac{\partial\theta}{\partial y}+\frac{\partial f}{\partial r}\frac{\partial^2r}{\partial y^2}+\frac{\partial^2f}{\partial \theta^2}\left(\frac{\partial\theta}{\partial y}\right)^2+\frac{\partial f}{\partial\theta}\frac{\partial^2\theta}{\partial y^2}.\]将以上两式相加, 可以得到\[\begin{split}
\frac{\partial^2f}{\partial x^2}+\frac{\partial^2f}{\partial y^2}=&\frac{\partial f^2}{\partial r^2}\left(\left(\frac{\partial r}{\partial x}\right)^2+\left(\frac{\partial r}{\partial y}\right)^2\right)+2\frac{\partial^2f}{\partial r\partial\theta}\left(\frac{\partial r}{\partial x}\frac{\partial\theta}{\partial x}+\frac{\partial r}{\partial y}\frac{\partial\theta}{\partial y}\right)\\+&\frac{\partial f}{\partial r}\left(\frac{\partial^2r}{\partial x^2}+\frac{\partial^2r}{\partial y^2}\right)+\frac{\partial f^2}{\partial\theta^2}\left(\left(\frac{\partial\theta}{\partial x}\right)^2+\left(\frac{\partial\theta}{\partial y}\right)^2\right)+\frac{\partial f}{\partial\theta}\left(\frac{\partial^2\theta}{\partial x^2}+\frac{\partial^2\theta}{\partial y^2}\right)\\=&\frac{\partial f^2}{\partial r^2}\left(\left(\frac{\partial r}{\partial x}\right)^2+\left(\frac{\partial r}{\partial y}\right)^2\right)+\frac{\partial f}{\partial r}\left(\frac{\partial^2r}{\partial x^2}+\frac{\partial^2r}{\partial y^2}\right)+\frac{\partial^2 f}{\partial\theta^2}\left(\frac{\partial^2\theta}{\partial x^2}+\frac{\partial^2\theta}{\partial y^2}\right),
\end{split}\]进一步得到\[\frac{\partial^2f}{\partial x^2}+\frac{\partial^2f}{\partial y^2}=\frac{\partial f^2}{\partial r^2}\left(\frac{x^2}{r^2}+\frac{y^2}{r^2}\right)+\frac{\partial f}{\partial r}\left(\frac{x^2}{r^3}+\frac{y^2}{r^3}\right)+\frac{\partial^2 f}{\partial\theta^2}\left(\frac{x^2}{r^4}+\frac{y^2}{r^4}\right),\]
利用\(r^2=x^2+y^2\), 我们便得\[f_{xx}+f_{yy}=\frac{\partial^2f}{\partial x^2}+\frac{\partial^2f}{\partial y^2}=\frac{\partial^2f}{\partial r^2}+\frac{1}{r}\frac{\partial f}{\partial r}+\frac{1}{r^2}\frac{\partial^2f}{\partial\theta^2}.\qedhere\]
\end{solution}
\woe 设三元实函数\(f\)在不包含原点的一个区域内有两阶的连续偏导数. 作变量代换\[\begin{cases}
    x=r\cos\theta\sin\varphi,\\
    y=r\sin\theta\sin\varphi,\\
    z=r\cos\varphi,
\end{cases}\]其中\(r>0,\theta\in\left[0,2\pi\right),\varphi\in[0,2\pi]\). 试用\(f\)关于\(r,\theta,\varphi\)的二阶偏导数表示\(f_{xx}+f_{yy}+f_{zz}\).
\begin{solution}

\end{solution}
\woe 设\(m\geqslant1\), 证明: 复数\(z_0\)为非零多项式\(P\)的\(m\)重根当且仅当\(P(z_0)=P'(z_0)=\cdots=P^{(m-1)}(z_0)=0\), 但\(P^{(m)}(z_0)\ne 0\).
\begin{proof}

\end{proof}
\woe 当\(a\)为何值时, \(x^4-9x^2+ax+12=0\)有重根?
\begin{solution}
记\(f(x)=x^4-9x^2+ax+12\), 由于\[f'(x)=4x^3-18x+a,\quad f''(x)=12x^2-18,\quad f^{(3)}(x)=24x,\]现在来做这样一件事: 令\(f^{(3)}=0\)得\(x=0\), 但\(f(0)\ne 0\), 从而\(f(x)\)没有四重根; 令\(f''(x)=0\)得\(x=\pm\sqrt{\frac{3}{2}}\), 在令\(f'\left(\pm\sqrt{\frac{3}{2}}\right)=0\)得\(a=\pm 6\sqrt{6}\), 将这\(x\)与对应的\(a\)代入\(f\), 得到\(f(x)\ne 0\), 从而\(f(x)\)没有三重根. 即若\(f(x)\)有重根, 必为二重根.

易见\(x=0\)不是\(f(x)\)的根. 设\(g(x)=\frac{9x^2-x^4-12}{x}\), 则\(a=g(x)\)的根即为\(f(x)=0\)的解, 特别地, 当\(a\)等于\(g(x)\)的极值时, \(f(x)\)有重根, 重根即为\(g(x)\)的极值点.

由\(g'(x)=9-3x^2+\frac{12}{x^2}\)可知\(x=\pm 2\)为\(g(x)\)的极值点, 代入\(a=g(x)\), 可知\(a=\pm 4\). 于是\(a=4\)时, \(x=2\)为\(f(x)=0\)的重根, \(a=-4\)时, \(x=-2\)为\(f(x)=0\)的重根.
\end{solution}
\woe 设\(\alpha>1,M>0,[a,b]\)上的实函数\(f\)满足\[|f(x)-f(y)|\leqslant M|x-y|^\alpha,\qquad\forall x,y\in[a,b].\]证明: \(f\)在\([a,b]\)上恒为常数.
\begin{solution}
易见\(f\)在\([a,b]\)上连续, 给定\(x\), 取\(y=x+\Delta x\), 则有\[\left|\frac{f(x+\Delta x)-f(x)}{\Delta x}\right|\leqslant M\left|\Delta x\right|^{\alpha -1}\rightarrow 0,\quad\Delta x\rightarrow 0,\]从而\(f\)在\([a,b]\)上任一点得导数值为\(0\), 故而\(f\)为常数.
\end{solution}
\woe 设\(\alpha\in\mathbb{C},n\in\mathbb{Z}\), 证明:\[\frac{\dd}{\dd x}(x+\alpha)^n=n(x+\alpha)^{n-1},\qquad x\ne -\alpha.\]
\begin{proof}
先证\(\left(\ln(x+\alpha)\right)'=\frac{1}{x+\alpha}\), 设\(y=\ln(x+\alpha)\), 则\(\ee^y=x+\alpha\), 那么\[\left(\ln(x+\alpha)\right)'=\frac{1}{\dd x/\dd y}=\frac{1}{\ee^y}=\frac{1}{x+\alpha},\]那么\(\frac{\dd}{\dd x}(x+\alpha)^n=\frac{\dd}{\dd x}\exp(n\ln(x+\alpha))=\exp\left(n\ln(x+\alpha)\right)\cdot\frac{n}{x+\alpha}=n(x+\alpha)^{n-1}\).
\end{proof}
\end{quiza}
\begin{quizb}
\woe 设\(P_1,P_2,Q_1,Q_2\)均为非零多项式, 满足\(\frac{1}{P(t)}=\frac{Q_1(t)}{P_1(t)}+\frac{Q_2(t)}{P_2(t)}\), 其中\(P(t)=P_1(t)P_2(t)\). 若\(f,f_1,f_2\)为\(I\)上的光滑函数, 满足\(P_1(\DD)f_1(x)=P_2(\DD)f_2(x)=f(x)\), 证明: \(F(x)=Q_1(\DD)f_1(x)+Q_2(\DD)f_2(x)\)满足\(P(\DD)F(x)=f(x)\).
\begin{proof}
由题设知\(Q_1(t)P_2(t)+Q_2(t)P_1(t)=1\), 于是\[\begin{split}
P(\DD)F(x)&=P_1(\DD)P_2(\DD)\left(Q_1(\DD)f_1(x)+Q_2(\DD)f_2(x)\right)\\&=Q_1(\DD)P_2(\DD)P_1(\DD)f_1(x)+Q_2(\DD)P_1(\DD)P_2(\DD)f_2(x)\\&=Q_1(\DD)P_2(\DD)f(x)+Q_2(\DD)P_1(\DD)f(x)\\&=\left(Q_1(\DD)P_2(\DD)+Q_2(\DD)P_1(\DD)\right)f(x)=f(x).\qedhere
\end{split}\]
\end{proof}
\woe 设实函数\(y\)在\((0,+\infty)\)上二阶可导, 满足\((1+x^2)y''(x)+4xy'(x)+2y(x)=0.\) 令\(z(x)=\frac{1+x^2}{1+x}y(x)\). 试导出\(z\)的微分方程(比较两种方法: 对等式\(y(x)=\frac{1+x}{1+x^2}z(x)\)求导和对等式\((1+x)z(x)=(1+x^2)y(x)\)求导).
\begin{solution}
记\(w(x)=\frac{1+x}{1+x^2}\), 则\(y(x)=w(x)\cdot z(x)\), 并且\[y'(x)=w'(x)z(x)+w(x)z'(x),\quad y''(x)=w''(x)z(x)+2w'(x)z'(x)+w(x)z''(x),\]代入微分方程得到\[(1+x^2)w(x)z''(x)+\left[2(1+x^2)w'(x)+4xw(x)\right]z'(x)+\left[(1+x^2)w''(x)+4xw'(x)+2w(x)\right]z(x)=0,\]
结合\(w'(x)=\frac{1-2x-x^2}{(1+x^2)^2},w''(x)=\frac{2x^3+6x^2-6x-2}{(1+x^2)^3}\), 带入方程即得\[(1+x)z''(x)+2z'(x)=0.\]

另外, 由于\((1+x)z(x)=(1+x^2)y(x)\), 于是有\[\begin{split}
z(x)+(1+x)z'(x)&=2xy(x)+(1+x^2)y'(x),\\
(1+x)z''(x)+2z'(x)&=(1+x^2)y''(x)+4xy'(x)+2y(x)=0,
\end{split}\]显然后者计算量更小.
\end{solution}
\end{quizb}
\section{复指数函数, 正弦函数和余弦函数}
\precis{用级数定义复指数函数,用微分方程定义正弦和余弦函数,Euler公式}
\begin{quiza}
\woe 计算\(\sin x,\cos x\)在\(x=\frac{\pi}{2},\frac{\pi}{3},\frac{\pi}{4}\)等点的值.
\begin{solution}

\end{solution}
\woe 证明\(\pi>2\sqrt{2}\). 
\begin{proof}

\end{proof}
\woe 证明: \(\forall x\in\left(0,\frac{\pi}{2}\right)\), 成立\(\sin x<x<4\sin\frac{x}{2}-\sin x\).
\begin{proof}

\end{proof}
\woe 计算\(\left(x^3\ee^{x}\sin x\right)^{(100)}\).
\begin{solution}
记\(f(x)=\ee^x\sin x\), 由Leibniz公式有\[\left(x^3f(x)\right)^{(100)}=C_{100}^0x^3f^{(100)}(x)+C_{100}^13x^2f^{(99)}(x)+C_{100}^26xf^{(98)}(x)+C_{100}^36f^{(97)}(x),\]我们依次计算\[\begin{split}
\left(\ee^x\sin x\right)^{(100)}&=\left(\mathrm{Re}\,\ee^{x+\ii x}\right)^{(100)}
\end{split}\]
\end{solution}
\woe 证明: 不恒为零的\(n\)阶三角多项式\(a_0+\sum_{k=1}^{n}(a_k\cos kx+b_k\sin kx)\)在\(\left[0,\frac{\pi}{2}\right)\)内至多只有\(2n\)个零点(含重数).
\begin{proof}

\end{proof}
\woe 利用上一题中的结论证明习题2.7.\(\boldsymbol{\mathcal{B}}\)第5题中的等式:\[\sin \pi x=(2n+1)\sin\frac{\pi x}{2n+1}\prod_{k=1}^{n}\left(1-\frac{\sin^2\dfrac{\pi x}{2n+1}}{\sin^2\dfrac{k\pi}{2n+1}}\right).\]
\end{quiza}
\begin{quizb}
\woe 求\(\sum_{n=1}^{\infty}\frac{\sin nx}{n!}\).
\begin{solution}
\(\sum_{n=1}^{\infty}\frac{\sin nx}{n!}=\mathrm{Im}\sum_{n=1}^{\infty}\frac{\ee^{\ii nx}}{n!}=\mathrm{Im}\left(\ee^{\ee^{\ii x}}-1\right)=\mathrm{Im}\ee^{\cos x+\ii\sin x}=\ee^{\cos x}\sin\left(\sin x\right).\)
\end{solution}
\woe 设\(z\in\mathrm{C}\), 证明: \(\lim_{n\rightarrow+\infty}\left(1+\frac{z}{n}\right)^n=\ee^z\).
\begin{proof}

\end{proof}
\woe 不借助复指数函数证明下列等式.
\begin{gather*}
(\sin t)'=\cos t,\quad (\cos t)'=\sin t,\quad\forall t\in\mathbb{R},\\
\sin (-t)=-\sin t,\quad \cos(-t)=\cos t,\quad\forall t\in\mathbb{R},\\
\sin (\alpha+\beta)=\sin\alpha\cos\beta+\cos\alpha\sin\beta,\quad\forall\alpha,\beta\in\mathbb{R},\\
\cos(\alpha+\beta)=\cos\alpha\cos\beta-\sin\beta\sin\beta,\quad\forall\alpha,\beta\in\mathbb{R}.
\end{gather*}
\woe 考虑Euler公式\(\sin x=x\prod_{n=1}^{\infty}\left(1-\frac{x^2}{n^2\pi^2}\right)\). 两端展开后, 形式上我们有\[\begin{split}
    \sum_{n=1}^{\infty}\frac{(-1)^n}{(2n+1)!}x^{2n+1}&=x-\sum_{k=1}^{\infty}\frac{1}{k^3}\frac{x^3}{\pi^2}+\cdots+\sum_{k_1=1}^{\infty}\sum_{k_2=k_1+1}^{\infty}\frac{1}{k_1^2k_2^2}\frac{x^5}{\pi^5}\\&+(-1)^n\sum_{1\leqslant k_1<k_2<\cdots<k_n}^{\infty}\frac{1}{k_1^2k_2^2\cdots k_n^2}\frac{x^{2n+1}}{\pi^{2n}}+\cdots
\end{split}\]比较系数得到\[\sum_{1\leqslant k_1<k_2\cdots k_n}^{\infty}\frac{1}{k_1^2k_2^2\cdots k_n^2}=\frac{\pi^{2\pi}}{(2n+1)!},\quad n\geqslant 1.\]特别地,\[\sum_{k=1}^{\infty}\frac{1}{k^2}=\frac{\pi^2}{6},\quad\sum_{1\geqslant k_1<k_2}\frac{1}{k_1^2k_2^2}=\frac{\pi^4}{120}.\]试严格证明上述关系式.
利用前述Euler公式, 形式上我有\[x\prod_{n=1}^{\infty}\left(1+\frac{x}{n^2\pi^2}\right)=\frac{\sin\mathrm{i}x}{i}=\frac{\ee^{\mathrm{i}\cdot\mathrm{i}x}-\ee^{\mathrm{-i}\cdot\mathrm{i}x}}{\mathrm{i}\cdot 2\mathrm{i}}=\frac{\ee^x-\ee^{-x}}{2},\quad\forall x\in\mathbb{R}.\]试严格证明\[\frac{\ee^{x}-\ee^{-x}}{2}=x\prod_{n=1}^{\infty}\left(1+\frac{x^2}{n^2\pi^2}\right),\quad\forall x\in\mathbb{R}.\]
\woe 证明\[\frac{\ee^{x}+\ee^{-x}}{2}=\prod_{n=0}^{\infty}\left(1+\frac{x^2}{\pi^2\left(n+\dfrac{1}{2}\right)^2}\right),\forall x\in\mathbb{R}.\]
\woe 仿习题\(\boldsymbol{\mathcal{A}}\)第6题给出一些新的形式
\end{quizb}