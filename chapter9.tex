\chapter{函数列与函数项级数}
\section{函数列与函数项级数的一致收敛及其性质}
\precis{函数列与函数项级数的一致收敛性及其性质,Cauchy准则}
\begin{quiza}
\woe 利用有限覆盖定理证明Dini定理.
\begin{proof}
由于\(f_n(x)\rightarrow f(x)(n\rightarrow+\infty)\), 则对于\(\forall\varepsilon>0,x_0\in [a,b],\exists N_{(x_0)}\in\mathbb{N}\)使得\[\left|f_{N_{(x_0)}}(x)-f(x)\right|<\frac{1}{3}\varepsilon,\]则\[\left|f_{N_{x_0}}(x)-f(x)\right|\leqslant\left|f_{N_{x_0}}(x)-f(x)\right|+\left|f_{N_{x_0}}(x)-f_{N_{x_0}}(x_0)\right|+\left|f(x)-f(x_0)\right|<\varepsilon,\]又因为对于固定的\(x_0\in [a,b]\), 当\(n>N_{x_0}\)时, 有\[\left|f_n(x)-f(x)\right|\leqslant\left|f_{N_{x_0}}(x)-f(x)\right|<\varepsilon,\]取区间\([a,b]\)的一个开覆盖\[H=\{U(x_i,\delta_{x_i})\big| x_i\in[a,b],i\in\Lambda\},\]其中\(\delta_{x_i}\)满足\(\exists N_{x_i}\), 当\(n>N_{x_i},\, x\in U(x_i,\delta_{x_i})\)时有\[\left|f_n(x)-f(x)\right|<\varepsilon,\]利用有限覆盖定理选取有限覆盖即证结论.
\end{proof}
\woe 试考察本节例题中那些函数列(函数项级数)的一致收敛性可用Dini定理解决.
\woe 设函数\(f\in C^{\infty}(\mathbb{R})\), 而且\(\forall x\in\mathbb{R},n\geqslant 1\), 有\(\left|f^{(n)}(x)-f^{(n-1)}(x)\right|<\frac{1}{n^2}\). 证明: \(\lim_{n\rightarrow+\infty}f^{(n)}(x)=C\ee^x\), 其中\(C\)为一常数.
\begin{proof}
由题设可知对于任意\(x\in\bbr\)有\[\sum_{n=1}^{\infty}\left|f^{(n)}(x)-f^{(n-1)}(x)\right|<\sum_{n=1}^{\infty}\frac{1}{n^2}=\frac{\pi^2}{6},\]从而\(\sum_{n=1}^{\infty}\left|f^{(n)}(x)-f^{(n-1)}(x)\right|\)收敛. 于是\(\forall\varepsilon>0,\exists N\)使得\[\left|\sum_{n=N}^{\infty}\left|f^{(n)}(x)-f^{(n-1)}(x)\right|\right|<\varepsilon,\]任取\(k>j>N\), 则有\[\begin{split}
&\left|f^{(k)}(x)-f^{(j)}(x)\right|=\left|f^{(k)}(x)-f^{(k-1)}(x)+f^{(k-1)}(x)-\cdots+f^{(j+1)}(x)-f^{(j)}(x)\right|\\\leqslant&\left|f^{(k)}(x)-f^{(k-1)}(x)\right|+\left|f^{(k-1)}(x)-f^{(k-2)}(x)\right|+\cdots+\left|f^{(j+1)}(x)-f^{(j)}(x)\right|<\varepsilon,
\end{split}\]
从而\(\{f^{(n)}(x)\}\)一致收敛, 设\(\lim_{n\rightarrow+\infty}f^{(n)}(x)=F(x)\), 则有\(\lim_{n\rightarrow+\infty}f^{(n+1)}(x)=F'(x)\), 从而\(F(x)=F'(x)\), 容易解得\(F(x)=C\ee^x\).
\end{proof}
\woe 计算\(\lim_{n\rightarrow+\infty}\left(1+\frac{1^2}{n^3}\right)\left(1+\frac{2^2}{n^3}\right)\cdots\left(1+\frac{n^2}{n^3}\right).\)
\begin{solution}
取对数. 注意到\(-\frac{1}{2}x^2<\ln(1+x)-x<0(\forall x\in (0,1))\), 所以\[-\frac{1}{2}\sum_{k=1}^{n}\frac{k^4}{n^6}<\sum_{k=1}^{n}\left[\ln\left(1+\frac{k^2}{n^3}\right)-\frac{k^2}{n^3}\right]<0,\]而\[\lim_{n\rightarrow\infty}\sum_{k=1}^{n}\frac{k^4}{n^6}=\lim_{n\rightarrow\infty}\frac{1}{n}\cdot\sum_{k=1}^{n}\frac{k^4}{n^4}=\lim_{n\rightarrow\infty}\frac{1}{n}\int_{0}^{1}x^4\dd x=0.\]所以\(\lim_{n\rightarrow\infty}\sum_{k=1}^{n}\ln\left(1+\frac{k^2}{n^3}\right)=\lim_{n\rightarrow\infty}\sum_{k=1}^{n}\frac{k^2}{n^3}=\frac{1}{3},\) 即\[\lim_{n\rightarrow+\infty}\left(1+\frac{1^2}{n^3}\right)\left(1+\frac{2^2}{n^3}\right)\cdots\left(1+\frac{n^2}{n^3}\right)=\ee^{1/3}.\qedhere\]
\end{solution}
\tcbline
还有另一种行之有效的方法, 使用不等式\[\frac{k}{n+k}<\ln\left(1+\frac{k}{n}\right)<\frac{k}{n},\]我们有\[\frac{k}{n^3+k}<\ln\left(1+\frac{k}{n^3}\right)<\frac{k}{n^3},\]于是\[\sum_{k=1}^{n}\frac{k^2}{n^3+n^2}<\sum_{k=1}^{n}\frac{k^2}{n^3+k^2}<\sum_{k=1}^{n}\ln\left(1+\frac{k^2}{n^3}\right)<\sum_{k=1}^{n}\frac{k^2}{n^3},\]而\(\sum_{k=1}^{n}k^2=\frac{n(n+1)(2n+1)}{6}\), 代入即得\(\lim_{n\rightarrow\infty}\sum_{k=1}^{n}\ln\left(1+\frac{k^2}{n^3}\right)=\frac{1}{3}\).
\woe 设\(\{g_n\}\)是\([a,b]\)上一致有界的可测函数列, \(f_n(x)=\alpha_n+\int_{a}^{x}g_n(t)\dd t(x\in [a,b];n\geqslant 1)\), \(\{\alpha_n\}\)收敛. 进一步, 以下条件之一成立:
\begin{compactenum}[(i)]
\item \(\{g_n\}\)在\([a.b]\)上一致收敛到\(g\), 且对每个\(n\geqslant 1,\lim_{x\rightarrow a^+}\frac{1}{x-a}\int_{a}^{x}g_n(x)\dd t=g_n(a)\).
\item \(g_n\)在\([a.b]\)上几乎处处收敛到\(g\), 且\(\lim_{x\rightarrow a^+}\frac{1}{x-a}\int_{a}^{x}g(t)\dd t=g(a)\).
\end{compactenum}
证明: \(\{f_n\}\)在\([a,b]\)上一致收敛到某个函数\(f\), 且\(f\)在点\(a\)的右导数等于\(g(a)\).
\begin{proof}

\end{proof}
\end{quiza}
\begin{quizb}
\woe 设\(\sum_{n=1}^{\infty}u_n(\cdot)\)在\([a,b]\)上收敛. 若对任何\(\varepsilon>0\)及\(N\geqslant 1\), 存在区间\((a_1,b_1),(a_2,b_2),\cdots,(a_k,b_k)\)以及\(n_1,n_2,\cdots,n_k\geqslant N\)使得\(\bigcup_{j=1}^{k}(a_j,b_j)\supset [a,b]\), 且\[\left|\sum_{l=n_j}^{\infty}u_l(x)\right|\leqslant\varepsilon,\quad\forall x\in(a_j,b_j)\cap[a,b];j=1,2,\cdots,k,\]则称\(\sum_{n=1}^{\infty}u_n(\cdot)\)在\([a,b]\)上\textbf{拟一致收敛}.

证明: 若\(u_n\)都在\([a,b]\)上连续\((n\geqslant 0)\), 则\(\sum_{n=0}^{\infty}u_n\)在\([a,b]\)上连续当且仅当它在\([a,b]\)上拟一致收敛.
\begin{proof}
	
\end{proof}
\woe 试将上一题的结果推广到\(\mathbb{R}^n\)中.
\woe 设\(\{f_k\}\)是\(\mathbb{R}^n\)中的实连续函数列, 它在\(\mathbb{R}^n\)上逐点收敛于\(f\). 试证明以下结论:\begin{compactenum}[(i)]
\item \(F=\bigcup_{\substack{p<q\\p,q\in\mathbb{Q}}}\left[f^{-1}(p,q)\backslash\left(f^{-1}(p,q)\right)^\circ\right]\)是\(f\)上不连续点全体.
\item 对于\(p<q\),\[\begin{split}
f^{-1}\left((p,q)\right)&=\left\lbrace \boldsymbol{x}\in\mathbb{R}^n\big|\exists m,k\geqslant 1,\mathrm{s.t.}\, p+\frac{1}{m}\leqslant f_j(\boldsymbol{x})\leqslant q-\frac{1}{m},\forall j\geqslant k\right\rbrace\\
&=\bigcup_{m=1}^{\infty}\bigcap_{k=1}^{\infty}\bigcup_{j=k}^{\infty}\left\lbrace\boldsymbol{x}\in\mathbb{R}^n\big| p+\frac{1}{m}\leqslant f_j(\boldsymbol{x})\leqslant q-\frac{1}{m}\right\rbrace. 
\end{split}\]
\item 集合\(F\)是一列无处稠密集的并.
\item 函数\(f\)一定有连续点.
\end{compactenum}
\woe 设\(f\)在区间\((a,b)\)处处可导, 证明: \(f'\)必有连续点.
\woe 设\(\{a_n\}\)是递增数列, \(a_1>1\). 求证: 级数\(\sum_{n=1}^{\infty}\frac{a_{n+1}-a_n}{a_n\ln a_{n+1}}\)收敛的充要条件是\(\{a_n\}\)有界. 又问级数通项中的\(a_n\)能不能换成\(a_{n+1}\)?
\begin{proof}
充分性. 若\(\{a_n\}\)有界, 设\(a_n\leqslant M\), 则\[\sum_{n=1}^{m}\frac{a_{n+1}-a_n}{a_n\ln a_{n+1}}\leqslant\sum_{n=1}^{m}\frac{a_{n+1}-a_n}{a_1\ln a_1}=\frac{a_{m+1}-a_1}{a_1\ln a_1}\leqslant\frac{M}{a_1\ln a_1},\]由此可知\(\sum_{n=1}^{m}\frac{a_{n+1}-a_n}{a_n\ln a_{n+1}}\)收敛.

必要性. 若\(\sum_{n=1}^{m}\frac{a_{n+1}-a_n}{a_n\ln a_{n+1}}\)收敛, 由于\[\ln a_{n+1}-\ln a_n=\ln\left(1+\frac{a_{n+1}-a_n}{a_n}\right)\leqslant\frac{a_{n+1}-a_n}{a_n},\]所以\(\frac{b_{n+1}-b_n}{b_{n+1}}\leqslant\frac{a_{n+1}-a_{n}}{a_n\ln a_{n+1}}\), 其中\(b_n=\ln a_n\), 因此\(\sum_{n=0}^{\infty}\frac{b_{n+1}-b_n}{b_{n+1}}\)收敛.

有Cauchy收敛准则, 存在自然数\(m\), 使得对一切自然数\(p\), 有\[\frac{1}{2}>\sum_{n=m}^{m+p}\frac{b_{n+1}-b_n}{b_{n+1}}\geqslant\sum_{n=m}^{m+p}\frac{b_{n+1}-b_n}{b_{m+p+1}}=1-\frac{b_m}{b_{m+p+1}}.\]由此可知\(\{b_n\}\)有界, 因为\(p\)是任意的, 所以\(\{a_n\}\)有界.

题中级数的分母\(a_n\)不能换成\(a_{n+1}\). 如取\(a_n=\ee^{n^2}\)无界, 但\(\sum_{n=1}^{\infty}\frac{a_{n+1}-a_n}{a_n\ln a_{n+1}}\)收敛.
\end{proof}
\woe 对于二元函数\(f\), 考虑以下各种情形中两种运算的次序交换问题. 利用已学的结果, 你可以给出那些情形的结果:
\begin{center}
\begin{tabular}{c|c|c|c}
\hline
&关于\(y\)的极限&关于\(y\)的积分&关于\(y\)的导数\\\hline
关于\(x\)的极限&\makecell{\(\lim_{x\rightarrow x_0}\lim_{y\rightarrow y_0}f(x,y)=\)\\\(\lim_{y\rightarrow y_0}\lim_{x\rightarrow x_0}f(x,y)\)}&\makecell{\(\lim_{x\rightarrow x_0}\int_{c}^{d}f(x,y)\dd y=\)\\\(\int_{c}^{d}\lim_{x\rightarrow x_0}f(x,y)\dd y\)}&\makecell{\(\lim_{x\rightarrow x_0}\frac{\partial}{\partial y}f(x,y)=\)\\\(\frac{\partial}{\partial y}\lim_{x\rightarrow x_0}f(x,y)\)}\\\hline
关于\(x\)的积分& &\makecell{\(\int_{a}^{b}\left(\int_{c}^{d}f(x,y)\dd y\right)\dd x=\)\\\(\int_{c}^{d}\left(\int_{a}^{b}f(x,y)\dd x\right)\dd y\)}&\makecell{\(\int_{a}^{b}\frac{\partial}{\partial y}f(x,y)\dd x=\)\\\(\frac{\partial}{\partial y}\int_{a}^{b}f(x,y)\dd x\)} \\\hline
关于\(x\)的导数& & &\makecell{\(\frac{\partial}{\partial x}\left(\frac{\partial}{\partial y}f(x,y)\right)=\)\\\(\frac{\partial}{\partial y}\left(\frac{\partial}{\partial x}f(x,y)\right)\)}\\\hline
\end{tabular}
\end{center}
\woe 设\(a_n>0.\sum_{n=1}^{\infty}\frac{1}{a_n}\)收敛, 证明: \(\sum_{n=1}^{\infty}\frac{n}{a_1+a_2+\cdots+a_n}\)收敛.
\begin{proof}
只需指出\(n=2m(m\in\mathbb{N})\)时有\(\sum_{k=1}^{n}\frac{k}{a_1+a_2+\cdots+a_k}\leqslant 4\sum_{k=1}^{n}\frac{1}{a_k}\)即可. 为此, 对\(n\)个数\(a_1,a_2,\cdots,a_n\)作重新排列为\(b_1\leqslant b_2\leqslant \cdots\leqslant b_n\), 则对正整数\(p\geqslant 1\), 我们有\begin{gather*}
a_1+a_2+\cdots+a_{2p}\geqslant a_1+a_2+\cdots+a_{2p-1}\geqslant b_1+b_2+\cdots+b_{2p-1}\geqslant pb_{p},\\
\frac{2p-1}{a_1+\cdots+a_{2p-1}}\leqslant \frac{2p-1}{pb_p}<\frac{2}{b_p},\quad\frac{2p}{a_1+\cdots+a_{2p}}\leqslant\frac{2}{b_p},\\
\frac{2p-1}{a_1+\cdots+a_{2p-1}}+\frac{2p}{a_1+\cdots+a_{2p}}<\frac{4}{b_p}.
\end{gather*}由此可得\begin{gather*}
\sum_{p=1}^{N}\left(\frac{2p-1}{a_1+a_2+\cdots+a_{2p-1}}+\frac{2p}{a_1+\cdots+a_{2p}}\right)<\sum_{p=1}^{N}\frac{4}{b_p},\\
\sum_{k=1}^{n}\frac{k}{a_1+\cdots+a_k}\leqslant\sum_{p=1}^{n}\frac{4}{b_p}=4\sum_{k=1}^{n}\frac{1}{a_k}.
\end{gather*}令\(n\rightarrow\infty\)即证.
\end{proof} 
\woe 推广第5题. 比如考察单调增加趋于无穷的函数\(f\)以及\(\sum_{n=1}^{\infty}\left(f(S_n)-f(S_{n-1})\right)\); 又或者考察在\((0,M)\)内单调下降且\(F(0^+)=+\infty\)的函数\(F\)以及\(\sum_{n=1}^{\infty}\left(F(S_n^{-1})-F(S_{n-1}^{-1})\right)\).
\end{quizb}
\section{函数项级数一致收敛性的判别法}
\precis{Weierstrass判别法,Abel判别法,Dirichlet判别法}
\begin{theorem}{}{AD}
考虑\(E\subseteq\mathbb{R}^n\)上的函数项级数\(\sum_{k=1}^{\infty}u_k(\cdot)v_k(\cdot)\). 设\(\{v_k(\cdot)\}\)在\(E\)上一致有界, 对每个\(\boldsymbol{x}\in E\), 数列\(\{v_k(\boldsymbol{x})\}\)单调.\begin{compactenum}[(i)]
\item \textbf{(Abel判别法)} 若\(\sum_{k=1}^{\infty}u_k(\cdot)\)在\(E\)上一致收敛, 则\(\sum_{k=1}^{\infty}u_k(\cdot)v_k(\cdot)\)在\(E\)上一致收敛.
\item \textbf{(Dirichlet判别法)} 若\(\sum_{k=1}^{\infty}u_k(\cdot)\)的部分和函数列\(\left\lbrace\sum_{j=1}^{k}u_j(\cdot)\right\rbrace\)在\(E\)上一致有界, \(\{v_k(\boldsymbol{x})\}\)在\(E\)上一致收敛到零, 则\(\sum_{k=1}^{\infty}u_k(\cdot)v_k(\cdot)\)在\(E\)上一致收敛.
\end{compactenum}
\end{theorem}
\begin{quiza}
\woe 考察\(\sum_{n=1}^{\infty}\frac{\cos nx}{n^2}\)的连续性, 可微性.
\begin{solution}
易见\(\left|\frac{\cos nx}{n^2}\right|\leqslant\frac{1}{n^2}\), 由Weierstrass判别法可知\(\sum_{n=1}^{\infty}\frac{\cos nx}{n^2}\)在\(\bbr\)上一致收敛, 因而和函数在\(\bbr\)上连续.

考察级数\(\sum_{n=1}^{\infty}\frac{\cos nx}{n^2}\)形式求导后的级数\(-\sum_{n=1}^{\infty}\frac{\sin nx}{n}\), 对于\(x\in(0,2\pi)\), 有\[\left|\sum_{n=1}^{N}\sin nx\right|=\left|\frac{\cos(x/2)-\cos(N+1/2)x}{2\sin(x/2)}\right|\leqslant\frac{1}{|\sin(x/2)|},\quad x\ne 2k\pi,\]由Dirichlet判别法易见\(-\sum_{n=1}^{\infty}\frac{\sin nx}{n}\)关于\(x\in(0,2\pi)\)内闭一致收敛, 从而\(\sum_{n=1}^{\infty}\frac{\cos nx}{n^2}\)在\((0,2\pi)\)上连续可导. 

记\(S(x)=\sum_{n=1}^{\infty}\frac{\cos nx}{n^2},\,S'(x)=-\sum_{n=1}^{\infty}\frac{\sin nx}{n}\). 下面考察\(S(x)\)在\(x=2k\pi\)处得可微性, 只需讨论\(x=0\)的情况即可. 由于\[-\sum_{n=1}^{\infty}\frac{\sin nx}{n}=\frac{x-\pi}{2},\quad 0<x<2\pi,\]事实上, 由Dirichlet判别法可知\(\sum_{n=1}^{\infty}\frac{\ee^{\ii nx}}{n}\)收敛, 于是由Abel定理,\[\begin{split}
\sum_{n=1}^{\infty}\frac{\ee^{\ii nx}}{n}&=\lim_{t\rightarrow 1^-}\sum_{n=1}^{\infty}\frac{\ee^{\ii nx}t^n}{n}=\int_{0}^{1}\sum_{n=1}^{\infty}t^{n-1}\ee^{\ii nx}\dd t=\int_{0}^{1}\frac{\ee^{\ii x}}{1-t\ee^{\ii x}}\dd t\\&=\int_{0}^{1}\frac{\ee^{\ii x}-t}{(t-\cos x)^2+\sin^2x}\dd t=-\ln\left(2\sin\frac{x}{2}\right)+\frac{\ii(\pi-x)}{2},
\end{split}\]于是\[\sum_{n=1}^{\infty}\frac{\sin nx}{n}=\frac{\pi-x}{2},\sum_{n=1}^{\infty}\frac{\cos nx}{n}=-\ln\left(2\sin\frac{x}{2}\right),\qquad x\in\left(0,\frac{\pi}{2}\right),\]
则有\[\lim_{x\rightarrow 0^+}\frac{x-\pi}{2}=-\frac{\pi}{2},\quad \lim_{x\rightarrow 2\pi^-}\frac{x-\pi}{2}=\frac{\pi}{2},\]这意味着\[\lim_{x\rightarrow 0^+}S'(x)=-\frac{\pi}{2},\quad \lim_{x\rightarrow 0^-}S(x)=\frac{\pi}{2},\]而由Lagrange中值定理有\[\lim_{x\rightarrow 0^+}\frac{S(x)-S(0)}{x}=\lim_{x\rightarrow 0^+}S'(\xi),\quad \left(0<\xi<x\right),\]由于\(\lim_{x\rightarrow 0^+}S'(x)=-\frac{\pi}{2}\), 从而\(\lim_{x\rightarrow0^+}\frac{S(x)-S(x)}{x}=-\frac{\pi}{2}\), 同理可知\(\lim_{x\rightarrow0^-}\frac{S(x)-S(x)}{x}=\frac{\pi}{2}\). 即\(S(x)\)在\(x=0\)处单侧导数不等, 从而和函数在\(x=2k\pi\)处不可微.
\end{solution}
\woe 求实数\(a\)的取值范围,  使得\(\sum_{n=20}^{\infty}\frac{(-1)^nx}{n^{ax}+(-1)^nx}\)关于\(x\in\left[1,+\infty\right)\)一致收敛.
\begin{solution}
当\(a\leqslant 0\)时, 由于\(\lim_{n\rightarrow+\infty}\frac{(-1)^nx}{n^{ax}+(-1)^nx}\ne 0\), 从而\(a>0\).

考虑级数\(\sum_{n=20}^{+\infty}\frac{(-1)^nx}{n^{ax}}\), 由Dirichlet判别法可知其在题设范围内一致收敛. 于是\[\frac{(-1)^nx}{n^{ax}}-\frac{(-1)^nx}{n^{ax}+(-1)^nx}=\frac{x}{n^{ax}\left(n^{ax}+(-1)^nx\right)}\sim\frac{x}{n^{2ax}},\]由此知\(a>\frac{1}{2}\)时原级数关于\(x\in\left[1,+\infty\right)\)一致收敛. 
\end{solution}
\woe 求实数\(a\)的取值范围, 使得函数项级数\(\sum_{n=2}^{\infty}\frac{\sin nx}{n^{ax}+\sin nx}\)关于\(x\in\left[\pi,+\infty\right)\)一致收敛.
\begin{solution}
当\(a\leqslant 0\)时, 由于\(\lim_{n\rightarrow+\infty}\frac{\sin nx}{n^{ax}+\sin nx}\ne 0\), 从而\(a>0\).

\end{solution}
\woe 证明定理 \reff{Th:AD}, 即Abel判别法和Dirichlet判别法.
\begin{proof}
我们先证明这样一件事: 设\(b_1\geqslant b_2\geqslant\cdots\geqslant b_n\)或\(b_1\leqslant b_2\leqslant\cdots\leqslant b_n\), 记\(A_k=\sum_{i=1}^{k}a_i\), 如果\(\left|A_k\right|\leqslant M(k=1,2,\cdots,n)\), 那么\(\left|\sum_{k=1}^{n}a_kb_k\right|\leqslant M(|b_1|+2|b_n|)\).

由Abel变换, 有\[\begin{split}
\left|\sum_{k=1}^{n}a_kb_k\right|&=\left|A_nb_n-\sum_{k=1}^{m-1}A_k(b_{k+1}-b_k)\right|\leqslant\left|A_n\right|\left|b_n\right|+\sum_{k=1}^{n-1}\left|A_k\right|\left|b_{k+1}-b_k\right|\\&\leqslant M\left(\sum_{k=1}^{n-1}\left|b_k-b_{k+1}\right|+|b_n|\right)= M\left(|b_1-b_n|+|b_n|\right)\leqslant M(|b_1|+2|b_n|).
\end{split}\]这正是Abel引理. 

Abel判别法. 设对任意\(\boldsymbol{x}\in E\)与\(n\in\bbn_+\)有\(\left|v_n(\boldsymbol{x})\right|\leqslant M\). 因为\(\sum_{k=1}^{\infty}u_k(\boldsymbol{x})\)在\(E\)上一致收敛, 所以\(\forall\varepsilon>0\), 存在\(N\)使得\(n>N\)时有\(\left|\sum_{k=n+1}^{n+p}u_k(\boldsymbol{x})\right|<\frac{\varepsilon}{3M}\)对任意\(\boldsymbol{x}\in E\)与\(p\in\bbn_+\)成立, 于是由Abel引理, 即得\[\left|\sum_{k=n+1}^{n+p}u_k(\boldsymbol{x})v_k(\boldsymbol{x})\right|\leqslant\frac{\varepsilon}{3M}\left(|v_{n+1}(\boldsymbol{x})|+2|v_{n+p}(\boldsymbol{x})|\right)<\varepsilon,\]因而级数\(\sum_{n=1}^{\infty}u_k(\boldsymbol{x})v_k(\boldsymbol{x})\)在\(E\)上一致收敛.

Dirichlet判别法. 设对任意\(\boldsymbol{x}\in E\)与\(n\in\bbn_+\)有\(\left|\sum_{k=1}^{n}u_k(\boldsymbol{x})\right|\leqslant M\). 则对\(n+1\leqslant m\leqslant n+p\), 有\[\left|\sum_{k=n+1}^{m}u_k(\boldsymbol{x})\right|=\left|\sum_{k=1}^{m}u_k(\boldsymbol{x})-\sum_{k=1}^{n}u_k(\boldsymbol{x})\right|\leqslant\left|\sum_{k=1}^{m}u_k(\boldsymbol{x})\right|+\left|\sum_{k=1}^{n}u_k(\boldsymbol{x})\right|\leqslant 2M,\]另外, 由于\(\{v_k(\boldsymbol{x})\}\)在\(E\)上一致收敛到零, 则对任意\(\varepsilon\), 有\(N\)使\(n>N\)时有\(|v_n(\boldsymbol{x})|<\frac{\varepsilon}{6M}\), 于是结合Abel引理有
\[\left|\sum_{k=n+1}^{n+p}u_k(\boldsymbol{x})v_k(\boldsymbol{x})\right|\leqslant 2M\left(|v_{n+1}(\boldsymbol{x})|+2|v_{n+p}(\boldsymbol{x})|\right)<\varepsilon,\]因而级数\(\sum_{n=1}^{\infty}u_k(\boldsymbol{x})v_k(\boldsymbol{x})\)在\(E\)上一致收敛.
\end{proof}
\woe 设\(S(x)=\sum_{n=1}^{\infty}\frac{\sin 2^nx}{2^n}\). 证明: 对任意\(j,k\in\mathbb{Z}\), \(S\)在点\(\frac{j}{2^k}\)不可导.
\begin{proof}
易见\(S(x)\)的一般项\(\frac{\sin 2^nx}{2^n}\)都是可导的, 对于任意\(k\in\bbz\), 置\(x=\frac{t}{2^k}\), 我们有\[\sum_{n=1}^{\infty}\frac{\sin 2^nx}{2^n}=\sum_{n=1}^{\infty}\frac{\sin 2^{n-k}t}{2^n}=\frac{1}{2^k}\sum_{n=1}^{\infty}\frac{\sin 2^{n-k}t}{2^{n-k}}=\frac{1}{2^{k}}\left(\sum_{n=1}^{k}\frac{\sin 2^{n-k}t}{2^{n-k}}+\sum_{n=1}^{\infty}\frac{\sin 2^nt}{2^n}\right),\]这意味着, 我们只需证\(S(x)\)在点\(j\)处不可导.

记\(u_k(x)=\frac{\sin 2^kx}{2^k}\), 取序列\(j_n=j+\frac{\pi}{2^n}\), 则\(\lim_{n\rightarrow+\infty}j_n=j\).  若\[\lim_{n\rightarrow+\infty}\frac{S(j_n)-S(j)}{j_n-j}\]不存在, 则\(S(x)\)在\(j\)处不可导, 注意到\(u_k(x)\)以\(\frac{\pi}{2^{k-1}}\)为周期, 于是有\[u_k(j_n)=\frac{\sin\left(2^kj+2^{k-n}\pi\right)}{2^k}=u_k(j),\quad k=n+1,n+2,\cdots,\]则\[\frac{u_k(j_n)-u_k(j)}{j_k-j}=\begin{cases}
\frac{\sin\left(2^kj+2^{k-n}\pi\right)-\sin 2^kj}{2^{k-n}\pi},\quad &k=1,2,\cdots,n\\
0,&k=n+1,n+2,\cdots
\end{cases}\]于是有\[\begin{split}
&\frac{S(j_n)-S(j)}{j_n-j}=\sum_{k=1}^{\infty}\frac{u_k(j_n)-u_k(j)}{j_n-j}=\sum_{k=1}^{n}\frac{u_k(j_n)-u_k(j)}{j_n-j}\\=&\sum_{k=1}^{n}2^{n-k}\pi\left(\sin\left(2^kj+2^{k-n}\pi\right)-\sin 2^kj\right)\\=&\sum_{k=1}^{n}\pi^2\cos\left(2^kj+\xi_k\right),\quad \xi_k\in\left(0,2^{k-n}\pi\right)\subset\left(0,\pi\right),
\end{split}\]
下证\(\lim_{k\rightarrow+\infty}\cos\left(2^kj+\xi_k\right)\ne 0\). 否则对于充分大的\(k\), 存在\(k_1,k_2\in\bbz\)以及任意小的\(\varepsilon_1,\varepsilon_2\)使得\[2^kj+\xi_k=k_1\pi+\frac{\pi}{2}+\varepsilon_1,\quad 2^{k-1}j+\xi_{k-1}=k_2\pi+\frac{\pi}{2}+\varepsilon_2,\]二倍右式减去左式得\[2\xi_{k-1}-\xi_k=\left(2k_2-k_1\right)\pi+\frac{\pi}{2}+2\varepsilon_2-\varepsilon_1,\]进一步有\[\frac{2\xi_{k-1}-\xi_k-\pi/2}{\pi}=2k_2-k_1+\frac{2\varepsilon_2-\varepsilon_1}{\pi},\]注意\(\frac{2\varepsilon_2-\varepsilon_1}{\pi}\)可以任意小, 而\[-\frac{3}{2}-\frac{2\varepsilon_2-\varepsilon_1}{\pi}<\frac{2\xi_{k-1}-\xi_k-\pi/2}{\pi}<\frac{3}{2}-\frac{2\varepsilon_2-\varepsilon_1}{\pi},\]这意味着\(2k_2-k_1\)只能取\(-1,0,1\)这三个值,
%%%%%%%%%
%如果\(S'(j)\)存在则\[\lim_{x\rightarrow j}\frac{1}{(x-j)^2}\int_{j}^{x}\left(S(x)-S(j)\right)\dd x=\lim_{x\rightarrow j}\frac{S(x)-S(j)}{2(x-j)}=\frac{1}{2}S'(0),\]另一方面\[\begin{split}
%&\frac{1}{(x-j)^2}\int_{j}^{x}\left(S(x)-S(j)\right)\dd x=\frac{1}{(x-j)^2}\int_{j}^{x}\left(\sum_{n=1}^{\infty}\frac{\sin 2^nx}{2^n}-\sum_{n=1}^{\infty}\frac{\sin 2^nj}{2^n}\right)\dd x\\=&\sum_{n=1}^{\infty}\left(\frac{\cos 2^n j-\cos 2^n x}{4^n(x-j)^2}-\frac{\sin 2^nj}{2^n\left(x-j\right)}\right)=:M,
%\end{split}\]对函数\(f(x)=\cos 2^nx\)在\(x=j\)处Taylor展开有\[\cos 2^n x=\cos 2^nj-2^n(x-j)\sin 2^nj-\frac{4^n(x-j)^2}{2!}\cos 2^nj+(x-j)^3\sum_{p=3}^{\infty}\frac{f^{(p)}(j)}{p!}(x-j)^{p-3},\]将此代入\(M\), 即得\[\begin{split}
%M=\sum_{n=1}^{\infty}\left(\frac{1}{2}\cos 2^nj-(x-j)\sum_{p=3}^{\infty}\frac{f^{(p)}(j)}{4^np!}(x-j)^{p-3}\right)\rightarrow\sum_{n=1}^{\infty}\frac{\cos 2^nj}{2},\quad x\rightarrow j,
%\end{split}\]后者一般项不趋于零, 从而\(M\)发散, 矛盾. 因而结论得证.
%%%%%%%%%
\end{proof}
\end{quiza}
\begin{quizb}
\woe 设\(S(x)=\sum_{n=1}^{\infty}\frac{\sqrt{x}}{\sqrt{n}}\ee^{-nx}\), 求\(S\)的定义域并讨论其连续性, 可微性.
\begin{solution}
我们先证明这样一个命题, 事实证明, 该命题对于今后证明某函数项级数非一致收敛是十分方便的: 

只要\(x\geqslant\delta>0\), 结合\[\ee^{nx}\geqslant 1+nx+\frac{1}{2}(nx)^2\geqslant \frac{(nx)^2}{2}\Rightarrow \ee^{-nx}\leqslant\frac{2}{(nx)^2},\]有\[\frac{\sqrt{x}}{\sqrt{n}}\ee^{-nx}\leqslant\frac{\sqrt{x}}{\sqrt{n}}\cdot\frac{2}{n^2x^2}=\frac{2}{n^{5/2}x^{3/2}}\leqslant\frac{2}{n^{5/2}\delta^{3/2}},\]由Weierstrass判别法可知\(S(x)\)在\(\left[\delta,+\infty\right)\)上一致收敛. 

另一方面, 记\(u_n=\frac{\sqrt{x}}{\sqrt{n}}\ee^{-nx}\), 取\(x_n=\frac{1}{n}\), 则\(u_n(x_n)=\frac{1}{\ee n}\), 易见\(\sum_{n=1}^{\infty}u_n(x_n)\)发散, 从而\(S(x)\)在\((0,\delta)\)上不一致收敛. 
\end{solution}
\woe 设\(x=-a_n\)是方程\(1+x+\frac{x^2}{2!}+\frac{x^3}{3!}+\cdots+\frac{x^{2n-1}}{(2n-1)!}=0\)的\textcolor{green!50!black}{实数}解, \(f(x)=\sum_{n=1}^{\infty}a_nx^p\ee^{-a_nx}\), 其中\(p\in\mathbb{R}\)为常数, 证明: \begin{compactenum}[(1)]
\item 存在正常数\(C_1,C_2\)使得\(C_1n\leqslant a_n\leqslant C_2n(n=1,2,\cdots)\);
\item \(f\)在\((0,+\infty)\)内连续.
\end{compactenum}
\begin{proof}
(1)原方程有且仅有一实根\(x_n\)(可见6.3 \(\boldsymbol{\mathcal{B}}\)第\textbf{9}题), 且\(x_n<0\), 否则\(x_n\geqslant 0\), 此时\[1+x_n+\frac{x_n^2}{2!}+\frac{x_n^3}{3!}+\cdots+\frac{x_n^{2n-1}}{(2n-1)!}>0\]由于\(a_n=-x_n\), 故\(a_n>0\). 由Taylor公式\[\ee^{x}=1+x+\frac{x^2}{2!}+\cdots+\frac{x^{2n-1}}{(2n-1)!}+\frac{\ee^{\xi}x^{2n}}{(2n)!},\quad \xi\text{\, 介于\(0\)于\(x\)之间},\]
将\(x=-a_n\)代入上式, 得到\[\ee^{-a_n}=\frac{\ee^{\xi}a_n^{2n}}{(2n)!},\quad\xi\in(-a_n,0)\]注意到\(\frac{\ee^{-a_n}a_n^{2n}}{(2n)!}\leqslant\frac{\ee^{\xi}a_n^{2n}}{(2n)!}\leqslant\frac{a_n^{2n}}{(2n)!}\), 于是\[\frac{\ee^{-a_n}a_n^{2n}}{(2n)!}\leqslant\ee^{-a_n}\Rightarrow a_n\leqslant\sqrt[2n]{(2n)!},\]下面我们证明一个有用的不等式\[\left(\frac{n}{\ee}\right)^n<n!<\ee\left(\frac{n}{2}\right)^n,\]由\(\sqrt{k(n-k)}\leqslant\frac{n}{2}\), 则\(\frac{1}{2}\left(\ln k+\ln(n-k)\right)\leqslant\ln\frac{n}{2}\), 从而\[\sum_{k=1}^{n-1}\ln k\leqslant(n-1)\ln\frac{n}{2}\Rightarrow (n-1)!\leqslant\left(\frac{n}{2}\right)^n,\]即有\(\frac{n}{2}\cdot (n-1)!\leqslant\left(\frac{n}{2}\right)^n\), 于是\(n!\leqslant 2\left(\frac{n}{2}\right)^n<\ee\left(\frac{n}{2}\right)^n\). 另外, 令\(x_n=\left(\frac{n}{\ee}\right)^n\), 于是有\[\frac{x_n}{x_{n-1}}=\frac{n^n}{(n-1)^{n-1}\ee}=\frac{\left(1+\displaystyle\frac{1}{n-1}\right)^{n-1}n}{\ee}<n,\]并且\(x_1=\frac{1}{\ee}<1\), 于是\(x_n=x_1\cdot\frac{x_2}{x_1}\cdot\cdots\cdot\frac{x_n}{x_{n-1}}<n!\), 即得\(\left(\frac{n}{\ee}\right)^n<n!\).

利用上述不等式, 有\[a_n\leqslant\sqrt[2n]{(2n)!}<\sqrt[2n]{\ee\left(\frac{2n}{2}\right)^{2n}}=\sqrt[2n]{\ee}\cdot n<\sqrt{\ee}\cdot n,\]另一方面,\[\ee^{-a_n}\leqslant\frac{a_n^{2n}}{(2n)!}\Rightarrow a_n\geqslant\sqrt[2n]{\ee^{-a_n}(2n)!}>\sqrt[2n]{\ee^{-\sqrt{\ee} n}}\cdot \frac{2n}{\ee}=2\ee^{-\sqrt{\ee}/2-1}n,\]即得\(2\ee^{-\sqrt{\ee}/2-1}n\leqslant a_n\leqslant\sqrt{\ee}\cdot n.\)

(2)\(\{a_n\}\)为正并且严格单调递增.
\end{proof}
\end{quizb}
设\(u_n(x)(n=1,2,\cdots)\)是定义在区间\(I\)上的函数, 且满足下列条件之一, 则函数项级数\(\sum_{n=1}^{\infty}u_n(x)\)在\(I\)上不一致收敛.
\begin{compactenum}[(1)]
\item 存在数列\(\{x_n\}\subset I\), 使得\(\sum_{n=1}^{\infty}u_n(x_n)\)发散.
\item \(\forall x\in I\), 有\(u_n(x)>0(n=1,2,\cdots)\), 且存在数列\(\{x_n\}\subset I\)使得\[\varliminf_{n\rightarrow+\infty}n^\lambda u_n(x_n)>0,\quad \lambda\leqslant 1.\]
\item \(u_n(x)=\frac{a_n(x)}{b_n(x)}(x\in I,n=1,2,\cdots)\), 对于每个\(x\in I\), \(a_n(x)\)与\(b_n(x)\)均为关于\(n\)的单调增加的正值函数, 且存在数列\(\{x_n\}\subset I\)使得\[\varliminf_{n\rightarrow+\infty}\frac{na_n(x_n)}{b_{2n}(x_n)}>0.\]
\begin{proof}

\end{proof}
\end{compactenum}

\section{幂级数与函数的幂级数展开}
\precis{幂级数的收敛半径,Abel第一定理,Cauchy-Hadamard公式,幂级数的性质,Abel第二定理,函数的幂级数展开,Taylor级数,Maclaurin级数,直接法,间接法,幂级数在复数域内的性质,非切向极限,实解析函数,复解析函数,复区域上函数的复导数,Cauchy-Riemann条件}
\begin{quiza}
\woe 举例说明存在幂级数使得其收敛域分别为\[\{0\},\quad\left(-1,1\right),\quad\left[-1,1\right),\quad\left(-1,1\right],\quad\left[-1,1\right],\quad\left(-\infty,+\infty\right).\]
\begin{solution}
级数\(\sum_{n=0}^{\infty}n!x^n\)仅在\(x=0\)处收敛, 因为除此之外的任何\(x\)都使级数的一般项不趋于零.

容易验证其他题设的收敛域可以对应以下级数:\[\sum_{n=1}^{\infty}\frac{x^{2n}}{n},\quad\sum_{n=1}^{\infty}\frac{x^{n}}{n},\quad\sum_{n=1}^{\infty}\frac{(-1)^nx^{n}}{n},\quad\sum_{n=1}^{\infty}\frac{x^{2n}}{n^2},\quad\sum_{n=0}^{\infty}\frac{x^{n}}{n!}.\qedhere\]
\end{solution}
\woe 考虑收敛半径为1的幂级数\(\sum_{n=0}^{\infty}a_nx^n\). 举例或说明:\begin{compactenum}[(1)]
\item 是否存在例子使得\(\sum_{n=0}^{\infty}a_nx^n\)及其形式求导后的级数在\([-1,1]\)上收敛?
\item 是否存在例子使得\(\sum_{n=0}^{\infty}a_nx^n\)在\([-1,1]\)上收敛, 而其形式求导后的级数的收敛域是\(\left(-1,1\right]\)?
\item 当\(\sum_{n=0}^{\infty}a_nx^n\)的收敛域是\(\left(-1,1\right]\)时, 使得级数形式求导\(k\)次后的级数的收敛域依然为\(\left(-1,1\right]\), 这样的\(k\)可以用多大?
\end{compactenum}
\begin{solution}
(1)易见级数\(\sum_{n=1}^{\infty}\frac{x^n}{n^3}\)的收敛半径为\(1\), 形式求导后得到\(\sum_{n=1}^{\infty}\frac{x^{n-1}}{n^2}\)其收敛域为\([-1,1]\).

(2)

(3)
\end{solution}
\woe 试将以下函数展开成Maclaurin级数, 并求其收敛域:\vspace{8pt}\\
\begin{tabular}{lcl}
\((1)\,\frac{x^2+x+4}{x^2+3x+2}\);&\qquad\qquad&\((2)\,\frac{x}{x^2+2x+2}\);\vspace{0.3cm}\\
\((3)\,\frac{1}{x^4+4x^2+4}\);&&\((4)\,\ee^x\cos x\);\vspace{0.3cm}\\
\((5)\,\int_{0}^{x}\frac{\ee^t-1}{t}\dd t\);&&\((6)\,\ln(x+\sqrt{1+x^2})\).\vspace{0.3cm}\\
\end{tabular}
\begin{solution}
(1)由\[\frac{x^2+x+4}{x^2+3x+2}=1+\frac{3}{1+x}-\frac{4}{2+x}=1+3\sum_{n=0}^{\infty}(-1)^nx^n-2\sum_{n=0}^{\infty}(-1)^n\left(\frac{x}{2}\right)^n,\]级数的收敛域为\((-1,1)\).

(2)由\[\begin{split}
\frac{x}{x^2+2x+2}&=\frac{x}{(x+1+\ii)(x+1-\ii)}=\frac{\ii x}{2}\left(\frac{1}{x+1+\ii}-\frac{1}{x+1-\ii}\right)\\&=\sum_{n=0}^{\infty}(-1)^n\frac{\ii}{2}\left(\left(\frac{1}{1+\ii}\right)^{n+1}-\left(\frac{1}{1-\ii}\right)^{n+1}\right)x^n
\end{split}\]
\end{solution}
\woe 试给出\(\arctan x\)在点1处的Taylor展开式及其收敛域.
\begin{solution}
由于\(\left(\arctan x\right)'=\frac{1}{1+x^2}=\sum_{n=0}^{\infty}(-1)^nx^{2n}\), 于是\[\arctan x=\int_{0}^{x}\sum_{n=0}^{\infty}(-1)^nx^{2n}\dd x=\sum_{n=0}^{\infty}\frac{(-1)^{n+1}x^{2n+1}}{2n+1},\]
\end{solution}
\woe 设\(f\)的Maclaurin展开式在点\(x=1\)处绝对收敛, 证明\(f^2\)的Maclaurin展开式在点\(x=-1\)处绝对收敛.
\begin{proof}
记\(f(x)=\sum_{n=0}^{\infty}a_nx^n\), 有\(\sum_{n=0}^{\infty}\left|a_n\right|<+\infty\), 则\[f^2(x)\big|_{x=-1}=\left(\sum_{n=0}^{\infty}(-1)^na_n\right)^2\leqslant\left(\sum_{n=0}^{\infty}|a_n|\right)^2<+\infty.\qedhere\]
\end{proof}
\woe 证明: 对任何\(n\geqslant 1,\,(1-x)^n\ee^{1/(1-x)}\)的Maclaurin展开式的收敛半径为1, 但在点\(x=1\)处发散.
\begin{proof}

\end{proof}
\woe 考虑\(\frac{1}{1-x}\)的Maclaurin展开式的余项, 可得\[\int_{0}^{x}\frac{(n+1)(x-t)^n}{(1-t)^{n+2}}\dd t=\frac{x^{n+1}}{1-x},\quad x\in\left[0,1\right).\]试利用上式估计\((1+x)^\alpha\)的\(n\)阶Maclaurin展开式的余项\[r_n(x)=\int_{0}^{x}(n+1)(x-t)^n\binom{\alpha}{n+1}(1+t)^{\alpha-n-1}\dd t,\quad x\in(-1,1).\]进而, 证明\((1+x)^\alpha\)在\((-1,1)\)可展开成Maclaurin级数.
\woe 设\(\sum_{n=0}^{\infty}a_nx^n\)的收敛半径大于\(1\), 其和函数为\(f\), 满足\(\min_{x\in [-1,1]}\left|f(x)\right|\geqslant 1\). 设\(g=\frac{1}{f}\). 依次证明:
\begin{quizs}
\item 存在常数\(C>0\)使得\[\left|\frac{f^{(n)}(x)}{n!}\right|\leqslant \frac{C^n}{(1-\left|x\right|)^n},\quad n\geqslant 0,\left|x\right|\leqslant 1.\]
\item 记\(A=C+1\), 则\[\left|\frac{g^{(n)}(x)}{n!}\right|\leqslant\frac{A^{n+1}}{(1-\left|x\right|)^n},\quad n\geqslant 0,\left|x\right|\leqslant 1.\]
\item \(g\)在点0处可以展开成幂级数.
\end{quizs}
\begin{proof}

\end{proof}
\woe 设\(R\)为正实数, 若\(f\)在\((-R,R)\)内实解析, 其Maclaurin展开式的收敛半径是\(r\in \left(0,R\right]\). 问: 在收敛域内, \(f\)的Maclaurin展开式是否等于\(f\).
\begin{solution}

\end{solution}
\end{quiza}
\begin{quizb}
\woe 举例说明存在收敛半径为1的幂级数\(\sum_{n=0}^{\infty}a_nx^n\), 它任意次逐项求导后的级数均在点\(x=1\)处收敛. 进一步, 证明: 此类级数也一定在点\(-1\)处收敛.
\begin{proof}

\end{proof}
\woe 举例说明存在收敛半径为1的幂级数\(\sum_{n=0}^{\infty}a_nx^n\), 它任意次逐项积分后得到的级数均在点\(x=1\)处发散. 进一步, 证明: 此类级数也一定在点\(-1\)处发散.
\begin{proof}

\end{proof}
\woe 设\(x_0\in\left(0,\frac{\pi}{2}\right)\), 试考察\(\tan x\)在点\(x_0\)处的Taylor展开式的收敛半径.
\begin{solution}

\end{solution}
\woe 对于Riemann \(\zeta\)函数\(\zeta(x)=\sum_{n=1}^{\infty}\frac{1}{n^x}\). 证明: \(\zeta\)在\((1,+\infty)\)上实解析.
\woe 考察是否存在\(n\geqslant 1\), 使得\((1-x)^n\sin\frac{1}{1-x}\)的Maclaurin展开式在点\(x=1\)处收敛.
\woe 证明: \((a,b)\)上的实解析函数是一个复区域\(D\)上的复解析函数在\((a,b)\)上的限制.
\woe 设\(\varphi\in C_c^{\infty}(\mathbb{R})\)满足\(\mathrm{supp}\,\varphi\subseteq [-2,2]\)以及\(\varphi\big|_{[-1,1]}=1\). 对于实数列\(\{a_k\}_{k=0}^{\infty}\), 定义\[f(x)=\sum_{n=0}^{\infty}\frac{a_n}{n!}\varphi(a_nx)x^n,\quad\forall x\in\mathbb{R}.\]证明: \(f\in C^{\infty}(\mathbb{R})\), 且\(f^{(n)}(0)=a_n(\forall n\geqslant 0)\). 此结果表明任何幂级数都是某个函数的Taylor展开式.
\end{quizb}
\section{幂级数的应用}
\precis{数项级数的计算,幂级数与三角级数,Abel和,Ces\'{a}ro和,Tauber型定理,母函数,Bernoulli多项式,Bernoulli数,幂级数的抽象应用,Hardy-Littlewood定理}
\begin{quiza}
\woe 利用幂级数计算级数\(\sum_{n=1}^{\infty}\frac{1}{(2n-1)(2n+1)(2n+3)}\).
\begin{solution}
由于\[\frac{1}{(2n-1)(2n+1)(2n+3)}=\frac{1}{8}\frac{1}{2n-1}-\frac{1}{4}\frac{1}{2n+1}+\frac{1}{8}\frac{1}{2n+3},\]考虑幂级数\[\sum_{n=1}^{\infty}\frac{x^{2n+3}}{(2n-1)(2n+1)(2n+3)}=\sum_{n=1}^{\infty}\left(\frac{x^4}{8}\frac{x^{2n-1}}{2n-1}-\frac{x^2}{4}\frac{x^{2n+1}}{2n+1}+\frac{1}{8}\frac{x^{2n+3}}{2n+3}\right),\]设\(S(x)=\sum_{n=1}^{\infty}\frac{x^{2n-1}}{2n-1}\), 则\(S'(x)=\sum_{n=1}^{\infty}x^{2n-2}=\frac{1}{1-x^2}\), 由此可得\[S(x)=\frac{1}{2}\ln\frac{1+x}{1-x},\quad\sum_{n=1}^{\infty}\frac{x^{2n+1}}{2n+1}=S(x)-x,\quad\sum_{n=1}^{\infty}\frac{x^{2n+3}}{2n+3}=S(x)-x-\frac{x^3}{3},\]于是\[\sum_{n=1}^{\infty}\frac{1}{(2n-1)(2n+1)(2n+3)}=\lim_{x\rightarrow 1^-}\frac{x^4}{8}S(x)-\frac{x^2}{4}\left(S(x)-x\right)+\frac{1}{8}\left(S(x)-x-\frac{x^3}{3}\right)=\frac{1}{12}.\qedhere\]
\end{solution}
\woe 计算\(\sum_{n=0}^{\infty}\frac{(-1)^n}{4n+1}\).
\begin{solution}
由于\(\sum_{n=0}^{\infty}(-1)^nx^{4n}=\frac{1}{1+x^4}\), 于是\[\sum_{n=0}^{\infty}\frac{(-1)^n}{4n+1}=\int_{0}^{1}\sum_{n=0}^{\infty}(-1)^nx^{4n}\dd x=\int_{0}^{1}\frac{1}{1+x^4}\dd x=\frac{\pi+\ln\left(1+\sqrt{2}\right)}{4\sqrt{2}}.\qedhere\]
\end{solution}
\woe 计算\(\sum_{n=1}^{\infty}\frac{1}{n^4}\).
\begin{solution}
在2.7\(\boldsymbol{\mathcal{B}}\) 第\textbf{5}题中, 我们得到了\[\sin\pi x=\pi x\prod_{n=1}^{\infty}\left(1-\frac{x^2}{n^2}\right),\]即有\(\frac{\sin \pi x}{\pi x}=\prod_{n=1}^{\infty}\left(1-\frac{x^2}{n^2}\right),\)注意到\[\frac{\sin \pi x}{\pi x}=1-\frac{\pi ^2 x^2}{6}+\frac{\pi ^4 x^4}{120}-o\left(x^5\right),\quad x\rightarrow 0,\]而根据根与系数的关系, 对右侧的无穷乘积有\[\prod_{n=1}^{\infty}\left(1-\frac{x^2}{n^2}\right)=1-x^2\sum_{n=1}^{\infty}\frac{1}{n^2}+x^4\sum_{i\ne j}\frac{1}{i^2\cdot j^2}+o\left(x^5\right),\quad x\rightarrow 0,\]于是有\(\sum_{n=1}^{\infty}\frac{1}{n^2}\frac{1}{n^2}=\frac{\pi^2}{6},\sum_{i\ne j}\frac{1}{i^2\cdot j^2}=\frac{\pi^4}{120}\), 而\[\sum_{n=1}^{\infty}\frac{1}{n^4}=\left(\sum_{n=1}^{\infty}\frac{1}{n^2}\right)^2-2\sum_{i\ne j}\frac{1}{i^2\cdot j^2}=\frac{\pi^4}{36}-\frac{\pi^4}{60}=\frac{\pi^4}{90}.\qedhere\]
\end{solution}
\woe 计算\(\sum_{n=1}^{\infty}\frac{(n+1)H_n}{2^n},\sum_{n=1}^{\infty}\frac{H_n}{2^n},\sum_{n=1}^{\infty}\frac{H_n}{2^n(n+1)}\), 其中\(H_n=\sum_{k=1}^{n}\frac{1}{k}\).
\begin{solution}
注意到\[\sum_{n=1}^{\infty}\frac{(n+1)}{2^n}\sum_{k=1}^{n}\frac{1}{k}=\sum_{n=1}^{\infty}\frac{1}{n}\sum_{k=n}^{\infty}\frac{k+1}{2^k},\]而\[\begin{split}
\sum_{k=n}^{\infty}\frac{k+1}{2^k}&=\sum_{k=1}^{\infty}\frac{k+1}{2^k}-\sum_{k=1}^{n-1}\frac{k+1}{2^k}=\left.\left(\sum_{k=1}^{\infty}x^{k+1}-\sum_{k=1}^{n-1}x^{k+1}\right)'\right|_{x=1/2}\\&=\left.\left(\frac{x^2}{1-x}-\frac{x^2(1-x^{n-1})}{1-x}\right)'\right|_{x=1/2}=\frac{2+n}{2^{n-1}}.
\end{split}\]于是\[\begin{split}
\sum_{n=1}^{\infty}\frac{(n+1)}{2^n}\sum_{k=1}^{n}\frac{1}{k}&=\sum_{n=1}^{\infty}\frac{1}{n}\sum_{k=n}^{\infty}\frac{k+1}{2^k}=\sum_{n=1}^{\infty}\frac{2+n}{n2^{n-1}}=\sum_{n=1}^{\infty}\frac{1}{2^{n-1}}+4\sum_{n=1}^{\infty}\frac{1}{n2^n}\\&=2+4\sum_{n=1}^{\infty}\int_{0}^{1/2}x^{n-1}\dd x=2+4\int_{0}^{1/2}\frac{1}{1-x}\dd x=2+4\ln 2.
\end{split}\]
同理\[\begin{split}
\sum_{n=1}^{\infty}\frac{H_n}{2^n}&=\sum_{n=1}^{\infty}\frac{1}{n}\sum_{k=n}^{\infty}\frac{1}{2^n}=\sum_{n=1}^{\infty}\frac{1}{n}\left(\sum_{k=1}^{\infty}\frac{1}{2^n}-\sum_{k=1}^{n-1}\frac{1}{2^n}\right)=\sum_{n=1}^{\infty}\frac{1}{n\cdot 2^{n-1}}\\&=2\left(\sum_{n=1}^{\infty}\int_{0}^{1/2}x^{n-1}\dd x\right)=2\int_{0}^{1/2}\frac{1}{1-x}\dd x=2\ln 2.\end{split}\]以及\[\begin{split}
\sum_{n=1}^{\infty}\frac{H_n}{2^n(n+1)}&=\sum_{n=1}^{\infty}\frac{1}{n}\sum_{k=n}^{\infty}\frac{1}{2^k(k+1)}=\sum_{n=1}^{\infty}\frac{1}{n}\left(\sum_{k=1}^{\infty}\frac{1}{2^k(k+1)}-\sum_{k=1}^{n-1}\frac{1}{2^k(k+1)}\right)\\&=2\sum_{n=1}^{\infty}\frac{1}{n}\left(\int_{0}^{1/2}\frac{x}{1-x}\dd x-\int_{0}^{1/2}\frac{x-x^n}{1-x}\dd x\right)=2\sum_{n=1}^{\infty}\frac{1}{n}\int_{0}^{1/2}\frac{x^n}{1-x}\dd x\\&=2\int_{0}^{1/2}\frac{1}{1-x}\int_{0}^{x}\frac{1}{1-t}\dd t\dd x=2\int_{0}^{1/2}\frac{\ln (1-x)}{x-1}\dd x=\left(\ln 2\right)^2.\qedhere\end{split}\]
\end{solution}
\woe 证明: 若级数\(\sum_{n=0}^{\infty}a_n\) Ces\'{a}ro可和, 则\(\left\lbrace\frac{a_n}{n+1}\right\rbrace \)有界.
\begin{proof}
记\(S_n=\sum_{k=1}^{N}a_k,\,\sigma_n=\frac{1}{n+1}\sum_{k=0}^{n}S_k\), 由题设\(\sigma=\lim_{n\rightarrow+\infty}\sigma_n\).  由于\[\frac{S_{n}}{n+1}=\frac{(n+1)\sigma_n-n\sigma_{n-1}}{n+1}=\sigma_n-\frac{n}{n+1}\sigma_{n-1}\rightarrow 0,\quad n\rightarrow+\infty,\]于是\[\frac{a_n}{n+1}=\frac{S_n-S_{n-1}}{n+1}=\frac{S_n}{n+1}-\frac{n}{n+1}\frac{S_{n-1}}{n}\rightarrow 0,\quad n\rightarrow+\infty,\]从而\(\left\lbrace\frac{a_n}{n+1}\right\rbrace \)有界.
\end{proof}
\woe 举例说明存在Abel可和但不Ces\'{a}ro可和的级数.
\begin{solution}
\(\sum_{n=0}^{\infty}(-1)^n(n+1)\). 容易验证该级数Abel可和, 但不Ces\'{a}ro可和.
\end{solution}
\woe 证明:\[\begin{split}
\pi&=\frac{1}{64}\sum_{n=0}^{\infty}\frac{(-1)^n}{1024^n}\left(-\frac{32}{4n+1}-\frac{1}{4n+3}+\frac{256}{10n+1}-\frac{64}{10n+3}-\right.\\&\left.\frac{4}{10n+5}-\frac{4}{10n+7}+\frac{1}{10n+9}\right).\end{split}\]
\begin{proof}
	我们将上式从左到右分拆成七个级数, 分别记为\(a_1,a_2,\cdots a_7\).
	
	我们先计算\(a_1=\sum_{n=0}^{\infty}\frac{(-1)^{n+1}32}{1024^n(4n+1)}\), 令\[f(x):=2^{5/2}\cdot 32\sum_{n=0}^{\infty}\frac{(-1)^{n+1}x^{4n+1}}{4n+1},\quad x\in(0,1),\]则\(f\left(2^{-5/2}\right)=a\), 并且有\(f'(x)=-\frac{2^{5/2}\cdot 32}{1+x^4}\), 即有\[-2^{5/2}\cdot 32\int_{0}^{2^{-5/2}}\frac{1}{1+x^2}\dd x=a,\]同理可得\[a_2=\sum_{n=0}^{\infty}\frac{(-1)^{n+1}}{1024^n(4n+3)}=\int_{0}^{2^{-5/2}}\frac{-(2^{5/2})^3x^2}{1+x^4}\dd x,\]
	以及\[a_k=\sum_{n=0}^{\infty}\frac{(-1)^{n+i_k}q_k}{1024^n\left(10n+(2k-5)\right)}=(-1)^{i_k}2^{2k-5}q_k\int_{0}^{1/2}\frac{x^{2k-6}}{1+x^{10}}\dd x,\quad k=3,4,5,6,7,\]上式中的\(i_k,q_k\)由题设公式给定. 则有\(64\pi=\sum_{k=1}^{7}a_k.\)
\end{proof}
\end{quiza}
\begin{quizb}
\woe 计算积分\(\int_{0}^{1}\frac{\ln(1+x)}{x}\dd x.\)
\begin{solution}
由\(\ln(1+x)\)的幂级数展开, \[\ln(1+x)=x-\frac{x^2}{2}+\frac{x^3}{3}+\cdots+\frac{(-1)^{n+1}x^n}{n}+\cdots,\]有\[\int_{0}^{1}\frac{\ln(1+x)}{x}\dd x=\lim_{n\rightarrow+\infty}\int_{0}^{1}\left(1-\frac{x}{2}+\frac{x^2}{3}+\cdots+\frac{(-1)^nx^n}{n}\right)\dd x=\sum_{n=1}^{\infty}\frac{(-1)^{n+1}}{n^2},\]
我们有\(\sum_{n=1}^{\infty}\frac{1}{n^2}=\frac{\pi^2}{6}\), 于是\(\sum_{n=1}^{\infty}\frac{(-1)^{n+1}}{n^2}=\sum_{n=1}^{\infty}\frac{1}{n^2}-\frac{1}{2}\sum_{n=1}^{\infty}\frac{1}{n^2}=\frac{\pi^2}{12}.\)
\end{solution}
\woe 试讨论级数\(\sum_{n=0}^{\infty}a_n\)和\(\sum_{n=0}^{\infty}b_n\)在各种收敛情况下(绝对收敛, 条件收敛, 发散), 它们的Cauchy乘积的收敛情况. 并给出证明或反例.
\begin{solution}

\end{solution}
\woe (小\(o\) Tauber定理)设\(\sum_{n=0}^{\infty}a_n\)的Abel和为A.证明:
\begin{quizs}
\item 若\(\lim_{n\rightarrow+\infty}na_n=0\), 则\(\sum_{n=0}^{\infty}a_n=A\).
\item \(\sum_{n=1}^{\infty}a_n\)收敛当且仅当\(\lim_{n\rightarrow+\infty}\frac{a_1+2a_2+\cdots+na_n}{n}=0.\)
\end{quizs}
\begin{proof}
(1)由于\(\lim_{n\rightarrow+\infty}na_n=0\), 可设\(\delta_n=\sup_{k\geqslant n}\{\left|k a_k\right|\}\), 则\(\{\delta_n\}\)单调递减趋于0. 令\[S(x)=\sum_{n=0}^{\infty}a_nx^n,\quad (0\leqslant x<1).\]对于任何正整数\(N\), 都有\[\begin{split}
\sum_{n=0}^{N}a_n-A&=\sum_{n=0}^{N}a_n-S(x)+S(x)-A=\sum_{n=1}^{N}a_n(1-x^n)-\sum_{n=N+1}^{\infty}a_nx^n+\left(S(x)-A\right)\\&=:I_1(x)+I_2(x)+I_3(x),
\end{split}\]对\(x\in[0,1)\), 有\[\begin{split}
&\left|I_1(x)\right|\leqslant (1-x)\sum_{n=1}^{N}|a_n|(1+x+\cdots+x^{n-1})\leqslant (1-x)\sum_{n=1}^{N}n|a_n|\leqslant (1-x)N\delta_1,\\
&\left|I_2(x)\right|\leqslant \sum_{n=N+1}^{\infty}|na_n|\frac{x^n}{n}\leqslant\frac{\delta_N}{N}\sum_{n=N+1}^{\infty}x^n=\frac{\delta_N}{N}\frac{x^{N+1}}{1-x}\leqslant\frac{\delta_N}{N(1-x)}.
\end{split}\]令\(x_N=1-\frac{\sqrt{\delta_N}}{N}\), 即\(N(1-x_N)=\sqrt{\delta_N}\). 易见当\(N\rightarrow+\infty\), \(x_N\rightarrow 1\). 可得\[|I_1(x_N)|\leqslant\delta_1\sqrt{\delta_N},\quad \left|I_2(x_N)\right|\leqslant\frac{\delta_N}{\sqrt{\delta_N}}=\sqrt{\delta_N},\quad I_3(x_N)=S(x_N)-A,\]结合\[\left|\sum_{n=0}^{N}a_n-A\right|\leqslant|I_1(x_N)|+|I_2(x_N)|+|I_3(x_N)|\leqslant (\delta_1+1)\sqrt{\delta_N}+\left|S(x_N)-A\right|,\]令\(N\rightarrow+\infty\)结论即证.

(2)必要性. 记\(A_k=a_1+a_2+\cdots+a_k\), 由题设\(\{A_n\}\)收敛到\(A\). 由Abel变换得到\[\sum_{k=1}^{n}ka_k=nA_n-\sum_{k=1}^{n-1}S_k,\Rightarrow\lim_{n\rightarrow+\infty}\frac{a_1+2a_2+\cdots+na_n}{n}=\lim_{n\rightarrow+\infty}\frac{1}{n}\left(nA_k-\sum_{k=1}^{n-1}A_k\right),\]由Stolz公式\(\lim_{n\rightarrow+\infty}\frac{1}{n}\left(nA_k-\sum_{k=1}^{n-1}A_k\right)=A-\lim_{n\rightarrow+\infty}\frac{A_n}{(n+1)-n}=A-A=0\). 从而必要性成立.

充分性. 记\(b_n=\frac{1}{n}\sum_{k=1}^{n}ka_k,b_0=0\), 则\(\lim_{n\rightarrow+\infty}b_n=0\), 由于\[na_n=nb_n-(n-1)b_{n-1}\Rightarrow a_n=b_{n}-b_{n-1}+\frac{b_{n-1}}{n},\]于是\(\sum_{n=1}^{\infty}a_n=\sum_{n=1}^{\infty}\frac{b_{n-1}}{n}\), 由\(\lim_{n\rightarrow+\infty}n\cdot\frac{b_{n-1}}{n}=0\)结合(1)可知\(\sum_{n=1}^{\infty}a_n\)收敛.
\end{proof}
\woe \textbf{Hardy-Littlewood(哈代-利特尔伍德)定理}: 设\(S_n\geqslant 0,\lim_{x\rightarrow 1^-}(1-x)\sum_{n=0}^{\infty}S_nx^n=A,\)则\[\lim_{n\rightarrow+\infty}\frac{1}{n}\sum_{k=0}^{n}S_k=A.\]请按以下步骤证明该定理.
\begin{compactenum}[(1)]
\item 当\(f\)为多项式时, 成立\begin{equation}\label{get}\tag{$\clubsuit$}
\lim_{x\rightarrow 1^-}(1-x)\sum_{n=0}^{\infty}S_nx^nf(x^n)=A\int_{0}^{1}f(x)\dd x.
\end{equation}
\item 对任何\(f\in C[0,1]\), \eqref{get}式成立.
\item 对任何分段常值函数, \eqref{get}式成立.
\item 取\(f(x)=\frac{1}{x}\chi_{[1/\ee,1]}(x), x_n=1-\frac{1}{n},\)证明\(\frac{1}{n}\sum_{k=0}^{n}S_k=(1-x_n)\sum_{n=1}^{\infty}S_kx_n^kf(x_n^k)\)并结束定理的证明.
\end{compactenum}
\begin{proof}
不妨设\(A=1\).
\begin{asparaenum}[\bfseries (1)]
\item 对于任何\(k\in\bbn_+\), 由于\(1-x^{k+1}=(1-x)(1+x+x^2+\cdots+x^k),\)我们有\[\lim_{x\rightarrow 1^-}(1-x^{k+1})\sum_{n=0}^{\infty}S_nx^{n(k+1)}=\lim_{x\rightarrow 1^-}(1-x)\sum_{n=0}^{\infty}S_nx^n=1,\]从而\[\lim_{x\rightarrow 1^-}(1-x)\sum_{n=0}^{\infty}S_nx^nx^{nk}=\frac{1}{k+1}=\int_{0}^{1}x^k\dd x,\]由此对于任何多项式\(f(x)\), 成立\(\lim_{x\rightarrow 1^-}(1-x)\sum_{n=0}^{\infty}S_nx^nf\left(x^n\right)=\int_{0}^{1}f(x)\dd x\).
\item 现任取\(f\in C[0,1]\), 则有多项式\(P_k\)使得\[\varepsilon_k\equiv\max_{x\in[0,1]}\left|P_k(x)-f(x)\right|\rightarrow 0,\quad k\rightarrow+\infty,\]则对于\(x\in(0,1)\), 有\[\begin{split}
&\left|(1-x)\sum_{n=0}^{\infty}S_nx^nf\left(x^n\right)-\int_{0}^{1}f(x)\dd x\right|\leqslant (1-x)\sum_{n=0}^{\infty}S_nx^n\left|f\left(x^n\right)-P_k\left(x^n\right)\right|\\+&\left|(1-x)\sum_{n=0}^{\infty}S_nx^nP_k\left(x^n\right)-\int_{0}^{1}P_k(x)\dd x\right|+\int_{0}^{1}\left|f(x)-P_k(x)\right|\dd x\\\leqslant&\varepsilon_k(1-x)\sum_{n=0}^{\infty}S_nx^n+\left|(1-x)\sum_{n=0}^{\infty}S_nx^nP_k\left(x^n\right)-\int_{0}^{1}P_k(x)\dd x\right|+\varepsilon_k,\quad \forall k\geqslant 0,
\end{split}\]令\(x\rightarrow 1^-\)得到\[\varlimsup_{x\rightarrow 1^-}\left|(1-x)\sum_{n=0}^{\infty}S_nx^nP_k\left(x^n\right)-\int_{0}^{1}P_k(x)\dd x\right|\leqslant 2\varepsilon_k,\quad\forall k\geqslant 0.\]在令\(k\rightarrow+\infty\)得到\[\lim_{x\rightarrow 1^-}(1-x)\sum_{n=0}^{\infty}S_nx^nf\left(x^n\right)=\int_{0}^{1}f(x)\dd x.\]
\item 现对分段常值函数证明上式. 任取\(0<a<b<1\), 我们先证\(f=\chi_{[a,b]}\)的情况. 取\(\varepsilon\in\left(0,\min\left\lbrace a,1-b,\frac{b-a}{3}\right\rbrace \right)\). 做分段连续函数\(f^{\varepsilon}\)与\(f_{\varepsilon}\), 使得\(f^{\varepsilon}\)在\([0,a-\varepsilon]\)和\([b+\varepsilon,1]\)上的取值为\(0\), 在\([a,b]\)上为\(1\), 在余下的区间\([a-\varepsilon,a]\)和\([b,b+\varepsilon]\)上用直线连接. 而\(f_\varepsilon\)在\([0,a]\)与\([b,1]\)上为\(0\), 在\([a+\varepsilon,b-\varepsilon]\)上为1, 在余下的区间\([a,a+\varepsilon]\)和\([b-\varepsilon,b]\)上用直线连接. \[f^\varepsilon(x)=\begin{cases}
1, & x\in[a,b],\\
\frac{x-a+\varepsilon}{\varepsilon},\quad&x\in[a-\varepsilon,a],\\
\frac{b+\varepsilon-x}{\varepsilon},&x\in[b,b+\varepsilon],\\
0,&x\in[0,a-\varepsilon]\cup[b+\varepsilon,1],
\end{cases}
f_\varepsilon(x)=\begin{cases}
1, & x\in[a+\varepsilon,b-\varepsilon],\\
\frac{x-a}{\varepsilon},\quad&x\in[a,a+\varepsilon],\\
\frac{b-x}{\varepsilon},&x\in[b-\varepsilon,b],\\
0,&x\in[0,a]\cup[b,1],
\end{cases}\]则\(f^\varepsilon,f_\varepsilon\)在\([0,1]\)上连续, 且\(\forall x\in(0,1)\),\[f_\varepsilon(x)\leqslant f(x)\leqslant f^\varepsilon(x).\]从而\[(1-x)\sum_{n=0}^{\infty}S_nx^nf_\varepsilon\left(x^n\right)\leqslant (1-x)\sum_{n=0}^{\infty}S_nx^nf\left(x^n\right)\leqslant (1-x)\sum_{n=0}^{\infty}S_nx^nf^\varepsilon\left(x^n\right),\]令\(x\rightarrow 1^-\)得到\[\int_{0}^{1}f_\varepsilon(x)\dd x\leqslant\varliminf_{x\rightarrow 1^-}(1-x)\sum_{n=0}^{\infty}S_nx^nf\left(x^n\right)\leqslant\varlimsup_{x\rightarrow 1^-}(1-x)\sum_{n=0}^{\infty}S_nx^nf\left(x^n\right)\leqslant\int_{0}^{1}f^\varepsilon\dd x.\]在令\(\varepsilon\rightarrow 0^+\)即得\eqref{get}式对于\(f=\chi_{[a,b]}\)成立. 同理或者利用已证结果可以证明当\(f\)为\(\chi_{[0,a]},\chi_{[b,1]},\)\\\(\chi_{(a,b)},\chi_{[a,b)}\)这样的区间时, \eqref{get}式也成立, 从而当\(f\)为分段常值函数时, \eqref{get}式成立.
\item 取\(f(x)=\frac{1}{x}\chi_{[1/\ee,1]}(x), x_n=1-\frac{1}{n},\)即\(f(x)=\begin{cases}
1/x,\quad&x\in[1/\ee,1],\\
0,&x\in[0,1/\ee)
\end{cases},\)便有\[\lim_{n\rightarrow+\infty}\frac{1}{n}\sum_{k=0}^{\infty}S_k\left(1-\frac{1}{n}\right)^kf\left(\left(1-\frac{1}{n}\right)^k\right)=\int_{0}^{1}f(x)\dd x=1,\]注意到\(\left(1-\frac{1}{n}\right)^n<\frac{1}{\ee}<\left(1-\frac{1}{n}\right)^{n-1},(n\geqslant 2)\) 我们有\(\lim_{n\rightarrow+\infty}\frac{1}{n}\sum_{k=0}^{n}S_k=1\), 结论得证.\qedhere
\end{asparaenum}
\end{proof}
\woe  利用\(\sum_{k=0}^{n}C_{2k}^kC_{2(n-k)}^{n-k}=4^n\)证明: \(\sum_{k=0}^{n}\frac{C_{2k}^kC_{2n-2k}^{n-k}}{(2k+1)(2n-2k+1)}=\frac{4^{2(n+1)}}{8(n+1)^2C_{2n+2}^{n+1}}\). 进一步, 计算\(\arcsin^2x\)和\(\ln^2(x+\sqrt{1+x^2})\)的Maclaurin级数.
\begin{proof}

\end{proof}
\woe 试计算级数\(\sum_{n=0}^{\infty}\frac{(2n)!}{2^{2n}(2n+1)(n!)^2}.\)
\begin{solution}
记级数一般项为\(u_n\), 则
\[\begin{split}
u_n&=\frac{(2n)!}{2^{2n}(2n+1)(n!)^2}=\frac{(2n-1)!!}{(2n+1)(2n)!!}=\frac{2}{\pi}\cdot\frac{1}{2n+1}\cdot\int_{0}^{\frac{\pi}{2}}\sin^{2n}x\dd x\\&=\frac{2}{\pi}\int_{0}^{\frac{\pi}{2}}\frac{1}{\sin x}\int_{0}^{x}\cos t\cdot\sin^{2n}t\,\dd t\dd x,
\end{split}\]从而\[\begin{split}
\sum_{n=0}^{\infty}u_n&=\frac{2}{\pi}\int_{0}^{\frac{\pi}{2}}\frac{1}{\sin x}\int_{0}^{x}\cos t\sum_{n=0}^{\infty}\sin^{2n}t\,\dd t\dd x=\frac{2}{\pi}\int_{0}^{\frac{\pi}{2}}\frac{1}{\sin x}\int_{0}^{x}\cos t\cdot\frac{1}{1-\sin^2 t}\,\dd t\dd x\\&=\frac{2}{\pi}\int_{0}^{\frac{\pi}{2}}\csc x\int_{0}^{x}\sec t\,\dd t\dd x=\frac{2}{\pi}\int_{0}^{\frac{\pi}{2}}\sec t\dd t\int_{t}^{\frac{\pi}{2}}\csc x\dd x\\&=-\frac{2}{\pi}\int_{0}^{\frac{\pi}{2}}\frac{\ln\left(\tan\frac{t}{2}\right)}{\cos t}\dd t\xlongequal{\tan\frac{t}{2}=x}-\frac{4}{\pi}\int_{0}^{1}\frac{\ln x}{1-x^2}\dd x=\frac{\pi}{2}.\qedhere\end{split}\]
\end{solution}
\woe 设\(|x|<1\), 考虑\textbf{第一类完全椭圆积分}\(\, K(x)=\int_{0}^{\pi/2}\frac{\dd\theta}{\sqrt{1-x^2\cos^2\theta}}\). 依次证明:
\begin{quizs}
\item 对任何\(x\in(-1,1)\), 成立\(K(x)=\frac{\pi}{2}\sum_{n=0}^{\infty}\frac{\left(\CC_{2n}^{n}\right)^2x^{2n}}{4^{2n}}\).
\item 对任何\(x\in(-1,1),\,\theta\in\left[0,2\pi\right)\), 有\(\frac{1}{|1-x\ee^{\mathrm{i}\theta}|}=\sum_{k=0}^{\infty}\frac{\CC_{2k}^kx^k}{4^k}\ee^{\mathrm{i}k\theta}\sum_{j=0}^{\infty}\frac{\CC_{2j}^jx^j}{4^j}\ee^{-\mathrm{i}j\theta}\).\\提示: 可以证明两边平方后的等式相等.
\item 对任何\(x\in\left[0,1\right)\), 有\(\frac{1}{1+x}K\left(\frac{2\sqrt{x}}{1+x}\right)=\frac{\pi}{2}\sum_{n=0}^{\infty}\frac{\left(\CC_{2n}^n\right)^2x^{2n}}{4^{2n}}=K(x)\).
\end{quizs}
\begin{proof}
(1)由\[\frac{1}{\sqrt{1-x^2\cos^2\theta}}=\frac{\CC_{2n}^n}{4^n}x^{2n}\cos^{2n}\theta,\]又由之前结果\[\int_{0}^{\pi/2}\cos^{2n}\theta\dd \theta=\frac{(2n-1)!!}{(2n)!!}\frac{\pi}{2}=\frac{\CC^{n}_{2n}}{4^n}\frac{\pi}{2},\]于是\(K(x)=\frac{\pi}{2}\sum_{n=0}^{\infty}\frac{\left(\CC_{2n}^{n}\right)^2x^{2n}}{4^{2n}}\).

(2)

(3)
\end{proof}
\woe 对于\(a,b>0\), 令\(M(a,b)=\int_{0}^{\frac{\pi}{2}}\frac{\dd\theta}{\sqrt{a^2\cos^2\theta+b^2\sin^2\theta}}\). 另一方面, 令\(a_0=a,b_0=b\), 并归纳定义\(a_{n+1}=\sqrt{a_nb_n},b_{n+1}=\frac{a_n+b_n}{2}(n\geqslant 0)\). 依次证明:\begin{quizs}
\item \(\{a_n\},\{b_n\}\)收敛到同一值, 记为\(AGM(a,b)\), 我们称之为\(a,b\)的\textbf{算术几何平均}.
\item \(AGM\left(\sqrt{ab},\frac{a+b}{2}\right)=AGM(a,b),M\left(\sqrt{ab},\frac{a+b}{2}\right)=M(a,b)\).
\item 成立如下的\textbf{Gauss(高斯)公式}: \(AGM(a,b)=\frac{\pi}{2M(a,b)}\).
\end{quizs}
\begin{proof}
(1)

(2)

(3)
\end{proof}
\woe 设\(b_k\)和\(B_k(k\geqslant 0)\)为Bernoulli数与Bernoulli多项式, \(\zeta\)为Riemann \(\zeta\)函数.\begin{quizs}
\item 利用\(\frac{t\ee^t}{e^t-1}=t+\frac{t}{\ee^t-1},\) 证明: 当\(k\ne 1\)时, \(B_k(1)=b_k\). 另一方面, 有\(b_0=1,b_1=-\frac{1}{2},B_k(1)=\frac{1}{2}\).
\item 利用\(\frac{t\ee^{(1-x)t}}{\ee^t-1}=\frac{-t\ee^{-xt}}{\ee^{-t}-1}\)证明对任何\(k\geqslant 0\), 成立\(B_k(1-x)=(-1)^kB_k(x)\).
\item 利用\(\frac{-t\ee^{xt}}{\ee^{-t}-1}=t\ee^{xt}+\frac{t\ee^{xt}}{\ee^t-1}\) 证明对任何\(k\geqslant 1\), 成立\(B_k(-x)=(-1)^kB_k(x)+(-1)^kkx^{k-1}\).
\item 证明对任何\(k\geqslant 1\), 成立\(b_{2k+1}=0\).
\item 证明对任何\(k\geqslant 0\), \(B_k\left(\frac{1}{2}-x\right)=(-1)^k\left(\frac{1}{2^{k-1}}B_k(2x)-B_k(x)\right)\).
\item 证明对任何\(k\geqslant 0,\,B'_{k+1}(x)=(k+1)B_k(x)\), 进而\(\int_{0}^{x}B_k(s)\dd s=\frac{B_{k+1}(x)-B_{k+1}}{k+1}\).
\item 设\(n\geqslant 1\), 证明Euler公式:\(\zeta(2n)=(-1)^{n-1}\frac{2^{2n-1}b_{2n}}{(2n)!}\pi^{2n}\).
\\提示: 对于\(x\in[0,1]\), 令\(T_0(x)=0\),\[T_n(x)=\frac{(-1)^{n-1}}{2(2n)!}B_{2n}(x)-\frac{1}{(2\pi)^{2n}}H_{n}(2\pi x),\quad\forall n\geqslant 1,\]其中\[H_n(x)=\sum_{k=1}^{\infty}\frac{\cos kx}{k^{2n}},\qquad x\in[0,2\pi],\]特别地,\[H_1(x)=\zeta(2)-\frac{\pi x}{2}+\frac{x^2}{4},\quad x\in[0,2\pi].\]验证\(T''_n(x)=-T_{n-1}(x)\).
\end{quizs}
\begin{proof}

\end{proof}
\woe 利用Bernoulli数与Bernoulli多项式求以下函数的Maclaurin级数:
\[(1)\tan x,\qquad (2)\sec x,\qquad (3)\tanh x,\qquad (4)x\cot x(\text{在}x=0\text{处定义为}1).\qquad\qquad\qquad\qquad\]
\woe 对于\(n\)阶正定矩阵\(\boldsymbol{A}\), 证明存在唯一的正定矩阵\(\boldsymbol{B}\), 使得\(\ee^{\boldsymbol{B}}=\boldsymbol{A}\).
\begin{proof}

\end{proof}
\woe 是否有\(\delta>0\)以及\((0,\delta)\)上的\(n\)阶正定矩阵值函数\(\boldsymbol{Y}(\cdot)\), 在\((0,\delta)\)上满足\(\boldsymbol{Y}(t)<\boldsymbol{I}\), \(\boldsymbol{Y}'(t)=(\boldsymbol{I}-\boldsymbol{Y}^2(t))^{1/2}\)以及\(\lim_{t\rightarrow 0^+}\boldsymbol{Y}(t)=0\)?
\woe 考察各基本初等函数, 思考可以把它们的定义推广到什么样的\(n\)阶方阵?
\end{quizb}
\section{常微分方程初值问题的存在性}
\precis{Picard迭代,等度连续,Arzel\`{a}-Ascoli定理,非Lipschitz条件下解的存在性,积分方程的解,解的延伸}
\begin{quiza}
\woe 设\(\boldsymbol{f}:\mathbb{R}\times \mathbb{R}^n\rightarrow\mathbb{R}^n\)连续可微. 证明:\begin{quizs}
\item 设\(a<b<c\). 若在\([a,b]\)上, \(\boldsymbol{\varphi}'(t)=f\left(t,\boldsymbol{\varphi}(t)\right)\)且\(\boldsymbol{\varphi}(a)=\boldsymbol{x}_0\), 而在\([b,c]\)上, \(\boldsymbol{\psi'}(t)=f\left(t,\boldsymbol{\psi}(t)\right)\)且\(\boldsymbol{\psi}(b)=\boldsymbol{\varphi}(b)\), 若\(\boldsymbol{x}(t)=\begin{cases}
\boldsymbol{\varphi}(t),\quad &t\in[a,b],\\
\boldsymbol{\psi}(t)&t\in[b,c]
\end{cases}\)是方程组\(\begin{cases}
\boldsymbol{x}'(t)=f(t,\boldsymbol{x}(t)),\\
\boldsymbol{x}(a)=\boldsymbol{x}_0,
\end{cases},t\in[a,c]\)的解.
\item 方程组(9.5.2)的解的最大存在区间存在, 且为开区间.
\end{quizs}
\woe 证明方程\(x'(t)=\left(1-x^2(t)\right)\left(\cos t^2+\sin x(t)\right)\)的满足初值条件\(x(0)=0\)的解的最大存在区间是\((-\infty,+\infty)\).
\woe 证明\textbf{Gronwall-Bellman(格朗沃尔-贝尔曼)不等式}:

设\(\alpha\in\mathbb{R}\), 而\(\varphi\in C[t_0,T],\psi,\beta\in L^1[t_0,T], \psi\)非负, 满足\[\varphi(t)\leqslant\alpha+\int_{t_0}^{t}\left(\psi(s)\varphi(s)+\beta(s)\right)\dd s,\quad\forall t\in[t_0,T].\]证明:\[\varphi(t)\leqslant\alpha\exp\left(\int_{t_0}^{t}\psi(s)\dd s\right)+\int_{t_0}^{t}\exp\left(\int_{s}^{t}\psi(r)\dd r\right)\beta(s)\dd s,\quad\forall t\in[t_0,T].\]

提示: 考虑满足以下等式的函数\(\varPhi\):\[\varPhi(t)=\alpha+\int_{t_0}^{t}\left(\psi(s)\varphi(s)+\beta(s)\right)\dd s,\quad\forall t\in[t_0,T].\]
\begin{proof}
考虑满足以下等式的函数\(\varPhi\):\[\varPhi(t)=\alpha+\int_{t_0}^{t}\left(\psi(s)\varphi(s)+\beta(s)\right)\dd s,\quad\forall t\in[t_0,T],\]即\(\varphi(t)\leqslant \varPhi(t)\), 又有\[\varPhi'(t)=\psi(t)\varphi(t)+\beta(t)\leqslant\psi(t)\varPhi(t)+\beta(t)\Rightarrow\left(\varPhi'(t)-\psi(t)\varPhi(t)\right)<\beta(t),\]则有\[\frac{\dd}{\dd t}\left(\varPhi(t)\cdot\exp\left(-\int_{t_0}^{t}\psi(s)\dd s\right)\right)\leqslant \exp\left(-\int_{t_0}^{t}\psi(s)\dd s\right)\beta(t),\]积分得到\[\begin{split}
\int_{t_0}^{t}\frac{\dd}{\dd t}\left(\varPhi(t)\cdot\exp\left(-\int_{t_0}^{t}\psi(s)\dd s\right)\right)\dd t&=\varPhi(t)\cdot\exp\left(-\int_{t_0}^{t}\psi(s)\dd s\right)-\alpha\\&\leqslant\int_{t_0}^{t}\exp\left(-\int_{t_0}^{s}\psi(r)\dd r\right)\beta(s)\dd s,
\end{split}\]于是\[\varPhi(t)\leqslant\alpha\exp\left(\int_{t_0}^{t}\psi(s)\dd s\right)+\int_{t_0}^{t}\exp\left(\int_{s}^{t}\psi(r)\dd r\right)\beta(s)\dd s,\quad\forall t\in[t_0,T].\]由\(\varphi(t)\leqslant \varPhi(t)\)即得结论.
\end{proof}
\woe 设\(\boldsymbol{f}:\mathbb{R}\times\mathbb{R}^n\rightarrow\mathbb{R}^n\)连续可微, 且\(\boldsymbol{f}\left(t,\boldsymbol{x}\right)\)关于\(\boldsymbol{x}\)满足\textbf{线性增长条件}, 即存在常数\(K>0\)使得\[|\boldsymbol{f}(t,\boldsymbol{x})|\leqslant K\left(|x|+1\right),\quad\forall(t,\boldsymbol{x})\in\mathbb{R}\times\mathbb{R}^n.\]证明: 对于任何初值条件, 方程\(\boldsymbol{x}'(t)=\boldsymbol{f}(t,\boldsymbol{x}(t))\)的解的最大存在区间是\((-\infty,+\infty)\).
\begin{proof}

\end{proof}
\woe 设\(\delta>0,f\)是\((-\delta,\delta)\)内的连续可微函数, 满足\(f'(x)=f^4(x)-x^8\)以及\(f(0)=0\), 试证明\(f\in C^{\infty}(-\delta,\delta)\), 并讨论\(f\)取正负的情况. 进一步, 讨论\(f\)的解析性.
\end{quiza}
\begin{quizb}
\woe 构造函数\(f\)使得方程\(\begin{cases}
x'(t)=f\left(t,x(t)\right),\\
x(0)=0
\end{cases}\)的解的最大存在区间为\(-\infty,1\), 且其解\(x\)满足\(\varliminf_{x\rightarrow 1^-}x(t)=-\infty,\varlimsup_{x\rightarrow 1^1}x(t)=+\infty\).
\woe 设\(\{f_k\}\)是\(\bbr^n\)的紧子集\(E\)上的一列一致有界且等度连续的函数列, \(f\in C(E)\). 证明: 若以下条件之一成立, 则\(\{f_k\}\)本身在\(E\)上一致收敛到\(f\).
\begin{asparaenum}[(i)]
\item \(\{f_k\}\)的任何在\(E\)上逐点收敛的子列均收敛到\(f\).
\item 对于\(\{f_k\}\)的满足\[\lim_{j\rightarrow+\infty}\int_E f_{k_j}\left(\boldsymbol{x}\right)\dd\boldsymbol{x}=\int_Eg\left(\boldsymbol{x}\right)h\left(\boldsymbol{x}\right)\dd\boldsymbol{x},\forall h\in C(E)\]的子列\(\{f_{k_j}\}\)和相应的\(g\in C(E)\), 均成立\(g=f.\)
\end{asparaenum}
\woe 设\(T>0,\boldsymbol{f}:[0,T]\times\bbr^n\rightarrow\bbr^n\)连续, 且\(\boldsymbol{f_x}\left(t,\boldsymbol{x}\right)\)也在\([0,T]\times\bbr^n\)上连续且一致有界. 设\(\boldsymbol{x}_0\in\bbr^n\), 记\(\boldsymbol{x}(\cdot;\boldsymbol{x}_0)\)为方程\[\begin{cases}
\boldsymbol{x}'(t)=\boldsymbol{f}\left(t,\boldsymbol{x}(t)\right),\quad t\in[0,T],\\
\boldsymbol{x}(0)=\boldsymbol{x}_0
\end{cases}\]的解. 任取\(\boldsymbol{e}\in S^{n-1}\), 证明: 当\(\varepsilon\rightarrow 0^+\)时, \(\boldsymbol{X}_{\varepsilon}(\cdot;\boldsymbol{e})=\frac{\boldsymbol{x}(\cdot;\boldsymbol{x}_0+\varepsilon\boldsymbol{e})-\boldsymbol{x}(\cdot;\boldsymbol{x}_0)}{\varepsilon}\)在\([0,T]\)上一致收敛到方程\[\begin{cases}
\boldsymbol{X}'(t)=\boldsymbol{f_x}\left(t,\boldsymbol{x}\left(t;\boldsymbol{x}_0\right)\boldsymbol{X}(t)\right),\quad t\in[0,T],\\
\boldsymbol{X}(0)=\boldsymbol{e}
\end{cases}\]的解\(\boldsymbol{X}(\cdot)\).
\woe 设\(T>0,\boldsymbol{x}_0\in\bbr^n,\boldsymbol{f}:[0,T]\times\bbr^n\times\bbr^m\rightarrow\bbr^n\)连续, 且\(\boldsymbol{f_x}\left(t,\boldsymbol{x},\boldsymbol{u}\right),\boldsymbol{f_u}\left(t,\boldsymbol{x},\boldsymbol{u}\right)\)也在\([0,T]\times\bbr^n\times\bbr^m\)上连续且一致有界. 设\(\boldsymbol{\varphi}\in C\left([0,T];\bbr^n\right)\), 记\(\boldsymbol{x}\left(\cdot;\boldsymbol{\varphi}\right)\)为方程\[\begin{cases}
\boldsymbol{x}'(t)=f\left(t,\boldsymbol{x}(t),\boldsymbol{\varphi}(t)\right),\quad t\in[0,T],\\
\boldsymbol{x}(0)=\boldsymbol{x}_0
\end{cases}\]的解. 任取\(\boldsymbol{\psi}\in C\left([0,T];\bbr^m\right)\), 证明: 当\(\varepsilon\rightarrow0^+\)时, \(\boldsymbol{Y}_{\varepsilon}(\cdot)=\frac{\boldsymbol{x}(\cdot;\boldsymbol{\varphi}+\varepsilon\boldsymbol{\psi})-\boldsymbol{x}(\cdot;\boldsymbol{\varphi})}{\varepsilon}\)在\([0,T]\)上一致收敛到方程\[\begin{cases}
\boldsymbol{Y}'(t)=\boldsymbol{f_x}\left(t,\boldsymbol{x}(t;\boldsymbol{\varphi}),\boldsymbol{\varphi}(t)\right)\boldsymbol{Y}(t)+\boldsymbol{f_u}\left(t,\boldsymbol{x}(t;\boldsymbol{\varphi}),\boldsymbol{\varphi}(t)\right)\boldsymbol{\psi}(t),\quad t\in[0,T],\\
\boldsymbol{Y}(0)=\boldsymbol{0}
\end{cases}\]的解\(\boldsymbol{Y}(\cdot)\).
\woe 试讨论方程\(f'(x)=f^4(x)-x^8\)在任何初值条件下解的最大存在区间为整个\(\bbr\).
\end{quizb}