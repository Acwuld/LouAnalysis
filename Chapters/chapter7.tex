\chapter{微分问题}
\section{隐函数存在定理}
\precis{隐函数存在定理,曲面的切平面,法向量}
\begin{quiza}
\woe 证明方程\(\ee^z+\sin z+\ee^y+y-\sin x+x=2\)在点\((0,0,0)\)附近确定一个隐函数\(z=z(x,y)\), 并计算\(\frac{\partial z}{\partial x},\frac{\partial^2 z}{\partial x\partial y}\).
\begin{solution}
记\(F(x,y,z)=\ee^z+\sin z+\ee^y+y-\sin x+x-2\), 由隐函数定理可得\[\frac{\partial z}{\partial x}=-\frac{F_x}{F_z}=\frac{\cos x-1}{\ee^z+\cos z},\quad\frac{\partial z}{\partial y}=-\frac{F_y}{F_z}=-\frac{\ee^y+1}{\ee^z+\cos z}\]进一步得到\[\frac{\partial^2z}{\partial x\partial y}=\frac{\partial}{\partial y}\left(\frac{\partial z}{\partial x}\right)=\frac{(1-\cos x)(\ee^z+\cos z)}{(\ee^z+\cos z)^2}\frac{\partial z}{\partial y}=\frac{(\cos x-1)(\ee^z+\cos z)(1+\ee^y)}{(\ee^z+\cos z)^3}.\qedhere\]
\end{solution}
\woe 证明方程组\[\begin{cases}
    x+y+u+v=0,\\ x^2+y^2+u^4+v^4=2
\end{cases}\]在\((-1,0,1,0)\)附近确定一个隐函数\(\left(\begin{matrix}
u\\ v
\end{matrix}\right)=\left(\begin{matrix}
u(x,y)\\v(x,y)
\end{matrix}\right)\). 进一步, 计算\(\frac{\partial u}{\partial x},\frac{\partial^2u}{\partial x^2}.\)
\begin{proof}

\end{proof}
\woe 证明方程组\[\begin{cases}
    x+y+u+v=0,\\x^2+y^2+u^4+v^4=2
\end{cases}\]在\((-1,0,1,0)\)附近确定一个隐函数\(\left(\begin{matrix}
y\\ u
\end{matrix}\right)=\left(\begin{matrix}
y(x,u)\\u(x,v)
\end{matrix}\right)\). 进一步, 计算\(\frac{\partial u}{\partial x},\frac{\partial^2u}{\partial x^2}.\)
\begin{proof}

\end{proof}
\end{quiza}
\begin{quizb}
\woe 设\(n\geqslant 2\), \(\varOmega\)是\(\mathbb{R}^n\)中的区域, \(F:\varOmega\rightarrow\mathbb{R}\)连续可微, 且\(F_{\boldsymbol{x}}(\boldsymbol{x})\ne\boldsymbol{0}(\forall\boldsymbol{x}\in\varOmega)\). 又设\(\boldsymbol{x}_0\in\varOmega\)使得\(F(\boldsymbol{x}_0)=0\). 证明存在\(\delta>0\), 使得对于任何满足\(F(\boldsymbol{x}_1)=0\)的\(\boldsymbol{x}_1\in B_\delta(\boldsymbol{x}_0)\), 均存在曲面\(F(\boldsymbol{x})=0\)上连接\(\boldsymbol{x}_0,\boldsymbol{x}_1\)的\(C^1\)曲线\(\boldsymbol{\tau}\). 即存在\(\boldsymbol{\tau}\in C^1\left([0,1];\mathbb{R}^n\right)\)满足\(F\left(\boldsymbol{\tau}(t)\right)=0(\forall t\in[0,1])\)以及\(\boldsymbol{\tau}(0)=\boldsymbol{x}_0,\boldsymbol{\tau}(1)=\boldsymbol{x}_1\).
\begin{proof}
	
\end{proof}
\end{quizb}
\section{极值问题}
\precis{强制条件,极值问题,无条件极值,一阶必要条件,二阶必要条件,驻点,二阶必要条件,最小二乘法,线性拟合,条件极值,Lagrange乘子法,矩阵的诱导范数}
\begin{quiza}
\woe 设\(n\geqslant 2\), \(\boldsymbol{\xi,\eta}\in\mathbb{R}^n\). 试计算\(\max_{x\in S^{n-1}}\{\inp{\boldsymbol{\xi,x}},\inp{\boldsymbol{\eta,x}}\}\).
\begin{solution}

\end{solution}
\woe  设\(1\leqslant p,q\leqslant+\infty\), 对\(\boldsymbol{A}\in\mathbb{R}^{m\times n}\), 定义\(\left\|\boldsymbol{A}\right\|_{p,q}=\max_{\left\|x\right\|_q=1}\left\|\boldsymbol{Ax}\right\|_p\). 证明: \(\left\|\cdot\right\|_{p,q}\)定义了一个范数.
\begin{proof}

\end{proof}
\woe 设\(\boldsymbol{A}\)为\(n\)阶正定矩阵, 证明\(\inp{\boldsymbol{x,y}}_{\boldsymbol{A}}=\boldsymbol{y}^{\mathrm{T}}\boldsymbol{Ax}\)定义了\(\mathbb{R}^n\)中的一个内积.
\begin{proof}
由于\[\inp{\boldsymbol{x,y}}_{\boldsymbol{A}}=\boldsymbol{y}^{\top}\boldsymbol{Ax}=\left(\boldsymbol{y}^{\top}\boldsymbol{Ax}\right)^\top=\boldsymbol{x}^{\top}\boldsymbol{Ay}=\inp{\boldsymbol{y,x}}_{\boldsymbol{A}},\]也有\(\inp{\boldsymbol{x+y,z}}_{\boldsymbol{A}}=\inp{\boldsymbol{x,z}}_{\boldsymbol{A}}+\inp{\boldsymbol{y,z}}_{\boldsymbol{A}}\), \(\inp{c\boldsymbol{x,y}}_{\boldsymbol{A}}=c\inp{\boldsymbol{x,y}}_{\boldsymbol{A}}\), 其中\(c\)为一常数. 结合\(\boldsymbol{A}\)正定可知对于任意\(\boldsymbol{x}\ne\boldsymbol{0}\), 有\(\inp{\boldsymbol{x,x}}_{\boldsymbol{A}}=\boldsymbol{x}^\top\boldsymbol{Ax}>0\), 从而上述的运算定义了一个内积.
\end{proof}
\woe 设\(n\geqslant 2\), \(\boldsymbol{A}\)为\(n\)阶实对称矩阵, \(\boldsymbol{\xi}_1\in S^{n-1}\)满足\(\lambda_1\equiv \boldsymbol{\xi}_1^{\mathrm{T}}\boldsymbol{A}\boldsymbol{\xi}_1=\max_{x\in S^{n-1}}\boldsymbol{x}^{\mathrm{T}}\boldsymbol{A}\boldsymbol{x}.\)考虑在\(\boldsymbol{x}\in S^{n-1}\)以及\(\boldsymbol{x}\cdot\boldsymbol{\xi}_1=0\)的约束条件下最大化\(\boldsymbol{x}^{\mathrm{T}}\boldsymbol{A}\boldsymbol{x}\)的最大值问题, 证明存在\(\lambda_2\leqslant\lambda_1\)以及\(\boldsymbol{\xi}_2\in S^{n-1}\), 使得\(\boldsymbol{A\xi}_2=\lambda_2\boldsymbol{\xi}_2,\boldsymbol{\xi}_2\cdot\boldsymbol{\xi}_1=0\). 一般地, 证明存在\(\lambda_1\geqslant \lambda_2\geqslant\cdots\geqslant\lambda_n\)以及两两正交的\(\boldsymbol{\xi}_1,\boldsymbol{\xi}_2,\cdots,\boldsymbol{\xi}_n\in S^{n-1}\)使得\(\boldsymbol{A}\boldsymbol{\xi}_k=\lambda_k\boldsymbol{\xi}_k(1\leqslant k\leqslant n)\).
\begin{proof}
由于\(\boldsymbol{A}\)实对称, 故存在正交矩阵\(\boldsymbol{P}\)使得\[\boldsymbol{P}^\top\boldsymbol{AP}=\left(\begin{matrix}
\lambda_1'&&&\\
&\lambda_2'&&\\
&&\ddots&\\
&&&\lambda_n'
\end{matrix}\right),\]其中\(\lambda_i'(1\leqslant i\leqslant n)\)是\(\boldsymbol{A}\)的特征值.
\end{proof}
\woe 设\(\boldsymbol{A}\)为\(m\times n\)实矩阵, 证明\(\left\|\boldsymbol{A}\right\|=\max_{\substack{|\boldsymbol{z}|=1\\ \boldsymbol{z}\in\mathbb{C}^{n}}}|\boldsymbol{Az}|\).
\begin{proof}

\end{proof}
\woe 设\(\boldsymbol{A}\)为\(n\)阶实方阵, \(\lambda\)是它的(复)特征值. 证明: \(|\lambda|\leqslant\left\|\boldsymbol{A}\right\|\).
\begin{proof}

\end{proof}
\woe 设\(\boldsymbol{x}\in\mathbb{R}^n\), 证明\(|\boldsymbol{x}|=\max_{\substack{\left\|\boldsymbol{A}\right\|\\\boldsymbol{A}\in\mathbb{R}^{n\times n}}}|\boldsymbol{Ax}|.\)
\begin{proof}

\end{proof}
\woe 是推导两空间直线\[\frac{x-x_i}{l_i}=\frac{y-y_i}{m_i}=\frac{z-z_i}{n_i},\quad i=1,2\]间的距离公式, 其中\[(l_1m_2-l_2m_1)^2+(m_1n_2-m_2n_1)^2+(n_1l_2-n_2l_1)^2>0.\]
%\begin{solution}
%记题设直线分别为\(L_1,L_2\), 取\(\boldsymbol{x_1}\in L_1\), \(\boldsymbol{x_2}\in L_2\), 则\(L_1\)于\(L_2\)之间的距离\(d = \min\left\|\boldsymbol{x}_1-\boldsymbol{x}_2\right\|\), 于是不妨设\[\boldsymbol{x}_1=\left(l_1t_1+x_1,m_1t_1+y_1,n_1t_1+z_1\right),\quad \boldsymbol{x}_2=\left(l_2t_2+x_2,m_2t_2+y_2,n_2t_2+z_2\right),\]记\(D(t_1,t_2)=\left(l_1t_1+x_1-l_2t_2-x_2\right)^2+\left(m_1t_1+y_1-m_2t_2-y_2\right)^2+\left(n_1t_1+z_1-n_2t_2-z_2\right)^2\), 原问题即为求\(D(t_1,t_2)\)的最小值, 分别对\(t_1,t_2\)求偏导有\[\begin{split}
%\frac{\partial D}{\partial t_1}&=2A_1t_1-2Bt_2+2l_1(x_1-x_2)+2m_1(y_1-y_2)+2n_1(z_1-z_2),\\
%\frac{\partial D}{\partial t_2}&=2A_2t_2-2Bt_1-2l_2(x_1-x_2)-2m_2(y_1-y_2)-2n_2(z_1-z_2),
%\end{split}\]
%其中\(A_i=l_i^2+m_i^2+n_i^2(i=1,2),B=l_1l_2+m_1m_2+n_1n_2\). 令\(\frac{\partial D}{\partial t_1}=\frac{\partial D}{\partial t_2}=0\), 并且注意到\[A_1A_2-B_2^2=(l_1m_2-l_2m_1)^2+(m_1n_2-m_2n_1)^2+(n_1l_2-n_2l_1)^2>0,\]
%由Cramer法则可得\[t'_1=\frac{BC_2-A_2C_1}{A_1A_2-B^2},\quad t'_2=\frac{A_1C_2-BC_1}{A_1A_2-B^2}\]其中\(C_i=l_i(x_1-x_2)+m_i(y_1-y_2)+n_i(z_1-z_2)(i=1,2)\). 将\(t'_1,t'_2\)代入\(D(t_1,t_2)\)得到
%\end{solution}
\begin{solution}
记题设直线分别为\(L_1,L_2\), 取\(\boldsymbol{x_1}\in L_1\), \(\boldsymbol{x_2}\in L_2\), 则\(L_1\)于\(L_2\)之间的距离\(d = \min\left|\boldsymbol{x}_1-\boldsymbol{x}_2\right|\), 不妨设\[\boldsymbol{a}_i=\left(x_i,y_i,z_i\right)^\top,\quad \boldsymbol{d}_i=\left(l_i,m_i,n_i\right)^\top,\quad i=1,2,\]于是\(\boldsymbol{x}_1=\boldsymbol{a}_1+t_1\boldsymbol{d}_1,\boldsymbol{x}_2=\boldsymbol{a}_2+t_2\boldsymbol{d}_2\), 从而设\[\begin{split}
D(t_1,t_2)&=\left|\boldsymbol{x}_1-\boldsymbol{x}_2\right|^2=\left|\boldsymbol{a}_1-\boldsymbol{a}_2+t_1\boldsymbol{d}_1-t_2\boldsymbol{d}_2\right|^2=\inp{\boldsymbol{a}_1-\boldsymbol{a}_2+t_1\boldsymbol{d}_1-t_2\boldsymbol{d}_2,\boldsymbol{a}_1-\boldsymbol{a}_2+t_1\boldsymbol{d}_1-t_2\boldsymbol{d}_2}\\&=\left|\boldsymbol{d}_1\right|^2t_1^2-2\inp{\boldsymbol{d}_1,\boldsymbol{d}_2}t_1t_2+\left|\boldsymbol{d}_2\right|^2t_2^2+2\inp{\boldsymbol{a}_1-\boldsymbol{a}_2,\boldsymbol{d}_1}t_1-2\inp{\boldsymbol{a}_1-\boldsymbol{a}_2,\boldsymbol{d}_2}t_2+\left|\boldsymbol{a}_1-\boldsymbol{a}_2\right|^2,
\end{split}\]
分别记\(A_i=\left|\boldsymbol{d}_i\right|^2,B=\inp{\boldsymbol{d}_1,\boldsymbol{d}_2},C_i=\inp{\boldsymbol{a}_1-\boldsymbol{a}_2,\boldsymbol{d}_i}(i=1,2)\), 这样就有\[D(t_1,t_2)=A_1t_1^2-2Bt_1t_2+A_2t_2^2+2C_1t_1-2C_2t_2+\left|\boldsymbol{a}_1-\boldsymbol{a}_2\right|^2,\]
分别对\(t_1,t_2\)求偏导有\[\frac{\partial D}{\partial t_1}=2A_1t_1-2Bt_2+2C_1,\quad
\frac{\partial D}{\partial t_2}=2A_2t_2-2Bt_1-2C_2,\]
令\(\frac{\partial D}{\partial t_1}=\frac{\partial D}{\partial t_2}=0\), 并且注意到\[A_1A_2-B_2^2=(l_1m_2-l_2m_1)^2+(m_1n_2-m_2n_1)^2+(n_1l_2-n_2l_1)^2>0,\]
由Cramer法则可得\[t'_1=\frac{BC_2-A_2C_1}{A_1A_2-B^2},\quad t'_2=\frac{A_1C_2-BC_1}{A_1A_2-B^2},\]
代入\(D(t1,t2)\)开方即得结果.
\end{solution}
\woe 计算两空间直线\(\frac{x-1}{1}=\frac{y-3}{2}=\frac{z-5}{3},\frac{x-7}{4}=\frac{y-9}{5}=\frac{z-11}{6}\)的公垂线.
\begin{solution}

\end{solution}
\woe 试推导点\((x_0,y_0,z_0)\)与平面\(Ax+By+Cz+D=0(A^2+B^2+C^2\ne 0)\)的距离公式.
\begin{solution}
我们先用条件极值的方法. 设\[f(\alpha,\beta,\gamma)=(\alpha-x_0)^2+(\beta-y_0)^2+(\gamma-z_0)^2,\]即求\(f\)在约束条件\(A\alpha+B\beta+C\gamma+D=0\)下的极值. 设\[F(\alpha,\beta,\gamma,\lambda)=(\alpha-x_0)^2+(\beta-y_0)^2+(\gamma-z_0)^2+\lambda\left(A\alpha+B\beta+C\gamma+D\right),\]则令\[\begin{cases}
F_{\alpha}=2\left(\alpha-x_0\right)+A\lambda=0,\\
F_{\beta}=2\left(\beta-y_0\right)+B\lambda=0,\\
F_{\gamma}=2\left(\gamma-z_0\right)+C\lambda=0,\\
F_{\lambda}=A\alpha+B\beta+C\gamma+D=0,
\end{cases}\]
可以解得\[\lambda=\frac{2(Ax_0+By_0+Cz_0+D)}{A^2+B^2+C^2},\alpha=x_0-\frac{A\lambda}{2},\beta=y_0-\frac{B\lambda}{2},\gamma=z_0-\frac{C\lambda}{2},\]此时得到\[f_{\min}=\frac{\left(Ax_0+By_0+Cz_0+D\right)^2}{A^2+B^2+C^2},\]即得点\((x_0.y_0,z_0)\)到平面\(Ax+By+Cz+D=0\)的距离为\(\sqrt{f_{\min}}=\frac{\left|Ax_0+By_0+Cz_0+D\right|}{\sqrt{A^2+B^2+C^2}}\).
\tcbline
令一种简单的方法是考虑几何关系, 记\((x_0,y_0,z_0)\)为\(P_1\), 取平面上一点\(P_2(a,b,c)\), 则考虑向量\(\overrightarrow{P_1P_2}=\left(a-x_0.b-y_0,c-z_0\right)\)与平面法向量\((A,B,C)\)夹角的余弦值(记较小的那个角为\(\theta\))可得点到平面距离\[d=\left|\overrightarrow{P_1P_2}\right|\cos\theta=\left|\overrightarrow{P_1P_2}\right|\frac{\left|\overrightarrow{P_1P_2}\cdot\left(A,B,C\right)\right|}{\left|\overrightarrow{P_1P_2}\right|\sqrt{A^2+B^2+C^2}}=\frac{\left|Ax_0+By_0+Cz_0+D\right|}{\sqrt{A^2+B^2+C^2}}.\qedhere\]
\end{solution}
\woe 设\(f(x,y)=xy-x\ln x+x-\ee^y(x>0,y\in\mathbb{R})\). 证明: \(f(x,y)\leqslant 0(\forall x>0,y\in\mathbb{R})\).
\begin{proof}

\end{proof}
\end{quiza}
\begin{quizb}
\woe 试构造\(\bbr^2\)上的二元实函数\(f\)使得点\(\boldsymbol{0}\)不是它的极小值点, 但对任何\(\alpha\in[0,2\pi],g(t)=f(t\cos\alpha,t\sin\alpha)\)在\(t=0\)取得严格极小值. 进一步, 能否取到这样的一个\(f\)使得它在\(\bbr^2\)上连续? 请证明你的结论.
\begin{solution}

\end{solution}
\woe 举例说明存在\(n\)阶方阵, 使得其所有特征值的绝对值都严格小于\(\left\|\boldsymbol{A}\right\|\).
\begin{solution}

\end{solution}
\woe 若\(\boldsymbol{A}\)是\(n\)阶方阵, \(\left\|\boldsymbol{A}\right\|<1\). 证明: \(\left(\boldsymbol{I}-\boldsymbol{A}\right)^{-1}=\sum_{k=0}^{\infty}\boldsymbol{A}^k.\)
\begin{proof}

\end{proof}
\woe 若\(\boldsymbol{A}\)是\(n\)阶可逆矩阵, 取\(\alpha>0\)足够小使得\(\boldsymbol{I}-\alpha\boldsymbol{AA}^\mathrm{T}\)正定. 令\[\boldsymbol{B}_0=\alpha\boldsymbol{A}^\mathrm{T},\boldsymbol{B}_{k+1}=\boldsymbol{B}_k\left(2\boldsymbol{I}-\boldsymbol{AB}_k\right)(k\geqslant 0).\]证明: \(\{\boldsymbol{AB}_k\}\)是单调增加的正定矩阵, 其极限为\(\boldsymbol{I}\). 特别地, \(\lim_{k\rightarrow+\infty}\boldsymbol{B}_k=\boldsymbol{A}^{-1}\).
\begin{proof}

\end{proof}
\woe 试对于\(\boldsymbol{A}\)为复矩阵的情形, 定义诱导范数并建立相关性质.
\woe 试改编以往遇到过的一个习题, 将一维情形的结果推广到矩阵情形.
\end{quizb}
\section{常系数线性微分方程}
\precis{一阶常系数线性微分方程,矩阵指数函数,高阶常系数线性微分方程,特征方程,算子法}
\begin{quiza}
\woe 求以下方程的通解:\vspace{8pt}\\
\begin{tabular}{lcl}
\((1)\,y''(x)-3y'(x)+2y(x)=0\);&\qquad\qquad&\((2)\,y''(x)-3y'(x)+2y(x)=x\);\vspace{0.3cm}\\
\((3)\,y''(x)-3y'(x)+2y(x)=x\ee^{3x}\);&&\((4)\,y''(x)-3y'(x)+2y(x)=x\ee^{2x}\);\vspace{0.3cm}\\
\((5)\,y''(x)-3y'(x)+2y(x)=x^2\ee^{x}\cos x\);&&\((6)\,y''(x)-3y'(x)+2y(x)=x^2\ee^{x}\sin 2x\).\vspace{0.3cm}\\
\end{tabular}
\begin{solution}
由特征方程\(\lambda^2-3\lambda+2=0\)得\(\lambda_1=1,\lambda_2=2\), 从而齐次方程\((1)\)的通解为\[y=C_1\ee^x+C_2\ee^{2x},\]根据线性方程得通解结构, 为得到后续方程得通解, 只需要得到这些方程得特解. 使用算子法, 即得各个特解如下:

\noindent (2) \(y^*=\frac{1}{(\DD-1)(\DD-2)}x=\left(\frac{1}{\DD-2}-\frac{1}{\DD-1}\right)x=\left(\frac{1}{2}+\frac{3}{4}\DD\right)x=\frac{1}{2}x+\frac{3}{4}.\)\vspace{0.5em}

\noindent (3)\(y^*=\frac{1}{(\DD-1)(\DD-2)}x\ee^{3x}=\ee^{3x}\frac{1}{(\DD+2)(\DD+1)}x=\ee^{3x}\left(\frac{1}{2}-\frac{3}{4}\DD\right)x=\ee^{3x}\left(\frac{1}{2}x-\frac{3}{4}\right).\)\vspace{0.5em}

\noindent (4)\(y^*=\frac{1}{(\DD-1)(\DD-2)}x\ee^{2x}=\ee^{2x}\frac{1}{\DD(\DD+1)}x=\ee^{2x}\frac{1}{\DD}\left(1-\DD\right)x=\ee^{2x}\left(\frac{1}{2}x^2-x\right).\)\vspace{0.5em}
\end{solution}
\woe 求以下方程的通解:\vspace{8pt}\\
\begin{tabular}{lcl}
\((1)\,y''(x)-4y'(x)+4y(x)=0\);&\qquad\qquad&\((2)\,y''(x)-4y'(x)+4y(x)=x\);\vspace{0.3cm}\\
\((3)\,y''(x)-4y'(x)+4y(x)=x\ee^{3x}\);&&\((4)\,y''(x)-4y'(x)+4y(x)=x\ee^{2x}\);\vspace{0.3cm}\\
\((5)\,y''(x)-4y'(x)+4y(x)=x^2\ee^{2x}\cos x\);&&\((6)\,y''(x)-4y'(x)+4y(x)=x^2\ee^{x}\sin x\).\vspace{0.3cm}\\
\end{tabular}
\begin{solution}
由特征方程\(\lambda^2-4\lambda+4=0\)得\(\lambda_1=\lambda_2=2\), 从而齐次方程\((1)\)的通解为\[y=C_1\ee^{2x}+C_2x\ee^{2x},\]根据线性方程得通解结构, 为得到后续方程得通解, 只需要得到这些方程得特解. 使用算子法, 即得各个特解如下:

\noindent (2)\(y^*=\left(\frac{1}{\DD-2}\right)^2x=\frac{1}{4}(1+\DD)x=\frac{1}{4}x+\frac{1}{4}.\)\vspace{0.5em}

\noindent (3)\(y^*=\left(\frac{1}{\DD-2}\right)^2x\ee^{3x}=\ee^{3x}\left(\frac{1}{\DD+1}\right)^2x=\ee^{3x}(1-2\DD)x=\left(x-2\right)\ee^{3x}.\)\vspace{0.5em}

\noindent (4)\(y^*=\left(\frac{1}{\DD-2}\right)^2x\ee^{2x}=\ee^{2x}\frac{1}{\DD^2}x=\frac{1}{6}x^3\ee^{2x}.\)\vspace{0.5em}
\end{solution}

\woe 求以下方程的通解:\vspace{8pt}\\
\begin{tabular}{lcl}
\((1)\,y''(x)+y'(x)+2y(x)=0\);&\qquad\qquad&\((2)\,y''(x)+y'(x)+2y(x)=x^3+1\);\vspace{0.3cm}\\
\((3)\,y''(x)+y'(x)+2y(x)=x\ee^{3x}\);&&\((4)\,y''(x)+y'(x)+2y(x)=x\sin 2x\);\vspace{0.3cm}\\
\((5)\,y''(x)+y'(x)+2y(x)=x^2\ee^{2x}\cos x\);&&\((6)\,y''(x)+y'(x)+2y(x)=x^2\ee^{x}\sin 2x\).\vspace{0.3cm}\\
\end{tabular}
\begin{solution}
由特征方程\(\lambda^2+\lambda+2=0\)得\(\lambda_1=\)
\end{solution}


\woe 求以下方程的通解:\vspace{8pt}\\
\begin{tabular}{lcl}
\((1)\,y^{(5)}(x)+2y'''(x)+y'(x)=0\);&\qquad\qquad&\((2)\,y^{(5)}(x)+2y'''(x)+y'(x)=(x+1)^2\);\vspace{0.3cm}\\
\((3)\,y^{(5)}(x)+2y'''(x)+y'(x)=x\sin 2x\);&&\((4)\,y^{(5)}(x)+2y'''(x)+y'(x)=x\sin x\);\vspace{0.3cm}\\
\((5)\,y^{(5)}(x)+2y'''(x)+y'(x)=x^2\ee^{2x}\cos x\);&&\((6)\,y^{(5)}(x)+2y'''(x)+y'(x)=x^2\ee^x\sin x\).\vspace{0.3cm}\\
\end{tabular}
\begin{solution}
由特征方程\(\lambda^5+2\lambda^3+\lambda=0\)得\(\lambda_1=0,\lambda_2=\ii,\lambda_3=-\ii\), \(\lambda_2,\lambda_3\)均为二重根. 于是齐次方程(1)的通解为\[y=C_1+C_2\cos x+C_3x\cos x+C_4\sin x+C_5x\sin x,\]
\end{solution}
\woe 求方程\(y''(x)-y(x)=1\)满足\(y(0)=1,y'(0)=2\)的特解.
\woe 试研究当\(b\)为何值时, 对所有\(y_0,y_1\in\mathbb{R}\), 方程\(y''(x)+y(x)=0\)满足\(y(0)=y_0,y(b)=y_1\)的特解总是存在.
\woe 若\(n\)阶方阵\(\boldsymbol{A},\boldsymbol{B}\)可交换, 证明: \(\ee^{\boldsymbol{A}+\boldsymbol{B}}=\ee^{\boldsymbol{A}}\ee^{\boldsymbol{B}}\).
\woe 设\(n\geqslant 2,\lambda\in\mathbb{C},n\)阶方阵\(\boldsymbol{A}\)为\[\begin{pmatrix}
\lambda&1&0&\cdots&0&0\\
0&\lambda&1&\cdots&0&0\\
\vdots&\cdots&\vdots&&\vdots&\vdots\\
0&0&0&\cdots&\lambda&1\\
0&0&0&\cdots&0&\lambda\\
\end{pmatrix}.\]试计算\(\ee^{x\boldsymbol{A}}\).
\end{quiza}
\begin{quizb}
\woe 设\(P_1,P_2\)是次数依次为\(m,n\geqslant 1\)的多项式, \(Q_1,Q_2\)是次数小于\(m,n\)的非零多项式, 满足\(\frac{1}{P(t)}=\frac{Q_1(t)}{P_1(t)}+\frac{Q_2(t)}{P_2(t)}\), 其中\(P(t)=P_1(t)P_2(t)\). 若\(f\)为\(I\)上的连续函数, \(f_1,f_2\)依次为区间\(I\)上的\(m,n\)次连续可微函数, 满足\(P_1(\mathrm{D})f_1(x)=P_2(\mathrm{D})f_2(x)=f(x)\). 令\(F(x)=P_1(\mathrm{D})f_1(x)+P_2(\mathrm{D})f_2(x)\). 证明: \(F\)是\(m+n\)阶连续可导函数, 进而\(P(\mathrm{D})F(x)=f(x)\).
\woe 举例说明上题中, \(P_1(\mathrm{D})f_1\)可以不是\(m+n\)阶连续可导的.
\end{quizb}
\section{导数的其他应用}
\precis{Newton切线法,平方收敛,平面曲线的曲率和曲率半径,一元实函数的草图,拐点}
\begin{quiza}
\woe 设\(a>0\), 用Newton法通过求解\(f(x)=\ee^x-a\)的零点计算\(\ln a\). 若取初值为\(1\), 试讨论Newton迭代法的收敛性, 并讨论误差估计.
\begin{solution}

\end{solution}
\woe 对于\(n\geqslant 2\)以及\(a>0\), 讨论通过Newton法计算\(f(x)=x^{10}-a\)的零点的可行性与误差估计.
\begin{solution}

\end{solution}
\woe 计算椭圆\(\frac{x^2}{4}+\frac{y^2}{9}=1.\)上各点的曲率.
\begin{solution}
我们先推到由参数方程确定的平面曲线的曲率计算公式. 对于\(y_0=f(x_0)\), 当\(f''(x_0)\ne 0\)时曲率为\(K=\frac{\left|f''(x_0)\right|}{\left(1+\left|f''(x_0)\right|^2\right)^{3/2}}\).先设平面曲线有如下的参数方程\(\begin{cases}
x=\varphi(t),\\
y=\phi(t)
\end{cases}\),则有\[y'(x)=\frac{\dd y}{\dd x}=\frac{\dd y/\dd t}{\dd x/\dd t}=\frac{\phi'(t)}{\varphi'(t)},\]进一步有\[y''(x)=\frac{\dd}{\dd x}\left(\frac{\dd y}{\dd x}\right)=\frac{\dd}{\dd t}\left(\frac{\dd y}{\dd x}\right)\cdot\frac{1}{\dd x/\dd t}=\frac{\phi''(t)\varphi'(t)-\phi'(x)\varphi''(x)}{\left(\varphi'(x)\right)^3},\]将其代入\(K\)得到\[K=\frac{\left|\phi''(t)\varphi'(t)-\phi'(x)\varphi''(x)\right|}{\left(\left(\varphi'(t)\right)^2+\left(\phi'(t)\right)^2\right)^{3/2}}.\]一般地, 椭圆\(\frac{x^2}{a^2}+\frac{y^2}{b^2}=1\)的参数方程可设为\(\left(a\cos\theta,b\sin\theta\right)\), 代入上述公式即得\[K=\sqrt{\frac{a^2 b^2}{\left(a^2 \sin ^2(t)+b^2 \cos ^2(t)\right)^3}}.\qedhere\]
\end{solution}
\woe 设\(y=\sqrt[3]{\frac{x^4}{x+1}}\), 试讨论函数的单调性, 极值, 凸性, 拐点, 求出它的渐近线, 并画出它的简图.
\begin{solution}
首先在\(x=-1\)处\(y(x)\)无定义, 易见其为铅直渐近线. 直接计算得到\[y'(x)=\frac{x^3(3x+4)}{3(1+x)^2}\cdot\left(\frac{x^4}{x+1}\right)^{-2/3},\quad x\ne -1,\]于是\(y(x)\)有驻点\(-\frac{4}{3},0\). 关于函数的单调性与极值等信息如下表:
\begin{center}
\begin{tabular}{c|c|c|c|c|c|c|c}
\Xhline{1.2pt}
\phantom{\Large\(\int\)}&\(\left(-\infty,-\frac{4}{3}\right)\)&\(-\frac{3}{4}\)&\(\left(-\frac{4}{3},-1\right)\)&\(-1\)&\(\left(-1,0\right)\)&\(0\)&\(\left(0,+\infty\right)\)\\\hline
\(f'\)&+&0&\(-\)&不存在&\(-\)&0&+\\\hline
\(f\)&严格单增&极大值点&严格单减&渐近线&严格单减&极小值点&严格单增\\\Xhline{1.2pt}
\end{tabular}
\end{center}
又有\[y''(x)=\frac{4x^2}{9(1+x)^3}\cdot\left(\frac{x^4}{x+1}\right)^{-2/3},\quad x\ne -1,\]可见\(y(x)\)在\((-\infty,-1)\)上凹而在\((-1,+\infty)\)上凸. 并且\[\lim_{x\rightarrow\infty}\frac{y(x)}{x}=1,\lim_{x\rightarrow\infty}\left(y(x)-x\right)=\frac{1}{3},\]即\(y\)有渐近线\(f(x)=x-\frac{1}{3}\)其简图如下:
\end{solution}
\woe 设\(y=\ln\frac{x^2+x+1}{x^2-x+1}\), 试讨论函数的单调性, 极值, 凸性, 拐点, 求出它的渐近线, 并画出它的简图.
\begin{solution}
先设\(g(x)=\frac{x^2+x+1}{x^2-x+1}\), 有\(g(0)=1>0\), 当\(x\ne 0\)时, \[g(x)=\frac{x^2+x+1}{x^2-x+1}=1+\frac{2x}{x^2-x+1}=1\frac{2}{x+1/x-1},\]\(x>0\)时\(x+\frac{1}{x}>2\), 从而\(g(x)>0\); \(x<0\)时\(x+\frac{1}{x}\leqslant -2,\frac{2}{x+1/x-1}<-1\), 也有\(g(x)>0\). 从而\(y(x)\)的定义域为\(\bbr\). 直接计算得到\[y'(x)=\frac{-2x^2+2}{x^4+x^2+1},\]于是\(y(x)\)有驻点\(\pm1\). 关于函数的单调性与极值等信息如下表:
\begin{center}
\begin{tabular}{c|c|c|c|c|c}
\Xhline{1.2pt}
&\(\left(-\infty,-1\right)\)&\(-1\)&\(\left(-1,1\right)\)&\(1\)&\(\left(1,+\infty\right)\)\\\hline
\(f'\)&\(-\)&0&\(+\)&0&\(-\)\\\hline
\(f\)&严格单减&极小值点&严格单增&极大值点&严格单减\\\Xhline{1.2pt}
\end{tabular}
\end{center}
\noindent
又有\[y''(x)=\frac{4x(x^4-2x^2-2)}{(x^4+x^2+1)^2},\]令\(y''(x)=0\), 得到三个实根分别为\(0,\pm\sqrt{1+\sqrt{3}}\). 关于函数凹凸性有下表:
\begin{center}
\begin{tabular}{c|c|c|c|c}
\Xhline{1.2pt}
\phantom{\Large\(\int\)}&\(\left(-\infty,-\sqrt{1+\sqrt{3}}\right)\)&\(\left(-\sqrt{1+\sqrt{3}},0\right)\)&\(\left(0,\sqrt{1+\sqrt{3}}\right)\)&\(\left(\sqrt{1+\sqrt{3}},+\infty\right)\)\\\hline
\(f''\)&\(-\)&\(+\)&\(-\)&\(+\)\\\hline
\(f\)&严格凹&严格凸&严格凹&严格凸\\\Xhline{1.2pt}
\end{tabular}
\end{center}
\noindent
 又有\[\lim_{x\rightarrow\infty}\ln\frac{x^2+x+1}{x^2-x+1}=0,\]从而\(y(x)\)有水平渐近线\(y=0\), 其简图如下.\[?\]
\end{solution}
\end{quiza}
\begin{quizb}
\woe 试给出由参数方程确定的平面曲线曲率的计算公式.
\begin{solution}
参见本节\(\boldsymbol{\mathcal{A}}\)中的第\textbf{3}小题.
\end{solution}
\woe 设\(\varphi\)为\((0,+\infty)\)内的凹函数, 证明:\[\lim_{x\rightarrow+\infty}\left(\varphi(2x)-2\varphi(x)\right)=-\infty\]当且仅当\(y=\varphi(x)\)当\(x\rightarrow+\infty\)时没有渐近线.
\end{quizb}