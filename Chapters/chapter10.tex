\chapter{反常积分与含参变量积分}
\section{基于Riemann积分的反常积分}
\precis{瑕积分,无穷积分,反常重积分,非负函数反常积分的收敛性,Cauchy准则,绝对收敛与条件收敛,Abel判别法,Dirichlet判别法,概率积分,Cauchy主值积分,数项级数收敛的Cauchy积分判别法}
\begin{quiza}
\woe 设\(f\)在\(\left(0,1\right]\)上非负单调, 证明: \(\lim_{n\rightarrow+\infty}\frac{1}{n}\sum_{k=1}^{n}f\left(\frac{k}{n}\right)=\int_{0}^{1}f(x)\dd x\), 这里右端的积分可能是\(+\infty\).
\begin{proof}
不妨设\(f\)单调递减, 则有\[\int_{k/n}^{(k+1)/n}f(x)\dd x\leqslant \frac{1}{n}f\left(\frac{k}{n}\right)\leqslant\int_{(k-1)/n}^{k/n}f(x)\dd x,\quad k=1,2,\cdots,n-1,\]于是有\[\int_{1/n}^{1}f(x)\dd x\leqslant\frac{1}{n}\sum_{k=1}^{n-1}f\left(\frac{k}{n}\right)\leqslant\int_{0}^{1-1/n}f(x)\dd x,\]令\(n\rightarrow+\infty\)即得结论.
\end{proof}
\woe 讨论以下积分的敛散性\footnote[1]{注: 在讨论含参数的问题时, 一开始不妨对参数的分类分得细一点, 最后再作适当的合并.}:\vspace{8pt}\\
\begin{tabular}{lcl}
\((1)\,\int_{0}^{+\infty}\frac{\ln x}{x^a(x-1)^{2b}}\dd x\);&\qquad\qquad\qquad&\((2)\,\int_{0}^{+\infty}\sin(x^p+\frac{1}{x^q})\dd x,\quad p,q>0\);\vspace{0.3cm}\\
\((3)\,\int_{0}^{+\infty}\frac{\sin \pi x}{x^\alpha|x-1|^\beta}\dd x\);&&\((4)\,\int_{0}^{+\infty}\frac{\dd x}{x^p\sqrt[3]{\sin^2x}}\);\vspace{0.3cm}\\
\((5)\,\int_{0}^{+\infty}\left(\ln x\right)\sin (x^p)\dd x\).&&\\
\end{tabular}
\begin{solution}
(1)当\(a>0\)时, \(x=0\)是瑕点, \(b>0\)时, \(x=1\)可能是瑕点. 为此不妨设\(0<\delta<1\), 有\[\int_{0}^{+\infty}\frac{\ln x}{x^a(x-1)^{2b}}\dd x=\int_{0}^{\delta}\frac{\ln x}{x^a(x-1)^{2b}}\dd x+\int_{\delta}^{1}\frac{\ln x}{x^a(x-1)^{2b}}+\int_{1}^{+\infty}\frac{\ln x}{x^a(x-1)^{2b}}=:I_1+I_2+I_3,\]先考察\(I_1\)得敛散性, 易见\(I_1\)的敛散性于\(\int_{0}^{\delta}\frac{\ln x}{x^a}\dd x\)相同, 由于\[\int_{0}^{\delta}\frac{\ln x}{x^a}\dd x=-\int_{1/\delta}^{+\infty}\frac{\ln t}{t^{2-a}}\dd t,\]易见后者在\(a<1\)时收敛. 从而\(a<1\)时\(I_1\)收敛.

对于\(I_2\), 其敛散性与\(\int_{\delta}^{1}\frac{\ln x}{(x-1)^{2b}}\)同, 注意到当\(b\leqslant\frac{1}{2}\)时有\[\frac{\ln x}{(x-1)^{2b}}\rightarrow 0\left(b<\frac{1}{2}\right),1\left(b=\frac{1}{2}\right),\quad x\rightarrow 1,\]此时\(x=1\)不是瑕点, 故\(I_2\)收敛. 否则\(b>\frac{1}{2}\)时, \(\frac{\ln x}{(x-1)^{2b}}\sim\frac{1}{(x-1)^{2b-1}}(x\rightarrow 1)\), 于是\(b<1\)时\(I_2\)收敛.

对于\(I_3\), 有\(\frac{\ln x}{x^a(x-1)^{2b}}\sim\frac{\ln x}{x^{a+2b}}(x\rightarrow+\infty)\), 从而\(a+2b>1\)时\(I_3\)收敛.

综上, 当\(a<1,b<1\)且\(a+2b>1\)时原积分收敛.

(2)

(3)

(4)易见\(x=k\pi(k\in\bbz_+)\)都是瑕点, 由于\(\sin x\sim\)

(5)
\end{solution}
\woe 说明以下积分收敛并求其值:\vspace{8pt}\\
\begin{tabular}{lcl}
\((1)\,\int_{0}^{+\infty}\frac{1}{(1+x^2)(1+x^\alpha)}\dd x\);&\qquad\qquad\qquad&\((2)\,\int_{0}^{\pi}x\ln\sin x\dd x\);\vspace{0.3cm}\\
\((3)\,\int_{0}^{\pi}\ln\left(1-2r\cos x+r^2\right)\dd x\);&&\((4)\,\int_{0}^{+\infty}\ee^{-\left(x^2+1/x^2\right)}\dd x\);\\
\end{tabular}
\begin{solution}
(1)易见无论\(\alpha\)为何值, 都有\(\frac{1}{(1+x^2)(1+x^\alpha)}\leqslant\frac{1}{1+x^2}\), 而\(\int_{0}^{+\infty}\frac{1}{1+x^2}\dd x\)收敛, 从而原积分收敛. 置\(x=\frac{1}{t}\),得到\[\int_{0}^{+\infty}\frac{1}{(1+x^2)(1+x^\alpha)}\dd x=\int_{0}^{+\infty}\frac{t^\alpha}{(1+t^2)(1+t^\alpha)}\dd t,\]于是\(\int_{0}^{+\infty}\frac{1}{(1+x^2)(1+x^\alpha)}=\frac{1}{2}\int_{0}^{+\infty}\frac{1+x^\alpha}{(1+x^2)(1+x^\alpha)}=\frac{\pi}{4}\).

(2)注意到\(\lim_{x\rightarrow 0^+}x\ln\sin x=0\), 故\(0\)不是瑕点, 而\[\lim_{x\rightarrow \pi^-}\sqrt{\pi-x}\cdot x\ln\sin x=0,\]可知\(\int_{0}^{\pi}x\ln\sin x\dd x\)收敛. 置\(x=\pi-t\), 有\[I:=\int_{0}^{\pi}x\ln\sin x\dd x=\int_{0}^{\pi}(\pi-t)\ln\sin t\dd t=\pi\int_{0}^{\pi}\ln\sin x\dd x-I,\]从而\(\int_{0}^{\pi}x\ln\sin x\dd x=\frac{\pi}{2}\int_{0}^{\pi}\ln\sin x\dd x=-\frac{\pi^2}{2}\ln 2.\)

(3)先考虑\(\left|r\right|<1\)的情形. 把\(r\)看成参变量, 记\(I(r)=\int_{0}^{\pi}\ln\left(1-2r\cos x+r^2\right)\dd x\)并对\(r\)求导, 得到\[I'(r)=\int_{0}^{\pi}\frac{-2\cos x+2r}{1-2r\cos x+r^2}\dd x,\]由此可得\(I'(0)=-2\int_{0}^{\pi}\cos x\dd x=0\). 现设\(r\ne 0\), 则\[I'(r)=\frac{1}{r}\int_{0}^{\pi}\left(1-\frac{1-r^2}{1-2r\cos x+r^2}\right)\dd x=\frac{1}{r}\left(\pi-(1-r^2)\int_{0}^{\pi}\frac{\dd x}{1-2r\cos x+r^2}\right),\]做变换\(t=\tan\left(x/2\right)\), 可得\[(1-r^2)\int_{0}^{\pi}\frac{\dd x}{1-2r\cos x+r^2}=\frac{1+r}{1-r}\int_{0}^{+\infty}\frac{2\dd t}{1+\displaystyle\left(\frac{1+r}{1-r}\right)^2}=\left.2\arctan\frac{1+r}{1-r}t\right|^{+\infty}_0=\pi.\]这里用到了\(\left|r\right|<1\)的条件, 从而\(I'(r)\equiv 0\), 易见\(I(0)=0\), 从而\(I(r)\equiv 0\).

在考虑\(\left|r\right|>1\)的情形. 令\(\rho=1/r\), 则\(\left|\rho\right|<1\). 于是\[I(r)=I\left(\frac{1}{\rho}\right)=\int_{0}^{\pi}\ln\left(1-2\rho\cos x+\rho^2\right)\dd x-\int_{0}^{\pi}\ln \rho^2\dd x=-2\pi\ln\left|\rho\right|=2\pi\ln\left|r\right|.\]

当\(r=\pm 1\)时, 利用上题结果可知\[\int_{0}^{\pi/2}\ln\sin x\dd x=\int_{0}^{\pi/2}\ln\cos x\dd x=-\frac{\pi}{2}\ln 2.\]可以得到\(I(\pm 1)=0\). 即得\(I(r)=\begin{cases}
0,&|r|\leqslant 1,\\
2\pi\ln|r|,\quad&|r|>1.
\end{cases}\)

(4)易见\(\int_{0}^{+\infty}\ee^{-(x^2+1/x^2)}\dd x\)收敛, 置\(x=\frac{1}{t}\)得到\[\int_{0}^{+\infty}\ee^{-(x^2+1/x^2)}\dd x=\int_{0}^{+\infty}\frac{1}{t^2}\ee^{-(t^2+1/t^2)}\dd t,\]于是有\[\begin{split}
\frac{1}{2}\int_{0}^{+\infty}\ee^{-(x^2+1/x^2)}&=\frac{1}{2}\int_{0}^{+\infty}\ee^{-\left((x-\frac{1}{x})^2+2\right)}\left(1+\frac{1}{x^2}\right)\dd x=\frac{1}{2}\int_{0}^{+\infty}\ee^{-\left((x-\frac{1}{x})^2+2\right)}\dd\left(x-\frac{1}{x}\right)\\&=\frac{1}{2}\int_{-\infty}^{+\infty}\ee^{-t^2-2}\dd t=\frac{\sqrt{\pi}}{2\ee^2}.\qedhere
\end{split}\]
\end{solution}
\woe 利用\(\sum_{n=1}^{\infty}\frac{\cos nx}{n}=-\ln\left(2\sin\frac{x}{2}\right),x\in\left(0,2\pi\right)\)说明\(\int_{0}^{\pi}\ln\sin x\dd x=-\pi\ln 2.\)
\woe 设\(f\)在\(\left[0,+\infty\right)\)非负, 严格单调下降, 证明: \(\int_{0}^{x}f(t)\sin t\dd t>0\). 特别地, 对于任何\(x>0\)成立\(\int_{0}^{x}\ee^{-t}\sin t\dd t>0\).
\begin{proof}
	
\end{proof}
\woe 若\(f(0^+)\)存在, \(\forall A>0,\int_{A}^{+\infty}\frac{f(x)}{x}\dd x\)收敛, 试计算Frullani积分\(\int_{0}^{+\infty}\frac{f(ax)-f(bx)}{x}\dd x\), 其中\(a,b>0\).
\begin{solution}
我们先回顾该积分的计算方法.
\tcbline
设\(f(0^+)\)与\(f(+\infty)\)都存在, 且对任何\(0<\delta<\varDelta<+\infty,\,\int_{\delta}^{\varDelta}\frac{f(x)}{x}\dd x\)存在. 则\[\begin{split}
&\int_{\delta}^{\varDelta}\frac{f(ax)-f(bx)}{x}\dd x=\int_{\delta}^{\varDelta}\frac{f(ax)}{x}\dd x-\int_{\delta}^{\varDelta}\frac{f(bx)}{x}\dd x\\
&=\int_{a\delta}^{a\varDelta}\frac{f(t)}{t}\dd t-\int_{b\delta}^{b\varDelta}\frac{f(t)}{t}\dd t=\int_{a\delta}^{b\delta}\frac{f(t)}{t}\dd t-\int_{a\varDelta}^{b\varDelta}\frac{f(t)}{t}\dd t
\end{split}\]我们记\(J_1=\int_{a\delta}^{b\delta}\frac{f(t)}{t}\dd t,\,J_2=\int_{a\varDelta}^{b\varDelta}\frac{f(t)}{t}\dd t.\) 于是由积分第一中值定理\[\begin{split}
J_1&=\int_{a\delta}^{b\delta}\frac{f(t)}{t}\mathrm{d}t=f(\xi_1)\int_{a\delta}^{b\delta}\frac{1}{t}\mathrm{d}t=f(\xi_1)\ln\frac{b}{a},\qquad a\delta\leqslant\xi_1\leqslant b\delta,\\
J_2&=\int_{a\varDelta}^{b\varDelta}\frac{f(t)}{t}\mathrm{d}t=f(\xi_2)\int_{a\varDelta}^{b\varDelta}\frac{1}{t}\mathrm{d}t=f(\xi_1)\ln\frac{b}{a},\qquad a\varDelta\leqslant\xi_2\leqslant b\varDelta.
\end{split}\]这时当\(\delta\rightarrow 0^+,\,\varDelta\rightarrow+\infty\)时, 得到\(\xi_1\rightarrow 0^+,\xi_2\rightarrow+\infty\), 即得到\[\int_0^{+\infty}\frac{f(ax)-f(bx)}{x}\mathrm{d}x=\left(f(0^+)-f(+\infty)\right)\ln\frac{b}{a}.\]
\tcbline
由题设\(f(0^+)\)存在, \(\forall A>0,\int_{A}^{+\infty}\frac{f(x)}{x}\dd x\)收敛, 则\[J_2=\int_{a\varDelta}^{b\varDelta}\frac{f(t)}{t}\mathrm{d}t=\int_{A}^{b\varDelta}\frac{f(t)}{t}\mathrm{d}t-\int_{A}^{a\varDelta}\frac{f(t)}{t}\mathrm{d}t\rightarrow 0,\quad\varDelta\rightarrow+\infty,\]从而\(\int_0^{+\infty}\frac{f(ax)-f(bx)}{x}\mathrm{d}x=f(0^+)\ln\frac{b}{a}.\)

同理若\(f(+\infty)\)存在, \(\forall\varepsilon>0,\int_{0}^{\varepsilon}\frac{f(x)}{x}\dd x\)收敛, 则\[J_1=\int_{a\delta}^{b\delta}\frac{f(t)}{t}\mathrm{d}t=\int_{a\delta}^{\varepsilon}\frac{f(t)}{t}\mathrm{d}t-\int_{b\delta}^{\varepsilon}\frac{f(t)}{t}\mathrm{d}t\rightarrow0,\quad\delta\rightarrow 0^+.\]从而\(\int_0^{+\infty}\frac{f(ax)-f(bx)}{x}\mathrm{d}x=f(+\infty)\ln\frac{a}{b}.\)
\end{solution}
\woe 若\(f(+\infty)\)存在, \(\forall\varepsilon>0,\int_{0}^{\varepsilon}\frac{f(x)}{x}\dd x\)收敛, 试计算Frullani积分\(\int_{0}^{+\infty}\frac{f(ax)-f(bx)}{x}\dd x\), 其中\(a,b>0\).
\begin{solution}
参见上一题的解答.
\end{solution}
\woe 设\(f\in C\left[0,+\infty\right),f>0\). 若\(\int_{0}^{+\infty}\frac{1}{f(x)}\dd x\)收敛. 证明: \(\lim_{A\rightarrow+\infty}\frac{1}{A^2}\int_{0}^{A}f(x)\dd x=+\infty\).
\begin{proof}
首先, 对于任意\(A>0\), 总存在\(a>0\)使得\(\frac{1}{f(x)}\)在区间\((a,a+1)\)上满足\(\frac{1}{f(x)}<\frac{1}{A^3}\), 否则\[\int_{a}^{a+N}f(x)\dd x=\sum_{n=a}^{a+N-1}\int_{n}^{n+1}\frac{1}{f(x)}\dd x\geqslant \frac{N}{A^3}\rightarrow+\infty,\quad N\rightarrow+\infty,\]与积分\(\int_{0}^{+\infty}\frac{1}{f(x)}\dd x\)收敛矛盾. 从而对任意\(A\), 存在区间\((a,a+1)\)使得\(f(x)>A^3\), 进而\[\frac{1}{A^2}\int_{0}^{A}f(x)\dd x\geqslant\frac{1}{A^2}\int_{a}^{a+1}f(x)\dd x> A\rightarrow+\infty,\quad A\rightarrow+\infty.\qedhere\]
\end{proof}
\woe 证明: \(\lim_{n\rightarrow+\infty}\int_{0}^{+\infty}\frac{\cos\left(nx+1/x\right)}{1+x^2}\dd x=0\).
\begin{proof}
由于\(\cos\left(nx+1/x\right)=\cos(nx)\cos\left(1/x\right)-\sin(nx)\sin(1/x)\), 记\[I_n=\int_{0}^{+\infty}\frac{\cos(1/x)}{1+x^2}\cdot\cos(nx)\dd x,\quad J_n=\int_{0}^{+\infty}\frac{\sin(1/x)}{1+x^2}\cdot\sin(nx)\dd x,\]易见对于任何\(n\in\bbn^+\), \(I_n,J_n\)都收敛. 我们只需要证明\(\lim_{n\rightarrow+\infty}\left(I_n-J_n\right)=0,\) 由分部积分公式\[\begin{split}
\int_{0}^{+\infty}\frac{\cos(1/x)}{1+x^2}\cdot\cos(nx)\dd x&=\left.\frac{\sin nx}{n}\frac{\cos(1/x)}{1+x}\right|_0^{+\infty}-\frac{1}{n}\int_{0}^{+\infty}\left(\frac{\cos(1/x)}{1+x^2}\right)'\sin nx\dd x\\&=-\frac{1}{n}\int_{0}^{+\infty}\frac{\sin(1/x)\cdot\frac{1+x^2}{x^2}-2x\cdot\cos(1/x)}{(1+x^2)^2}\cdot\sin nx\dd x
\end{split}\]
\end{proof}
\woe 推广习题8.3\(\boldsymbol{\mathcal{A}}\)第6题如下: 设\(f\in C[a.b]\)非负, 且在\(c\in[a,b]\)取得唯一的最大值\(f(c)>0\), \(\varphi\)在\([a,b]\)上绝对可积, 并在点\(c\)连续. 证明: \(\lim_{n\rightarrow+\infty}\frac{\displaystyle\int_{a}^{b}f^n(x)\varphi(x)\dd x}{\displaystyle\int_{a}^{b}f^n(x)\dd x}=\varphi(c)\).
\end{quiza}
\begin{quizb}
\woe 尝试给出比习题\(\boldsymbol{\mathcal{A}}\)第1题的题设更弱的条件, 使得结论仍然成立.
\woe 试讨论更一般的反常重积分的定义. 特别的, 考虑在定义反常重积分时, 瑕点集可以一般到什么程度.
\woe 求\(\sum_{n=1}^{\infty}x^{n^2}\)当\(x\rightarrow 1^-\)的阶.
\begin{solution}
易见\(\sum_{n=1}^{\infty}x^{n^2}\sim\sum_{n=0}^{\infty}x^{n^2}\), 由单调性, 当\(0<x<1\)时, 我们有\[x^{n^2}<\int_{(n-1)}^{n}x^{t^2}\dd t<x^{(n-1)^2},\quad (n\geqslant 1),\]因此\[\int_{0}^{+\infty}x^{t^2}\dd t\leqslant\sum_{n=0}^{\infty}x^{n^2}\leqslant 1+\int_{0}^{+\infty}x^{t^2}\dd t,\]而\[\int_{0}^{+\infty}x^{t^2}\dd t=\int_{0}^{+\infty}\exp\left(-t^2\ln\frac{1}{x}\right)\dd t=\frac{1}{\sqrt{\ln\frac{1}{x}}}\int_{0}^{+\infty}\ee^{-t^2}\dd t=\frac{1}{2}\sqrt{\frac{\pi}{\ln\frac{1}{x}}}.\]当\(x\rightarrow 1^-\)时, \(\ln x\sim x-1\), 因此\(\sum_{n=0}^{\infty}x^{n^2}\sim\frac{1}{2}\sqrt{\frac{\pi}{1-x}}\).
\end{solution}
\woe \vspace*{-\baselineskip*28/20}
\begin{quizs}
\item 举例说明, 存在恒正的\(f\)使得\(\int_{0}^{+\infty}f(x)\dd x\)收敛, 但\(\lim_{x\rightarrow+\infty}f(x)\)不存在.
\item 举例说明, 存在恒正的无穷次可导函数\(f\)使得\(\int_{0}^{+\infty}f(x)\dd x\)收敛, 但\(\varlimsup_{x\rightarrow+\infty}f(x)\)不存在.
\item 若\(f\)一致连续, 且\(\int_{0}^{+\infty}f(x)\dd x\)收敛, 证明:\(\lim_{x\rightarrow+\infty}f(x)=0.\)
\end{quizs}
\woe 证明: \(\lim_{n\rightarrow+\infty}\int_{0}^{\pi/2}\sin x^n\dd x=0\).
\begin{proof}
由于\[\int_{0}^{\pi/2}\sin x^n\dd x=\int_{0}^{1}\sin x^n\dd x+\int_{1}^{\pi/2}\sin x^n\dd x,\]而\begin{gather*}
0\leqslant\int_{0}^{1}\sin x^n\dd x\leqslant\int_{0}^{1}x^n\dd x=\frac{1}{n+1},\\
-\frac{2}{n}\leqslant\int_{1}^{\pi/2}\frac{1}{nx^{n-1}}\cdot nx^{n-1}\sin x^n\dd x\xlongequal{\text{积分第二中值定理}}\frac{1}{n}\int_{0}^{\xi}\dd\left(-\cos x^n\right)\leqslant\frac{2}{n}
\end{gather*}
依夹逼准则有\(\lim_{n\rightarrow+\infty}\int_{0}^{\pi/2}\sin x^n\dd x=0\).
\end{proof}
\woe 证明: \(\lim_{n\rightarrow+\infty}\int_{0}^{\pi/2}\cos x^n\dd x=1\). 
\begin{proof}
仿上法, 我们有\begin{gather*}
1\geqslant\int_{0}^{1}\cos x^n\dd x\geqslant\int_{0}^{1}\left(1-\frac{1}{2}x^{2n}\right)=1-\frac{1}{2(2n+1)}.\\
-\frac{2}{n}\leqslant\int_{1}^{\pi/2}\frac{1}{nx^{n-1}}\cdot nx^{n-1}\cos x^n\dd x=\frac{1}{n}\int_{0}^{\xi}\dd\left(\sin x^n\right)\leqslant\frac{2}{n}
\end{gather*}
依夹逼准则便得结论.
\end{proof}
\woe 设函数\(f:(0,+\infty)\rightarrow(0,+\infty)\)连续且严格单调, 证明: “\(g\)在\([0,1]\)上非负可积蕴含\(f\circ g\)在\([0,1]\)上非负可积”\(\,\)等价于\(\varlimsup_{x\rightarrow+\infty}\frac{f(x)}{x}<+\infty.\)
\woe 证明: \(f(x)=\int_{0}^{+\infty}\ee^{-xt^2}\sin t\dd t\)在\((0,+\infty)\)上有界.
\begin{proof}
记\(F(x,\alpha)=\int_{0}^{+\infty}\ee^{-xt^2}\sin\alpha t\dd t\), 原问题即证\(F(x,1)\)在\(x\in (0,+\infty)\)上有界. 由于\[\frac{\partial F}{\partial \alpha}=\int_{0}^{+\infty}t\ee^{-xt^2}\cos\alpha t\dd t,\]对任意的\(\alpha\in\bbr\)成立, 事实上, 由于\[\left|t\ee^{-xt^2}\cos\alpha t\right|\leqslant t\ee^{-xt^2},\]而\(\int_{0}^{+\infty}t\ee^{-xt^2}\dd t\)收敛, 故由Weierstrass判别法可知\(\int_{0}^{+\infty}t\ee^{-xt^2}\cos\alpha t\dd t\)对\(b\in\bbr\)一致收敛, 因而等式成立, 又有\[\frac{\partial F}{\partial \alpha}=\int_{0}^{+\infty}t\ee^{-xt^2}\cos\alpha t\dd t=\frac{1}{2x}-\frac{\alpha}{2x}\int_{0}^{+\infty}\ee^{xt^2}\sin\alpha t\dd t=\frac{1}{2x}-\frac{\alpha}{2x}F(x,\alpha),\]并且有\(F(x,0)=0\), 得到\[F(x,\alpha)=\exp\left(-\frac{\alpha^2}{4x}\right)\int_{0}^{\alpha}\frac{1}{2x}\exp\left(\frac{\alpha^2}{4x}\right)\dd\alpha,\]于是\(F(x,1)=\exp\left(-\frac{1}{4x}\right)\int_{0}^{1}\frac{1}{2x}\exp\left(\frac{\alpha^2}{4x}\right)\dd x\), 而\[\begin{split}
&\frac{1}{2x}\int_{0}^{1}\exp\left(\frac{\alpha^2}{4x}\right)\dd x=\frac{1}{\sqrt{x}}\int_{0}^{1/(2\sqrt{x})}\exp\left(u^2\right)\dd x\\\leqslant&\frac{1}{\sqrt{x}}\left(\int_{0}^{2}\exp\left(u^2\right)\dd u+\int_{1}^{1/(2\sqrt{x})}\left(2-\frac{1}{u^2}\right)\exp\left(u^2\right)\dd u\right)\\=&\frac{1}{\sqrt{x}}\left(\int_{0}^{1}\exp\left(u^2\right)\dd u+\int_{1}^{1/(2\sqrt{x})}\frac{\dd}{\dd u}\frac{\exp(u^2)}{u}\dd u\right)\\\leqslant&\frac{1}{\sqrt{x}}\left(C+2\sqrt{x}\exp\left(\frac{1}{4x}\right)\right),
\end{split}\]其中\(C=\int_{0}^{1}\ee^{u^2}\dd u\). 于是\(F(x,1)\leqslant\frac{C}{\sqrt{x}}\exp\left(-\frac{1}{4x}\right)+2\), 右侧当\(x\rightarrow 0^+\)时有有限极限,并且结合\(\boldsymbol{8.4 \mathcal{A}}\)第\(\boldsymbol{4}\)题可知\(f(x)>0\). 结论即证.
\end{proof}
\woe 设\(\int_{0}^{+\infty}f(x)\dd x\)收敛. 求\(\lim_{x\rightarrow+\infty}\frac{\displaystyle\int_{0}^{x}tf(t)\dd t}{x}\).
\begin{proof}
记\(I=\int_{0}^{+\infty}f(x)\dd x,F(x)=\int_{0}^{x}f(t)\dd t\), 下证\[\lim_{x\rightarrow+\infty}\frac{1}{x}\int_{0}^{x}F(t)\dd t=I.\]由于\(\int_{0}^{+\infty}f(x)\dd x\)收敛, 可以找到\(x_0>0\), 使得对\(\forall \varepsilon>0,x\geqslant x_0\)有\(\left|F(x)-I\right|<\frac{\varepsilon}{2}\). 于是当\(x>x_0\), 有\[\begin{split}
\frac{1}{x}\int_{0}^{x}F(t)\dd t-I&=\frac{1}{x}\int_{0}^{x}\left(F(t)-I\right)\dd t\\&=\frac{1}{x}\int_{0}^{x_0}\left(F(t)-I\right)\dd t+\left(1-\frac{x_0}{x}\right)\cdot\frac{1}{x-x_0}\int_{x_0}^{x}\left(F(t)-I\right)\dd t,
\end{split}\]因此\[\left|\frac{1}{x}\int_{0}^{x}F(t)\dd t-I\right|<\frac{1}{x}\left|\int_{0}^{x_0}\left(F(t)-I\right)\dd t\right|+\frac{1}{x-x_0}\int_{x_0}^{x}\left|F(t)-I\right|\dd t.\]右端第二项\(<\frac{\varepsilon}{2}\); 当\(x\)充分大时, 第一项同样也小于\(\frac{\varepsilon}{2}\), 于是有\[\left|\frac{1}{x}\int_{0}^{x}F(t)\dd t-I\right|<\varepsilon.\]另一方面\[\int_{0}^{x}tf(t)\dd t=\int_{0}^{x}tF'(t)\dd t=xF(x)-\int_{0}^{x}F(t)\dd t\]于是\[\lim_{x\rightarrow+\infty}\frac{\displaystyle\int_{0}^{x}tf(t)\dd t}{x}=\lim_{x\rightarrow+\infty}\frac{\displaystyle xF(x)-\int_{0}^{x}F(t)\dd t}{x}=0.\qedhere\]
\end{proof}
\woe 设\(f\)是\(\left[0,+\infty\right)\)上非负可导函数. \(f(0)=0,f'(x)\leqslant\frac{1}{2},\int_{0}^{+\infty}f(x)\dd x\)收敛. 求证: 对任何\(p>1,\int_{0}^{+\infty}f^p(x)\dd x\)也收敛, 且\(\int_{0}^{+\infty}f^p(x)\dd x\leqslant\left(\int_{0}^{+\infty}f(x)\dd x\right)^{(p+1)/2}\).
\begin{proof}
令\(g(t)=\left(\int_{0}^{t}f(x)\dd x\right)^{(p+1)/2}-\int_{0}^{t}f^p(x)\dd x\), 则\(g(t)\)可导,\[g'(t)=f(t)\left(\frac{p+1}{2}\left(\int_{0}^{t}f(x)\dd x\right)^{(p-1)/2}-f^{p-1}(t)\right).\]令\(h(t)=\left(\frac{p+1}{2}\right)^{2/(p-1)}\int_{0}^{t}f(x)\dd x-f^2(t)\), 则有\(h'(t)=f(t)\left(\left(\frac{p+1}{2}\right)^{2/(p-1)}-2f'(t)\right)\).

由于\(\beta>1,f'(x)\leqslant\frac{1}{2}\), 我们有\(h'(t)\geqslant 0\). 这说明\(h(t)\)单调递增, 从\(h(0)=0\)得\(h(t)\geqslant 0\), 因此\(g'(t)\geqslant 0\), 从\(g(0)=0\)得\(g(t)\geqslant 0,\) 即\[\int_{0}^{t}f^p(x)\dd x\leqslant\left(\int_{0}^{t}f(x)\dd x\right)^{(p+1)/2},\]令\(t\rightarrow+\infty\)即得结论.
\end{proof}
\woe 设\(\alpha\in(0,1),\beta\in\left(0,1\right],\sum_{n=1}^{\infty}\frac{\sin n^\alpha}{n^\beta}\)的收敛性如何?
\begin{solution}

\end{solution}
\woe 设\(\alpha>1,\beta\in\left(0,1\right]\), 对于\(\sum_{n=1}^{\infty}\frac{\sin n^\alpha}{n^\beta}\)的收敛性, 你能够说些什么?
\woe 设\(\alpha>0,\beta>0\), 对于\(\sum_{n=1}^{\infty}\frac{1}{n^\beta\sin n^\alpha}\)的的收敛性, 你能够说些什么?
\woe 设\(f\in C^1[0,1],f(0)=f(1)=0\). 证明: \(\left(\int_{0}^{1}f(x)\dd x\right)^2\leqslant\frac{1}{12}\int_{0}^{1}|f'(x)|^2\dd x\). 进一步, 如何把条件放宽到\(f\in C_0^1(0,1)\)? 其中\(C_0^1(0,1)=\left\lbrace f\in C[0,1]\cap C^1(0,1)\big|f(0)=f(1)=0\right\rbrace \).
\begin{proof}
由分部积分可知\[\int_{0}^{1}f(x)\dd x=xf(x)\big|_0^1-\int_{0}^{1}xf'(x)\dd x=-\int_{0}^{1}xf'(x)\dd x,\]同理\(\int_{0}^{1}f(x)\dd x=-\int_{0}^{1}(x-1)f'(x)\dd x.\)从而\[\begin{split}
\left(\int_{0}^{1}f(x)\dd x\right)^2&=\left(\frac{1}{2}\int_{0}^{1}(2x-1)f'(x)\right)^2\\
&\leqslant\frac{1}{4}\int_{0}^{1}(2x-1)^2\dd x\int_{0}^{1}\left|f'(x)\right|^2\dd x\leqslant \frac{1}{12}\int_{0}^{1}\left|f'(x)\right|^2\dd x.\qedhere
\end{split}\]
\end{proof}
\end{quizb}
\section{含参变量反常积分的一致收敛性及判别法}
\precis{含参变量反常积分的一致收敛性,Cauchy准则,Weierstrass判别法,Abel判别法,Dirichlet判别法}
\begin{quiza}
\woe 证明以下含参变量积分关于所考虑的参数内闭一致收敛, 但非一致收敛:\vspace{8pt}\\
\begin{tabular}{lcl}
\((1)\int_{0}^{+\infty}x^{\alpha-1}\ee^{-x}\dd x,\,\alpha>0\);&&\((2)\int_{0}^{+\infty}x^{\alpha-1}\ee^{-x}\ln^3x\dd x,\,\alpha>0\);\vspace{0.3cm}\\
\((3)\int_{0}^{1}x^{p-1}(1-x)^{q-1}\dd x,\, p,q>0\);&&\((4)\int_{0}^{1}x^{p-1}(1-x)^{q-1}\left(\ln x\right)\ln^2(1-x)\dd x,\,p,q>0\);\vspace{0.3cm}\\
\((4)\int_{0}^{+\infty}\alpha\ee^{-\alpha x}\dd x,\,\alpha>0\);&&\((6)\int_{0}^{+\infty}\alpha\ee^{-\alpha^2x^2},\,\alpha>0\).\vspace{0.3cm}
\end{tabular}
\begin{proof}
(1)

(2)

(3)

(4)

(5)

(6)
\end{proof}
\woe 考察以下含参变量积分关于所考虑的参数的一致收敛性\footnote{习题中讨论一致收敛性需包括内闭一致收敛性, 或者说需要讨论关于一致收敛性能够得到的最好结果.}:\vspace{8pt}\\
\begin{tabular}{lcl}
\((1)\int_{0}^{\pi}\ln\left(1-2\alpha\cos x+\alpha^2\right)\dd x,\,\alpha\in\mathbb{R}\);&&\((2)\int_{0}^{1}x^{\alpha}\sin\frac{1}{x^\beta}\dd x,\,\alpha>-2,\beta>0\);\vspace{0.3cm}\\
\((3)\int_{1}^{+\infty}\frac{y^2-x^2}{(x^2+y^2)^2}\dd x,\, y\in\mathbb{R}\);&&\((4)\int_{0}^{1}\frac{y^2-x^2}{(x^2+y^2)^2}\dd x,\, y>0\);\vspace{0.3cm}
\end{tabular}
\begin{solution}
(1)

(2)

(3)

(4)
\end{solution}
\woe 设\(f\in C(0,+\infty),\int_{0}^{+\infty}t^\lambda f(t)\dd t\)当\(\lambda=a,\lambda=b\)时都收敛. 证明: \(\int_{0}^{+\infty}t^\lambda f(t)\dd t\)关于\(\lambda\in[a,b]\)一致收敛.
\begin{proof}

\end{proof}
\end{quiza}
\begin{quizb}
\woe 考察使得以下积分收敛的参数\(\alpha,p,q\)以及积分一致收敛的范围:\vspace{8pt}\\
\begin{tabular}{lcl}
\((1)\int_{0}^{1}x^\alpha\sin\left(x^p+\frac{1}{x^q}\right)\dd x\);&&\((2)\int_{1}^{+\infty}x^\alpha\sin\left(x^p+\frac{1}{x^q}\right)\).
\end{tabular}
\begin{solution}
内容...
\end{solution}
\woe 试讨论变量代换对于反常积分收敛性与一致收敛性的影响.
\end{quizb}
\section{含参变量积分的性质}
\begin{theorem}{}{chang}
设\(f:\left[a,+\infty\right)\times\varOmega\rightarrow\mathbb{R}\)其中\(\varOmega\subseteq\mathbb{R}^n\)为区域, 且存在\(\boldsymbol{y}_0\in\varOmega\), 使得\(\int_{a}^{+\infty}f(x,\boldsymbol{y}_0)\dd x\)收敛. 对几乎所有的\(x\in\left[a,+\infty\right),f(x,\cdot)\)在\(\varOmega\)内(连续)可微. 对任何\(A>a\), 存在\(g_A\in L[a,A]\)使得\[\left|f_{\boldsymbol{y}}(x,\boldsymbol{y})\right|\leqslant g_A(x),\qquad\forall\boldsymbol{y}\in\varOmega,\mathrm{a.e.}x\in[a,A].\]进一步,\(\int_{a}^{+\infty}f_{\boldsymbol{y}}(x,\boldsymbol{y})\dd x\)关于\(\boldsymbol{y}\in\varOmega\)一致收敛, 则对任何\(\boldsymbol{y}\in\varOmega\),\(\int_{a}^{+\infty}f(x,\boldsymbol{y})\dd x\)收敛, 而\(\varphi(\boldsymbol{y})=\int_{a}^{+\infty}f(x,\boldsymbol{y})\dd x\)在\(\Omega\)内(连续)可微且\[\varphi_{\boldsymbol{y}}(\boldsymbol{y})=\int_{a}^{+\infty}f_{\boldsymbol{y}}(x,\boldsymbol{y})\dd x,\qquad\forall\boldsymbol{y}\in\varOmega.\]另外, 当\(\varOmega\)为有界凸区域时, \(\int_{a}^{+\infty}f(x,\boldsymbol{y})\dd x\)关于\(\boldsymbol{y}\in\varOmega\)一致收敛.
\end{theorem}
\precis{含参变量积分的极限与连续性,含参变量积分的可微性,含参变量积分的积分,积分的计算}
\begin{quiza}
\woe 求极限:\(\lim_{n\rightarrow+\infty}\int_{0}^{\pi/2}\frac{\sin^n x+\cos^nx}{\sin^{n+1}x+\cos^{n+1}x}\dd x.\)
\begin{solution}
因为有\[\int_{0}^{\pi/2}\frac{\sin^n x+\cos^nx}{\sin^{n+1}x+\cos^{n+1}x}\dd x\xlongequal{\tan x=t}\int_{0}^{+\infty}\frac{t^n+1}{\left(t^{n+1}+1\right)\sqrt{t^2+1}}\dd t\]同时有\(\frac{t^n+1}{\left(t^{n+1}+1\right)\sqrt{t^2+1}}\leqslant\frac{2}{\sqrt{1+t^2}}\), 依Lebesgue控制收敛定理有\[\lim_{n\rightarrow+\infty}\int_{0}^{+\infty}\frac{t^n+1}{\left(t^{n+1}+1\right)\sqrt{t^2+1}}\dd t=\int_{0}^{1}\frac{1}{\sqrt{t^2+1}}\dd t+\int_{1}^{+\infty}\frac{1}{t\sqrt{t^2+1}}\dd t=2\ln(\sqrt{2}+1).\]
或者, 注意到\(0\leqslant x\leqslant\frac{\pi}{4},\) 有\(0\leqslant \sin x\leqslant\cos x\leqslant 1,\sec x\geqslant 1\), 于是\[\int_{0}^{\pi/4}\frac{\sin^nx+\cos^nx}{\sin^{n+1}x+\cos^{n+1}x}\dd x\geqslant\int_{0}^{\pi/4}\frac{\sin^nx+\cos^nx}{\sin^nx\cos x+\cos^{n+1}x}\dd x=\int_{0}^{\pi/4}\frac{1}{\cos x}\dd x,\]另一方面,\[
\int_{0}^{\pi/4}\frac{\sin^nx+\cos^nx}{\sin^{n+1}x+\cos^{n+1}x}\dd x\leqslant\int_{0}^{\pi/4}\frac{\sin^nx+\cos^nx}{\cos^{n+1}x}\dd x=\int_{0}^{\pi/4}\frac{1}{\cos x}\dd x+\int_{0}^{\pi/4}\tan^nx\sec x\dd x,\]而\[\int_{0}^{\pi/4}\tan^nx\sec x\dd x\leqslant\int_{0}^{\pi/4}\tan^nx\sec^2 x\dd x=\frac{1}{n+1}.\]依夹逼准则可知\[\lim_{n\rightarrow+\infty}\int_{0}^{\pi/4}\frac{\sin^nx+\cos^nx}{\sin^{n+1}x+\cos^{n+1}x}=\int_{0}^{\pi/4}\sec x\dd x=\ln(\sqrt{2}+1).\]注意到被积函数关于\(x=\frac{\pi}{2}\)对称, 这就证明了结论. 
\end{solution}
\woe 通过引入参数并利用积分号下求导计算\(\int_{0}^{1}\frac{\ln (1+x)}{1+x^2}\dd x\).
\begin{solution}
记\(I(\alpha)=\int_{0}^{1}\frac{\ln(1+\alpha x)}{1+x^2}\dd x\), 则有\(I(0)=0, I(1)=\int_{0}^{1}\frac{\ln(1+x)}{1+x^2}\dd x.\)显然含参积分关于\(\alpha\)一致收敛, 于是有\[I'(\alpha)=\int_{0}^{1}\frac{1}{1+x^2}\frac{x}{1+\alpha x}\dd x=\frac{1}{1+\alpha^2}\left(\frac{\ln 2}{2}+\frac{\pi}{4}\alpha-\ln(1+\alpha)\right),\]于是\[I(1)=\int_{0}^{1}I'(\alpha)\dd \alpha=\int_{0}^{1}\frac{1}{1+\alpha^2}\left(\frac{\ln 2}{2}+\frac{\pi}{4}\alpha\right)\dd \alpha-I(1),\]于是\(I(1)=\frac{1}{2}\int_{0}^{1}\frac{1}{1+\alpha^2}\left(\frac{\ln 2}{2}+\frac{\pi}{4}\alpha\right)\dd\alpha=\frac{\pi}{8}\ln 2.\)
\end{solution}
\woe 对应于定理\reff{Th:chang}, 给出并证明含参变量无穷积分被积函数关于参数仅在单点可微时的结果.
\woe 对应于定理\reff{Th:chang}, 给出可能含有瑕点的含参变量积分可微性的结果并证明.
\woe 证明\(\int_{0}^{\pi}\frac{\sin x}{x^\alpha(\pi-\alpha)^{2-\alpha}}\dd x\)是区间\((0,2)\)内关于\(\alpha\)的无界, 连续函数.
\begin{proof}

\end{proof}
\woe 计算\(\lim_{n\rightarrow+\infty}\int_{0}^{2}\frac{x^n\ln x}{1+x^n}\dd x\)并说明计算过程合理.
\begin{solution}
由Weierstrass判别法可知原积分关于参数\(n\)一致收敛, 从而\[\begin{split}
&\lim_{n\rightarrow+\infty}\int_{0}^{2}\frac{x^n\ln x}{1+x^n}\dd x=\lim_{n\rightarrow+\infty}\left(\int_{0}^{1}\frac{x^n\ln x}{1+x^2}\dd x+\int_{1}^{2}\frac{x^n\ln x}{1+x^n}\dd x\right)\\=&\int_{0}^{1}\lim_{n\rightarrow+\infty}\frac{x^n\ln x}{1+x^n}\dd x+\int_{1}^{2}\lim_{n\rightarrow+\infty}\frac{x^n\ln x}{1+x^n}\dd x\\=&\int_{1}^{2}\ln x\dd x=2\ln 2-1.\qedhere
\end{split}\]
\end{solution}
\woe 设\(a,b>0\), 利用Frullani积分计算\(\int_{0}^{1}\frac{x^b-x^a}{\ln x}\dd x\).
\begin{solution}
置\(\ln x=-t\), 于是\[\int_{0}^{1}\frac{x^b-x^a}{\ln x}\dd x=\int_{0}^{+\infty}\frac{\ee^{-t(a+1)}-\ee^{-t(b+1)}}{t}\dd t=\ln\frac{b+1}{a+1}.\qedhere\]
\end{solution}
\woe 考察下列积分与例10.3.5的联系, 并尝试以各种方法计算这些积分:
\begin{quizs}
\item \(\int_{0}^{\pi/2}\ln(a^2-\sin^2x)\dd x,\quad a\geqslant 1.\)
\item \(\int_{0}^{\pi}\ln(1+a\cos x)\dd x\quad |a|\leqslant 1\).
\item \(\int_{0}^{\pi}\ln(a^2\cos^2 x+b^2\sin^2x)\dd x,a^2+b^2>0.\)
\end{quizs}
\begin{solution}
依靠例10.3.5可以知道\(\int_{0}^{\pi}\ln\left(2-2\cos x\right)\dd x=0.\) 于是也有\(\int_{0}^{\pi}\ln(2+2\cos x)\dd x=0\), 从而\[\int_{0}^{\pi}\ln\left(1-\cos x\right)\dd x=\int_{0}^{\pi}\ln\left(1+\cos x\right)=-\pi\ln 2,\]于是有\[\begin{split}
&\int_{0}^{\pi}\ln\left(1-\cos x\right)\dd x+\int_{0}^{\pi}\ln\left(1+\cos x\right)\dd x=\int_{0}^{\pi}\ln\left(1-\cos^2x\right)\dd x\\=&2\int_{0}^{\pi/2}\ln\left(1-\sin^2x\right)\dd x=-2\pi\ln 2.
\end{split}\]

(1)记\(F(a)=\int_{0}^{\pi/2}\ln\left(a^2-\sin^2x\right)\dd x\), 则有\[\begin{split}
F'(a)&=\int_{0}^{\pi/2}\frac{2a}{a^2-\sin^2x}\dd x=\int_{0}^{\pi/2}\frac{2a}{(a^2-1)\sin^2x+a^2\cos^2x}\dd x\\
&=\int_{0}^{\pi/2}\frac{2a\sec^2x}{(a^2-1)\tan^2x+a^2}\dd x=\int_{0}^{\pi/2}\frac{2a}{(a^2-1)\tan^2x+a^2}\dd\left(\tan x\right)=\frac{\pi}{\sqrt{a^2-1}},
\end{split}\]于是\[F(a)=\int\frac{\pi}{\sqrt{a^2-1}}\dd a=\pi\ln(a+\sqrt{a^2-1})+C,\]由前述分析知\(F(1)=-\pi\ln 2\), 从而\(C=-\pi\ln 2\), 即\(F(a)=\pi\ln\left(a+\sqrt{a^2-1}\right)-\pi\ln 2\).

(2)记\(G(x)=\int_{0}^{\pi}\ln\left(1+a\cos x\right)\dd x\), 有\(G(x)=\int_{0}^{\pi}\ln\left(1-a\cos x\right)\dd x\), 这意味着我们可以只讨论\(a>0\)的情况, 同时它也带给我们了另一个便利. 这在之后的不久便能体会到. 易见\(F(a)=\int_{0}^{\pi/2}\ln\left(a^2-\cos^2x\right)\dd x\), 当\(0<a\leqslant 1\)时, \(\frac{1}{a}\geqslant 1\), 于是\[\begin{split}
&F\left(\frac{1}{a}\right)=\int_{0}^{\pi/2}\ln\left(\frac{1-a^2\cos^2x}{a^2}\right)\dd x\\=&\frac{1}{2}\int_{0}^{\pi}\left(\ln(1+a\cos x)+\ln(1-a\cos x)-\ln a^2\right)\dd x=G(x)-\pi \ln a,
\end{split}\]由(1)的结果可得\[G(x)=\pi\ln\left(1+\sqrt{1-a^2}\right).\]

(3)由于\[\begin{split}
\int_{0}^{\pi}\ln(a^2\cos^2x+b^2\sin^2x)\dd x&=\int_{0}^{\pi/2}\ln(a^2\cos^2x+b^2\sin^2x)\dd x+\int_{\pi/2}^{\pi}\ln(a^2\cos^2x+b^2\sin^2x)\dd x\\&=\int_{0}^{\pi/2}\ln(a^2\cos^2x+b^2\sin^2x)\dd x+\int_{0}^{\pi/2}\ln(a^2\sin^2x+b^2\cos^2x)\dd x,
\end{split}\]
不妨记\(I(a,b)=\int_{0}^{\pi/2}\ln(a^2\sin^2x+b^2\cos^2x)\dd x\). 若\(a=b>0\), 则有\[I(a,a)=\int_{0}^{\pi/2}\ln a^2\dd x=\pi\ln a,\]当\(a,b>0\)时有\[\begin{split}
&\frac{\partial I(a,b)}{\partial a}=\int_{0}^{\pi/2}\frac{2a\sin^2x}{a^2\sin^2x+b^2\cos^2x}\dd x=2a\int_{0}^{\pi/2}\frac{\tan^2x}{a^2\tan^2x+b^2}\dd x\\=&2a\int_{0}^{+\infty}\frac{t^2}{(a^2t^2+b^2)(1+t^2)}\dd t=\frac{2a}{a^2-b^2}\int_{0}^{+\infty}\left(\frac{1}{1+t^2}-\frac{b^2}{a^2t^2+b^2}\right)\dd t\\=&\frac{2a}{a^2-b^2}\left.\left(\arctan t-\frac{b}{a}\arctan\frac{a}{b}t\right)\right|^{+\infty}_0=\frac{\pi}{a+b},
\end{split}\]于是\(I(a,b)=\pi\ln(a+b)+C(b)\), 令\(a=b\)带入可得\[C(a)=I(a,a)-\pi\ln\left(2a\right)=\pi\ln a-\pi\left(\ln 2+\ln a\right)=-\pi\ln 2,\]即\(C(b)=-\pi\ln 2\), 得到\(I(a,b)=\pi\ln\frac{a+b}{2}.\)

当\(a>0,b=0\)时, 有\[I(a,0)=\int_{0}^{\pi/2}\ln\left(a^2\sin^2x\right)\dd x=2\int_{0}^{\pi/2}\left(\ln a+\ln\sin x\right)\dd x=2\left(\frac{\pi}{2}\ln a+\left(-\frac{\pi}{2}\ln 2\right)\right)=\pi\ln\frac{a}{2},\]同理\(a=0,b>0\)时\(I(0,b)=\pi\ln\frac{b}{2}\). 综上所述当\(a^2+b^2>0\)时有\(I(a,b)=\pi\ln\frac{|a|+|b|}{2}\), 显然对于\(\int_{0}^{\pi/2}\ln\left(a^2\cos^2x+b^2\sin^2x\right)\dd x\)有同样的结果, 故\[\int_{0}^{\pi}\ln\left(a^2\cos^2x+b^2\sin^2x\right)\dd x=2\pi\ln\frac{|a|+|b|}{2}.\qedhere\]
\end{solution}
\textbf{例 10.3.5} 设\(\alpha\in(-\infty,+\infty)\). 计算\(\int_{0}^{\pi}\ln\left(1-2\alpha\cos x+\alpha^2\right)\dd x\).
\begin{solution}
首先, 从含参积分的瑕点定义来看, 当\(\alpha\)取值于整个\(\bbr\)时, \([0,\pi]\)上所有点都是积分的瑕点. 但当我们把\(\alpha\)限制在\(\bbr\)上的有界集上时, 积分可能的瑕点只有\(0\)和\(\pi\). 我们有\[\sin^2x\leqslant 1-2\alpha\cos x+\alpha^2\leqslant 2+2\alpha^2,\]因此,\[\left|\ln\left(1-2\alpha\cos x+\alpha^2\right)\right|\leqslant \left|\ln\sin^2x\right|+\ln\left(2+2\alpha^2\right),\quad \forall x\in(0,\pi),\]于是由Weierstrass判别法, \(\int_{0}^{\pi}\ln\left(1-2\alpha\cos x+\alpha^2\right)\dd x\)关于\(\alpha\in(-\infty,+\infty)\)内闭一致收敛. 于是利用一致收敛性或Lebesgue控制收敛定理,\[F(\alpha)=\int_{0}^{\pi}\ln\left(1-2\alpha\cos x+\alpha^2\right)\dd x\]在\(-\infty,+\infty\)上连续. 我们有\[\begin{split}
F(\alpha)&=F(-\alpha)=\frac{1}{2}\left(F(\alpha)+F(-\alpha)\right)=\frac{1}{2}\int_{0}^{\pi}\ln\left((1+\alpha^2)-4\alpha^2\cos^2x\right)\dd x\\&=\frac{1}{2}\int_{0}^{\pi}\ln\left((1+\alpha^2)^2-2\alpha^2\cos2x-2\alpha^2\right)\dd x=\frac{1}{4}\int_{0}^{2\pi}\ln\left(1+\alpha^4-2\alpha^2\cos x\right)\dd x\\&=\frac{1}{2}\int_{0}^{\pi}\ln\left(1+\alpha^4-2\alpha^2\cos x\right)\dd x=\frac{1}{2}F(\alpha^2).
\end{split}\]因此, \(F(\pm 1)=0\). 更一般地, 反复利用上式可得\[F(\alpha)=\frac{1}{2^n}F\left(\alpha^{2^n}\right),\quad \forall\left|\alpha\right|<1,n\geqslant 1.\]所以\[F(\alpha)=\lim_{n\rightarrow+\infty}\frac{1}{2^n}F\left(\alpha^{2^n}\right)=0,\quad F(0)=0,\quad \forall\left|\alpha\right|<1.\]最后, 结合\(F\left(\frac{1}{\alpha}\right)=-2\pi\ln\left|\alpha\right|+F(\alpha)\)可得\(F(\alpha)=\begin{cases}
2\pi\ln\left|\alpha\right|,\quad &\left|\alpha\right|>1,\\
0,&\left|\alpha\right|<1.
\end{cases}\)
\end{solution}
\woe 证明\(\lim_{\alpha\rightarrow+\infty}\int_{0}^{+\infty}\frac{\sin x}{x\sqrt{x}}\ee^{-\alpha x}\dd x=0.\)
\begin{proof}

\end{proof}
\woe 计算\(\lim_{n\rightarrow+\infty}\int_{0}^{1}\frac{\cos(\pi x)-\ee^{-n^2x^2}}{n\arcsin x}\frac{\dd x}{\ln (1+x)}\)并说明计算过程成立的理由.
\begin{solution}

\end{solution}
\woe 计算\(\lim_{n\rightarrow+\infty}\sqrt{n}\int_{0}^{+\infty}\frac{\cos x}{(1+x^2)^n}\dd x\)并说明计算过程的正确性.
\begin{solution}

\end{solution}
\woe 证明或证伪: 当\(n\rightarrow+\infty\)时, \(\int_{0}^{\pi}\ee^{xt}\left(\frac{\cos\left(n+1/2\right)t}{\ln(1+t)}-\frac{\cos nt}{t}\right)\dd t\)关于\(x\in[0,1]\)一致收敛于零.
\woe 设\(f\)是\(\left[0,1\right)\)上的连续可微函数, \(f(0)=0,f'\in L^2(0,1)\). 证明: \[\int_{0}^{1}\frac{|f(x)|^2}{x^2}\dd x\leqslant 4\int_{0}^{1}|f'(x)|^2\dd x.\]
\begin{proof}
记\(g(x)=\begin{cases}
\frac{f(x)}{x},\quad &x\in(0,1),\\
f'(0),&x=0
\end{cases}\), 那么原不等式等价于\[\int_{0}^{1}\left(g(x)\right)^2\dd x\leqslant 4\int_{0}^{1}\left(g(x)+xg'(x)\right)^2\dd x,\]即\[0\leqslant\int_{0}^{1}\left(3g^2+8xgg'+4x^2(g')^2\right)\dd x=2\int_{0}^{1}\left(g^2+2xgg'\right)\dd x+\int_{0}^{1}\left(g+2xg'\right)^2\dd x,\]
而\(\int_{0}^{1}\left(g^2+2xgg'\right)\dd x=\left(xg^2\right)\Big|_0^1=g^2(1)\geqslant 0\), 原不等式得证.
\end{proof}
\woe 设\(f\)为以\(\pi\)为周期且在\([0,\pi]\)上可积的偶函数. 证明:
\begin{quizs}
\item \(\int_{0}^{+\infty}f(x)\frac{\sin x}{x}\dd x=\int_{0}^{\pi/2}f(x)\dd x\).
\item \(\int_{0}^{+\infty}f(x)\left(\frac{\sin x}{x}\right)^2\dd x=\int_{0}^{\pi/2}f(x)\dd x\).
\end{quizs}
\begin{proof}
(1)我们分三步证明.
\begin{asparaenum}[(i)]
\item\textbf{收敛性.} 首先, \(\int_{0}^{2\pi}\frac{f(x)\sin x}{x}\dd x\)是一个定积分. 其次, 由题设, 可见\[\int_{2\pi}^{2k\pi}f(x)\sin x\dd x=0,\quad\forall k\geqslant 1.\]由此不难得到对任何\(A>2\pi\), 成立\[\left|\int_{2\pi}^{A}f(x)\sin x\dd x\right|\leqslant\int_{0}^{\pi}\left|f(x)\right|\dd x.\]于是由Dirichlet判别法, \(\int_{2\pi}^{+\infty}\frac{f(x)\sin x}{x}\dd x\)收敛, 即原积分收敛.
\item\textbf{\(f\)为三角多项式情形.} 若\(f=\sum_{k=0}^{m}a_k\cos (2kx)\), 则\[\begin{split}
\int_{0}^{+\infty}\frac{f(x)\sin x}{x}\dd x=&\int_{0}^{+\infty}\sum_{k=0}^{m}\frac{a_k\cos(2kx)\sin x}{x}\dd x\\=&\frac{1}{2}\int_{0}^{+\infty}\sum_{k=0}^{m}a_k\frac{\sin\left((2k+1)x\right)-\sin\left((2k-1)x\right)}{x}\dd x\\=&\frac{\pi}{4}\sum_{k=0}^{m}a_k\left(\mathrm{sgn}(2k+1)-\mathrm{sgn}(2k-1)\right)=\frac{\pi a_0}{2}=\int_{0}^{\pi/2}f(x)\dd x.
\end{split}\]
\item\textbf{一般情形.} 一般地, 由于\(f\)是以\(2\pi\)为周期的连续函数, 根据Weierstrass逼近定理, 存在一列三角多项式\(T_n\)使得\[\int_{0}^{2\pi}\left|f(x)-T_n\right|\dd x\rightarrow 0,\quad n\rightarrow+\infty.\]令\[g_n(x)=\frac{T(x)+T(-x)}{2}+\frac{1}{2\pi}\int_{0}^{2\pi}\left(f(t)-T_n(t)\right)\dd t.\]则\(g_n\)具有形式\(g_n(x)=\sum_{k=0}^{m_n}a_{n,k}\cos(2kx)\), 满足\[\int_{0}^{2\pi}\left(g_n(x)-f(x)\right)\dd x=0,\]且\[\varepsilon_n\equiv\int_{0}^{\pi}\left|f(x)-g_n(x)\right|\dd x\rightarrow 0,\quad n\rightarrow+\infty.\]记\[G_n(x)=\int_{0}^{x}\left(f(t)-g_n(t)\right)\sin t\dd t,\quad x\geqslant 0.\]则\[G_n(2k\pi)=0,\quad\left|G_n(x)\right|\leqslant\int_{0}^{\pi}\left|f(t)-g_n(t)\right|\dd t=\varepsilon_n.\]我们有\[\begin{split}
&\left|\int_{0}^{2\pi}\frac{\left(f(x)-g_n(x)\right)\sin x}{x}\dd x\right|\int_{0}^{2\pi}\left|f(x)-g_n(x)\right|\dd x=2\varepsilon_n,\\
&\left|\int_{2\pi}^{+\infty}\frac{\left(f(x)-g_n(x)\right)\sin x}{x}\dd x\right|=\left|\frac{G_n(x)}{x}\Big|_{2\pi}^{+\infty}+\int_{2\pi}^{+\infty}\frac{G_n(x)}{x^2}\dd x\right|\\=&\left|\int_{2\pi}^{+\infty}\frac{G_n(x)}{x^2}\dd x\right|\leqslant\varepsilon_n\int_{2\pi}^{+\infty}\frac{1}{x^2}\dd x\leqslant \varepsilon_n.
\end{split}\]
于是\[\begin{split}
&\left|\int_{0}^{+\infty}\frac{f(x)\sin x}{x}\dd x-\int_{0}^{\pi/2}f(x)\dd x\right|=\left|\int_{0}^{+\infty}\frac{\left(f(x)-g_n(x)\right)\sin x}{x}\dd x-\int_{0}^{\pi/2}\left(f(x)-g_n(x)\right)\dd x\right|\\\leqslant & \left|\int_{0}^{2\pi}\frac{\left(f(x)-g_n(x)\right)\sin x}{x}\dd x\right|+\left|\int_{2\pi}^{+\infty}\frac{\left(f(x)-g_n(x)\right)\sin x}{x}\dd x\right|+\left|\int_{0}^{\pi/2}\left(f(x)-g_n(x)\right)\dd x\right|\\\leqslant &2\varepsilon_n+\varepsilon_n+\varepsilon_n.
\end{split}\]令\(n\rightarrow+\infty\)即得结论.
\end{asparaenum}

(2)情况类似.
\begin{asparaenum}[(i)]
\def\sgn{\mathrm{sgn}}
\item \textbf{收敛性. }易见\(\int_{0}^{2\pi}f(x)\left(\frac{\sin x}{x}\right)^2\dd x\)是一个定积分, 而\(\int_{2\pi}^{+\infty}f(x)\left(\frac{\sin x}{x}\right)^2\dd x\)绝对收敛, 因此, 原积分收敛.
\item \textbf{\(f\)为三角多项式情形. }设\(f=\sum_{k=0}^{m}a_k\cos(2k\pi x)\). 注意到\[\begin{split}
&\int_{0}^{+\infty}\cos(2kx)\left(\frac{\sin x}{x}\right)^2\dd x=\int_{0}^{+\infty}\frac{\cos(2kx)\sin 2x-2k\sin(2kx)\sin^2x}{x}\dd x\\
=&\int_{0}^{+\infty}\frac{\sin\left(2(k+1)x\right)-\sin(2(k-1)x)-2k\sin(2kx)+k\sin\left(2(k+1)x\right)+k\sin\left(2(k-1)x\right)}{2x}\dd x\\
=&\frac{\pi}{2}\left(\sgn(k+1)-\sgn(k-1)-2k\sgn(k)+k\sgn(k+1)+k\sgn(k-1)\right)=\begin{cases}
\pi/2,\quad &k=0,\\
0,\quad &k\geqslant 1.
\end{cases}
\end{split}\]此时\(\int_{0}^{+\infty}f(x)\left(\frac{\sin x}{x}\right)^2\dd x=\int_{0}^{\pi/2}f(x)\dd x\)成立.
\item \textbf{一般情形. }一般地, 对于满足题意的函数\(f\), 存在一列三角多项式\(g_n(x)=\sum_{k=0}^{m_n}a_{n,k}\cos(2kx)\), 使得\[\max_{x\in[0,\pi]}\left|g_n(x)\right|\leqslant M_f:=\sup_{x\in[0,\pi]}\left|f(x)\right|,\]且\[\varepsilon_n\equiv\int_{0}^{\pi}\left|f(x)-g_n(x)\right|\dd x\rightarrow 0,\quad n\rightarrow+\infty.\]取自然数列\(\{k_n\}\)使得\(k_n\rightarrow+\infty\), 且\(k_n^2\varepsilon_n\leqslant 1\). 我们有\[\left|\int_{0}^{+\infty}\left(f(x)-g_n(x)\right)\left(\frac{\sin x}{x}\right)^2\dd x\right|\leqslant\int_{0}^{k_n\pi}\left|f(x)-g_n(x)\right|\dd x+2M_f\int_{k_n\pi}^{+\infty}\frac{1}{x^2}\dd x=k_n\varepsilon_n+\frac{2M_f}{k_n\pi}.\]于是\[\int_{0}^{+\infty}f(x)\left(\frac{\sin x}{x}\right)^2\dd x=\lim_{n\rightarrow+\infty}\int_{0}^{+\infty}g_n(x)\left(\frac{\sin x}{x}\right)^2\dd x=\lim_{n\rightarrow+\infty}\int_{0}^{\pi/2}g_n(x)\dd x=\int_{0}^{\pi/2}f(x)\dd x.\]由此即得结论.\qedhere
\end{asparaenum}
\end{proof}
\tcbline
本题还有其他方法. 直接利用换元法与\(\csc x\)的展开式, 简述步骤如下.

\tcbline
\woe 设\(f\)为以\(2\pi\)为周期在\([0,2\pi]\)上可积的偶函数. 证明:\[\int_{0}^{+\infty}f(x)\frac{\sin x}{x}\dd x=\frac{1}{2}\int_{0}^{\pi}f(x)(1+\cos x)\dd x.\]
\begin{proof}

\end{proof}
\woe 证明\(\int_{0}^{1}\frac{1}{x^x}\dd x=\sum_{n=1}^{\infty}\frac{1}{n^n}\).
\begin{proof}
由\(\int_{0}^{1}\frac{1}{x^x}\dd x=\int_{0}^{1}\frac{1}{\ee^{x\ln x}}\dd x\), 并且\(\int_{0}^{1}\frac{1}{\ee^{x\ln x}}\dd x=\int_{0}^{1}\left(\sum_{n=0}^{\infty}\frac{(-1)^n(x\ln x)^n}{n!}\right)\dd x\), 设\[f(x)=\left|x\ln x\right|,\quad x\in[0,1],\]即\(f(x)=-x\ln x\), 易见\(f(x)\)有最大值\(\frac{1}{\ee}\), 从而上述的被积函数中的级数有优级数\(\sum_{n=0}^{\infty}\frac{\ee^n}{n!}\), 由Weierstrass判别法可知该级数一致收敛, 从而可逐项积分, 即\[\int_{0}^{1}\frac{1}{x^x}\dd x=\sum_{n=0}^{\infty}\frac{(-1)^n}{n!}\int_{0}^{1}x^n\ln^n x\dd x,\]现在来计算\(\int_{0}^{1}x^n\ln^n x\dd x\), 置\(x^{n+1}=\ee^{-y}\)得到\[\int_{0}^{1}x^n\ln^nx\dd x=\frac{(-1)^n}{(n+1)^{n+1}}\int_{0}^{+\infty}y^n\ee^{-y}\dd y,\]将右侧的积分记为\(\Gamma(n)\), 有\[\Gamma(n)=\int_{0}^{+\infty}y^n\ee^{-y}\dd y=\left.-y^n\ee^{-y}\right|_0^{+\infty}+n\int_{0}^{+\infty}y^{n-1}\ee^{-y}\dd y=n\Gamma(n-1),\]于是\(\sum_{n=0}^{\infty}\frac{(-1)^n}{n!}\cdot(-1)^n\frac{n!}{(n+1)^{n+1}}=\sum_{n=1}^{\infty}\frac{1}{n^n}.\)
\end{proof}
\woe 对于\(E\subseteq\mathbb{R}^n\), 用\(L\ln L(E)\)表示满足\(\int_E|f(\boldsymbol{x})|\ln\left(1+|f(\boldsymbol{x})|\right)\dd\boldsymbol{x}<+\infty\)的函数\(f\)的全体. 现设\(a>0,f:[0,1]\rightarrow\left[a,+\infty\right), f\in L\ln L[0,1]\). 证明:\[\int_{0}^{1}f(x)\ln f(x)\dd x\geqslant\int_{0}^{1}f(x)\dd x\int_{0}^{1}\ln f(x)\dd x.\]
\begin{proof}

\end{proof}
\woe 试利用\(\frac{1}{2}+\sum_{k=1}^{n}\cos kx=\frac{\sin\left(n+1/2\right)x}{2\sin(x/2)}\)证明:\[\frac{\pi}{2}=\lim_{n\rightarrow+\infty}\int_{0}^{\pi}\frac{\sin\left(n+1/2\right)x}{2\sin(x/2)}\dd x=\lim_{n\rightarrow+\infty}\int_{0}^{\pi}\frac{\sin\left(n+1/2\right)x}{x}=\int_{0}^{+\infty}\frac{\sin x}{x}\dd x.\]
\begin{proof}

\end{proof}
\end{quiza}
\begin{quizb}
\woe 设\(\alpha>0\), 求\(n\rightarrow+\infty\)时\(\int_{n\pi}^{(n+1)\pi}\frac{x}{1+x^\alpha\sin^2x}\dd x\)的阶.
\begin{solution}
我们记\[I_n=\int_{n\pi}^{(n+1)\pi}\frac{x}{1+x^\alpha\sin^2x}\dd x=\int_{0}^{\pi}\frac{x+n\pi}{1+(x+n\pi)^\alpha\sin^2x}\dd x,\]一方面\[\begin{split}
I_n&\geqslant 2\int_{0}^{\pi/2}\frac{n\pi}{1+(n+1)^\alpha\pi^\alpha\sin^2x}\dd x\geqslant 2\int_{0}^{\pi/2}\frac{n\pi}{1+(n+1)^\alpha\pi^\alpha x^2}\dd x\\&=\frac{2n\pi}{(n+1)^{\alpha/2}\pi^{\alpha/2}}\arctan\left(\frac{(n+1)^{\alpha/2}\pi^{\alpha/2+1}}{2}\right)=O(n^{1-\alpha/2}).
\end{split}\]另一方面\[\begin{split}
I_n&\leqslant 2\int_{0}^{\pi/2}\frac{(n+1)\pi}{1+n^\alpha\pi^{\alpha}\sin^2x}\dd x\leqslant 2\int_{0}^{\pi/2}\frac{(n+1)\pi}{1+n^{\alpha}\pi^{\alpha}(2x/\pi)^2}\dd x\\
&=\frac{(n+1)\pi}{n^{\alpha/2}\pi^{\alpha/2-1}}\arctan(n^{\alpha/2}\pi^{\alpha/2})=O(n^{1-\alpha/2}).\end{split}\]
故\(\int_{n\pi}^{(n+1)\pi}\frac{x}{1+x^\alpha\sin^2x}\dd x=O(n^{1-\alpha/2})\).
\end{solution}
\woe 计算\(\int_{0}^{+\infty}\frac{(1-x^2)\arctan x^2}{x^4+4x^2+1}\dd x.\)
\begin{solution}
注意到\(\arctan x^2=\int_{0}^{x^2}\frac{1}{1+y^2}\dd y\), 于是有\[\begin{split}
&\int_{0}^{+\infty}\frac{(1-x^2)\arctan x^2}{x^4+4x^2+1}\dd x=\int_{0}^{+\infty}\frac{1-x^2}{x^4+4x^2+1}\left(\int_{0}^{x^2}\frac{1}{1+y^2}\dd y\right)\dd x\\=&\int_{0}^{+\infty}\int_{\sqrt{y}}^{+\infty}\frac{1-x^2}{x^4+4x^2+1}\cdot\frac{1}{1+y^2}\dd x\dd y
\end{split}\]
先处理\(\int_{\sqrt{y}}^{+\infty}\frac{1-x^2}{x^4+4x^2+1}\dd x\), 有\[\frac{1-x^2}{x^4+4x^2+1}=\frac{A}{x^2+2-\sqrt{3}}-\frac{B}{x^2+2+\sqrt{3}},\]其中\(A=\frac{\sqrt{3}-1}{2},B=\frac{\sqrt{3}+1}{2}\), 于是有
\[\begin{aligned}
	\int_{\sqrt{y}}^{+\infty}{\frac{1-x^2}{x^4+4x^2+1}}\mathrm{dd}x&=\int_{\sqrt{y}}^{+\infty}{\frac{A}{x^2+2-\sqrt{3}}}-\frac{B}{x^2+2+\sqrt{3}}\mathrm{dd}x\\
	&=\frac{A}{\sqrt{2-\sqrt{3}}}\mathrm{arc}\tan \frac{x}{\sqrt{2-\sqrt{3}}}-\frac{B}{\sqrt{2+\sqrt{3}}}\mathrm{arc}\tan \frac{x}{\sqrt{2+\sqrt{3}}}|_{\sqrt{y}}^{+\infty}\\
	&=\frac{\sqrt{2}}{2}\left( \mathrm{arc}\tan \left( \sqrt{2-\sqrt{3}}\cdot \sqrt{y} \right) -\mathrm{arc}\tan \left( \sqrt{2+\sqrt{3}}\cdot \sqrt{y} \right) \right) .\\
\end{aligned}\]
我们记\(I(\alpha)=\int_{0}^{+\infty}\frac{\arctan\left(\alpha\sqrt{y}\right)}{1+y^2}\dd y(\alpha\geqslant 0)\), 易见其关于参数\(\alpha\)一致收敛,  于是\[\int_{0}^{+\infty}\int_{\sqrt{y}}^{+\infty}\frac{1-x^2}{x^4+4x^2+1}\cdot\frac{1}{1+y^2}\dd x\dd y=\frac{\sqrt{2}}{2}\left(I\left(\sqrt{2-\sqrt{3}}\right)-I\left(\sqrt{2+\sqrt{3}}\right)\right).\]
另一方面,易见\(I(0)=0\),\[I'(\alpha)=\int_{0}^{+\infty}\frac{\sqrt{y}}{(1+y^2)(1+\alpha^2y)}\dd y=\int_{0}^{+\infty}\frac{2y^2}{(1+y^4)(1+\alpha^2y^2)}\dd y,\]记\(f(z)=\frac{2z^2}{(1+z^4)(1+\alpha^2z^2)}\), 其在上半平面内有三个极点,\[a_1=\ee^{\pi\ii/4},\quad a_2=\ee^{3\pi\ii/4},\quad a_3=\frac{\ii}{\alpha},\]容易计算\(\underset{z=a_k}{\mathrm{Res}}f(z)=\frac{2z^2}{4z^3(1+\alpha^2z^2)+2\alpha^2z(1+z^4)}\Big|_{z=a_k}\)分别为\[r_1:=-\frac{\ee^{3\pi\ii/4}}{2\left(1+\alpha^2\ee^{\pi\ii/2}\right)},\quad r_2:=-\frac{\ee^{\pi\ii/4}}{2\left(1+\alpha^2\ee^{3\pi\ii/2}\right)},\quad r_3:=\frac{\ii\alpha}{1+\alpha^4},\]于是\(I'(\alpha)=\pi\ii\sum_{k=1}^{3}r_k=\frac{\left(\sqrt{2}\alpha^2-2\alpha+\sqrt{2}\right)\pi}{2(1+\alpha^4)}\), 注意到\[\int\frac{\alpha}{1+\alpha^4}\dd\alpha=\frac{1}{2}\arctan\alpha^2+C,\]以及\[\int\frac{\alpha^2+1}{1+\alpha^4}\dd\alpha=\int\frac{1+1/\alpha^2}{\left(\alpha-1/\alpha\right)^2+2}\dd\alpha=\frac{1}{\sqrt{2}}\arctan\left(\frac{\alpha-1/\alpha}{\sqrt{2}}\right)+C.\]
于是\[\begin{split}
&I\left(\sqrt{2-\sqrt{3}}\right)-I\left(\sqrt{2+\sqrt{3}}\right)=\int_{\sqrt{2+\sqrt{3}}}^{\sqrt{2-\sqrt{3}}}I'(\alpha)\dd\alpha\\
=&\frac{\pi}{2}\left(\arctan\left(\frac{\alpha-1/\alpha}{\sqrt{2}}\right)-\arctan\alpha^2\right)\Big|_{\sqrt{2+\sqrt{3}}}^{\sqrt{2-\sqrt{3}}}\\
=&\frac{\pi}{2}\left(2\arctan\frac{\sqrt{2-\sqrt{3}}-\sqrt{2+\sqrt{3}}}{\sqrt{2}}-\left(\arctan\sqrt{2-\sqrt{3}}-\arctan\sqrt{2+\sqrt{3}}\right)\right)\\
=&\frac{\pi}{2}\left(-\frac{\pi}{2}+\frac{\pi}{3}\right)=-\frac{\pi}{12}.
\end{split}\]
所以\(\int_{0}^{+\infty}\frac{(1-x^2)\arctan x^2}{x^4+4x^2+1}\dd x=-\frac{\sqrt{2}}{2}\cdot\frac{\pi^2}{12}=-\frac{\sqrt{2}\pi^2}{24}.\)
\end{solution}
\woe 设\(\alpha\in[0,1]\), 计算\(\int_{0}^{\pi}\ln|a^2-\sin^2x|\dd x.\)
\begin{solution}

\end{solution}
\woe 试寻求一些线性算子的积分不等式, 给出与之对偶的不等式.
\woe 构造\([0,1]\)上的线性函数\(f\), 使得\(\overline{\{\left(x,f(x)\right)|x\in[0,1]\}}=[0,1]\times [0,1]\).
\woe 设\(p>1,f\in L^p(0,+\infty)\)且\(f\)非负,证明\textbf{Hardy不等式}:\[\int_{0}^{+\infty}\left(\frac{1}{x}\int_{0}^{x}f(t)\dd t\right)^p\dd x\leqslant\left(\frac{p}{p-1}\right)^p\int_{0}^{+\infty}f^p(x)\dd x,\]其中\(\left(\frac{p}{p-1}\right)^p\)为最佳常数, 等号成立当且仅当\(\int_{0}^{+\infty}f(x)\dd x=0\).
\begin{proof}

\end{proof}
\woe 设\(p>1,f\in L^p(0,+\infty)\)且\(f\)非负. 又设\(r>0,r\ne 1\). 证明Hardy不等式的推广: 若\(r>1\), 则\[\int_{0}^{+\infty}x^{-r}\left(\int_{0}^{x}f(t)\dd t\right)^p\dd x\leqslant\left(\frac{p}{r-1}\right)^p\int_{0}^{+\infty}x^{p-r}f^p(x)\dd x.\]若\(0<r<1\), 则\[\int_{0}^{+\infty}x^{-r}\left(\int_{x}^{+\infty}f(t)\dd t\right)^p\dd x\leqslant\left(\frac{p}{r-1}\right)^p\int_{0}^{+\infty}x^{p-r}f^p(x)\dd x.\]
\begin{proof}

\end{proof}
\woe 设\(p,q>1\)为对偶数, \(f\in L^q(0,+\infty)\)且\(f\)非负. 令\(F(f)(x)=\int_{x}^{+\infty}\frac{f(t)}{t}\dd t.\)利用Hardy不等式以及对偶关系证明: \(\left\|F(f)\right\|_{L^q(0,+\infty)}\leqslant\frac{p}{p-1}\left\|f\right\|_{L^q(0,+\infty)}\).
\begin{proof}

\end{proof}
\end{quizb}
\section{Euler积分}
\precis{\(\Gamma\)函数,\(\Gamma\)函数的递推公式,log-凸,Stirling公式及其改进,Euler公式,Gauss叠乘定理,倍元公式,Bohr-Mollerup定理,余元公式,B函数,B函数与\(\Gamma\)函数的关系,多重对数函数,双\(\Gamma\)函数,利用Euler积分计算}
\begin{quiza}
\woe 求实数\(a\)的取值范围, 使得积分\(\int_{0}^{+\infty}\frac{1}{x^a(x+1)(x+2)(x+3)}\dd x\)收敛, 并计算该积分.
\begin{solution}
由于\[\frac{1}{x^a(x+1)(x+2)(x+3)}\sim\frac{1}{6x^a},x\rightarrow 0^+,\quad \frac{1}{x^a(x+1)(x+2)(x+3)}\sim\frac{1}{x^{a+3}},x\rightarrow+\infty,\]从而\(a\in(-2,1)\)时原积分收敛.又\[\frac{1}{(x+1)(x+2)(x+3)}=\frac{1}{2}\left(\frac{1}{x+1}-\frac{2}{x+2}+\frac{1}{x+3}\right),\]如果\(a\in(0,1)\), 那么根据例10.4.2的已知结果\[\int_{0}^{+\infty}\frac{1}{(1+x)^a}\dd x=\int_{0}^{+\infty}\frac{1}{(1+x)x^{1-a}}\dd x=\frac{\pi}{\sin a\pi},\]有\[\begin{split}
	&\int_{0}^{+\infty}\frac{1}{x^a(x+1)(x+2)(x+3)}\dd x=\frac{1}{2}\int_{0}^{+\infty}\left(\frac{1}{x^a(1+x)}-\frac{2}{x^a(x+2)}+\frac{1}{x^a(x+3)}\right)\dd x\\=&\frac{1}{2}\int_{0}^{+\infty}\frac{1}{x^a(1+x)}\dd x-\int_{0}^{+\infty}\frac{1}{2^ax^a(x+1)}\dd x+\frac{1}{2}\int_{0}^{+\infty}\frac{1}{3^ax^a(x+1)}\dd x\\=&\frac{1}{2}\frac{\pi}{\sin a\pi}-\frac{1}{2^a}\frac{\pi}{\sin a\pi}+\frac{1}{2\cdot 3^a}\frac{\pi}{\sin a\pi}=\frac{\pi}{2\sin a\pi}\left(1-\frac{1}{2^{a-1}}+\frac{1}{3^a}\right),
\end{split}\]

当\(a=0\)时, 可以算得\[\begin{split}
	&\int_{0}^{A}\frac{1}{(x+1)(x+2)(x+3)}\dd x=\frac{1}{2}\int_{0}^{A}\left(\frac{1}{x+1}-\frac{2}{x+2}+\frac{1}{x+3}\right)\dd x\\=&\frac{1}{2}\left(\ln(A+1)-2\ln(A+2)+\ln(A+3)+2\ln 2-\ln 3\right)\rightarrow \ln 2-\frac{\ln 3}{2},\quad A\rightarrow+\infty,
\end{split}\]类似得, \(a=-1\)时有\[\begin{split}
&\int_{0}^{A}\frac{x}{(x+1)(x+2)(x+3)}\dd x=\frac{1}{2}\int_{0}^{A}\left(\frac{4}{x+2}-\frac{3}{x+3}-\frac{1}{x+1}\right)\dd x\\=&\frac{1}{2}\left(4\ln(A+2)-3\ln(A+3)-\ln(A+1)-4\ln 2+3\ln 3\right)\rightarrow\frac{3}{2}\ln 3-2\ln 2,\quad A\rightarrow+\infty.
\end{split}\]

现在我们来计算\(a\in (-2,0)\)的情况, 前面的计算已经表明了\(a=-1\)的情形, 事实证明, 这是有益的, 因为下面的计算无法显示\(a\)为整数时的值. 方便起见, 令\(b=-a\), 则\(b\in(0,2)\)原积分也对应变为\(\int_{0}^{+\infty}\frac{x^b}{(x+1)(x+2)(x+3)}\dd x.\)

设\(F(z)=\frac{1}{(z+1)(z+2)(z+3)}\), 考虑围道积分\(\int_Cz^bF(z)\dd z\), 其中\(C\)是如下图所示的区域.
\begin{figure}[H]
	\centering
	\begin{tikzpicture}
		[decoration={
			markings,
			mark=at position 0.5 with {\arrow{>}}}
		]
		\draw[-latex] (-2.5,0)--(2.5,0);
		\draw[-latex] (0,-2.5)--(0,2.5);
		\node at (2.3,-0.2){{\footnotesize $x$}};
		\node at (0.2,2.3){{\footnotesize $y$}};
		\fill (-1,0)circle(1pt);
		\node at (-1,-0.25) {{\footnotesize $-1$}};
		
		\node at (1.8,1.5){{\footnotesize $C_R$}};
		\node at (0.2,0.2){{\footnotesize $C_r$}};
		\node at (-0.15,-0.15){{\footnotesize $O$}};
		
		\draw[red] (15:0.5) arc(15:60:0.5);
		\draw[<-,red] (60:0.5) arc(60:135:0.5);
		\draw[<-,red] (135:0.5) arc(135:215:0.5);
		\draw[<-,red] (215:0.5) arc(215:300:0.5);
		\draw[<-,red] (300:0.5) arc(300:345:0.5);
		
		
		\draw[->,blue] (4:2) arc(4:49:2);
		\draw[->,blue] (49:2) arc(49:125:2);
		\draw[->,blue] (125:2) arc(125:225:2);
		\draw[->,blue] (225:2) arc(225:325:2);
		\draw[blue] (325:2) arc(325:356:2);
		
		\draw[-,purple,postaction={decorate}] (15:0.5)--(4:2);
		\draw[-,green,postaction={decorate}] (356:2)--(345:0.5);
	\end{tikzpicture}
\end{figure}
则\[\begin{split}
	\int_Cz^bF(z)\dd z&=\int_{r}^{R}x^bF(x)\dd x+\int_{C_R}z^bF(z)\dd z+\int_{R}^{r}\left(x\ee^{2\pi\ii}\right)^bF(x)\dd x+\int_{C_r}z^bF(z)\dd z\\&=\left(1-\ee^{2\pi\ii b}\right)\int_{r}^{R}x^bF(x)\dd x+\int_{C_r}z^bF(z)\dd z+\int_{C_R}z^bF(z)\dd z,
\end{split}\]由于\(x\rightarrow 0\)时\(x^{b+1}F(x)\rightarrow 0\), 所以
\[\left|\int_{C_r}z^bF(z)\dd z\right|=\left|\int_{C_r}z^{b+1}F(z)\frac{\dd z}{z}\right|\leqslant \max_{C_r}\left|z^{b+1}F(z)\right|\frac{2\pi r}{r}=2\pi\max_{C_r}\left|z^{b+1}F(z)\right|\rightarrow 0,\quad r\rightarrow 0,\]同理结合\(x\rightarrow+\infty\)时\(x^{b+1}F(x)\rightarrow 0\)可得\[\left|\int_{C_R}z^bF(z)\dd z\right|=\left|\int_{C_R}z^{b+1}F(z)\frac{\dd z}{z}\right|\leqslant \max_{C_R}\left|z^{b+1}F(z)\right|\frac{2\pi R}{R}\rightarrow 0,R\rightarrow+\infty,\]于是有\[\int_C z^bF(z)\dd z=(1-\ee^{2\pi\ii b})\int_{r}^{R}x^bF(x)\dd x=2\pi\ii\sum\mathrm{Res}\left(z^bF(z)\right),\quad (0<\arg z<2\pi),\]令\(r\rightarrow 0^+,R\rightarrow+\infty\)得到\[\int_{0}^{+\infty}x^bF(x)\dd x=\frac{2\pi\ii}{1-\ee^{2\pi\ii b}}\sum\mathrm{Res}\left(z^bF(z)\right),\]又
\[1-\ee^{2\pi\ii b}=\ee^{\pi\ii b}\left(\ee^{-\pi\ii b}-\ee^{\pi\ii b}\right)=(-1)^{b+1}\left(2\ii\sin b\pi\right),\]最后得到\[\int_{0}^{+\infty}x^bF(x)\dd x=-\frac{\pi}{\sin b\pi}\sum\mathrm{Res}\left((-z)^bF(z)\right),\]\(F(z)\)在负实轴上只有\(-1,-2,-3\)三个一阶极点, 易得\[\sum\mathrm{Res}\left((-z)^bF(z)\right)=\frac{1}{2}-2^b+\frac{3^b}{2},\]于是\(\int_{0}^{+\infty}x^bF(x)\dd x=-\frac{\pi}{\sin b\pi}\left(\frac{1}{2}-2^b+\frac{3^b}{2}\right)\), 即\[\int_0^{+\infty}\frac{1}{x^a(x+1)(x+2)(x+3)}\dd x=\frac{\pi}{2\sin a\pi}\left(1-\frac{1}{2^{a-1}}+\frac{1}{3^a}\right).\qedhere\]
\end{solution}
\woe 试计算如下积分:\vspace{8pt}\\
\begin{tabular}{lcl}
\((1)\int_{0}^{1}\frac{\ln (1+x)}{x}\dd x\);&\qquad\qquad\qquad&\((2)\int_{0}^{1}\frac{\ln(1+x^2)}{x}\dd x\);\vspace{0.3cm}\\
\((3)\int_{0}^{1}\frac{\ln(1+x+x^2)}{x}\dd x\);&&\((4)\int_{0}^{1}\frac{\ln (1-x+x^2)}{x}\dd x\).
\end{tabular}
\begin{solution}
(1)我们有\[\int_{0}^{1}\frac{\ln(1+x)}{x}\dd x=\int_{0}^{1}\frac{\ln(1-x^2)}{x}\dd x-\int_{0}^{1}\frac{\ln(1-x)}{x}\dd x=-\frac{1}{2}\int_{0}^{1}\frac{\ln(1-x)}{x}\dd x,\]注意到\[\int_{0}^{1}\frac{\ln(1-x)}{x}\dd x=\lim_{p\rightarrow 0^+}\lim_{q\rightarrow 1}\frac{\partial}{\partial q}\int_{0}^{1}x^{p-1}(1-x)^{q-1}\dd x=\lim_{p\rightarrow 0^+}\lim_{q\rightarrow 1}\frac{\partial}{\partial q}\frac{\Gamma(p)\Gamma(q)}{\Gamma(p+q)},\]而\[\frac{\partial}{\partial q}\frac{\Gamma(p)\Gamma(q)}{\Gamma(p+q)}=\frac{\Gamma'(q)\Gamma(p)\Gamma(p+q)-\Gamma'(p+q)\Gamma(p)\Gamma(q)}{\Gamma^2(p+q)}=\frac{\Gamma(p)\Gamma(q)}{\Gamma(p+q)}\left(\psi(q)-\psi(p+q)\right),\]从而
\[\begin{split}
	&\lim_{p\rightarrow 0^+}\lim_{q\rightarrow 1}\frac{\partial}{\partial q}\frac{\Gamma(p)\Gamma(q)}{\Gamma(p+q)}=\lim_{p\rightarrow 0^+}\lim_{q\rightarrow 1}\frac{\Gamma(p)\Gamma(q)}{\Gamma(p+q)}\left(\psi(q)-\psi(p+q)\right)\\=&\lim_{p\rightarrow 0^+}\lim_{q\rightarrow 1}\frac{1}{p}\left(\psi(q)-\psi(p+q)\right)=-\psi'(1)=-\sum_{n=1}^{\infty}\frac{1}{n^2}=-\frac{\pi^2}{6},
\end{split}\]
于是\(\int_{0}^{1}\frac{\ln(1+x)}{x}\dd x=\frac{\pi^2}{12}.\) 注意在这过程之中我们也得到了\(\int_{0}^{1}\frac{\ln(1-x)}{x}\dd x=-\frac{\pi^2}{6}\), 而她的作用不久就会让我们感觉到.

(2)置\(x^2=t\)并结合上一问的结果\[\int_{0}^{1}\frac{\ln(1+x^2)}{x}\dd x=\int_{0}^{1}\frac{\ln(1+t)}{2t}\dd t=\frac{\pi^2}{24}.\]

(3)注意到\[(1-x^3)=(1-x)(1+x+x^2),\quad (1+x^3)=(1+x)(1-x+x^2),\]这启示我们计算\(\int_{0}^{1}\frac{\ln(1\pm x^3)}{x}\dd x\), 事实上我们有\[\int_{0}^{1}\frac{\ln(1+x^3)}{x}\dd x=\int_{0}^{1}\frac{\ln(1-x^6)}{x}\dd x-\int_{0}^{1}\frac{\ln(1-x^3)}{x}\dd x=-\frac{1}{2}\int_{0}^{1}\frac{\ln(1-x^3)}{x}\dd x,\]又置\(x^3=t\), 发现\[\int_{0}^{1}\frac{\ln(1-x^3)}{x}\dd x=\frac{1}{3}\int_{0}^{1}\frac{\ln(1-t)}{t}\dd t=-\frac{\pi^2}{18},\]这也意味着\(\int_{0}^{1}\frac{\ln(1+x^3)}{x}\dd x=\frac{\pi^2}{36},\)于是有\[\int_{0}^{1}\frac{\ln(1+x+x^2)}{x}\dd x=\int_{0}^{1}\frac{\ln(1-x^3)}{x}\dd x-\int_{0}^{1}\frac{\ln(1-x)}{x}\dd x=\frac{\pi^2}{9}.\]

(4)根据上述分析有\[\int_{0}^{1}\frac{\ln(1-x+x^2)}{x}\dd x=\int_{0}^{1}\frac{\ln(1+x^3)}{x}\dd x-\int_{0}^{1}\frac{\ln(1+x)}{x}\dd x=-\frac{\pi^2}{18}.\qedhere\]
\end{solution}
\woe 设\(n\geqslant 2\), 试用多种方法证明\(\prod_{k=0}^{n-1}\sin(x+\frac{k\pi}{n})=\frac{\sin nx}{2^{n-1}}\).
\begin{proof}
	
\end{proof}
\woe 设\(n\geqslant 2\), 证明: \(\prod_{k=1}^{n-1}\sin\frac{k\pi}{n}=\frac{n}{2^{n-1}}\).
\begin{proof}
我们我们直接使用下一题的结论, 注意到\[\sin\frac{\pi}{n}=\frac{\pi}{\Gamma\left(\frac{1}{n}\right)\Gamma\left(1-\frac{1}{n}\right)},\sin\frac{2\pi}{n}=\frac{\pi}{\Gamma\left(\frac{2}{n}\right)\Gamma\left(1-\frac{2}{n}\right)},\cdots,\sin\frac{(n-1)\pi}{n},\]于是有\[\prod_{k=1}^{n-1}\sin\frac{k\pi}{n}=\frac{\pi^{n-1}}{\left(\prod_{k=1}^{n-1}\Gamma\left(\frac{k}{n}\right)\right)^2}=\frac{n}{2^{n-1}}.\qedhere\]
\end{proof}
\woe 设\(n\geqslant 2\), 证明: \(\prod_{k=1}^{n-1}\Gamma\left(\frac{k}{n}\right)=\frac{1}{\sqrt{n}}(2\pi)^{(n-1)/2}.\)
\begin{proof}
由Gauss叠乘定理: \[\Gamma(s)\Gamma\left(s+\frac{1}{k}\right)\cdots\Gamma\left(s+\frac{k-1}{k}\right)=\left(2\pi\right)^{(k-1)/2}k^{1/2-ks}\Gamma(ks),\]得到\[\prod_{k=1}^{n-1}\Gamma\left(\frac{k}{n}\right)=\lim_{s\rightarrow 0^+}(2\pi)^{(n-1)/2}n^{1/2-ns}\frac{\Gamma(ns)}{\Gamma(s)}=\frac{1}{\sqrt{n}}(2\pi)^{(n-1)/2}.\qedhere\]
\end{proof}
\woe 计算\(\int_{0}^{1}\ln\Gamma(x)\dd x.\)
\begin{solution}
换元, 然后使用余元公式:\[\begin{split}
\int_{0}^{1}\ln\Gamma(x)\dd x&=\int_{0}^{1}\ln\Gamma(1-x)\dd x=\frac{1}{2}\int_{0}^{1}\ln\Gamma(x)\ln\Gamma(1-x)\dd x\\&=\frac{1}{2}\int_{0}^{1}\ln\frac{\pi}{\sin\pi x}\dd x=-\frac{\pi}{2}\int_{0}^{1}\ln\sin\pi x\dd x=\frac{\ln2\pi}{2}\qedhere
\end{split}\]
\end{solution}
\woe 设\(f\in C^1(\mathbb{R})\)以1为周期, \(f(x)+f\left(x+\frac{1}{2}\right)=f(2x)\). 证明: \(f\equiv 0\).
\begin{proof}
\(f'\)为\(\bbr\)上的连续周期函数, 所以它有连续模\[\omega(r)=\sup_{x,y\in\bbr \atop |x-y|\leqslant r}\left|f'(x)-f'(y)\right|,\quad \forall r\geqslant 0,\]\(\omega(r)\)在\([0,+\infty)\)上连续单调增加, 且\(\omega(0)=0\).

由题设有\[f'(x)+f'\left(x+\frac{1}{2}\right)=2f'(2x),\quad\forall x\in\bbr,\]于是对于任何\(\alpha>0\), 有\[\begin{split}
&\omega(2\alpha)=\sup_{x,y\in\bbr \atop |x-y|\leqslant 2\alpha}\left|f'(x)-f'(y)\right|=\sup_{x,y\in\bbr \atop |x-y|\leqslant\alpha}\left|f'(2x)-f'(2y)\right|\\=&\frac{1}{2}\sup_{x,y\in\bbr \atop |x-y|\leqslant \alpha}\left|f'(x)+f'\left(x+\frac{1}{2}\right)-\left(f'(y)+f'\left(y+\frac{1}{2}\right)\right)\right|\leqslant\omega(\alpha),
\end{split}\]反复利用上式得到\[\omega(\alpha)\leqslant\omega\left(\frac{\alpha}{2}\right)\leqslant\cdots\leqslant\omega\left(\frac{\alpha}{2^n}\right),\quad\forall n\geqslant 1,\]令\(n\rightarrow+\infty\), 得到\[\omega(\alpha)=\lim_{n\rightarrow+\infty}\omega\left(\frac{\alpha}{2^n}\right)=\omega(0)=0,\]所以\(f'(x)\)是常数. 因为\(f(x)\)为可微的周期函数, 因为\(f'(x)\)有零点. 于是\(f'(x)\equiv 0\). 所以\(f\)为常数, 易见\(f\left(\frac{1}{2}\right)=0\), 从而\(f\equiv 0\).
\end{proof}
\end{quiza}
\begin{quizb}
\woe 证明: \(\int_{0}^{+\infty}\frac{\ee^{-x}-\cos x}{x}\dd x=0\).
\begin{proof}
记\(I(\alpha)=\int_{0}^{+\infty}\ee^{-\alpha x}\frac{\ee^{-x}-\cos x}{x}\dd x(\alpha\geqslant0)\), 易见其关于参数\(\alpha\)一致收敛, 于是有\[\begin{split}
I'(\alpha)&=\int_{0}^{+\infty}\ee^{-\alpha x}\left(\cos x-\ee^{-x}\right)\dd x=\left.\left(\frac{\ee^{-\alpha x}(\sin x-\alpha\cos x)}{\alpha^2+1}+\frac{\ee^{-(\alpha+1)x}}{\alpha+1}\right)\right|_0^{+\infty}\\&=\frac{\alpha}{1+\alpha^2}-\frac{1}{\alpha+1}.
\end{split}\]
注意到\(\lim_{\alpha\rightarrow+\infty}I(\alpha)=0\), 于是\[\int_{0}^{+\infty}\frac{\ee^{-x}-\cos x}{x}\dd x=I(0)=-\int_{0}^{+\infty}\frac{\alpha}{1+\alpha^2}-\frac{1}{1+\alpha}\dd \alpha=0.\qedhere\]
\end{proof}
\woe 计算\(\int_{0}^{+\infty}\frac{\cos x-\cos x^2}{x}\dd x\).
\begin{solution}
记\(I(a)=\int_{0}^{+\infty}\frac{\exp(-x^a)-\cos(x^a)}{x}\dd x\), 置\(x^a=t\), 利用上一问的结果, 得到\[I(a)=\frac{1}{a}\int_{0}^{+\infty}\frac{\ee^{-t}-\cos t}{t}\dd x\equiv 0,\]于是由\(I(1)=I(2)\)可得\[\int_{0}^{+\infty}\frac{\cos x-\cos x^2}{x}\dd x=\int_{0}^{+\infty}\frac{\exp(-x)-\exp(-x^2)}{x}\dd x,\]我们只需要计算右侧的积分, 即有\[\begin{split}
	&\int_{0}^{+\infty}\frac{\exp(-x)-\exp(-x^2)}{x}\dd x=\int_{0}^{+\infty}\left(\exp(-x)-\exp(-x^2)\right)\dd\left(\ln x\right)\\=&\int_{0}^{+\infty}\ee^{-x}\ln x\dd x-2\int_{0}^{+\infty}x\exp(-x^2)\ln x\dd x=\frac{1}{2}\int_{0}^{+\infty}\ee^{-x}\ln x\dd x=\frac{1}{2}\Gamma'(1)=-\frac{\gamma}{2}.
\end{split}\]从而\(\int_{0}^{+\infty}\frac{\cos x-\cos x^2}{x}\dd x=-\frac{\gamma}{2}.\)
\end{solution}
\woe 计算\(\int_{-\infty}^{+\infty}\sin t^2\dd t\).
\begin{solution}
换元, 使用累次积分并运用已知结果.
\[\begin{split}
\int_{-\infty}^{+\infty}\sin t^2\dd t&=2\int_{0}^{+\infty}\sin t^2\dd t\xlongequal{t^2=x}\int_{0}^{+\infty}\frac{\sin x}{\sqrt{x}}\dd x=2\int_{0}^{+\infty}\sin x\int_{0}^{+\infty}\frac{\ee^{-xt^2}}{\sqrt{\pi}}\dd t\dd x\\&=\frac{2}{\sqrt{\pi}}\int_{0}^{+\infty}\int_{0}^{+\infty}\ee^{-xt^2}\sin x\dd x\dd t=\frac{2}{\sqrt{\pi}}\int_{0}^{+\infty}\frac{1}{1+t^4}\dd t=\sqrt{\frac{\pi}{2}}.\qedhere
\end{split}\]
\end{solution}
\woe 计算\(\lim_{x\rightarrow 0^+}\sum_{n=1}^{\infty}\frac{nx^2}{1+n^3x^3}\).
\begin{solution}
我们证明这样一个有趣的命题:设函数\(g\in[0,+\infty)\)单调递减, 并且Riemann积分\(\int_{0}^{+\infty}g(x)\dd x\)收敛. 则对于任何满足条件\(|f(x)|\leqslant g(x)\)的函数\(f\in[0,\infty)\)有\(\lim_{x\rightarrow0^+}x\sum_{n=1}^{\infty}f(nx)=\int_{0}^{+\infty}f(x)\dd x\).

如果我们证明了这个命题, 那么对于本问, 取\(f(x)=\frac{x}{1+x^3}\), 就有\[\lim_{x\rightarrow 0^+}x\sum_{n=1}^{\infty}\frac{nx}{1+n^3x^3}=\lim_{x\rightarrow0^+}x\sum_{n=1}^{\infty}f(nx)=\int_{0}^{+\infty}\frac{x}{1+x^3}\dd x=\frac{2\pi}{3\sqrt{3}}.\qedhere\]

\end{solution}
下面我们将证明这个命题.
\begin{proof}
考察\(M_n:=h\sum_{n=1}^{\infty}f(nh)-\int_{0}^{+\infty}f(x)\dd x\), 设\(N,L\)是某些正整数, 我们将\(M_n\)表示为\[\begin{split}
M_n&=h\sum_{n=1}^{N}f(nh)+h\sum_{n=N+1}^{\infty}f(nh)-\int_{0}^{L}f(x)\dd x-\int_{L}^{+\infty}f(x)\dd x\\&=\left(h-\frac{L}{N}\right)\sum_{n=1}^{N}f(nh)+h\sum_{n=N+1}^{\infty}f(nh)+\left(\frac{1}{N}\sum_{n=1}^{N}f(nh)-\int_{0}^{L}f(x)\dd x\right)-\int_{L}^{+\infty}f(x)\dd x.
\end{split}\]
\begin{asparaenum}[\bfseries (i)]
\item 因为\(g(x)\)在\([0,+\infty)\)上Riemann可积, 所以对于任何给定的\(\varepsilon>0\), 存在最小的正整数\(L=L(\varepsilon)\)使得\[\int_{L-1}^{+\infty}g(x)\dd x<\frac{\varepsilon}{4}.\]下文中固定此\(L\).
\item 对于任何给定的\(h\in(0,1)\), 可取正整数\(N\)满足\(Nh\leqslant L\leqslant (N+1)h\), 于是\[nh\in\left(\frac{Ln}{N+1},\frac{Ln}{N}\right]\subset\left(\frac{n-1}{N}L,\frac{n}{N}L\right],\quad 1\leqslant n\leqslant N,\]由题设可知, 积分\(\int_{0}^{L}f(x)\dd x\)存在, 所以\[\frac{L}{N}\sum_{n=1}^{N}f(nh)\rightarrow\int_{0}^{L}f(x)\dd x,\quad N\rightarrow+\infty,\]注意由\(N\)的取法可知\(\frac{L}{N}\in\left[h,\frac{N+1}{N}h\right)\), 所以当\(h\rightarrow0^+\)时\(N\rightarrow+\infty\), 于是当\(0<h<h_0\)(其中\(h_0=h_0(\varepsilon)\))时, 有\[\left|\frac{L}{N}\sum_{n=1}^{N}f(nh)-\int_{0}^{L}f(x)\dd x\right|<\frac{\varepsilon}{4}.\]
\item 因为\(\frac{L}{N}\in\left[h,\frac{N+1}{N}h\right)\), 所以\[\left|h-\frac{L}{N}\right|\leqslant\frac{N+1}{N}h-h=\frac{h}{N},\]并且由\(|f(x)|<g(x)\), 我们有\[\left|\left(h-\frac{L}{N}\right)\sum_{n=1}^{N}f(nh)\right|\leqslant\left|h-\frac{L}{N}\right|\sum_{n=1}^{N}g(nh)\leqslant\frac{h}{N}\sum_{n=1}^{N}g(nh).\]因为\(g(x)\)在\([0,+\infty)\)上单调递减, 所以\[h\sum_{n=1}^{N}g(nh)<\int_{0}^{Nh}g(x)\dd x<\int_{0}^{+\infty}g(x)\dd x,\]并且因为\(\int_{0}^{+\infty}g(x)\dd x\)收敛, 所以当\(0<h<h_0\)(其中\(h_0\)足够小)时也有\(\frac{1}{N}\int_{0}^{+\infty}g(x)\dd x<\frac{\varepsilon}{4}.\)于是由前式得知, 当\(0<h<h_0\)时\[\left|\left(h-\frac{L}{N}\right)\sum_{n=1}^{N}f(nh)\right|<\frac{\varepsilon}{4}.\]
\item 另外, 依步骤(i), 我们有\[\left|h\sum_{n=N+1}^{\infty}f(nh)\right|\leqslant h\sum_{n=N+1}^{\infty}g(nh)\leqslant\int_{L-1}^{+\infty}g(x)\dd x<\frac{\varepsilon}{4}.\]
\item 最后, 综上述诸估计, 我们得到对于任何\(\varepsilon>0\), 当\(0<h<h_0\)时\(|M_n|<\varepsilon.\) 因此结论得证.\qedhere
\end{asparaenum}
\end{proof}
\woe 计算\(\int_{0}^{+\infty}\frac{\ln x}{x^2-1}\dd x\).
\begin{solution}
易见
\[\begin{split}
\int_{0}^{+\infty}\frac{\ln x}{x^2-1}\dd x&=2\int_{0}^{1}\frac{\ln x}{x^2-1}\dd x=-2\int_{0}^{1}\sum_{n=1}^{\infty}x^{2n}\ln x\dd x\\&=-2\sum_{n=1}^{\infty}\int_{0}^{1}x^{2n}\ln x\dd x=2\sum_{n=1}^{\infty}\frac{1}{(2n+1)^2}=\frac{\pi^2}{4}.\qedhere
\end{split}\]
\end{solution}
\woe 设\(p,q>1\)为对偶数, \(\sum_{n=1}^{\infty}|a_n|^p\)和\(\sum_{n=1}^{\infty}|b_n|^p\)收敛. 证明如下离散的Hard-Hilbert不等式:\[\sum_{m=1}^{\infty}\sum_{n=1}^{\infty}\frac{a_mb_n}{m+n}\leqslant\frac{\pi}{\sin(\pi/p)}\left(\sum_{n=1}^{\infty}|a_n|^p \right)^{1/p}\left(\sum_{n=1}^{\infty}|b_n|^p\right)^{1/q}.\]
\begin{proof}
	
\end{proof}
\woe 计算\(\psi\)在\(\frac{1}{2},\frac{1}{3},\frac{2}{3},\frac{1}{4},\frac{3}{4},\frac{1}{6},\frac{5}{6}\)等点的值.
\begin{solution}
我们将利用以下两个公式\begin{gather}
\sum_{j=0}^{k}\psi\left(x+\frac{j}{k}\right)=k\left(\psi(kx)-\ln k\right),\quad\forall x>0.\tag{$\heartsuit$}\label{c8psi1}\\
\psi(x)-\psi(1-x)=-\pi\cot\pi x,\quad\forall x\in(0,1).\tag{$\spadesuit$}\label{c8psi2}
\end{gather}
在式\eqref{c8psi1}中令\(k=2\)得到\[\psi(x)+\psi\left(x+\frac{1}{2}\right)=2\left(\varphi(2x)-\ln 2\right),\]令\(x=\frac{1}{2}\), 即有\(\psi\left(\frac{1}{2}\right)+\psi\left(1\right)=2\left(\psi(1)-\ln 2\right)\), 由此解得\(\psi\left(\frac{1}{2}\right)=\psi(1)-2\ln 2=-\gamma-2\ln 2.\)

令\(k=3\),在分别令\(x=\frac{1}{6},\frac{1}{3}\),有\[\begin{split}
\psi\left(\frac{1}{6}\right)+\psi\left(\frac{5}{6}\right)&=2\psi\left(\frac{1}{2}\right)-3\ln 3,\\
\psi\left(\frac{1}{3}\right)+\psi\left(\frac{2}{3}\right)&=2\psi\left(1\right)-3\ln 3,\\
\end{split}\]结合\eqref{c8psi2}式, 即有\[\begin{split}
\psi\left(\frac{1}{6}\right)-\psi\left(\frac{5}{6}\right)&=-\sqrt{3}\pi,\\
\psi\left(\frac{1}{3}\right)-\psi\left(\frac{2}{3}\right)&=-\frac{\pi}{\sqrt{3}},
\end{split}\]联立解得\[
\begin{split}
\psi\left(\frac{1}{3}\right)&=-\gamma-\frac{3}{2}\ln 3-\frac{\pi}{2\sqrt{3}},\quad \psi\left(\frac{2}{3}\right)=-\gamma-\frac{3}{2}\ln 3+\frac{\pi}{2\sqrt{3}},\\
\psi\left(\frac{1}{6}\right)&=-\gamma-2\ln2-\frac{3}{2}\ln 3-\frac{\sqrt{3}}{2}\pi,\psi\left(\frac{5}{6}\right)=-\gamma-2\ln2-\frac{3}{2}\ln 3+\frac{\sqrt{3}}{2}\pi,
\end{split}\]
随即令\(k=4\), \(x=\frac{1}{4}\), 结合\eqref{c8psi2}式, 有\[\psi\left(\frac{1}{4}\right)+\psi\left(\frac{3}{4}\right)=3\psi(1)-\psi\left(\frac{1}{2}\right)-8\ln 2,
\psi\left(\frac{1}{4}\right)-\psi\left(\frac{3}{4}\right)=-\pi,\]解得\[\psi\left(\frac{1}{4}\right)=-\gamma-3\ln 2-\frac{\pi}{2},\psi\left(\frac{3}{4}\right)=-\gamma-3\ln 2+\frac{\pi}{2}.\qedhere\]
\end{solution}
\woe 证明\(\Gamma\)函数在定义域内解析.
\begin{proof}
	
\end{proof}
\woe 证明例9.4.7中的结论.
\end{quizb}

\section{变分法初步}
\precis{最优解的必要条件,Euler-Lagrange方程,特殊情形Euler-Lagrange方程的求解,捷线问题,最优解的充要条件,存在性问题简介,Poincar\'{e}不等式,弱收敛,强收敛,Clarkson不等式,凸集分离定理,Mazur定理,Riesz表示定理}

\begin{quiza}
\woe 证明: 对于任何\(a,b>0\), 存在唯一的\(r>0\)以及\(\theta\in\left(0,2\pi\right]\)使得\(a=r\left(\theta-\sin\theta\right),b=r\left(1-\cos \theta\right)\).
\begin{proof}
	
\end{proof}
\woe 证明(10.5.28)式.
\woe 设\(f\in C[a,b]\). 若存在\(h_1,h_2\in L^1[a,b]\)使得\[f(x)=f(a)+\int_{a}^{x}h_j(t)\dd t,\forall x\in[a,b],j=1,2.\]证明: \(\int_{a}^{b}\left|h_1(x)-h_2(x)\right|\dd x=0.\)
\woe 设\(1<q<+\infty,E\subseteq\mathbb{R}^n\)为测度非零的可测集, \(G\)由(10.5.50)式给出. 证明: \(\forall g\in L^q(E)\), 存在\(g_{\varepsilon}\in G\)使得\(\left\|g_{\varepsilon}-g\right\|_{L^q(E)}<\varepsilon\).
\end{quiza}
\begin{quizb}
\woe 举例说明, 对于\(\boldsymbol{A}\in C([a,b];\mathbb{R}^{(2m)\times(2m)}),\boldsymbol{A}^\top(x)=\boldsymbol{A}(\forall x\in [a,b])\), 条件\[\int_{a}^{b}\left(\begin{matrix}
\boldsymbol{\varphi}(x)\\\boldsymbol{\varphi}'(x)
\end{matrix}\right)^\top\boldsymbol{A}^\top\left(\begin{matrix}
\boldsymbol{\varphi}(x)\\\boldsymbol{\varphi}'(x)
\end{matrix}\right)\dd x\geqslant 0,\qquad\forall\boldsymbol{\varphi}\in C_0^1([a,b];\mathbb{R}^m)\]并不蕴涵\[\boldsymbol{A}\geqslant 0,\qquad\forall x\in [a,b].\]
\end{quizb}