\chapter{序列极限}
\section{数列极限}
\precis{数列极限,无穷级数,无穷乘积,数列极限的性质,夹逼准则}
\begin{quiza}
\woe  用定义证明\(\lim_{n\rightarrow +\infty}\frac{3n^3+4n^2-100}{2n^3-9n-11}=\frac{3}{2}.\)
\begin{proof}
\(\forall \varepsilon >0\), 要使得\[\left|\frac{3n^3+4n^2-100}{2n^3-9n-11}-\frac{3}{2}\right|<\varepsilon,\]即\[\left|\frac{4n^2+\frac{27}{2}n-\frac{167}{2}}{2n^3-9n-11}\right|<\varepsilon.\]当\(n>100\)时\[\left|\frac{4n^2+\frac{27}{2}n-\frac{167}{2}}{2n^3-9n-11}\right|<\left|\frac{4n^2+n^2}{2n^3-n^3}\right|=\left|\frac{5}{n}\right|<\varepsilon,\]于是\(\lim_{n\rightarrow +\infty}\frac{3n^3+4n^2-100}{2n^3-9n-11}=\frac{3}{2}.\)
\end{proof}
\woe 用定义证明\(\lim_{n\rightarrow +\infty}\lg \left(1+\frac{1}{n}\right)=0.\)
\begin{proof}
注意到\(x>0\)时\(\lg(1+x)<x\), 于是由\(\forall \varepsilon>0\), 取\(N=\left[\frac{1}{\varepsilon}\right]\), 则\(n>N\)时有\[\left|\lg\left(1+\frac{1}{n}\right)\right|<\frac{1}{n}<\varepsilon.\]从而\(\lim_{n\rightarrow +\infty}\lg \left(1+\frac{1}{n}\right)=0.\)
\end{proof}
\end{quiza}
\begin{quizb}
\woe 证明:\(\lim_{n\rightarrow +\infty}a_n=0\)的充要条件是存在单调下降且无正下确界的正数列\(\{w_n\}\)使得\(\forall n\geqslant 1\)成立\(\left|a_n\right|\leqslant w_n\).
\woe 设\(0\leqslant p\leqslant k-1\), 求\( \lim_{n\rightarrow +\infty}\frac{C_{kn}^{p}+C_{kn}^{p+k}+\cdots+C_{kn}^{p+(n-1)k}}{2^{kn}}\).
\begin{solution}
假设有这样一套装置, 用\(kn\)个开关共同控制\(k\)个灯泡: 开关或开或闭, 灯泡按\(0,1,\cdots,k-1\)的编号来编号. 对于编号为\(p(p=0,1,\cdots,k-1)\)的灯泡来说, 当且仅当闭合开关总数模\(k\)余\(p\)时亮起. 这编号\(p\)的灯泡亮起的概率即为\(\frac{\displaystyle\sum_{i=0}^{n-1}C_{kn}^{p+ik}}{2^{kn}}\).

很显然, 这\(k\)个灯泡中有且仅有一个灯泡亮起. 当\(n\rightarrow\infty\)时, 每个灯泡都有相同的机会亮起, 编号为\(p\)的灯泡亮起的概率也为\(1/k\). 即得\[\lim_{n\rightarrow +\infty}\frac{C_{kn}^{p}+C_{kn}^{p+k}+\cdots+C_{kn}^{p+(n-1)k}}{2^{kn}}=\frac{1}{k}.\]这道题还有其他解法见技巧书.
\end{solution}
\end{quizb}
\section{无穷大量, 无穷小量, Stolz 公式}
\precis{无穷大量,无穷小量,小\(o\),大\(O\),等价,同阶,不定型,\(Stolz-Ces\acute{a}ro\)定理}
\begin{theorem}{Stolz-Ces\'{a}ro定理}{C20}
  设\(\{x_n\}\)和\(\{y_n\}\)是两个实数列, 若\begin{compactenum}[(i)]
    \item \(\{y_n\}\)严格单增,
    \item \( \lim_{n\rightarrow\infty}y_n=+\infty\),
    \item \( \lim_{n\rightarrow+\infty}\frac{x_{n+1}-x_n}{y_{n+1}-y_n}=\ell\), 其中\(\ell \)可能为有限数, \(+\infty\)或\(-\infty\),   
  \end{compactenum}则\[\lim_{n\rightarrow+\infty}\frac{x_n}{y_n}=\ell.\]  
\end{theorem}
\begin{theorem}{}{C21}
  设\(\{x_n\}\)和\(\{y_n\}\)是两个实数列, 若\begin{compactenum}[(i)]
    \item \(\{y_n\}\)严格单调减少,
    \item \( \lim_{n\rightarrow+\infty}x_n=\lim_{n\rightarrow+\infty}y_n=0\),
    \item \( \lim_{n\rightarrow+\infty}\frac{x_n-x_{n+1}}{y_{n}-y_{n+1}}=\ell\), 其中\(\ell \)可能为有限数, \(+\infty\)或\(-\infty\),   
  \end{compactenum}则\[\lim_{n\rightarrow+\infty}\frac{x_n}{y_n}=\ell.\]  
\end{theorem}
\begin{quiza}
\woe 证明: \(\{a_n\}\)的极限为\(L\)当且仅当\(\{a_{2n}\}\)与\(\{a_{2n+1}\}\)的极限\(L\), 其中\(L\)为有限数, \(+\infty\), \(-\infty\)或\(\infty\).
\begin{proof}
必要性. 由于\(\lim_{n\rightarrow\infty}a_n=L\). 若\(|L|<+\infty\). 则对\(\forall\varepsilon>0\), 有\(N>0\)使得\(n>N\)时有\[|a_n-L|<\varepsilon,\]特别地, \(2n>N\)时也有\[|a_{2n}-L|<\varepsilon,\quad|a_{2n+1}-L|<\varepsilon.\]若\(L=+\infty\), 则对\(\forall A>0\), 有\(N>0\)使得\(n>N\)时有\(a_n>A\), 特别地, 当\(2n>N\)时也有\(a_n>A\), 其余情况类似. 

充分性. 先说明\(L\)有限的情况, 由于\(\lim_{n\rightarrow+\infty}a_{2n}=L\), 则对\(\forall\varepsilon>0\), 有\(N_1\in\bbz_+\)使得\(n>N_1\)时有\[|a_{2n}-L|<\varepsilon,\]同理由\(\lim_{n\rightarrow+\infty}a_{2k-1}=L\), 有\(N_2\in\bbz_+\)使得\(n>N_2\)时有\[|a_{2n+1}-L|<\varepsilon,\]取\(N=(2N_1)\vee(2N_2+1)\), 那么当\(n>N\)时, 都有\(|a_n-L|<\varepsilon\). \(L\)无穷时情况类似.
\end{proof}
\woe 设\(p\)为有理数, 证明\(\lim_{n\rightarrow+\infty}\frac{(n+1)^p-n^p}{n^{p-1}}=p\).
\begin{proof}
\(\lim_{n\rightarrow+\infty}\frac{(n+1)^p-n^p}{n^{p-1}}=\lim_{n\rightarrow+\infty}n\left(\left(1+\frac{1}{n}\right)^p-1\right)=p.\)\qedhere
\end{proof}
\woe 证明定理\reff{Th:C21}.
\begin{proof}
当\(\ell\in\mathbb{R}\)时, 由条件知\(\forall\varepsilon>0\), 均存在\(N_1\in\mathbb{N}_+\)使得\(n>N_1\)时有\[\ell-\varepsilon<\frac{x_{n}-x_{n+1}}{y_n-y_{n+1}}<\ell+\varepsilon,\]注意到\(\{y_n\}\)单调递减, 故有\[(\ell-\varepsilon)(y_{n-1}-y_n)<x_{n-1}-x_{n}<(\ell+\varepsilon)(y_{n-1}-y_n),\]取\(m>n>N_1\), 并将上面的不等式中的\(n\)依次换成\(n+1,n+2,\cdots,m\), 然后把这些不等式相加得到\[(\ell-\varepsilon)(y_n-y_m)<x_n-x_m<(\ell+\varepsilon)(y_n-y_m),\]也即\[\left|\frac{x_n-x_m}{y_n-y_m}-\ell\right|<\varepsilon,\quad\forall m>n>N_1,\]现令\(m\rightarrow\infty\), 则由\(\lim_{m\rightarrow\infty}x_m=\lim_{m\rightarrow\infty}y_m=0\)得到\[\left|\frac{x_n}{y_n}-\ell\right|\leqslant \varepsilon,\qquad\forall n>N_1,\]这说明\(\lim_{n\rightarrow\infty}\frac{x_n}{y_n}=\ell.\)

若\(\ell=+\infty\), 则当\(n\)充分大时有\(\frac{x_{n}-x_{n-1}}{y_{n}-t_{n-1}}>0\), 故此时\(\{x_n\}\)严格单调递减趋于0. 另外, 我们有\[\textcolor{red}{\lim_{n\rightarrow\infty}\frac{y_n-y_{n-1}}{x_{n}-x_{n-1}}=0,}\]从而由上一段讨论知\(\lim_{n\rightarrow\infty}\frac{y_n}{x_n}=0.\) 因此\(\lim_{n\rightarrow\infty}\frac{x_n}{y_n}=+\infty.\)

若\(\ell=-\infty\), 我们记\(z_n=-x_n\), 于是\(\lim_{n\rightarrow\infty}z_n=0\)且\[\lim_{n\rightarrow\infty}\frac{z_n-z_{n-1}}{y_n-y_{n-1}}=+\infty,\]由上一段的讨论知\(\lim_{n\rightarrow\infty}\frac{z_n}{y_n}=+\infty\), 从而\(\lim_{n\rightarrow\infty}\frac{x_n}{y_n}=-\infty.\)
\end{proof}
\woe 设\(x_1=a,\,x_2=b,\,x_{n+2}=\frac{1}{2}(x_{n+1}+x_n)\,(n=1,2,3,\cdots)\). 试证明\(\{x_n\}\)收敛, 并求其极限.
\begin{solution}
递推数列有特征方程\[\lambda^2=\frac{1}{2}(\lambda+1),\quad\text{即}\quad 2\lambda^2-\lambda-1=0,\]解得\(\lambda_1=1,\lambda_2=-\frac{1}{2}\). 即设原数列有通项公式\(x_n=C_1+C_2\lambda_2^n\). 代入\(x_1=a,x_2=b\)解得\(C_1=\frac{2b-a}{3},C_2=\frac{4b-4a}{3}.\) 即\(x_n=\frac{2b-a}{3}+\frac{4b-4a}{3}\cdot\left(-\frac{1}{2}\right)^n.\) 易见\(\{x_n\}\)收敛, 极限为\(\frac{2b-a}{3}\).
\end{solution}
\woe 设\(\{x_n\}\)满足\(\lim_{n\rightarrow+\infty}x_{n}\sum_{k=1}^{n}x_k^2=1\), 证明\(\lim_{n\rightarrow+\infty}\sqrt[3]{3n}x_{n}=1\).
\begin{proof}
设\(S_n=\sum_{k=1}^{n}x_k^2\), 显然\(\{S_n\}\)单调递增. 并且\(\frac{1}{x_n}\sim S_n(n\rightarrow+\infty)\), 若\(\lim_{n\rightarrow+\infty}S_n=S\)有限, 则有\(\lim_{n\rightarrow+\infty}x_n=1/S\ne 0\), 这与\(S_n\)收敛矛盾, 从而\[\lim_{n\rightarrow+\infty}S_n=+\infty,\quad\lim_{n\rightarrow+\infty}x_n=0.\]
结合Stolz定理, 我们有\[\lim_{n\rightarrow\infty}\frac{1}{(3n)x_n^3}=\lim_{n\rightarrow\infty}\frac{S_n^3}{3n}=\lim_{n\rightarrow\infty}\frac{S_n^3-S_{n-1}^3}{3},\]于是我们仅需证明\(\lim_{n\rightarrow\infty}\left(S_n^3-S_{n-1}^3\right)=3\). 注意到\[\begin{split}
&S_n^3-S_{n-1}^3=(S_n-S_{n-1})\left(S_n^2+S_nS_{n-1}+S_{n-1}^2\right)\\=&x_n^2\left(S_n^2+S_n\left(S_n-x_n^2\right)+\left(S_n-x_n^2\right)^2\right)=3\left(x_nS_n\right)^2-3x_n^4S_n+x_n^6\\=&3\left(x_n\sum_{k=1}^{n}x_k^2\right)^2-3a_n^2\left(a_n\sum_{k=1}^{n}a_k^2\right)+x_n^6\rightarrow3\quad (n\rightarrow +\infty).\qedhere
\end{split}\]
\end{proof}
\woe 设\(a\)和\(d\)是给定的正数. 对于\(n=1,2,3,\cdots\), 由等差数列\(a,\,a+d,\,\cdots,a+(n-1)d\) 形成的算数平均数\(A_n\)和几何平均数\(G_n\), 试求\(\lim_{n\rightarrow+\infty}\frac{G_n}{A_n}\).
\begin{solution}
由题设\(G_n=\sqrt[n]{\prod_{k=0}^{n-1}\left(a+kd\right)},\,A_n=\frac{\displaystyle\sum_{k=0}^{n-1}\left(a+kd\right)}{n}=a+\frac{n-1}{2}\), 记\(t=\frac{a}{d}\). 于是\[\begin{split}
&\lim_{n\rightarrow+\infty}\frac{G_n}{A_n}=\lim_{n\rightarrow+\infty}\frac{\sqrt[n]{\displaystyle\prod_{k=0}^{n-1}\left(a+kd\right)}}{(n-1)d/2}\cdot\frac{(n-1)d/2}{a+(n-1)d/2}=\lim_{n\rightarrow+\infty}\sqrt[n]{\prod_{k=0}^{n-1}(t+k)}\cdot\frac{2}{n-1}\\=&\lim_{n\rightarrow+\infty}\sqrt[n]{\frac{\Gamma(t+n)}{\Gamma(t)}}\cdot\frac{2}{n-1}=\lim_{n\rightarrow+\infty}\sqrt[n]{\frac{1}{\Gamma(t)}}\sqrt[n]{\left(\frac{t+n}{\ee}\right)^n\sqrt{2\pi(t+n)}}\cdot\frac{2}{n-1}\\=&\lim_{n\rightarrow+\infty}\frac{t+n}{\ee}\frac{2}{n+1}=\frac{2}{\ee}.\qedhere
\end{split}\]
\end{solution}
\woe 令\( a_{n,k}=\frac{k}{n-k+1}(n\geqslant k\geqslant 1)\). 证明:
\begin{quizcs}
\item 对于固定的\(k\geqslant 1\), \(\{a_{n,k}\}\)是无穷小.
\item \( A_n=\prod_{k=1}^{n}a_{n,k}\)不是无穷小.
\end{quizcs}
\begin{proof}
易见\[\lim_{n\rightarrow+\infty}a_{n,k}=\lim_{n\rightarrow+\infty}\frac{k}{n-k+1}=0,\]并且\[\lim_{n\rightarrow+\infty}A_n=\lim_{n\rightarrow+\infty}\prod_{k=1}^{n}a_{n,k}=1,\]从而结论得证.
\end{proof}
\woe 设\(E\subseteq\mathbb{R}\)为非空集. 证明: 存在\(E\)中点列\(\{x_k\}\)使得\(\lim_{k\rightarrow+\infty}x_k=\sup E\), 这里\(\sup E\)可以是\(+\infty\).\\\textbf{注: }对于\(f:D\rightarrow\mathbb{R}\), 满足\(\lim_{k\rightarrow+\infty}f(x_k)=\underset{x\in D}{\sup}f(x)\left(\underset{x\in D}{\inf}f(x)\right)\)的\(D\)中的点列\(\{x_k\}\)称为\(f\)在\(D\)上的\textbf{极大化序列(极小化序列)}.
\begin{proof}
若\(\sup E=+\infty\), 则对于任意给定的正数\(M\), 都有\(x_1\in E, x_1\geqslant M\); 同理对于\(M+1\), 也有\(x_2\in E, x_2\geqslant M+1\), 依次类推, 此时有\(\lim_{k\rightarrow+\infty}x_k=+\infty=\sup E\).

否则\(\sup E<+\infty\), 任取\(E\)的一个上界\(M\)与\(x_0\in E\), 令\(E_1=[x_0,M]\), \(T=(x_0+M)/2\), 这样区间\(E_1\)被\(T\)二分, 优先考察右侧的集合\([T,M]\), 若\([T,M]\cap E\ne\varnothing\), 则令\(E_2=[T,M]\), 否则令\(E_2=[x_0,T]\), 继续将\(E_2\)二分, 并仿上法选择\(E_3\), 以此类推. 易见\(\{E_k\}\)形成一列闭区间套, 设\(x_k\in E_k\), 则\(\lim_{k\rightarrow+\infty}x_k=\sup E\).
\end{proof}
\end{quiza}
\begin{quizb}
\woe \textbf{(Toeplitz(特普利茨)定理)} 设有无穷下三角矩阵\((t_{mn})_{n\geqslant m}\):
    \[\left(\begin{matrix}
      t_{11}& & & \\
      t_{21}&t_{22}&&\\
      t_{31}&t_{32}&t_{33}&\\
      \vdots&\vdots&\vdots&\ddots\\
    \end{matrix}\right)\]满足下列条件
\begin{compactenum}[(i)]
\item 每一列元素趋于零, 即\(\lim_{n\rightarrow+\infty}t_{nm}=0\).
\item 各行元素的绝对值之和有界, 即\(\forall n\in \mathbb{Z}_+\), \[\left|t_{n1}\right|+\left|t_{n2}\right|+\cdots+\left|t_{nn}\right|\leqslant K<+\infty.\]  
\end{compactenum}
记\(y_n=t_{n1}x_1+t_{n2}x_2+\cdots+t_{nn}x_n\).证明:\begin{quizs}
\item 若\(\lim_{n\rightarrow+\infty}x_{n}=0\), 则\(\lim_{n\rightarrow+\infty}y_{n}=0\).
\item 记\(T_{n}=t_{n1}+t_{n2}+\cdots+t_{nn}\). 若\(\lim_{n\rightarrow+\infty}T_{n}=1\), 且\(\lim_{n\rightarrow+\infty}x_{n}=a\)有限, 则\(\lim_{n\rightarrow+\infty}y_{n}=a\).
\begin{proof}
(1)由\(\lim_{n\rightarrow+\infty}x_{n}=0\)知存在\(M>0\)使得\[|x_n|<M,\qquad n=1,2,\cdots,\]并且对\(\forall\varepsilon>0,\exists N\in\mathbb{N}_+\), 使得\(n>N\)时, 有\[|x_n|<\frac{\varepsilon}{2K}.\]

对于上述的\(\varepsilon\)与\(N\), 由\(\lim_{n\rightarrow\infty}t_{nm}=0\)可知\(\exists N_1\in\mathbb{N}_+\), 使得\(N_1>N\), 且当\(n>N_1\)时有\[|t_{nm}|<\frac{\varepsilon}{2NM},\qquad m=1,2,\cdots,N,\]从而当\(n>N_1\)时, 由\(y_n\)的定义得\[\begin{split}
|y_n|&=|t_{n1}x_1+t_{n2}x_2+\cdots+t_{nn}x_n|\leqslant|t_{n1}x_1|+|t_{n2}x_2|+\cdots+|t_{nn}x_n|\\&\leqslant\frac{\varepsilon}{2NM}\cdot NM+\left(|t_{n(N+1)}|+|t_{n(N+2)}|+\cdots+|t_{nn}|\right)\frac{\varepsilon}{2K}\\&<\frac{\varepsilon}{2}+\frac{\varepsilon}{2}=\varepsilon.
\end{split}\]即\(\lim_{n\rightarrow\infty}y_n=0.\)

(2)由题设知\(\lim_{n\rightarrow\infty}\left(x_n-a\right)=0\), 而\[
z_n:=t_{n1}(x_1-a)+t_{n2}(x_2-a)+\cdots+t_{nn}(x_n-a)=y_n-aT_n,\]由前面的结论知\(\lim_{n\rightarrow n\rightarrow}z_n=0\), 于是\(\lim_{n\rightarrow\infty}y_n=\lim_{n\rightarrow\infty}z_n+a\lim_{n\rightarrow\infty}T_n=a.\)
\end{proof}
\end{quizs}
\woe 试考察Stolz定理和Toeplitz定理的关系.
\tcbline
由Toeplitz定理可以推导出Stolz定理\reff{Th:C20}, 理由如下.

记号如同问题 \(\boldsymbol{1}\). 令上述的无穷下三角矩阵的元素为\(t_{nm}=\frac{y_{m+1}-y_{m}}{y_{m+1}}\), 容易验证其满足定理条件, 又记\(a_n=\frac{x_{n+1}-x_n}{y_{n+1}-y_n}\), 于是\[b_n:=t_{n1}a_1+t_{n2}a_2+\cdots+t_{nn}a_n=\frac{x_{n+1}-x_1}{y_{n+1}}.\]由Toeplitz定理知\(\lim_{n\rightarrow\infty}b_n=\ell\), 易见\(\lim_{n\rightarrow\infty}\frac{x_n}{y_n}=\ell.\)
%\tcblower
%\textcolor{red}{
%\woe 设\(a_1\in(0,1)\), 且有\[a_{n+1}=a_n+\left(\frac{a_n}{n}\right)^k,\quad n=1,2,\cdots,\]其中\(k>1\). 证明: \(\{a_n\}\)有界.}
\end{quizb}
\section{Euclid空间中的基本概念}
\precis{线性空间,赋范线性空间,度量空间,Eucild距离,平行四边形法则,范数,内点,邻域,外点,边界点,聚点,开集,闭集,闭包,稠密,无处稠密/疏朗集,内积空间}
\begin{definition}{}{C22}
  设\(X\)为非空集, 映射\(d:X\times X\rightarrow\mathbb{R}\)称为\(X\)上的\textbf{度量}, 又称\textbf{距离}, 如果对任何\(x,y,z\in X\)成立:\begin{compactenum}[(i)]
    \item \textbf{非负性.} \(d(x,y)\geqslant 0\), 且等号当且仅当在\(x=y\)时成立.
    \item \textbf{对称性.} \(d(x,y)=d(y,x)\).
    \item \textbf{三角不等式.} \(d(x,z)\leqslant d(x,y)+d(y,z)\).
  \end{compactenum}
  此时, 称\((X,d)\)(或简称\(X\))为\textbf{度量空间}或\textbf{距离空间}, 称\(d(x,y)\)为\(x\)与\(y\)间的距离.
\end{definition}
\begin{quiza}
\woestar 设\(E\subseteq \mathbb{R}\)为非空有界闭集. 证明\(E\)有最大值最小值, 即\(\sup E\in E,\,\inf E\in E\).
\begin{proof}

\end{proof}
\woe \textbf{(距离的等价定义)}. 在定义\reff{Def:C22}中, 由(ii)----(iii)可以得到: (iii\('\)) 对任何\(x,y\in X\), 成立\(d(x,z)\leqslant d(y,x)+d(y,z).\)证明: (i)--(iii)与(i), (iii\('\))等价.
\woe 设\(E\)为\(\mathbb{R}\)中的无理数全体. 证明\(\overline{E}=\mathbb{R}\).
\woe 称\(\mathbb{R}^n\)中各分量均为有理数的点为\textbf{有理点}. 证明\(\mathbb{R}^n\)中有理点的全体的闭包为\(\mathbb{R}^n\).
\end{quiza}
\begin{quizb}
\woe 设\(E\subseteq \mathbb{R}^n\), 证明存在\(E\)的至多可列的子集\(F\), 使得\(\overline{F}\supseteq E\).
\woe 设\(E\)为\(\mathbb{R}^n\)中的闭集, 证明存在集合\(F\)使得\(F\)的导集为\(E\).
\woe 下表穷举了集合的内部、导集、边界和闭包. 请证明每一格中的结果, 对于非等式的情形, 举出不成立的例子. 进一步, 思考能否得出其他的关系式.
    \begin{table}[H]
      \centering
      \begin{tabular}{ccccc}
        \hline
        集合  &  内部  &  导集  &  边界  &  闭包\\\hline
        \(E^\circ\)  &  \((E^\circ)^\circ=E^\circ\)  &  \((E^\circ)'=\overline{E^\circ}\)  &  \(\partial(E^\circ)\subseteq\partial E\cap E'\)  &  \(\overline{E^\circ}\subseteq E'\)\\
        \(E'\)  &  \((E')^\circ=\overline{E}^\circ\supseteq E^\circ\)  &  \((E')'\subseteq E'\)  &  \(\partial (E')\subseteq \partial E\cap E'\)  &  \(\overline{E'}=E'\)\\
        \(\partial E\)  &  \((\partial E)^\circ=(\overline{E}\cap \overline{\mathscr{C}E})^\circ\)  &  \((\partial E)'\subseteq\partial E\cap E'\)  &  \(\partial(\partial E)\subseteq \partial E\)  &  \(\overline{\partial E}=\partial E\)\\
        \(\overline{E}\)  &  \((\overline{E})^\circ\supseteq E^\circ\)  &  \((\overline{E})'=E'\)  &  \(\partial(\overline{E})\subseteq \partial E\)  &  \(\overline{\left(\overline{E}\right)}=\overline{E}\)\\\hline
      \end{tabular}
    \end{table}
\woe 设\(X\)为线性空间, 称映射\(\inp{\cdot,\cdot}:X\times X\rightarrow \mathbb{R}\)为\(X\)上的内积, 如果它满足: 对于任何\(x,y,z\in X\)以及\(\alpha\in \mathbb{R}\),
\begin{compactenum}[(1)]
    \item \textbf{非负性.} \(\inp{x,x}\geqslant 0\), 且等号成立当且仅当\(x=0\)时成立.
    \item \textbf{线性性.} \(\inp{x,y+z}=\inp{x,y}+\inp{x,z}\).
    \item \textbf{关于数乘线性.} \(\inp{\alpha x,y}=\alpha\inp{x,y}\).
    \item \textbf{对称性.} \(\inp{x,y}=\inp{y,x}\).
\end{compactenum}
此时称\(X\)为\textbf{(实)内积空间}. 令\(\left\|x\right\|=\sqrt{\inp{x,x}}\). 证明\(\left\|\cdot\right\|\)为\(X\)上的范数---称为由该内积诱导的范数, 且该范数满足平行四边形法则:\[\left\|x+y\right\|^2+\left\|x-y\right\|^2=2\left\|x\right\|^2+2\left\|y\right\|^2.\]
\begin{proof}
容易验证按上述规则定义的范数满足非负性和齐次性, 下证其满足三角不等式, 在此之前, 我们先证明Cauchy-Schwarz不等式: 即\(\left|\inp{x,y}\right|\leqslant\left\|x\right\|\cdot\left\|y\right\|\), 等号成立当且仅当\(x,y\)线性相关.

首先, 当\(x=0\)或\(y=kx\)时(即\(x,y\)线性相关), 我们有\[\begin{split}
\left|\inp{x,y}\right|&=\left|\inp{0,y}\right|=\left|0\inp{0,y}\right|=0=\left\|0\right\|\cdot\left\|y\right\|=\left\|x\right\|\cdot\left\|y\right\|,\\
\left|\inp{x,y}\right|&=\left|\inp{x,kx}\right|=\left|k\inp{x,x}\right|=|k|\left\|x\right\|^2=\left\|x\right\|\cdot\left\|kx\right\|=\left\|x\right\|\cdot\left\|y\right\|.
\end{split}\]当\(x,y\)线性无关时, 即对任何实数\(k\), 都有\(y\ne kx\), 也即\(kx-y\ne 0\), 于是由内积的非负性有\[0<\inp{kx-y,kx-y}=\left\|x\right\|^2k^2-2\inp{x,y}k+\left\|y\right\|^2,\]这意味着关于\(k\)的一元二次方程\(\left\|x\right\|^2k^2-2\inp{x,y}k+\left\|y\right\|^2=0\)无实根, 即\(\Delta=4\inp{x,y}-4\left\|x\right\|\cdot\left\|y\right\|<0\), 即\(\left|\inp{x,y}\right|\leqslant\left\|x\right\|\cdot\left\|y\right\|\). 这个不等式也称为Cauchy-Buniakowsky-Schwarz(柯西-布尼亚科夫斯基-施瓦茨)不等式.

利用Cauchy-Schwarz不等式\[\left\|x+y\right\|^2=\left\|x\right\|^2+2\inp{x,y}+\left\|y\right\|^2\leqslant\left\|x\right\|^2+2\left\|x\right\|\left\|y\right\|+\left\|y\right\|^2=\left(\left\|x\right\|+\left\|y\right\|\right)^2,\]即\(\left\|x+y\right\|\leqslant\left\|x\right\|+\left\|y\right\|\).

易见\[\begin{split}
\left\|x+y\right\|^2+\left\|x-y\right\|^2&=\inp{x+y,x+y}+\inp{x-y,x-y}\\&=\inp{x,x}+2\inp{x,y}+\inp{y,y}+\inp{x,x}-2\inp{x,y}+\inp{y,y}\\&=2\inp{x,x}+2\inp{y,y}=2\left\|x\right\|^2+2\left\|y\right\|^2.\qedhere
\end{split}\]
\end{proof}
\woe 设\(X,\left\|\cdot\right\|\)为赋范线性空间, 若范数\(\left\|\cdot\right\|\)满足平行四边形法则, 令\(\inp{x,y}=\frac{\left\|x+y\right\|^2-\left\|x-y\right\|^2}{4}\). 验证\(\inp{\cdot,\cdot}\)是\(X\)上的内积.
\begin{solution}
即该范数满足\(\left\|x+y\right\|^2+\left\|x-y\right\|^2=2\left\|x\right\|^2+2\left\|y\right\|^2.\) 下面对上题所给内积性质进行注意验证:
\begin{asparaenum}[(1)]
    \item 非负性. 由\[\inp{x,x}=\frac{\left\|x+x\right\|^2-\left\|x-x\right\|^2}{4}=\left\|x\right\|^2\geqslant 0,\]且等号成立当且仅当\(x=0\). 故而非负性成立.
    \item 线性性.由\[\inp{x,y+z}=\frac{\left\|x+(y+z)\right\|^2-\left\|x-(y+z)\right\|^2}{4},\]结合平行四边形法则有\[\begin{split}
    &\frac{1}{4}\cdot\frac{\left\|x+(y+z)\right\|^2-\left\|x-(y+z)\right\|^2}{2}\\=&\frac{\left\|x+y+z\right\|^2+\left\|x+y-z\right\|^2-\left(\left\|x-y+z\right\|^2+\left\|x-y-z\right\|^2\right)}{4\times2}\\=&\frac{\left\|x+y\right\|^2+\left\|z\right\|^2-\left\|x-z\right\|^2-\left\|y\right\|^2}{4},\\
    &\frac{1}{4}\cdot\frac{\left\|x+(y+z)\right\|^2-\left\|x-(y+z)\right\|^2}{2}\\=&\frac{\left\|x+y+z\right\|^2+\left\|x+z-y\right\|^2-\left(\left\|x+z-y\right\|^2+\left\|x-y-z\right\|^2\right)}{4\times 2}\\=&\frac{\left\|x+z\right\|^2+\left\|y\right\|^2-\left\|x-y\right\|^2-\left\|z\right\|^2}{4},
    \end{split}\]上面两式相加, 便得\[\inp{x,y+z}=\frac{\left\|x+y\right\|^2-\left\|x-y\right\|^2}{4}+\frac{\left\|x+z\right\|^2-\left\|x-z\right\|^2}{4}=\inp{x,y}+\inp{x,z},\]故而线性性成立.
    \item 关于数乘线性与对称性是显然成立的.\qedhere
\end{asparaenum}
\end{solution}
\woe 试构造\(\mathbb{R}\)中一列非空集\(\{E_k\}\)使得\(E_{k+1}=E'_k\subset E_k (k\geqslant 1)\).
\end{quizb}
\section{Eucild空间中的基本定理}
\precis{确界存在定理,单调收敛定理,自然对数,常数\(e\),闭区间套定理,聚点原则,致密性定理,基本列,Cauchy准则,调和级数,有限覆盖定理,Loewner偏序,对角线法,闭集套定理,局部,紧集,完备性,列紧集,相对紧集,准紧集,Euler常数,Lebesgue数,Lebesgue覆盖定理}
\begin{theorem}{}{C23}
    设\(\{\boldsymbol{A}_k\}\)是\(\mathbb{S}^n\)中的Loewner偏序下的单调有界列, 则\(\{\boldsymbol{A}_k\}\)收敛.
    \tcbline*
    记\(\mathbb{S}^n\)为\(n\)阶实对称矩阵全体, 在其中引入\textbf{Loewner(勒夫纳)偏序}: 对于\(\boldsymbol{A,B}\in\mathbb{S}^n\), 定义\(\boldsymbol{A}\leqslant\boldsymbol{B}\)为: \(\boldsymbol{B-A}\)为半正定矩阵, 即对任何\(\boldsymbol{x}\in \mathbb{R}\), 成立\[\boldsymbol{x}^\top\boldsymbol{Ax}\leqslant\boldsymbol{x}^\top\boldsymbol{Bx}.\]
\end{theorem}
\begin{quiza}
\woe 设\(x_0\in(0,2),\,x_{n+1}=x_n(2-x_n)(n\geqslant 0)\). 证明: \(\lim_{n\rightarrow\infty}x_n=1\).\\ 由此, 对于\(c>0\), 任取\(y_0>0\)满足\(cy_0<1\), 令\(y_{n+1}=y_{n}(2-cy_n)(n\geqslant 0)\), 则\(\lim_{n\rightarrow\infty}y_n=\frac{1}{c}\).\\这一事实让我们有可能利用乘法来计算倒数的近似值.
\begin{proof}
无论\(x_0\)为何值时, 有\[x_{n+1}=x_n(2-x_n)\leqslant\left(\frac{x_n+2-x_n}{2}\right)^2=1.\]即\(\{x_n\}\)有界. 特别的, \(x_1\leqslant 1\), 等号成立当且仅当\(x_0=1\), 此时结论是平凡的, 否则由\(x_n<1\), 则有\[x_{n+1}=x_n(2-x_n)>x_n,\qquad n\geqslant 1,\]即\(\{x_n\}(n=1,2,\cdots)\)单调递增, 于是\(\{x_n\}\)存在, 对条件两侧取极限即得结论.

后者令\(x_n=cy_n\)即得结论.
\end{proof}
\woe 设\(x_0\geqslant 0\), \(x_{n+1}=\sqrt{x_n+2}(n\geqslant 0)\). 试分别用以下方法证明\(\lim_{n\rightarrow+\infty}x_n=2\).
\begin{quizcs}
\item 说明\(\{x_n\}\)单调有界.
\item 证明存在常数\(C>0\)使得\(|x_n-2|\leqslant\frac{C}{2^n}(\forall n\geqslant 1)\).      
\end{quizcs}
\begin{proof}
    (1)是容易说明的(数学归纳法). 对于(2), 我们取\(f(x)=\sqrt{x+2}(x\geqslant 0)\). 这样有\[f(2)=2,\qquad x_{n+1}=f(x_n),\]并且\[f'(x)=\frac{1}{2\sqrt{2+x}}<\frac{1}{2}.\]于是\[\begin{split}
      |x_n-2|&=\left|f(x_{n-1})-f(2)\right|=\left|f'(\eta)\right||x_{n-2}-2|\\&<\frac{1}{2}|x_{n-2}-2|<\cdots <\frac{1}{2^n}|x_0-2|.
    \end{split}\]
    或者,\[\begin{split}
      |x_n-2|&=|\sqrt{x_{n-1}+2}-2|=\left|\frac{x_{n-1}-2}{\sqrt{x_{n-2}+2}+2}\right|\leqslant \left|\frac{x_{n-2}-2}{2}\right|\\ &<\cdots<\frac{1}{2^n}|x_0-2|.
    \end{split}\]总之无论如何, 都有\(\lim_{n\rightarrow+\infty}x_n=2\).
\end{proof}
\woe 设\(a_0,b_0>0\), \( a_{n+1}=\frac{a_n+b_n}{2},\,b_{n+1}=\frac{2a_nb_n}{a_n+b_n}\). 证明: \(\{a_n\}\)和\(\{b_n\}\)均收敛且极限相同.
\begin{proof}
对\(a, b>0\), 由算数几何平均不等式有\(\frac{a+b}{2}\geqslant\sqrt{ab}\), 于是有\[\sqrt{ab}-\frac{2ab}{a+b}=\frac{\sqrt{ab}\left(a+b-2\sqrt{ab}\right)}{a+b}\geqslant 0.\]从而有\(\frac{a_n+b_n}{2}\geqslant \sqrt{a_nb_n}\geqslant\frac{2a_nb_n}{a_n+b_n}\), 即\(a_{n+1}\geqslant b_{n+1}(n=0,1,\cdots)\), 由于\[a_{n+1}=\frac{a_n+b_n}{2}\Rightarrow\frac{a_{n+1}}{a_n}=\frac{1+b_n/a_n}{2}\leqslant 1,\]
从而\(\{a_{n+1}\}\)单调递减, 同理可证\(\{b_{n+1}\}\)单调递增, 有\([b_{n+1},a_{n+1}]\subset [b_{n},a_{n}]\) 又\[a_{n+1}-b_{n+1}=\frac{a_n-b_n}{2(a_n+b_n)}\cdot(a_n-b_n)<\frac{1}{2}(a_n-b_n)<\cdots<\frac{1}{2^{n-1}}(a_1-b_1)\rightarrow 0,\quad n\rightarrow\infty,\]依Cauchy-Cantor闭区间套定理原命题得证.
\end{proof}
\woe 对于\(n\geqslant 1\), 考虑\[a_n=1+\frac{1}{2}+\cdots+\frac{1}{n}-\ln n,\quad b_n=1+\frac{1}{2}+\cdots+\frac{1}{n}-\ln(n+1).\]
证明: \(\{a_n\}\)和\(\{b_n\}\)均收敛, 且收敛到同一极限, 该极限称为\textbf{Euler(欧拉)}常数.
\begin{proof}
利用不等式\[\frac{1}{n+1}<\ln\left(1+\frac{1}{n}\right)<\frac{1}{n},\qquad\forall n\geqslant 1.\]由于
    \begin{gather*}
      a_{n+1}-a_{n}=\frac{1}{n+1}-\ln\left(1+\frac{1}{n}\right)<0.\\
      b_{n+1}-b_{n}=\frac{1}{n+1}-\ln\left(1+\frac{1}{n+1}\right)>0.\\
      a_n-b_n=\ln\left(1+\frac{1}{n}\right)>\frac{1}{n}>0, \lim_{n\rightarrow\infty}(a_n-b_n)=0.
    \end{gather*}
可见\(\forall n\geqslant 1\), 有\([b_{n+1},a_{n+1}]\subseteq [b_n,a_n]\). 依Cauahy-Cantor闭区间套定理得证结论.
\end{proof}
\woe 设\(c\geqslant -3\), \[x_1=\frac{c}{2},\qquad x_{n+1}=\frac{c}{2}+\frac{x_n^2}{2},\quad n=1,2,3,\cdots,\]问\(\{x_n\}\)何时收敛, 并求极限.
\begin{solution}
我们分情况对\(c\)的范围进行讨论.
\begin{asparaenum}[\bfseries (i)]
\item \(c> 1\). 此时由基本不等式有\[x_{n+1}=\frac{c+x_n^2}{2}\geqslant\sqrt{c}x_n\geqslant\cdots\geqslant \left(\sqrt{c}\right)^nx_1,\]
故\(c>1\)时\(\lim_{n\rightarrow\infty}x_n=+\infty\).
\item \(0\leqslant c\leqslant 1\). 此时有\(x_1=\frac{c}{2}<1\), 假设\(x_n<1\), 那么\(x_{n+1}<1\). 即由数学归纳法得\(x_n\)有界. 另一方面\[x_{n+1}=\frac{c}{2}+\frac{x_n^2}{2},\quad x_n=\frac{c}{2}+\frac{x_{n-1}^2}{2},\]两式相减得到\(x_{n+1}-x_{n}=\frac{x_n^2-x_{n-1}^2}{2}\), 注意到\(x_2>x_1=\frac{c}{2}\), 则由上式递推便可得到\(x_{n+1}>x_n\). 即\(\{x_n\}\)单增有界, 从而收敛. 并且, 若\(\{x_n\}\)收敛, 记\(\lim_{n\rightarrow\infty}x_n=X\), 则对递推公式两侧取极限得到\[A=\frac{c}{2}+\frac{A^2}{2}\Rightarrow A=1\pm\sqrt{1-c},\]结合\(x_n<1\), 故\(0<c\leqslant 1\)时\(\{x_n\}\)收敛, 且\(\lim_{n\rightarrow\infty}x_n=1-\sqrt{1-c}.\)
\item \label{c2x2f}\(-3\leqslant c< 0\). 易见此时对一切\(n\)有\(x_n\geqslant\frac{c}{2}\). 并且\(x_1=\frac{c}{2}<0\), 假设\(x_n<0\), 则有\[\left|x_n\right|\leqslant\frac{c}{2},\quad x_n^2\leqslant\frac{\left|c\right|}{4}\cdot|c|<|c|,\quad x_{n+1}=\frac{c}{2}+\frac{x_n^2}{2}<\frac{c+|c|}{2}=0.\]即此时\(\frac{c}{2}\leqslant x_n<0\). 仿上法, 分别考虑\(n=2k+1,2k-1\)以及\(n=2k+2,2k\)的情况, 有\begin{align}
x_{2k+1}-x_{2k-1}&=\frac{x_{2k}^2-x_{2k-2}^2}{2},\tag{$\blacklozenge$}\label{odd}\\
x_{2k+2}-x_{2k}&=\frac{x_{2k+1}^2-x_{2k-1}^2}{2}.\tag{$\Diamond$}\label{even}
\end{align}
事实上, 有\(x_3>x_1=\frac{c}{2}\), 故\(|x_3|<|x_1|,x_3^2<x_1^2\), 在\eqref{even}式中令\(k=1\), 得到\(x_4<x_2\), 即有\(|x_4|>|x_2|,x_4^2>x_2^2\), 在\eqref{odd}式中令\(k=2\), 又得到\(x_5>x_3\)等等. 总之有\[x_{2k+1}>x_{2k-1},\quad x_{2k+2}<x_{2k}.\]结合\(\{x_{2k}\},\{x_{2k+1}\}\)均有界, 从而二者极限存在, 记\(X_1=\lim_{k\rightarrow\infty}x_{2k+1},X_2=\lim_{k\rightarrow\infty}x_{2k}\). 结合题设递推公式得到
\begin{equation}\tag{$\spadesuit$}\label{x1x2}
X_1=\frac{c}{2}+\frac{X_2^2}{2},\quad X_2=\frac{c}{2}+\frac{X_1^2}{2},
\end{equation}
两式相减, 得到\((X_1-X_2)(X_1+X_2+2)=0\). 若\(X_1+X_2+2=0\), 即\(X_1=-2-X_2\), 代入\eqref{x1x2}中左式有\[X_2^2+2X_2+(c+4)=0,\]此时\(\Delta=4-4(4+c)\leqslant 0\), 即\(c<-3\)时, 上式在实数范围内无解, 因此只能有\(X_1=X_2\). 另外当\(c=-3\)时, 易见\(X_1=X_2=-1\). 故\(-3\leqslant c<0\)时\(\{x_n\}\)收敛, 结合步骤(ii)的分析可知\(\lim_{n\rightarrow\infty}x_n=1-\sqrt{1-c}.\)\qedhere
\end{asparaenum}
\end{solution}
\woe 证明: 对于任何\(n\geqslant 1\), 存在\(\theta_n\in (0,1)\)使得\(\ee=\sum_{k=0}^{n}\frac{1}{k!}+\frac{\theta_n}{n\cdot n!}\). 由此说明\(\ee\)是无理数.
\begin{proof}
由于\[\ee=\sum_{k=0}^{\infty}\frac{1}{k!}=\sum_{k=0}^{n}\frac{1}{k!}+\sum_{k=n+1}^{\infty}\frac{1}{k!}.\]又因为\[\begin{split}
      \sum_{k=n+1}^{\infty}\frac{1}{k!}&=\frac{1}{(n+1)!}+\frac{1}{(n+2)!}+\cdots+\frac{1}{m!}+\cdots\\&<\frac{1}{(n+1)!}\left(1+\frac{1}{n+2}+\frac{1}{(n+2)^2}+\cdots+\frac{1}{(n+2)^k}+\cdots\right)\\&=\frac{1}{(n+1)!}\left(\frac{1}{1-\frac{1}{n+2}}\right)=\frac{n+2}{(n+1)\cdot(n+1)!}\\&=\frac{1}{n\cdot n!}\cdot\frac{n(n+2)}{(n+1)^2}<\frac{1}{n\cdot n!}.
    \end{split}\]这几乎就完成了第一部分的证明.
    
下面说明\(\ee\)是无理数, 事实上, 若其为有理数, 则定能表示成既约分数的形式, 设其为\(p/q\), 有上述可知\[\ee=\frac{p}{q}=\sum_{k=0}^{q}\frac{1}{k!}+\sum_{k=q+1}^{\infty}\frac{1}{k!}=\sum_{k=0}^{q}\frac{1}{k!}+\frac{\theta_n}{q\cdot q!}.\]于是右侧\(\frac{\theta_n}{q\cdot q!}\)应是\(\frac{1}{q!}\)的倍数, 但这是不可能的.
\end{proof}
\woe 设\(E,\,F\)为\(\mathbb{R}\)中的非空集, 定义\(d(E,F)=\underset{x\in E,y\in F}{\inf}|\boldsymbol{x}-\boldsymbol{y}|.\)
\begin{quizs}
      \item 若\(E,F\)为非空闭集, 且\(E\)有界, 证明存在\(\boldsymbol{x}_0\in E,\boldsymbol{y}_0\in F\)使得\(|\boldsymbol{x}_0-\boldsymbol{y}_0|=d(E,F)\).
      \item 举例说明存在不相交的非空闭集\(E,F\in \mathbb{R}^n\)使得\(d(E,F)=0\).
\end{quizs}
\begin{proof}
(1)

(2)取\(E=\{(x,0)\big| x\in\bbr\},F=\{(x,\ee^x)\big|x\in\bbr\}\), 则\(E,F\)都是闭集且\(d(E,F)=0\).
\end{proof}
\woe 设\(\mathbb{R}^n\)中有界非空闭集\(\{E_k\}\)满足\(E_{k+1}\subseteq E_{k}(k\geqslant 1)\). 证明: \(\bigcap_{n=1}^\infty E_n\)为非空闭集.
\begin{proof}

\end{proof}
\woestar 梳理/证明以下结果, 并尝试对这些结论有一个直观的理解:
\begin{quizcs}
\item 收敛数列的子列收敛到同一极限.
\item 如果一个数列有两个子列收敛到不同极限, 则该数列发散.
\item 如果有界数列的收敛子列均收敛到同一个极限, 则该数列收敛.
\item 有界数列\(\{a_n\}\)不收敛到\(A\)当且仅当存在一个收敛到\(B\ne A\)的子列.
\item \(\bar{x}\)为\(E\)的聚点当且仅当存在\(E\)中两两不同的点列\(x_n\)使得\(x_n\rightarrow \bar{x}\).
\item \(\bar{x}\)为\(E\)的聚点当且仅当存在\(x_n\in E\backslash\{\bar{x}\}\)使得\(x_n\rightarrow \bar{x}\)
\end{quizcs}
\begin{proof}
	(1)设\(\{a_n\}\)收敛到\(a\), 则\(\forall\varepsilon>0\), 有\(N_1\)使得\(n>N_1\)时\(|a_n-a|<\varepsilon\),	若其有子列\(\{a_{n_k}\}\)收敛到\(b\ne a\), 则同理存在\(N_2\)使得\(n_k>N_2\)时\(|a_{n_k}-b|<\varepsilon\), 我们不妨取\(\varepsilon<\frac{|b-a|}{2}\), 则\(n>N_1+N_2\)时有\[|b-a|-\varepsilon\leqslant|a-b|-|a_n-a|\leqslant|a_n-a+a-b|<|a_n-b|<\varepsilon,\]矛盾. 直观的理解是收敛数列的子列收敛到同一极限, 否则将导致原数列发散.
	
	(2)
	
	(3)
	
	(4)
	
	(5)
	
	(6)
\end{proof}
\woe 设\(E\subseteq \mathbb{R}\). 证明存在至多可列集\(F\subseteq E\)使得\(E\)在\(E\)中稠密.
\woe 证明: 闭区间\([a,b]\)不能表示为两个不相交的非空闭集的并.
\begin{proof}
	反证法, 设\([a,b]=[a_1,b_1]\cup [a_2,b_2]\)并且\([a_1,b_1]\cap [a_2,b_2]=\varnothing\), 可设\(a_1<b_1<a_2<b_2\), 若\(a\ne a_1\), 则或者\(a>a_1\), 此时\(a_1\notin [a,b]\), 或者\(a<a_1\), 此时\(a\notin[a_1,b_1]\cup[a_2,b_2]\)均矛盾, 从而必有\(a=a_1\), 同理\(b=b_1\), 但\((b_1,a_2)\)非空, 易见\[[a_1,b_1]\cup [a_2,b_2]=[a,b]\backslash (b_1,a_2),\]这与假设矛盾, 从而结论成立.
\end{proof}
\woe 证明: \(\mathbb{R}\)中既开又闭的集只有空集和\(\mathbb{R}\).
\begin{proof}
参考下一题.	
\end{proof}
\woe 证明: \(\mathbb{R}^n\)中既开又闭的集只有空集和\(\mathbb{R}^n\).
\begin{proof}
	设\(S\)为\(\bbr^n\)中既开又闭的集合. 因\(S\)是闭集, 故\(\bbr^n\backslash S\)为开集. 于是\(\bbr^n=S\cup\left(\bbr^n\backslash S\right)\)是两个不相交开集的并. 由\(\bbr^n\)的连通性可知\(S\)与\(\bbr^n\backslash S\)中至少有一为空集, 故\(S=\bbr^n\)或\(\varnothing\).
\end{proof}
\end{quiza}
\begin{quizb}
\woe 设\(x_0\in (0,1),\, x_{n+1}=x_n(1-x_n)(n\geqslant 0)\). 不使用Stolz公式, 按以下过程证明:
\begin{quizcs}
  \item \(\lim_{n\rightarrow+\infty}x_n=0\),
  \item 对任何\(n\geqslant 1\), 成立\((n+1)x_n<1\),
  \item 从某一项开始, \(\{nx_n\}\)单调有界.
  \item \(\lim_{n\rightarrow+\infty}nx_n=1.\)
\end{quizcs}
\begin{proof}
(1)易见\(0<x_n<1\)且\(x_{n+1}<x_n\), 从而\(\{x_n\}\)有极限, 设\(\lim_{n\rightarrow+\infty}x_n=A\), 有\(A=A(1-A)\), 即得\(A=0\).

(2)当\(n=1\)时有\(x_1=x_0(1-x_0)\leqslant\left(\frac{x_0+1-x_0}{2}\right)^2=\frac{1}{4}<\frac{1}{2}\), 即成立\((1+1)x_1<1\). 假设对\(x_n\)成立\((n+1)x_n<1\), 则由\(x_{n+1}=x_n(1-x_n)\), 注意到\(f(x)=x(1-x)\)在\(\left(0,\frac{1}{n+1}\right)\)上单调递增, 则\(x_{n+1}=x_n(1-x_n)<\frac{1}{n+1}\left(1-\frac{1}{n+1}\right)=\frac{n}{n^2+2n+1}<\frac{1}{n+2}\), 由数学归纳法可知对任何\(n\geqslant 1\)成立\((n+1)x_n<1\).

(3)易见\(\{nx_n\}\)有界. 并且有\[(n+1)x_{n+1}-nx_n=(n+1)x_n(1-x_n)-nx_n=nx_n^2+x_n-x_n^2>0,\]即\(\{nx_n\}\)单调递增.

(4)由上述知\(\{nx_n\}\)有极限, 不妨记\(\lim_{n\rightarrow+\infty}nx_n=B\), 另记\(c_n=nx_n\). 由(2)知\[c_n<1-\frac{c_n}{n}.\]由于\(\{c_n\}\)有界, 则令\(n\rightarrow+\infty\)得到\(B\leqslant 1\). 另一方面, 若\(B<1\), 则存在\(k\)使得\(B<k<1\), 注意到\[-\frac{\ln\left(1-\displaystyle\frac{c_n}{n}\right)}{\ln\left(1+\displaystyle\frac{1}{n}\right)}\rightarrow B<k,\quad n\rightarrow+\infty,\]故\(n\)充分大时, 有\(1-\frac{c_n}{n}\geqslant\left(\frac{n+1}{n}\right)^{-k}\), 并且由\(x_{n+1}=x_n(1-x_n)\)知\(c_{n+1}=\frac{n+1}{n}c_n\left(1-\frac{c_n}{n}\right)\), 有\[\frac{c_{n+1}}{c_n}=\frac{n+1}{n}\left(1-\frac{c_n}{n}\right)\geqslant\left(\frac{n+1}{n}\right)^{1-k},\]故有\(C_0>0\)使得对充分大的\(n\)有\(c_n\geqslant C_0n^{1-k}\), 这与\(\{c_n\}\)有界矛盾. 故\(B=1\).
\end{proof}
\woe 在习题2.4 \(\boldsymbol{\mathcal{A}}\)的第5题中, \(c<-3\)会如何? 
\begin{solution}
我们继续习题2.4 \(\boldsymbol{\mathcal{A}}\)中第5题的讨论. 在下面的叙述中, 我们需要使用之前的结论.
\begin{asparaenum}[\bfseries (i)]
\setcounter{enumi}{3}
\item \(-4\leqslant c<-3\). 这种情况下, 与\textbf{(\ref{c2x2f})}十分类似. 同理, 我们可以得到\(\frac{c}{2}\leqslant x_n\leqslant 0\), 并且也有\(\{x_{2k}\}\)与\(\{x_{2k+1}\}\)收敛. 沿用之前的记号, 得到
\begin{gather*}
X_1=\frac{c}{2}+\frac{X_2^2}{2},\quad X_2=\frac{c}{2}+\frac{X_1^2}{2},\\
(X_1-X_2)(X_1+X_2+2)=0,
\end{gather*}
下面我们证明, 这种情况下, \(X_1\ne X_2\). 否则\(\{x_n\}\)收敛到\(X=X_1=X_2=1-\sqrt{1-c}\), 由于\(|X|>1\), 于是\(n\)充分大时有\[\left|\frac{x_n+X}{2}\right|>1,\]同时考虑到\[x_{n+1}=\frac{c}{2}+\frac{x_n^2}{2},\quad X=\frac{c}{2}+\frac{X^2}{2},\]两式相减得到\[\left|x_{n+1}-X\right|=\left|\frac{x_n+X}{2}\right|\cdot\left|x_n-X\right|\geqslant\left|x_n-X\right|,\]这表明若\(\{x_n\}\)收敛到\(X\), 必有\(N\in\bbz\)使得\(n>N\)时\(x_n=X\). 由于\(X=\frac{c}{2}+\frac{X^2}{2}>\frac{c}{2}=x_1\), 故\(N>1\). 选取满足\(n=N-1\)时成立\(x_n\ne X\)的\(N\), 于是\[\left|x_N-X\right|=\left|\frac{x_{N-1}+X}{2}\right|\cdot\left|x_{N-1}-X\right|,\]这表明\(x_{N-1}=-X\). 故\(\{x_n\}\)收敛当且仅当\(\exists n\)使得\(x_n=-X\). 由于\(X=1-\sqrt{1-c},-X=\sqrt{1-c}-1>1\), 而\(x_n<0\). 故此时\(\{x_n\}\)不收敛. 即此时有\(X_1+X_2+2=0\). \(\{x_n\}\)有两极限点. 易见其为方程\(x^2+2x+(4+c)=0\)的两个根. 特别的, \(c=-4\)时, \(X_1=0, X_2=-2\).
\item \(-8\leqslant c<-4\). 首先\(c=-8\)时. 易见此时收敛并且\(\lim_{n\rightarrow+\infty}x_n=4\). 当\(-8<c<-4\)时, 归纳易得\(|x_n|\leqslant -\frac{c}{2}<1+\sqrt{1-c}\), 即若\(\{x_n\}\)收敛必有\(X=1-\sqrt{1-c}\). 由上面的分析知, 这等价于\(\exists k\geqslant 2\)使得\(x_k=-X=\sqrt{1-c}-1\). 易见\(x_n\)可表成\(c\)得多项式, 即有\(x_k:=f_k(c)\), 由于\[f_k(-4)\leqslant 0<\sqrt{1-(-4)}-1,\quad f_k(-8)=4>\sqrt{1-(-8)}-1,\]故对每个\(k\geqslant 2\)都存在有限个\(c\in(-8,-4)\)使得\(x_k=f_k(c)=\sqrt{1-c}-1\). 故此时使\(\{x_n\}\)收敛的值为\((-8,-4)\)中的可数多个值.
\item \(c<-8\). 此时\(x_1=\frac{c}{2},x_2=\frac{c}{2}+\frac{x_1^2}{2}=\frac{c}{2}+\frac{c^2}{8}\), 并且此时\[\frac{c}{2}+\frac{c^2}{8}>4,\quad \left|\frac{c^2}{8}\right|=\left|\frac{c}{8}\right|\cdot|c|>|c|\Rightarrow\left|\frac{c}{2}+\frac{c^2}{8}\right|>\left|\frac{c}{2}\right|,\]即此时\(0<|x_1|<x_2\), 结合\[x_{n+1}-x_n=\frac{x_n^2-x_{n-1}^2}{2},\]可知此时\(\{x_n\}\)单增, 即有\(x_3>x_2>4\), 于是当\(n>3\)时, \[x_{n+1}-x_n=\frac{x_n+x_{n-1}}{2}\left(x_n-x_{n-1}\right)>4\left(x_n-x_{n-1}\right),\]这意味着\(\{x_n-x_{n-1}\}\)无界, 进而\(\{x_n\}\)无界. 故\(c<-8\)时\(\lim_{n\rightarrow\infty}x_n=+\infty.\)\qedhere
\end{asparaenum}
\end{solution}
\woe 设\([a,b]\)上函数\(\{f_n\}\)一致有界, 即存在常数\(M>0\)使得\[\left|f_n(x)\right|\leqslant M,\qquad\forall n\geqslant 1,\,x\in [a,b],\]证明存在子函数列\(\{f_{n_k}\}\)在\([a,b]\)的所有有理数点收敛.
\begin{proof}
设\([a,b]\)上的有理数集为\[a_1,a_2,\cdots,a_n,\cdots,\]由于\(\{f_n\}\)一致有界, 则对\(\forall x_0\in [a,b]\), \(\{f_n(x_0)\}\)都是\(\bbr\)中的有界列, 由Bolzano-Weierstrass定理可知, 其有收敛子列. 设对于\(a_1\), 子列\(\{f_{1_k}\}\)收敛, 则对于\(a_2\), 抽取\(\{f_{1_k}\}\)的子列\(\{f_{2_k}\}\), 则\(\{f_{2_k}\}\)对于\(a_1,a_0\)都收敛, 在\(f_{2_k}\)中选取子列\(f_{3_k}\)使得\(\{f_{3_k}(a_3)\}\)收敛, 并以此类推. 这就完成了证明.
\end{proof}
\woe 设整数\(m\geqslant 1\), 证明: \(\lim_{n\rightarrow+\infty}\sqrt{m+\sqrt{m+1+\sqrt{m+2+\cdots+\sqrt{m+n}}}}\)存在, 并记该极限为\(a_m\). 进一步, 计算\(\lim_{m\rightarrow+\infty}\left(a_m-\sqrt{m}\right)\).
\begin{solution}
记原连根式为\(A_n^{(m)}\), 关于\(n\)单增, 只需证其存在上界, 反复利用\[\sqrt{m+x}\leqslant\sqrt{m}\left(1+\frac{x}{2m}\right),\forall x\geqslant 0,\]令\(\alpha_m=\frac{1}{2\sqrt{m}}\), 可得\[A_n^{(m)}\leqslant\sqrt{m}\sum_{k=0}^{n}\alpha_m^k+\sum_{k=1}^{n}k\alpha_m^{k+1}\leqslant\sqrt{m}+\frac{\sqrt{m}\alpha_m}{1-\alpha_m}+\left(\frac{\alpha_m}{1-\alpha_m}\right)^2,\]故\(\lim_{n\rightarrow+\infty}A_n^{(m)}=a_m\)存在, 且\[\sqrt{m+\sqrt{m}}\leqslant a_m\leqslant\sqrt{m}+\frac{\sqrt{m}\alpha_m}{1-\alpha_m}+\left(\frac{\alpha_m}{1-\alpha_m}\right)^2,\]可得\(\lim_{n\rightarrow+\infty}\left(a_m-\sqrt{m}\right)=\frac{1}{2}\).
\end{solution}
\woe 证明: \(\lim_{n\rightarrow+\infty}\sqrt{1+2\sqrt{1+3\sqrt{1+4\sqrt{1+\cdots+n\sqrt{1+(n+1)}}}}}=3.\)
\begin{proof}
令\[f(x)=\sqrt{1+x\sqrt{1+(x+1)\sqrt{1+(x+2)\sqrt{1+\cdots}}}}\quad (x\geqslant 1),\]那么\(f(x)\)满足函数方程\[f^2(x)=1+xf(x+1).\]因为\((x+1)^2=1+x(x+2)\), 所以函数\(x+1\)是它的一个解. 我们来证明这是唯一解.

显然, 当\(x\geqslant 2\)时有\[f(x)\geqslant\sqrt{x\sqrt{x\sqrt{x\sqrt{x\cdots}}}}=\lim_{n\rightarrow\infty}x^{1/2+1/2^2+\cdots+1/2^n}=x>\frac{1}{2}(x+1).\]并且\[\begin{split}
f(x)&\leqslant\sqrt{(x+1)\sqrt{(x+2)\sqrt{(x+3)\sqrt{(x+4)\cdots}}}}\\
&<\sqrt{(x+1)\sqrt{2(x+1)\sqrt{4(x+1)\sqrt{8(x+1)\cdots}}}}\\
&=\sqrt{(x+1)\sqrt{(x+1)\sqrt{(x+1)\sqrt{(x+1)\cdots}}}}\sqrt{1\sqrt{2\sqrt{4\sqrt{8\cdots}}}},
\end{split}\]上式右边第一个因子等于\[\lim_{n\rightarrow\infty}(x+1)^{1/2+1/4+1/8+\cdots+1/2^n}=x+1,\]依据\(\sum_{k=2}^{\infty}(k-1)2^{-k}=1\), 第二个因子等于\[\lim_{n\rightarrow\infty}1^{1/2}2^{1/4}4^{1/8}\cdots(2^{n-1})^{1/2^n}=2,\]所以\[f(x)<2(x+1).\]合起来就是\[\frac{1}{2}(x+1)<f(x)<2(x+1).\]

在上面的不等式中易\(x\)为\(x+1\)得\[\frac{1}{2}(x+2)<f(x+1)<2(x+2),\]而由函数方程\(f^2(x)=1+xf(x+1)\)可知\[\frac{1}{2}+xf(x+1)<f^2(x)<2+xf(x+1).\]由上述两式推出\[\frac{1}{2}\left(1+x(x+2)\right)<f^2(x)<2(1+x(x+2)),\]亦即\[\frac{1}{2}(x+1)^2<f^2(x)<2(x+1)^2,\]因此得到\[\sqrt{\frac{1}{2}}(x+1)<f(x)<\sqrt{2}(x+1).\]重复上述过程, 也就是说, 首先在上式中易\(x\)为\(x+1\)得\[\sqrt{\frac{1}{2}}(x+2)<f(x+1)<\sqrt{2}(x+2),\]并且由函数方程可知\[\sqrt{\frac{1}{2}}+xf(x+1)<f^2(x)<\sqrt{2}+xf(x+1).\]于是从两式推出\[\sqrt{\frac{1}{2}}\left(1+x(x+2)\right)<f^2(x)<\sqrt{2}\left(1+x(x+2)\right),\]因而得到\[\sqrt[4]{\frac{1}{2}}(x+1)<f(x)<\sqrt[4]{2}(x+1).\]一般地, 应用数学归纳法可证\[\sqrt[2^k]{\frac{1}{2}}(x+1)<f(x)<\sqrt[2^k]{2}(x+1).\]在其中令\(k\rightarrow\infty\)即得\(x+1\leqslant f(x)\leqslant x+1\). 这样, 我们最终得到\(f(x)=x+1\). 对应于原极限知其为3.
\end{proof}
\woe 推广第\(\boldsymbol{5}\)题.
\begin{solution}
结合第\(\boldsymbol{5}\)题的证明可以得到如下结果:\[\sqrt{1+a\sqrt{1+(a+1)\sqrt{1+(a+2)\sqrt{1+\cdots}}}}=a+1,\qquad a\geqslant 1.\qedhere\]
\end{solution}
\woe 证明命题\reff{Th:C23}: 设\(\{\boldsymbol{A}_k\}\)是\(\mathbb{S}^n\)中Loewner偏序意义下的单调有界列, 则\(\{\boldsymbol{A}_k\}\)收敛.
\begin{proof}

\end{proof}
\woe 在Loewner偏序意义下, 设\(E\)是\(\mathbb{S}^n\)中的非空上有界集, 证明或否定: \(E\)一定有最小上界.
\woe 设\(x_0>0,\,x_{n+1}=\sqrt{2+x_n}(n\geqslant 0)\). 问: \(\lim_{n\rightarrow+\infty}4^n(2-x_n)\)是否存在? 该极限存在时, 能否得到极限的值?
\begin{solution}
当\(0<x_0< 2\)时, 令\(x_0=2\cos\alpha\left(0<\alpha<\frac{\pi}{2}\right)\), 则有
\(x_1=\sqrt{2+2\cos\alpha}=2\cos\frac{\alpha}{2}\), 归纳可得\[x_n=2\cos\frac{\alpha}{2^n},\]此时\(\lim_{n\rightarrow+\infty}4^n(2-x_n)=\lim_{n\rightarrow+\infty}4^n\left(2-2\cos\frac{\alpha}{2^n}\right)=\alpha^2.\)

当\(x_0\geqslant 2\)时, 令\(x_0=\frac{1}{\alpha}+\alpha(\alpha>0)\), 则有\(x_1=\sqrt{2+\alpha+\frac{1}{\alpha}}=\frac{1}{\sqrt{\alpha}}+\sqrt{\alpha}\), 归纳可得\[x_n=\alpha^{-1/2^n}+\alpha^{1/2^n},\]此时\(\lim_{n\rightarrow+\infty}4^n(2-x_n)=\lim_{n\rightarrow+\infty}4^n\left(2-\alpha^{-1/2^n}-\alpha^{1/2^n}\right)=\ln^2\alpha.\)
\end{solution}
\woe 一般地, 对于所求过的极限, 可考虑进一步的结果, 例如, 若\(\lim_{n\rightarrow+\infty}x_n=A\), 试试能否找到一个无穷大\(y_n\)使得\(\lim_{n\rightarrow+\infty}y_n\left(x_n-A\right)\)为一个非零常数.
\tcbline
这是可能的.
\tcbline
\woe 固定\(r\in(0,4)\), 任取\(x_0\in(0,1)\), 令\(x_{n+1}=rx_n(1-x_n)(n\geqslant 0)\). 问\(r\)取什么值时, 可以证明对于任何\(x_0\in(0,1)\), 按上述方式定义的\(\{x_n\}\)一定收敛?
  
试编写一计算机程序, 感受对于不同的\(r\), 当\(n\)足够大时\(x_n\)的收敛情况.
\begin{solution}
由\(0<x_0<1\), 如果\(x_n\in(0,1)\), 则\[0<x_{n+1}=rx_n(1-x_n)\leqslant r\left(\frac{x_{n}+1-x_{n}}{2}\right)^2=\frac{r}{4}<1,\]
由归纳法可知\(x_n\in(0,1)\). 下面我们分情况对\(r\)的范围进行讨论.
\begin{asparaenum}[\bfseries (i)]
\item \(0<r\leqslant 1\). 易见\(\{x_n\}\)单调有界,  设其收敛于\(A\), 则有\(A=rA(1-A)\), 则或者\(A=0\), 或者\(A=1-\frac{1}{r}\), 但此时\(1-\frac{1}{r}< 0(r\ne 1)\), 这是不可能的, 故\(A=0\).
\item \(r>1\)

\end{asparaenum}
\end{solution}
\woe 设\(\mathscr{U}\)和\(\mathscr{V}\)都是\(E\subseteq \mathbb{R}^n\)的覆盖. 称\(\mathscr{V}\)是\(\mathscr{U}\)的\textbf{加细}, 如果对于任何\(V\in\mathscr{V}\)存在\(U\in\mathscr{U}\)使得\(V\subseteq U\).
  
对于\(\lambda>0\), 若\(\{B_\lambda(x)|x\in E\}\)是\(\mathscr{U}\)的加细, 则称\(\lambda\)为覆盖\(\mathscr{U}\)的\textbf{Lebesgue数}. 易见, 若\(E\)的一个子集\(F\)的直径小于Lebesgue数\(\lambda\), 则可在\(\mathscr{U}\)中找到元素\(V\)包含\(F\).
  
试证明\textbf{Lebesgue覆盖定理}: 设\(\mathscr{U}\)为紧集\(E\subset\mathbb{R}\)的开覆盖, 则它存在Lebesgue数.
\begin{proof}

\end{proof}


\woe 考虑线性空间\[X=\left\lbrace \{x_n\}_{n=1}^{\infty}\left| \sum_{n=1}^{\infty}|x_n|^2<+\infty\right. \right\rbrace ,\]在其上定义\[||\{x_n\}||=\left(\sum_{n=1}^{\infty}|x_n|^2\right)^{1/2},\qquad d\left(\{x_n\},\{y_n\}\right)=||\{x_n-y_n\}||.\]证明\(||\cdot||\)是\(X\)上的范数, 并以此范数定义收敛性, 开闭集等. 证明: 
\begin{quizs}
\item 在\(X\)上, Cauchy准则成立.
\item 在\(X\)上, 致密性定理不成立, 即有界点列不一定有收敛子列.
\item 设\(\{x_k\}\)是\(X\)中的点列, 它的每一个分量都有界, 即记\(\boldsymbol{x}_k=\{x_{k,n}\}_{n=1}^{\infty}\)时, 对任何\(n\geqslant 1\), \(\{x_{k,n}\}_{k=1}^{\infty}\)是有界集. 证明: 存在\(\{\boldsymbol{x}_k\}\)的子列使得其每一个分量都收敛.
\item 在\(X\)上, 有限覆盖定理不成立.
\item 在\(X\)上, 闭区间套定理成立. 即若\(\{F_k\}\)是\(X\)中的一列单调下降且直径趋于零的闭集列, 则\( \bigcap_{n=1}^{\infty}E_n\)为单点集.
\item \(X\)中存在一列有界非空闭集\(E_1\supseteq E_2\supseteq E_3\supseteq \cdots\)使得\(\bigcap_{n=1}^{\infty}E_n\)为空集.
\end{quizs}
\begin{proof}

\end{proof}
\woe 回答下列问题.
\begin{quizs}
    \item 在\([-\infty,+\infty]\)中\footnote{以下说法在\(\mathbb{R}\)中不适用}, 若\(\lim_{n\rightarrow+\infty}a_n=+\infty\)或\(\lim_{n\rightarrow+\infty}a_n=-\infty\), 我们就称\(\{a_n\}\)收敛到\(+\infty\)或\(-\infty\). 试重新考察习题\(\boldsymbol{\mathcal{A}}\)中的第9题.
    \item 对于\(E\subseteq[-\infty,+\infty] \), 若存在\(a\in \mathbb{R}\)使得\(\left(a,+\infty\right]\subseteq E\left(\left[-\infty,a\right)\subseteq E\right)\), 则称\(+\infty(-\infty)\)为\(E\)的内点. 若\(U\subseteq[-\infty,+\infty]\)的所有点都是其内点, 则称\(U\)为\([-\infty,+\infty]\)的开集. 考虑在\([-\infty,+\infty]\)中, 哪些集合是紧集, 即\([-\infty,+\infty]\)中满足任何开覆盖都有有限子覆盖的集合.
\end{quizs}
\woe 设\(E\in\mathbb{R}\), 则\(E\)满足什么条件时, 成立: 对于\(E\)的任何闭覆盖, 均有有限子覆盖.
\begin{solution}
	
\end{solution}
\woestar 设\(U\)为\(\bbr\)中的\textcolor{red}{非空}开集, 证明:
\begin{quizcs}
    \item 若\(x_0\in U\), 则存在唯一的开区间\((\alpha,\beta)\)---这样的区间称为\(U\)的构成区间, 满足: \\ (i) \(x_0\in(\alpha,\beta)\subseteq U\); (ii) 若\(x_0\in (a,b)\subseteq U\), 则\((a,b)\subseteq (\alpha,\beta)\).
    \item \(U\)可以表示为至多可列个两两不交的构成区间的并.
\end{quizcs}
\begin{proof}
	(1)由于\(x_0\in U\)是\(U\)的内点, 所以集合\[A:=\{y\in U\big|\,y<x_0\,\text{且}\,(y,x_0)\subseteq U\}\]非空, 定义\(\alpha=\inf A\). 那么\(-\infty\leqslant\alpha<x_0\), 并且\((\alpha,x_0)\in U\). 如果\(\alpha\)是实数, 那么根据\(\alpha\)的定义, 必有\(\alpha\notin U\).
	
	同理, 定义集合\[B:=\{y\in U\big|x_0<y\,\text{且}\,(x_0,y)\subseteq U\},\]取\(\beta=\sup B\), 则由\(\alpha,\beta\)的定义可见\[x_0\in(\alpha,\beta)\subseteq U,\]并且若\(x_0\in(a,b)\subseteq U\), 则\((a,b)\subseteq (\alpha,\beta)\).
	
	(2)把全体具有这样性质的开区间所构成的集合记为\(\mathscr{G}\), 那么\(\mathscr{G}\)中任意两个元素作为开区间都不交. 从它的每一个元素中选取一个有理数与之对应, 全体这样的有理数集合记为\(S\), 那么\(\mathscr{G}\)与\(S\)一一对应, 而\(S\)是\(\bbq\)的子集, 从而是可数集, 从而结论得证.
\end{proof}
\woe 设\(\{x_n\}\)是在\([0,1]\)中稠密的点列, 任取\(\alpha\in\left(0,\frac{1}{2}\right),\,E=[0,1]\setminus\bigcup_{n=1}^{\infty}\left(x_n-\frac{\alpha}{2^n},x_n+\frac{\alpha}{2^n}\right)\). 证明\(E\)为非空的疏朗集.
\begin{proof}
	?
\end{proof}
\woe 证明: 闭区间\([a,b]\)不能表示为至多两个但至多可列个两两不交的非空闭集的并.
\begin{proof}
	
\end{proof}
\woe 证明: \(\mathbb{R}^n\)不能表示成一列无处稠密集的并.
\begin{proof}
	
\end{proof}
\woe 设\(\{V_k\}\)是一列在\(\mathbb{R}^n\)中稠密的开集, 证明: \(\bigcap_{k}V_k\)在\(\mathbb{R}\)中稠密.
\begin{proof}
	
\end{proof}
\end{quizb}
\section{上、下极限}
\precis{上极限,下极限, Stolz公式的推广}
\begin{quiza}
\woe 若\(x_n>0(n=1,2,\cdots)\), 且\(\varlimsup_{n\rightarrow+\infty}x_n\cdot\varlimsup_{n\rightarrow+\infty}\frac{1}{x_n}=1\). 证明序列\(\{x_n\}\)收敛.
\begin{proof}
由于\[\varlimsup_{n\rightarrow+\infty}x_n\cdot\varlimsup_{n\rightarrow+\infty}\frac{1}{x_n}=\frac{\displaystyle\varlimsup_{n\rightarrow+\infty}x_n}{\displaystyle\varliminf_{n\rightarrow+\infty}x_n}=1,\]从而有\(\varlimsup_{n\rightarrow+\infty}x_n=\varliminf_{n\rightarrow+\infty}x_n\), 故\(\{x_n\}\)收敛.
\end{proof}
\woe 设\(\{x_n\}\)满足\(0\leqslant x_{m+n}\leqslant x_{m}\cdot x_{n}\,(\forall m,n\geqslant 1)\). 证明\(\{\sqrt[n]{x_n}\}\)收敛.
\begin{proof}
易见\[0\leqslant x_n\leqslant x_{n-1}\cdot x_1\leqslant \cdots\leqslant x_1^n.\]从而\(\sqrt[n]{x_n}\leqslant x_1\), 故\(\varlimsup_{b\rightarrow+\infty}\sqrt[n]{x_n}\)有限. 给定正整数\(p\), 使得\(n=kp+r\), 其中\(k\geqslant 0,\, 0\leqslant r<p\), 于是\[x_n=x_{pk+r}\leqslant x_{pk}\cdot x_r\leqslant x_p^k\cdot x_r,\]即有\[\sqrt[n]{x_n}\leqslant \sqrt[n]{x_p^k}\sqrt[n]{x_r}\leqslant \sqrt[p]{x_p}\sqrt[n]{x_r},\]注意到\(x_r\in\{x_1,x_2,\cdots,x_{p-1}\}\)有界, 令\(n\rightarrow+\infty\)得到\[\sqrt[n]{x_n}\leqslant\sqrt[p]{x_p}\]对任意\(p\)都成立, 于是有\[\varlimsup_{n\rightarrow+\infty}\sqrt[n]{x_n}\leqslant\varliminf_{p\rightarrow+\infty}\sqrt[p]{x_p}.\]依上、下极限的性质, 反向不等式也成立, 从而目标序列收敛.
\end{proof}
\woe 举例说明, 在第2题中, \(\{\sqrt[n]{x_n}\}\)可以没有单调性.
\begin{solution}
\(x_1=2,x_2=1,x_3=2,x_4=1,x_5=2,\cdots\)
\end{solution}
\woe 设\(\{a_n\}\)满足\(a_m+a_n-1\leqslant a_{m+n}\leqslant a_m+a_n+1(\forall m,n\geqslant 1)\). 求证: 
\begin{quizs}
\item \(\lim_{n\rightarrow+\infty}\frac{a_n}{n}\)存在.
\item 设\(\lim_{n\rightarrow+\infty}\frac{a_n}{n}=q\), 则\(nq-1\leqslant a_n\leqslant nq+1\).
\end{quizs}
\begin{proof}

\end{proof}
\woe 设\(r\in (0,1)\,,x_0=0,\,x_{n+1}=r(1-x_n^2)(n=0,1,2,\cdots)\). 试探究\(\{x_n\}\)的敛散性.
\begin{solution}
易见\(0<x_n<1(n\geqslant 1)\)有界. 我们先证明这样一件事: 对于正整数\(m\), 只要\(k>2m+1\), 则\(x_k<x_{2m+1}\). 当\(m=0\)时, \(2m+1=1\), 易见\(k>1\)有\(x_k=r(1-x_{k-1}^2)<r=x_1\)成立, 假设对\(m\in\bbn_+\)成立, 考虑\(m+1\)时\[k>2(m+1)+1\Rightarrow k-2>2m+1,\]又由\[x_{k-1}=r(1-x_{k-2}^2)>r(1-x_{2m+1}^2)=x_{2m+2}\Rightarrow x_k=r(1-x_{k-1}^2)<r(1-x_{2m+2}^2)=x_{2(m+1)+1},\]从而由数学归纳法可知结论成立. 特别的, \(\{x_{2k+1}\}\)严格单调递减. 同理可证对于\(m\in\bbn\), 若\(k>2m\), 则\(x_k>x_{2m}\), 即得\(\{x_{2k}\}\)严格单调递增. 从而可设\(\lim_{k\rightarrow+\infty}x_{2k}=A,\lim_{k\rightarrow+\infty}x_{2k+1}=B.\)

对于\(x_{n+1}=r(1-x_n^2)\), 分别令\(n=2k-1,2k\)再令\(k\rightarrow+\infty\)可得\[A=r(1-B^2),\quad B=r(1-A^2),\]两式相减得到\(\left(A-B\right)\left(1-r(A+B)\right)=0\). 即或者\(A=B\), 或者\(1-r(A+B)=0\), 注意到
\[x_{2n+1}+x_{2n}=r(1-x_{2n}^2)+x_{2n}\Rightarrow A+B=r(1-A^2)+A,\]
代入\(1-r(A+B)=0\)有\[r^2A^2-rA-r^2+1=0\Rightarrow\left(rA-\frac{1}{2}\right)^2=r^2-\frac{3}{4},\]这意味着\(r^2<\frac{3}{4}\)时, 即\(0<r<\frac{\sqrt{3}}{2}\)时\(1-r(A+B)=0\)无解, 此时只能有\(A=B\), 结合式递推公式可知\(\lim_{n\rightarrow+\infty}x_n=\frac{\sqrt{4r^2+1}-1}{2r}.\)
考虑\(r=\frac{\sqrt{3}}{2}\)时, 有\[x_{2k+1}=r(1-x_{2k}^2)=r\left(1-r^2(1-x_{2k-1}^2)^2\right)\Rightarrow r^3B^4-2r^3B^2+B+r^3-r=0,\]得\(\frac{3\sqrt{3}}{8}B^4-\frac{3\sqrt{3}}{4}B^2+B-\frac{\sqrt{3}}{8}=0\), 即\[\frac{3\sqrt{3}}{8}\left(B-\frac{\sqrt{3}}{3}\right)^3(B+\sqrt{3})=0\Rightarrow B=\frac{\sqrt{3}}{3},\]同理可得\(A=\frac{\sqrt{3}}{3}\), 从而\(\lim_{n\rightarrow+\infty}x_n=\frac{\sqrt{3}}{3}\).

下证\(\frac{\sqrt{3}}{2}<r<1\)时\(\{x_n\}\)发散. 只需要证明\(A\ne B\). 由上述的分析注意到\[r^3B^4-2r^3B^2+B+r^3-r=0\Rightarrow (B^2r-r+B)(B^2r^2-Br-r^2+1)=0,\]
考虑\(B^2r^2-Br-r^2+1=0\)的解, 记\[C=\frac{1-\sqrt{4r^2-3}}{2r},\quad D=\frac{1+\sqrt{4r^2-3}}{2r},\]我们证明, 对于任何\(m\in\bbn_+\), 有\(x_{2m}<C,x_{2m+1}>D\). 但当\(\frac{\sqrt{3}}{2}<r<1\)时有\(0<C<\frac{\sqrt{3}}{3}<D<1\), 进而\(0\leqslant A\leqslant C<\frac{\sqrt{3}}{3}<D\leqslant B\leqslant 1\), 即得\(A\ne B\), 从而\(\{x_n\}\)发散. 事实上, 注意到\[
C^2=\frac{4r^2-\left(2+2\sqrt{4r^2-3}\right)}{4r^2}=1-\frac{D}{r},\quad D^2=\frac{4r^2-\left(2-2\sqrt{4r^2-3}\right)}{4r^2}=1-\frac{C}{r},
\]当\(m=0\)时, \(x_0=0<C,x_1=r>D.\) 假设对\(m\in\bbn_+\)成立, 考虑\(m+1\)时\[x_{2(m+1)}=r\left(1-x_{2m+1}^2\right)<r(1-D^2)=C,x_{2(m+1)+1}=r\left(1-x_{2(m+1)}^2\right)>r\left(1-C^2\right)=D,\]由数学归纳法可知结论成立. 综上, \(r\in\left(0,\frac{\sqrt{3}}{2}\right]\),\(\{x_n\}\)收敛, \(r\in\left(\frac{\sqrt{3}}{2},1\right)\)时\(\{x_n\}\)发散.
\end{solution}
\woe 设\(0<q<1,\,a_n,\,b_n\)满足: \(a_n=b_n-qa_{n+1}(n=1,2,\cdots)\), 且\(a_n,\,b_n\)有界. 求证: \(\lim_{n\rightarrow+\infty}b_n\)存在的充要条件是\(\lim_{n\rightarrow+\infty}a_n\)存在.
\begin{proof}

\end{proof}
\woe 在第6题中当\(q\notin (0,1)\)时, 结论会怎样?
\begin{solution}

\end{solution}
\woe 设\(a_{n}\geqslant 0\)满足\( a_{n+1}\leqslant a_n+\frac{1}{n^2}\), 证明: \(\{a_n\}\)收敛.
\begin{proof}
由题意知\[a_{n+1}\leqslant a_n+\frac{1}{n^2}\leqslant a_n+\frac{1}{n(n-1)}=a_{n}+\frac{1}{n-1}-\frac{1}{n}\Rightarrow a_{n+1}+\frac{1}{n}\leqslant a_{n}+\frac{1}{n-1},\]于是\(\left\lbrace a_n+\frac{1}{n-1}\right\rbrace\)递减且有下界. 故\(\left\lbrace a_n+\frac{1}{n-1}\right\rbrace\)收敛. 而\(a_n=a_n+\frac{1}{n-1}-\frac{1}{n-1}\)也收敛.
\end{proof}
\woe 利用上下极限证明\(\mathbb{R}\)中的Cauchy准则.
\begin{proof}

\end{proof}
\end{quiza}
\begin{quizb}
\woe 推广习题2.5 \(\boldsymbol{\mathcal{A}}\)中的第2题和第4题.
\begin{solution}

\end{solution}
\woe 设\(x\in\mathbb{R}\), 证明: \(\lim_{n\rightarrow+\infty}\left(1+\frac{x}{n}\right)^n=\ee^x\).
\begin{proof}
我们先证明这样一个事实: 设\(h:\bbr\rightarrow(0,+\infty)\)满足\(\lim_{x\rightarrow+\infty}h(x)=+\infty\), 也就是说, 不论实数\(A\)多么大, 都能找到实数\(B\)使得当\(x>B\)时, \(h(x)>A\)成立, 那么\[\lim_{x\rightarrow+\infty}\left(1+\frac{1}{h(x)}\right)^{h(x)}=\ee.\]记\(f(x)=\left(1+\frac{1}{h(x)}\right)^{h(x)}\), 考虑充分大的\(x\)使得\(h(x)>1\). 记\[[h(x)]=m,h(x)-m=\alpha\in[0,1),\]其中\([x]\)指不超过\(x\)的最大整数, 那么\[\begin{split}
\left(1+\frac{1}{m+1}\right)^{m+\alpha}&<f(x)\leqslant\left(1+\frac{1}{m}\right)^{m+\alpha},\\
\left(1+\frac{1}{m+1}\right)^{\alpha-1}e_{m+1}&<f(x)\leqslant\left(1+\frac{1}{m}\right)^\alpha e_{m}.
\end{split}\]其中\(e_m:=\left(1+\frac{1}{m}\right)^m\). 令\(x\rightarrow+\infty\), 这暗含着\(m\rightarrow+\infty\), 即得\(\lim_{x\rightarrow+\infty}f(x)=\ee.\)

回到题目的证明: \(x=0\)时的情形不需考虑. 设\(x\ne 0\). 记\(h_n=h_n(x)=\frac{x}{n}\), 则由上述结果有\[\lim_{n\rightarrow+\infty}\left(1+h_n\right)^{1/h_n}=\ee.\]两边取\(x\)次幂, 有\[\ee^x=\left(\lim_{n\rightarrow+\infty}\left(1+h_n\right)^{1/h_n}\right)^x,\]下面我们仅需说明\[\left(\lim_{n\rightarrow+\infty}\left(1+h_n\right)^{1/h_n}\right)^x=\lim_{n\rightarrow+\infty}\left(1+h_n\right)^{x/h_n},\]就完成了证明.

为此, 我们证明下面的定理: 设\(\alpha\in\bbr,f(x):=x^\alpha,x>0\), 则\(\lim_{y\rightarrow 0}f(x+y)=f(x).\) 这正是对幂函数连续性的刻画. 即证明\(\forall\varepsilon>0,\exists\delta,\)只要\(|y|<\delta\), 就有\[\left|f(x+y)-f(x)\right|<\varepsilon.\]设\(|y|<x\). 这保证\(x+y>0\), 有\[f(x+y)-f(x)=f(x)\left(\left(1+\frac{y}{x}\right)^\alpha-1\right),\]记\(h=\frac{y}{x}\), 并记\(m=[\alpha]+1\), 分两种情形讨论:\begin{compactenum}[(1)]
\item \(0\leqslant h<1.\) 此时\[0\leqslant (1+h)^\alpha-1\leqslant(1+h)^m-1=h\sum_{k=1}^{m}(1+h)^{m-k}\leqslant h\left(m\cdot 2^m\right).\]
\item \(-1\leqslant h<0.\) 此时\[0\leqslant 1-(1+h)^\alpha\leqslant 1-(1+h)^m=-h\sum_{k=1}^{m}(1+h)^{m-k}\leqslant -h(m\cdot 2^m).\]
\end{compactenum}
总之,\[\left|f(x+y)-f(x)\right|=f(x)\left|(1+h)^\alpha-1\right|\leqslant|h|\left(m\cdot 2^m\right).\]由此可见, \(\forall \varepsilon\in(0,1)\), 取\(\delta=\frac{\varepsilon x}{m\cdot 2^m}\), 就对于满足\(|y|<\delta\)的\(y\)成立\(|f(x+y)-f(x)|<\varepsilon\). 这就完成了证明.
\end{proof}
\woe 证明: 对任何\(x\in\mathbb{R}\)成立\(\ee^x=\sum_{n=0}^{\infty}\frac{x^n}{n!}\).
\begin{proof}
我们定义\(T_n(x)=\left(1+\frac{x}{n}\right)^n\), 由上一题的结论可知对于任何\(x\in\bbr\), \(\{T_n(x)\}\)都是Cauchy列. 再设\(S_n(x)=\sum_{k=0}^{n}\frac{x^k}{k!}\). 则对于\(m>n>2N\), 其中\(N=\left[|x|\right]+1\), 有\[\left|S_m(x)-S_n(x)\right|\leqslant\sum_{k=n+1}^{m}\frac{N^k}{k!}<\sum_{k=n+1}^{m}\frac{N^k}{(2N)^{k-2N}}=(2N)^{2N}\sum_{k=n+1}^{m}\left(\frac{1}{2}\right)^k<\frac{(2N)^{2N}}{2^n},\]所以对于一切实数\(x\), \(\{S_n(x)\}\)是Cauchy列.

另一方面, 当\(n>2\)时,\[S_n(x)-T_n(x)=\sum_{k=2}^{n}\frac{1}{k!}\left(1-\left(1-\frac{1}{n}\right)\cdots\left(1-\frac{k-1}{n}\right)\right)x^k,\]注意到当\(k\geqslant 2\)时,\[0<1-\left(1-\frac{1}{n}\right)\cdots\left(1-\frac{k-1}{n}\right)\leqslant\frac{1+\cdots+(k-1)}{n}=\frac{k(k-1)}{2n},\]得到\[\left|S_n(x)-T_n(x)\right|\leqslant\frac{1}{2n}\sum_{k=2}^{n}\frac{|x|^k}{(k-2)!}=\frac{x^2}{2n}S_{n-2}(|x|),\]从而对于一切实数\(x\), 数列\(\{S_n(x)\}\)与\(\{T_n(x)\}\)等价. 因此有\(\ee^x=\sum_{n=0}^{\infty}\frac{x^n}{n!}\).
\end{proof}
\woe 设\(\{x_n\}\)是正数列, 证明\(\varlimsup_{n\rightarrow\infty}n\left(\frac{1+x_{n+1}}{x_n}-1\right)\geqslant 1.\)
\begin{proof}
设\(\varlimsup_{n\rightarrow+\infty}n\left(\frac{1+x_{n+1}}{x_n}-1\right)<1\), 则\(\exists N>0\), 当\(n\geqslant N\)时,\[n\left(\frac{1+x_{n+1}}{x_n}\right)<1,\]此即\(\frac{1}{n+1}<\frac{x_n}{n}-\frac{x_{n+1}}{n+1}(n=N,N+1,\cdots,N+k-1,\cdots)\). 这是无穷多个不等式, 将前\(k\)个不等式相加得\[\frac{1}{N+1}+\cdots+\frac{1}{N+k}<\frac{x_N}{N}-\frac{x_{N+k}}{N+k}<\frac{x_N}{N}.\]此式应对一切\(k>1\)成立, 但实际上, 左端当\(k\rightarrow+\infty\)时, 极限为\(+\infty\), 矛盾.

事实上, 本题有如下推广: 设\(\{a_n\}\)为任意正数列, \(p\)是任意给定的正整数, 则\[\varlimsup_{n\rightarrow+\infty}\left(\frac{a_1+a_{n+p}}{a_n}\right)>\ee^p,\quad\varlimsup_{n\rightarrow+\infty}n\left(\frac{1+a_{n+p}}{a_n}-1\right)>p,\]上述不等式中的\(p\)不可用更大的数代替, 并且上述两个不等式等价.
\end{proof}
\woe 设\(f\)是\(\mathbb{R}\)上周期为1的实函数. 满足: \(\forall x,y\in\mathbb{R},\,x\ne y\), 成立\(|f(x)-f(y)|<|x-y|\). 任取\(x_0\in\mathbb{R}\), 定义\(x_{n+1}=f(x_n)(n\geqslant 0)\). 证明: \(\{x_n\}\)收敛, 且极限不依赖于\(x_0\)的选择.
\begin{proof}

\end{proof}
\end{quizb}
\section{正项级数}
\precis{正项级数,正项级数收敛的基本定理,比较判别法,Cauchy判别法,D'Alembert判别法,Raabe判别法,收敛得更慢与发散得更慢得级数}
\begin{quiza}
\woestar 设\(\{a_n\}\)为一正数列. 证明:\[\varliminf_{n\rightarrow\infty}\frac{a_{n+1}}{a_n}\leqslant\varliminf_{n\rightarrow\infty}\sqrt[n]{a_n}\leqslant\varlimsup_{n\rightarrow\infty}\sqrt[n]{a_n}\leqslant\varlimsup_{n\rightarrow\infty}\frac{a_{n+1}}{a_n}.\] 
\begin{proof}
设\(\varlimsup_{n\rightarrow\infty}\frac{a_{n+1}}{a_n}=a<\infty\), 证明\(\varlimsup_{n\rightarrow\infty}\sqrt[n]{a_n}\geqslant a\)
\end{proof}
\woe 讨论以下级数的收敛性: \vspace{8pt}\\
\begin{tabular}{lcl}
\((1)\,\sum_{n=1}^{\infty}\frac{n^3+2n+5}{n^4+4n+7}\);&\qquad\qquad\qquad&\((2)\,\sum_{n=1}^{\infty}\frac{1}{n^2-100n+1}\cos\frac{1}{n}\);\vspace{0.3cm}\\
\((3)\,\sum_{n=1}^{\infty}\ln\left(1+\frac{1}{n^2}\right)\);&&\((4)\,\sum_{n=1}^{\infty}\left(\frac{(2n-1)!!}{(2n)!!}\right)^{\alpha}\).
\end{tabular}
\begin{solution}
(1)由于\(\frac{n^3+2n+5}{n^4+4n+7}\sim\frac{1}{n}\), 故\(\sum_{n=1}^{\infty}\frac{n^3+2n+5}{n^4+4n+7}\)发散.

(2)由于\(\frac{1}{n^2-100n+1}\cos\frac{1}{n}\leqslant\frac{1}{n^2-100n+1}\sim\frac{1}{n^2}\), 故\(\sum_{n=1}^{\infty}\frac{1}{n^2-100n+1}\cos\frac{1}{n}\)收敛; 

(3)由于\(\ln\left(1+\frac{1}{n^2}\right)\leqslant\frac{1}{n^2}\), 故\(\sum_{n=1}^{\infty}\ln\left(1+\frac{1}{n^2}\right)\)收敛;

(4)注意到\[\begin{matrix}\vspace{3pt}
3=\frac{2+4}{2}>\sqrt{2\times 4},&\qquad &2=\frac{1+3}{2}>\sqrt{1\times 3},\\
5=\frac{4+6}{2}>\sqrt{4\times 6},&\qquad &4=\frac{3+5}{2}>\sqrt{3\times 5},\\
\vdots&\qquad&\vdots\\
2n-1=\frac{(2n-2)+2n}{2}>\sqrt{(2n-2)\cdot 2n},&\qquad&2n=\frac{(2n-1)+(2n+1)}{2}>\sqrt{(2n-1)(2n+1)},
\end{matrix}\]于是有\[\left(\frac{1}{2\sqrt{n}}\right)^\alpha<\left(\frac{(2n-1)!!}{(2n)!!}\right)^{\alpha}<\left(\frac{1}{\sqrt{2n+1}}\right)^\alpha,\]于是当\(\alpha>2\)时, \(\sum_{n=1}^{\infty}\left(\frac{(2n-1)!!}{(2n)!!}\right)^{\alpha}\)收敛, 其余情况发散.
\end{solution}
\woe 举例说明, 当\(\varlimsup_{n\rightarrow+\infty}\sqrt[n]{a_n}=1\)时,  正项级数\(\sum_{n=1}^{\infty}a_n\)既可能收敛也可能发散.
\begin{solution}
对于\(\sum_{n=1}^{\infty}\frac{1}{n}\)与\(\sum_{n=1}^{\infty}\frac{1}{n^2}\)均有\(\varlimsup_{n\rightarrow+\infty}\sqrt[n]{a_n}=1\), 但前者发散而后者收敛.
\end{solution}
\woe 设\(a_1\geqslant a_2\geqslant a_3\geqslant\cdots\geqslant 0\). 证明: 若\(\sum_{n=0}^{n}a_n\)收敛, 则\(\lim_{n\rightarrow\infty}na_n=0.\)
\begin{proof}
对于任何\(m\)与\(n>m\), 我们有\[(n-m)a_n<a_{m+1}+a_{m+2}+\cdots+a_n<\alpha_m,\]其中\(\alpha_m\)为该收敛级数的余式, 由此得\(na_n<\frac{n}{n-m}\alpha_m.\) 由于级数\(\sum_{n=0}^{n}a_n\)收敛, 故对于任给得\(\varepsilon>0\), 则存在\(M\)使得\(\alpha_{M}<\varepsilon\). 其次, 由于\(\lim_{n\rightarrow\infty}\frac{n}{n-M}=1\), 故存在\(N(N>M)\), 使得\(n\geqslant N\)时有\(\frac{n}{n-M}<2\). 于是, 当\(n\geqslant N\)时, 有\(0<na_n<2\varepsilon\), 因此\(\lim_{n\rightarrow\infty}na_n=0.\)
\end{proof}
\woe 设正项级数\(\sum_{n=1}^{\infty}a_n\)收敛, \(\lim_{n\rightarrow\infty}na_n=a\). 证明\(a=0\).
\begin{proof}
反证法. 若\(a\ne 0\), 则由\(a_n>0\)知\(a>0\), 结合\(\lim_{n\rightarrow\infty}na_n=a\)可知对于\(\forall 0<\varepsilon<a\), 存在\(N\), 使得\(n>N\)时有\[na_n>a-\varepsilon,\quad\text{即}\quad a_n>\frac{a-\varepsilon}{n},\]由于\(\sum_{n=1}^{\infty}a_n\)收敛, 由此推知\(\sum_{n=1}^{\infty}\frac{1}{n}\)收敛, 矛盾.
\end{proof}
\woe 讨论级数\(\left(\frac{1}{2}\right)^p+\left(\frac{1\cdot 3}{2\cdot 4}\right)^p+\left(\frac{1\cdot 3\cdot 5}{2\cdot 4\cdot 6}\right)^p+\cdots\)的收敛性.
\begin{solution}
该题与2.6 \(\boldsymbol{\mathcal{A}}\)第\(\boldsymbol{2}\)题的第(4)题是完全相同的. 这里给出另一种方法. 由于\(\frac{a_n}{a_{n+1}}=\left(\frac{2n+2}{2n+1}\right)^p,\) 于是\[\lim_{n\rightarrow\infty}n\left(\frac{a_n}{a_{n+1}}-1\right)=\lim_{n\rightarrow\infty}\frac{\left(\displaystyle\frac{2n+2}{2n+1}\right)^p-1}{\displaystyle\frac{1}{n}}=\lim_{n\rightarrow\infty}\frac{1+\displaystyle\frac{p}{2n+1}+o\left(\frac{1}{n}\right)-1}{\displaystyle\frac{1}{n}}=\frac{p}{2},\]由Rabba判别法, 知当\(p>2\)时, 级数收敛.
\end{solution}
\woe 研究数列\(\left\lbrace\frac{1}{1+\alpha}\,\frac{2}{2+\alpha}\,\cdots\,\frac{n}{n+\alpha}\right\rbrace \)当\(n\)趋于无穷时候的阶, 其中\(\alpha\)是一个非负常数.
\begin{solution}
当\(n\)趋于无穷时, 由String公式可得\[\prod_{k=1}^{n}\left(\frac{k}{k+\alpha}\right)=\frac{\Gamma(1+\alpha)\Gamma(n+1)}{\Gamma(n+\alpha+1)}\sim\frac{\Gamma(\alpha+1)\sqrt{2\pi n}(n/\ee)^n}{\sqrt{2\pi(n+\alpha)}\left((n+\alpha)/\ee\right)^{n+\alpha}}\sim\frac{\Gamma(\alpha+1)}{n^\alpha}.\qedhere\]
\end{solution}
\woe 研究级数\(\sum_{n=10}^{\infty}\frac{1}{n^p(\ln n)^q(\ln\ln n)^r}\)的敛散性, 其中\(p,q,r\)为实数.
\begin{solution}
当\(p>1\)时, 令\(p=1+2\varepsilon,\varepsilon>0\), 则\[\frac{n^{1+\varepsilon}}{n^p(\ln n)^q(\ln\ln n)^r}=\frac{n^{1+\varepsilon}}{n^{1+2\varepsilon}(\ln n)^q(\ln\ln n)^r}=\frac{1}{n^\varepsilon(\ln n)^q(\ln\ln n)^r}\rightarrow 0,\quad n\rightarrow+\infty,\]由\(\sum_{n=10}^{\infty}\frac{1}{n^{1+\varepsilon}}\)收敛可知\(\sum_{n=10}^{\infty}\frac{1}{n^p(\ln n)^q(\ln\ln n)^r}\)收敛.

当\(p<1\)时, 令\(p=1-2\varepsilon,\varepsilon>0\), 则\[\frac{n^{1-\varepsilon}}{n^p(\ln n)^q(\ln\ln n)^r}=\frac{n^{1-\varepsilon}}{n^{1-2\varepsilon}(\ln n)^q(\ln\ln n)^r}=\frac{n^{\varepsilon}}{(\ln n)^q(\ln\ln n)^r}\rightarrow+\infty,\quad n\rightarrow+\infty,\]由\(\sum_{n=10}^{\infty}\frac{1}{n^{1-\varepsilon}}\)发散可知\(\sum_{n=10}^{\infty}\frac{1}{n^p(\ln n)^q(\ln\ln n)^r}\)发散.

当\(p=1\)时.
\end{solution}
\end{quiza}
\begin{quizb}
\woe 给定\(0\leqslant l\leqslant 1\leqslant L\leqslant+\infty\), 是否存在收敛或发散的正项级数\(\sum_{n=0}^{\infty}a_n\)使得\(\varlimsup_{n\rightarrow+\infty}\frac{a_{n+1}}{a_n}=L\), 且\(\varliminf_{n\rightarrow+\infty}\frac{a_{n+1}}{a_n}=l\) ?
\begin{solution}
存在, 可设\(a_n=\begin{cases}
\frac{1}{2^n},n=2k,\\
\frac{1}{4^n},n=2k+1
\end{cases},k\in\bbz\), 于是\(\varlimsup_{n\rightarrow+\infty}\frac{a_{n+1}}{a_n}=+\infty\), 且\(\varliminf_{n\rightarrow+\infty}\frac{a_{n+1}}{a_n}=0\).
\end{solution}
\woe 利用与级数\(\sum_{n=3}^{\infty}\frac{1}{n\ln^p n}\)比较, 仿Raabe判别法给出一个判别定理.
\begin{solution}
根据例2.6.4, 我们已经知道, 级数\(\sum_{n=3}^{\infty}\frac{1}{n\ln^p n}\)当且仅当\(p>1\)时收敛. 于是我们有\[\begin{split}
&\text{正项级数}\sum_{n=0}^{\infty}a_n\text{收敛.}\\
\Longleftarrow & \exists p>1,M>0,N\geqslant 3,\mathrm{s.t.}\,\forall n\geqslant N,\quad a_n\leqslant\frac{M}{n\ln^pn}\\
\Longleftrightarrow & \exists p>1,M>0,N\geqslant 3,\mathrm{s.t.}\,\forall n\geqslant N,\quad p\leqslant\frac{\ln M-\ln n-\ln a_n}{\ln\left(\ln n\right)}\\
\Longleftrightarrow &\varliminf_{n\rightarrow\infty}\frac{-\ln n-\ln a_n}{\ln\left(\ln n\right)}>1\\
\Longleftrightarrow&\varliminf_{n\rightarrow\infty}\frac{\ln n+\ln a_n-\ln(n+1)-\ln a_{n+1}}{\ln\left(\ln(n+1)\right)-\ln\left(\ln n\right)}>1\\
\Longleftrightarrow&\varliminf_{n\rightarrow\infty}\displaystyle\frac{\displaystyle\ln\frac{n}{n+1}+\ln\displaystyle\frac{a_n}{a_{n+1}}}{\ln\left(\displaystyle\frac{\ln(n+1)}{\ln n}\right)}>1\Longleftrightarrow\varliminf_{n\rightarrow\infty}\displaystyle\frac{-\displaystyle\frac{1}{n}+\frac{a_n}{a_{n+1}}-1}{\displaystyle\frac{\ln(n+1)}{\ln n}-1}>1\\
\Longleftrightarrow&\varliminf_{n\rightarrow\infty}\frac{\ln n\left(n\left(\displaystyle\frac{a_n}{a_{n+1}}-1\right)-1\right)}{n\ln\left(1+\displaystyle\frac{1}{n}\right)}>1\Longleftrightarrow\varliminf_{n\rightarrow\infty}\ln n\left(n\left(\frac{a_n}{a_{n+1}}-1\right)-1\right)>1.
\end{split}\]
同理可证\(\varlimsup_{n\rightarrow\infty}\ln n\left(n\left(\frac{a_n}{a_{n+1}}-1\right)-1\right)<1\)时\(\sum_{n=0}^{\infty}a_n\)发散.
\end{solution}
\woe 试考察不可数个正数的和.
\end{quizb}
\section{任意项级数}
\precis{任意项级数,绝对收敛,条件收敛,Abel变换,Abel判别法,Dirichlet判别法,交错级数,Leibniz判别法,幂级数,幂级数的收敛半径,Cauchy-Hadamard公式,Cauchy乘积,Mertens定理,级数的重排,累级数,无穷乘积的收敛性}
\begin{theorem}{Mertens 定理}{C24}
设级数\(\sum_{n=1}^{\infty}a_n\)绝对收敛到\(A\), 级数\(\sum_{n=1}^{\infty}b_n\)收敛到\(B\), 则他们的Cauchy乘积\(\sum_{n=1}^{\infty}c_n\)收敛到\(AB\).
\end{theorem}
\begin{quiza}
\woe 讨论以下级数的收敛性(包括绝对收敛性):\vspace{8pt}\\
\begin{tabular}{lcl}
\((1)\,\sum_{n=1}^{\infty}\frac{1}{n}\ln\left(1+\frac{(-1)^{n-1}}{n}\right)\);&\qquad\qquad\qquad&\((2)\,\sum_{n=1}^{\infty}\frac{n^nx^n}{n!}\);\vspace{0.3cm}\\
\((3)\,\sum_{n=1}^{\infty}\frac{1}{\sqrt{n}}\sin\left(n+\frac{1}{n}\right)\);&&\((4)\,\sum_{n=1}^{\infty}\frac{\cos(n+(-1)^n)}{n}\).
\end{tabular}
\begin{solution}
(1)由于\[\frac{1}{n}\ln\left(1+\frac{(-1)^n}{n}\right)=\frac{(-1)^{n-1}}{n^2}-\frac{1}{2n^3}+o\left(\frac{1}{n^3}\right),\]故该级数收敛且绝对收敛.

(2)记\(a_{n}=\frac{n^n}{n!}\), 则\(\lim_{n\rightarrow\infty}\frac{a_{n+1}}{a_{n}}=\lim_{n\rightarrow\infty}\left(1+\frac{1}{n}\right)^n=\ee.\) 从而\(x\in\left(-\frac{1}{\ee},\frac{1}{\ee}\right)\)时收敛, 并且绝对收敛. 当\(x=\frac{1}{\ee}\)时, 有\[\frac{n^n}{n!\ee^n}=\frac{n^n}{\sqrt{2\pi n}(n/\ee)^n\ee^n}\sim\frac{1}{\sqrt{2\pi n}},\]从而发散. 当\(x=-\frac{1}{\ee}\), 易见其为Leibniz级数, 即得条件收敛.

(3)注意到\(\sin\left(n+\frac{1}{n}\right)\sim\sin n\), 而\[\begin{split}
\sum_{k=1}^{n}\sin k=\frac{\displaystyle\sum_{k=1}^{n}2\sin k\sin(1/2)}{2\sin(1/2)}=\frac{\displaystyle\sum_{k=1}^{n}\left(\cos\frac{2k-1}{2}-\cos\frac{2k+1}{2}\right)}{2\sin(1/2)}=\frac{\displaystyle\cos\frac{1}{2}-\cos\frac{2n+1}{2}}{2\sin(1/2)},
\end{split}\]故\(\sum_{n=1}^{\infty}\sin n\)部分和有界, \(\left\lbrace \frac{1}{\sqrt{n}}\right\rbrace \)单调递减趋于零, 故由Dirichlet判别法知该级数收敛. 但\[\left|\frac{1}{\sqrt{n}}\sin\left(n+\frac{1}{n}\right)\right|\sim\left|\frac{\sin n}{\sqrt{n}}\right|>\left|\frac{\sin n}{n}\right|,\]后者发散, 故为条件收敛.

(4)条件收敛.
\end{solution}
\woe 证明对于任何\(x\), 级数\(\sin x-\sin\sin x+\sin\sin\sin x-\sin\sin\sin\sin x+\cdots\)收敛.
\begin{proof}
不妨设\(a_0=x\in(0,\pi),a_{n+1}=\sin a_n(n\geqslant 1)\), 于是原级数即为\(\sum_{n=1}^{\infty}(-1)^{n+1}a_n\), 易见\(\{a_{n}\}\)单调递减趋于零, 由Leibniz判别法可知该级数收敛. 若\(a_0=x\in(-\pi,0)\), 则\(-x\in(0,\pi)\), 原级数即为\(\sum_{n=1}^{\infty}(-1)^{n}a_n\), 同理可获结论. 结合\(\sin x\)的周期性易见对于任何\(x\), 结论成立.
\end{proof}
\woe 设有级数\(\sum_{n=1}^{\infty}a_n\). 级数\(\sum_{n=1}^{\infty}A_n\)由级数\(\sum_{n=1}^{\infty}a_n\)加括号得到, 即\( A_n=\sum_{k=p_n}^{p_{n+1}-1}a_k\), 其中\(\{p_n\}\)是严格单增的正整数列, \(p_1=1\). 证明:
\begin{quizs}
\item 若\(\sum_{n=1}^{\infty}a_n\)收敛, 则\(\sum_{n=1}^{\infty}A_n\)收敛.
\item 若\(\lim_{n\rightarrow\infty}a_n=0\), \(\{p_{n+1}-p_n\}\)有界, \(\sum_{n=1}^{\infty}A_n\)收敛, 则\(\sum_{n=1}^{\infty}a_n\)收敛.
\end{quizs}
\woe 级数\(1+\frac{1}{2}+\frac{1}{3}-\frac{1}{4}-\frac{1}{5}-\frac{1}{6}+\frac{1}{7}+\frac{1}{8}+\frac{1}{9}+\cdots\)是否收敛?
\begin{solution}
易见\[1,1,1,-1,-1,-1,1,1,1,-1,\cdots,\]部分和有界, 而\(\left\lbrace\frac{1}{n}\right\rbrace \)单调递减趋于0, 依Dirichlet判别法知其收敛.
\end{solution}
\woe 设\(\sum_{n=1}^{\infty}a_n\)条件收敛, 证明: \(\sum_{n=1}^{\infty}a_n\)通过重排可以发散到\(+\infty\)以及\(-\infty\).
\begin{proof}
由\(\sum_{n=1}^{\infty}a_n\)条件收敛, 可知\(\sum_{n=1}^{\infty}a_n^+\)与\(\sum_{n=1}^{\infty}a_n^-\)都发散到正无穷, 且\(\lim_{n\rightarrow+\infty}a_n^+=\lim_{n\rightarrow+\infty}a_n^-=0\). 我们这样排列: 先按原来的顺序选取若干正项, 使其和大于1, 然后添加第一个负项, 然后在余下的正项中继续选取, 使得整个和大于2, 然后添加第二个负项, 依此类推. 由于\(\lim_{n\rightarrow+\infty}a_n^-=0\), 故\(n\)充分大时\(a_n^-\)可以任意小, 从而可知由上法得到的级数将发散到\(+\infty\), 相似的操作方式可使其发散到\(-\infty\).
\end{proof}
\woe 试利用Dirichlet判别法证明Abel判别法.
\begin{proof}
若数列\(\{b_n\}\)单调有界且\(\sum_{n=1}^{\infty}a_n\)收敛, 则由单调收敛定理可知\(\{b_n\}\)收敛, 不妨记\(\lim_{n\rightarrow+\infty}b_n=b\), 从而\(\{b_n-b\}\)单调趋于零. 根据Dirichlet判别法知级数\(\sum_{n=1}^{\infty}a_n(b_n-b)\)收敛, 于是由\[\sum_{n=1}^{\infty}a_nb_n=\sum_{n=1}^{\infty}a_n(b_n-b)+b\sum_{n=1}^{\infty}a_n\]可知, 级数\(\sum_{n=1}^{\infty}a_nb_n\)收敛.
\end{proof}
\woe 证明Mertens定理, 即定理\reff{Th:C24}.
\begin{proof}
记\(A_n=\sum_{k=1}^{n}a_n\,B_n=\sum_{k=1}^{n}b_n\,C_n=\sum_{k=1}^{\infty}c_k\). 我们要证\(\lim_{n\rightarrow+\infty}C_n=AB\), 易见\[\begin{split}
C_n&=a_1b_1+(a_1b_2+a_2b_1)+(a_1b_3+a_2b_2+a_3b_1)+\cdots+\left(a_1b_n+a_2b_{n-1}+\cdots a_nb_1\right)\\&=a_1\left(b_1+b_2+\cdots+b_n\right)+a_2(b_1+b_2+\cdots+b_{n-1})+\cdots+a_nb_1\\&=a_1B_n+a_2B_{n-1}+\cdots+a_nB_1.
\end{split}\]如果令\(\beta_n=B_n-B\), 则\(\beta_n\rightarrow 0(n\rightarrow+\infty)\), 于是\[C_n=A_nB-\gamma_n,\]其中\(\gamma_n=\sum_{\ell=1}^{n}a_1\beta_{n+1-\ell}=a_1\beta_n+a_2\beta_{n-1}+\cdots+a_n\beta_1,\)这样的问题就归结于证\(\lim_{n\rightarrow+\infty}\gamma_n=0\).

由于\(\lim_{n\rightarrow+\infty}\beta_n=0\), 故对任意的\(\varepsilon>0\), 存在\(N\in\bbn_+\), 当\(n>N\)时有\(|\beta_n|<\varepsilon\). 于是由\(\gamma_n\)的定义可得\[\begin{split}
|\gamma_n|&\leqslant\left|a_1\beta_n+\cdots a_{n-N}\beta_{N+1}\right|+\left|a_{n-N+1}\beta_{N}+\cdots+\cdots a_n\beta_1\right|\\&<\varepsilon M+\left|a_{n-N+1}\beta_{N}+\cdots+\cdots a_n\beta_1\right|,
\end{split}\]其中\(M=\sum_{n=1}^{\infty}|a_n|<+\infty\). 在上式中固定\(N\), 令\(n\rightarrow+\infty\), 即得\[\varlimsup_{n\rightarrow+\infty}|\gamma_n|\leqslant\varepsilon M,\]由此可知\(\lim_{n\rightarrow+\infty}\gamma_n=0\), 从而\(\lim_{n\rightarrow+\infty}C_n=AB.\)
\end{proof}
\woe 设\(\sum_{k=0}^{\infty}\sum_{j=0}^{\infty}a_{kj}\)绝对收敛, 证明: \(\sum_{j=0}^{\infty}\sum_{k=0}^{\infty}a_{kj}\)绝对收敛, 且\(\sum_{j=0}^{\infty}\sum_{k=0}^{\infty}a_{kj}=\sum_{k=0}^{\infty}\sum_{j=0}^{\infty}a_{kj}\).
\begin{proof}

\end{proof}
\woe 举例说明两个条件收敛的级数的Cauchy乘积有可能收敛, 也有可能发散.
\begin{solution}
取\(a_n=b_n=\frac{(-1)^n}{n}\), 则\(\sum_{n=1}^{\infty}a_n\)与\(\sum_{n=1}^{\infty}b_n\)条件收敛,

取\(a_n=b_n=\frac{(-1)^n}{\sqrt{n}}\), 则\(\sum_{n=1}^{\infty}a_n\)与\(\sum_{n=1}^{\infty}b_n\)条件收敛, 但这对于其Cauchy乘积\(\sum_{n=1}^{\infty}c_n\)有\[|c_n|=\sum_{i+j=n+1}\frac{1}{\sqrt{ij}}\geqslant\sum_{i+j=n+1}\frac{2}{i+j}=\frac{2n}{n+1}\geqslant 1,\]从而\(\sum_{n=1}^{\infty}c_n\)发散.
\end{solution}
\woe 按绝对收敛, 条件收敛, 发散分类, 穷举两个级数的Cauchy乘积收敛性的各种可能性. 并给出必要的证明和反例.
\woe 确定实数\(p\)的取值区间, 使在该区间内的二重级数\(\sum_{n=1}^{\infty}\sum_{m=1}^{\infty}\frac{1}{(m+n)^p}\)收敛.
\begin{solution}

\end{solution}
\woe 考察\(\sum_{n=1}^{\infty}\frac{\sin n\sin n^2}{n}\)的敛散性.
\begin{solution}
我们有\[\sin n\sin n^2=\frac{1}{2}\left(\cos (n^2-n)-\cos(n^2+n)\right),\]而\[\left|\frac{1}{2}\sum_{k=1}^{N}\left(\cos (k^2-k)-\cos(k^2+k)\right)\right|=\frac{1}{2}\left|\cos 0-\cos (N^2+N)\right|\leqslant 1,\]又\(\left\lbrace\frac{1}{n}\right\rbrace\)单调递减趋于0. 依Dirichlet判别法, 知\(\sum_{n=1}^{\infty}\frac{\sin n\sin n^2}{n}\)收敛.
\end{solution}
\end{quiza}
\begin{quizb}
\woe 利用Abel变换证明\textbf{钟开莱不等式}: 设\(a_1\geqslant a_2\geqslant\cdots\geqslant a_n>0\), \[\sum_{k=1}^{m}b_k\geqslant \sum_{k=1}^{m}a_k,\qquad\forall m=1,2,\cdots,n.\]则\(\sum_{k=1}^{n}a_k^2\leqslant\sum_{k=1}^{n}b_k^2.\)
\begin{proof}
记\(A_m=\sum_{k=1}^{m}a_k,B_m=\sum_{k=1}^{m}b_m\), 则由Abel变换\[\sum_{k=1}^{n}a_k^2=a_nA_n+\sum_{k=1}^{n-1}A_k(a_k-a_{k+1})\leqslant a_nB_n+\sum_{k=1}^{n-1}B_k(a_{k}-a_{k-1})=\sum_{k=1}^{n}a_kb_k,\]即\[\sum_{k=1}^{n}a_k^2\leqslant\sum_{k=1}^{n}a_kb_k\leqslant\sqrt{\sum_{k=1}^{n}a_k^2\cdot\sum_{k=1}^{n}b_k^2}\Rightarrow\sum_{k=1}^{n}a_k^2\leqslant\sum_{k=1}^{n}b_k^2,\]从而结论成立.
\end{proof}
\woe 设\(\sum_{n=1}^{\infty}A_n\)收敛. 试证明存在\(\{a_n\},\,\{b_n\}\)使得\(A_n=a_nb_n(n\geqslant 1)\), \(\sum_{n=1}^{\infty}a_n\)部分和有界, 而\(\{b_n\}\)单调下降趋于零.
\begin{solution}

\end{solution}
\woe 证明: \(\lim_{\alpha\rightarrow 0^+}\sum_{n=1}^{\infty}\frac{(-1)^{n-1}}{n^\alpha}=\frac{1}{2}\).
\begin{proof}
令\(a_n(\alpha)=\frac{1}{n^\alpha},B_0=0,\Delta B_{n-1}=B_n-B_{n-1}=(-1)^{n-1}\), 则\[\Delta a_n=\frac{1}{(n+1)^\alpha}-\frac{1}{n^\alpha},\]且\[B_n=\sum_{k=1}^{n}\left(B_k-B_{k-1}\right)=\sum_{k=1}^{n}(-1)^{k-1}=\frac{1-(-1)^{n}}{2},\]于是对\(\forall m\in\bbz^+\), 有\[\begin{split}
\sum_{n=1}^{m}\frac{(-1)^{n-1}}{n^\alpha}&=\sum_{n=1}^{m}a_n(\alpha)\Delta B_{n-1}=a_m(x)B_m-\sum_{n=1}^{m-1}B_n\Delta a_n(\alpha)\\
&=\frac{1}{m^\alpha}\frac{1-(-1)^n}{2}+\sum_{n=1}^{m-1}\frac{1-(-1)^n}{2}\left(\frac{1}{n^\alpha}-\frac{1}{(n+1)^\alpha}\right).
\end{split}\]
由于对于\(\forall\alpha\in(0,+\infty)\), 级数\(\sum_{n=1}^{\infty}\frac{(-1)^{n-1}}{n^\alpha}\)都收敛, 所以令\(m\rightarrow+\infty\), 由上面的等式可得\[\begin{split}
\sum_{n=1}^{\infty}\frac{(-1)^{n-1}}{n^\alpha}&=\sum_{n=1}^{\infty}\frac{1-(-1)^n}{2}\left(\frac{1}{n^\alpha}-\frac{1}{(n+1)^\alpha}\right)\\&=\frac{1}{2}\sum_{n=1}^{\infty}\left(\frac{1}{n^\alpha}-\frac{1}{(n+1)^\alpha}\right)-\frac{1}{2}\sum_{n=1}^{\infty}(-1)^n\left(\frac{1}{n^\alpha}-\frac{1}{(n+1)^\alpha}\right)\\&=\frac{1}{2}-\frac{1}{2}\sum_{n=1}^{\infty}(-1)^n\left(\frac{1}{n^\alpha}-\frac{1}{(n+1)^\alpha}\right).
\end{split}\]

下面证明级数\(\sum_{n=1}^{\infty}(-1)^n\left(\frac{1}{n^\alpha}-\frac{1}{(n+1)^\alpha}\right)\)在\(\left[0,+\infty\right)\)上一致收敛. 令\(b_n(\alpha)=-\Delta a_n(\alpha)(n\in\bbz^+)\), 则由\[b_n(\alpha)=\frac{1}{n^\alpha}-\frac{1}{(n+1)^\alpha}=\frac{1}{n^\alpha}\left(1-\frac{1}{\left(1+\displaystyle\frac{1}{n}\right)^\alpha}\right)\]可知, 对于任何\(\alpha\in\left[0,+\infty\right)\), \(\{b_n(\alpha)\}\)关于\(n\)单调减少, 并且当\(x\in[0,1]\)时, 有\[\left|b_n(\alpha)\right|\leqslant 1-\frac{1}{\displaystyle\left(1+\frac{1}{n}\right)^x}\leqslant 1-\frac{1}{\displaystyle 1+\frac{1}{n}}=\frac{1}{n+1},\]
当\(\alpha>1\)时, 有\[\left|b_n(\alpha)\right|\leqslant\frac{1}{n^\alpha}<\frac{1}{n},\]于是数列\(\{b_n(\alpha)\}\)在\(\left[0,+\infty\right)\)上一致收敛于零.故由Dirichlet判别法知该级数一致收敛, 于是\[\begin{split}
\lim_{\alpha\rightarrow 0^+}\sum_{n=1}^{\infty}\frac{(-1)^{n-1}}{n^\alpha}&=\lim_{\alpha\rightarrow0^+}\left(\frac{1}{2}-\frac{1}{2}\sum_{n=1}^{\infty}(-1)^n\left(\frac{1}{n^\alpha}-\frac{1}{(n+1)^\alpha}\right)\right)\\&=\frac{1}{2}-\frac{1}{2}\sum_{n=1}^{\infty}(-1)^n\lim_{\alpha\rightarrow0^+}\left(\frac{1}{n^\alpha}-\frac{1}{(n+1)^\alpha}\right)=\frac{1}{2}.\qedhere
\end{split}\]
\end{proof}
\woe 对于\(s>1\), \textbf{Riemann(黎曼) \(\zeta\)函数}定义为: \(\zeta(s)=\sum_{n=1}^{\infty}\frac{1}{n^s}\). 设\(\{p_n\}_{n=1}^{\infty}\)为所有素数, 证明\textbf{Euler乘积公式}: \(\zeta(s)=\prod_{n=1}^{\infty}\frac{1}{1-1/p_n^s}\).
\begin{proof}

\end{proof}
\woe Euler最早根据\(\sin \pi x\)的零点为\(0,\pm 1,\pm 2,\cdots\)而猜测如下的\textbf{Euler公式}: \[\sin \pi x=\pi x\prod_{n=1}^{\infty}\left(1-\frac{x^2}{n^2}\right).\]试按如下步骤证明Euler公式:
\begin{quizs}
\item 对于\(n\geqslant 0\)以及\(t\in (-1,1)\), 证明:\[\sin\left((2n+1)\arcsin t\right)=(2n+1)t\prod_{k=1}^{n}\left(1-\frac{t^2}{\sin^2\frac{k\pi}{2n+1}}\right).\]
\item 证明: \(|x|\leqslant n+\frac{1}{2}\)时,\[\sin \pi x=(2n+1)\sin\frac{\pi x}{2n+1}\prod_{k=1}^{n}\left(1-\frac{\sin^2\frac{\pi x}{2n+1}}{\sin^2\frac{k\pi}{2n+1}}\right).\]
\item 固定\(x\in\mathbb{R}\), 对于\(m>K>|\pi x|\), 成立
\[\begin{split}
\prod_{k=K}^{m}\left(1-\frac{x^2}{k^2}\right)\cdot\prod_{k=m+1}^{\infty}\left(1-\frac{\pi^2x^2}{4k^2}\right)&\leqslant\varliminf_{n\rightarrow+\infty}\prod_{k=K}^{n}\left(1-\frac{\sin^2\frac{\pi x}{2n+1}}{\sin^2\frac{k\pi}{2n+1}}\right)\\&\leqslant\varlimsup_{n\rightarrow+\infty}\prod_{k=K}^{n}\left(1-\frac{\sin^2\frac{\pi x}{2n+1}}{\sin^2\frac{k\pi}{2n+1}}\right)\\&\leqslant\prod_{k=K}^{m}\left(1-\frac{x^2}{k^2}\right). \end{split}\]
\item 证明对任何\(x\in\mathbb{R}\), 成立\[\lim_{n\rightarrow+\infty}\prod_{k=1}^{n}\left(1-\frac{\sin^2\frac{\pi x}{2n+1}}{\sin^2\frac{k\pi}{2n+1}}\right)=\prod_{k=1}^{\infty}\left(1-\frac{x^2}{k^2}\right).\]
\end{quizs}
\begin{proof}

\end{proof}
\woe 证明: \(\cos \pi x=\prod_{n=0}^{\infty}\left(1-\frac{x^2}{\left(n+1/2\right)^2}\right)\).
\begin{proof}

\end{proof}
\end{quizb}