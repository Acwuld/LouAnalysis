\begin{titlepage}
\maketitle %首页
\end{titlepage}
\restoregeometry
\begin{center}
\Large 感悟
\end{center}
\noindent \Large 今
\normalsize
天是2023年5月20日,一个特别而又普通的日子:为什么说特别,因为我是单身,为什么普通?还是这个原因。这一天我读完了这套数学分析的第一遍,并将部分习题抄录至此,但大概有百分之七十的习题并没有弄懂,你可能觉得是时间仓促导致,但事实并非如此:笔者从22年8月开始阅读,至今已十月有余。我听闻难事可让人一夜白头,然这十月后我的头发依旧乌黑(只是掉了几根),这样比起来楼红卫十分温柔,或者还有一个原因:我还没努力到白头的程度。

我每天都乐此不疲的发现让我头疼的题目,并不是我喜欢头疼,若非迫不得已,我宁愿躺在床上睡大觉。但我每天都很苦恼,这让我几乎无法入眠:大佬做题犹如作画,随便一笔我都必须揣摩良久,最后只能承认自己的愚蠢,这是十分值得伤心的事,我也有理由尽情悲伤。但我尽量幽默一点,因为学习与生活永远不可能让我一直保持笑脸,但幽默总能。

这本书极不适合初学者,当然你如果天赋异禀,那本书能让你飞黄腾达。笔者大一大二学过一点数学分析等之类的课程,看这套书就像牙齿掉光的老太太吃五仁月饼,不过值得一提的是,许多都是一股脑吞下的,以至于事后问我这什么味道:干巴巴,没什么味儿。

我几乎要放弃的时候非常感谢各位群友的帮助,是大家让我间歇性放弃找女朋友的念头而是改为买一只宠物蜈蚣,相信很快就会和大家见面的!另外,明天开始就要开始要第二遍了。届时会完善这些习题。
\pagenumbering{Roman}
\tableofcontents
\newpage
\pagenumbering{arabic}
\setcounter{page}{0}
\setcounter{chapter}{-1}
\newcounter{mistake}
\renewcommand{\themistake}{\arabic{mistake}}
\newcommand{\mist}[1]{\par\textcolor{red}{\refstepcounter{mistake}\textbf{\themistake.} \textbf{Page #1 }}\hspace{0.5em}}
\chapter{闲谈}
\section{勘误}

\mist{21} 第4题(3)中对于\(R(x)\)的系数为有理数时结论才可被证明, 当\(R(x)\)中包含无理系数时结论不一定成立.
\mist{27} (F)(O)(C)错印为 (F)(D)(C).
\mist{281} 最后一行\(g\)应为\(f\).
\mist{196} \(f^{(n)}(x)\)后\(-1\)的幂次应为\(n\)而非\(n-1\).
\mist{299} 第二个行列式第一行第六列的元素应为24而非0.
\mist{315} 第七题. 
\mist{23}定理8.2.7中大括号.
\section{符号说明}
\begin{longtable}{ll}
\hline
\(\bbn\)&自然数集\\\hline
\(\bbz,\bbz_+\)&整数集,正整数集\\\hline
\(\bbq,\bbq_+\)&有理数域, 正有理数集\\\hline
\(\bbr,\bbr_+\)&实数域, 正实数集\\\hline
\(\bbr^n\)&\(n\)维欧氏空间\\\hline
\(\bbc\)&复数域\\\hline
\(\bbc^n\)&\(n\)维复空间\\\hline
\(S^{n-1}\)&\(\bbr^n\)中的单位球面, 即\(\{\boldsymbol{x}\in\bbr^n\big||x|=1\}\)\\\hline
\(\mathbb{S}^n\)&\(n\)阶实对称矩阵全体\\\hline
\(B_r(\boldsymbol{x})\)&半径为\(r\)中心在\(\boldsymbol{x}\in\bbr^n\)中的开球\\\hline
\(\mathring{B}_r(\boldsymbol{x})\)&半径为\(r\)中心在\(\boldsymbol{x}\in\bbr^n\)中的去心开球\\\hline
\(Q_{\delta}(\boldsymbol{x})\)&边长为\(2\delta\)中心在\(\boldsymbol{x}\in\bbr^n\), 且各边平行于坐标轴的开正方体\\\hline
\(\boldsymbol{A}^\top,\boldsymbol{x}^\top\)&矩阵\(\boldsymbol{A}\), 向量\(\boldsymbol{x}\)的转置\\\hline
\(\boldsymbol{x}\cdot\boldsymbol{y}\)&\(\bbr^n\)中向量\(\boldsymbol{x}\)与\(\boldsymbol{y}\)的数量积, 也常常用\(\inp{x,y},\boldsymbol{x\top y}\)表示\\\hline
\(\inp{x,y}\)&内积空间中两个元素\(x,y\)的内积\\\hline
\(\left\|\boldsymbol{x}\right\|_p\)&\(\bbr^n\)中的向量\(\boldsymbol{x}=(x_1,x_2,\cdots,x_n)^\top\)的\(p-\)范数\(\left(\sum_{k=1}^{n}|x_k|^p\right)^p\)\phantom{\huge\(\int\)}\\\hline
\(\left\|\boldsymbol{A}\right\|_p\)&方阵\(\boldsymbol{A}\bbr^{n\times n}\)的诱导范数\(\left\|\boldsymbol{A}\right\|_p=\max_{\left\|\boldsymbol{x}\right\|=1}\left\|\boldsymbol{Ax}\right\|_p\)\\\hline
\(|\boldsymbol{x}|\)&\(\bbr^n\)中向量\(\boldsymbol{x}\)通常的范数, 即\(\left\|\boldsymbol{x}\right\|_2\)\\\hline
\(\left\|\boldsymbol{A}\right\|\)&方阵\(\boldsymbol{A}\bbr^{n\times n}\)通常的诱导范数\(\left\|\boldsymbol{A}\right\|_2\)\\\hline
\(|E|\)&\(\bbr^n\)中集合\(E\)的Lebesgue测度(或Jordan测度)---长度, 面积, 体积\\\hline
\(|E|^*,|E|_*\)&\(\bbr^n\)中集合\(E\)的Jordan外测度, 内测度\\\hline
\(m^*E,m_*E\)&\(\bbr^n\)中集合\(E\)的Lebesgue外测度, 内测度\\\hline
\(a^+.a_-\)&实数\(a\)的正部\(\left(|a|+a\right)/2\)与负部\(\left(|a|-a\right)/2\)\\\hline
\(a\vee b,a\wedge b\)&实数\(a,b\)的最大值和最小值\\\hline
\(\Re z,\Im z\)&复数\(z=a+b\ii\)的实部\(a\)和虚部\(b\), 其中\(a,b\)为实数\\\hline
\(\chi_E\)&集合\(E\)的特征函数, 既在\(E\)上取值为\(1\), 在其余点取值为\(0\)\\\hline
\(\exists\)&存在\\\hline
\(\forall\)&对于任意\\\hline
\(\gg,\ll\)&大大大于,大大小于\\\hline
\(\mathrm{a.e.}\)&几乎处处\\\hline
\(\mathrm{s.t.}\)&满足或使得\\\hline
\(\varnothing\)&空集\\\hline
\(\in,\ni,\notin\)&\(a\in E\)与\(E\ni a\)均表示\(a\)是\(E\)的元素, \(a\notin E\)表\(a\)不是\(E\)中的元素\\\hline
\(\subseteq,\supseteq\)&\(E\subseteq F\)与\(F\supseteq E\)均表示\(E\)包含于集合\(F\), 即\(F\)包含\(E\)\\\hline
\(E\{\varphi\in F\}\)&表示集合\(\{x\in E\big|\varphi(x)\in F\}\). 在\(E\)明确的情况下, 简记为\(\{\varphi\in F\}\)\\\hline
\(f(D)\)&当\(f\)是映射, \(D\)是集合时, 表示\(D\)的像集\(\{f(x)\big|x\in D\}\)\\\hline
\(\cap\)&集合的交. \(A\cap B\)表示同时属于\(A\)和\(B\)的所有元素组成的集合\\\hline
\(\cup\)&集合的并. \(A\cup B\)表示属于\(A\)或属于\(B\)的所有元素组成的集合\\\hline
\(\backslash\)&集合的差. \(A\backslash B\)表示属于\(A\)但不属于\(B\)的所有元素组成的集合\\\hline
\(\mathscr{C}\)&集合的补. \(\mathscr{C}E\)表示在全集\(X\)明确的情况下, \(E\)的补集\(X\backslash E\)\\\hline
\(E^{\circ}\)&集合\(E\)的内部, 即\(E\)的全体内点\\\hline
\(E'\)&集合\(E\)的导集, 即\(E\)的极限点(聚点)的全体\\\hline
\(\overline{E}\)&集合\(E\)的闭包\\\hline
\(\partial E\)&集合\(E\)的边界\\\hline
\(\subset\subset\)&集合的紧包含关系, \(E\subset\subset F\)当且仅当\(\overline{E}\)是\(F\)的紧子集\\\hline
\(\alpha E+\beta F\)&线性空间中, 集合的伸缩, 代数和与代数差等, 表示集合\(\{\alpha x+\beta y\big|x\in E,y\in F\}\)\\\hline
\(\sum,\prod\)&连加号, 连乘号\\\hline
\([x],{x}\)&实数\(x\)的正数部分(即不大于\(x\)的最大整数)与小数部分(即\(x-[x]\))\\\hline
\(\varlimsup,\varliminf\)&上极限, 下极限\\\hline
\(\upint,\lowint\)&上积分符号, 下积分符号\phantom{\Large\(\int\)}\\\hline
\(C_n^k\)&在\(n\)个元素中选取\(k\)个的组合数\\\hline
\(C^k(\Omega)\)&在\(\Omega\)上有\(k\)阶连续(偏)导数的函数全体\\\hline
\(C_c^k(\Omega)\)&在\(\Omega\)上有紧支集且有\(k\)阶连续(偏)导数的函数全体\\\hline
\(C^{k,\alpha}(\Omega)\)&在\(\Omega\)上\(k\)阶(偏)导数满足\(\alpha\)次H\''{o}lder条件的函数全体\\\hline
\(\mathscr{S} \)&速降函数全体\\\hline
\(\hat{f}\check{f}\)&函数\(f\)的Fourier变换, Fourier逆变换\\\hline
\(f*g\)&函数\(f\)与\(g\)的卷积\\\hline
\end{longtable}





